\section{Semantics}
\label{sec:semantics}

\begin{figure}
\[\begin{array}{c}
 \begin{array}{rrclcl}
  \text{Variables}    & \px, \py & ∈ & \Var        & \simeq & ℕ \\
  \text{Labels}       &     \lbl & ∈ & \Labels     & \simeq & ℕ \\
  \text{Values}       &      \pv & ∈ & \Val        & ::=    & \Lam{\px}{\pe} \\
  \text{Expressions}  &      \pe & ∈ & \Exp        & ::=    & \slbl \px \mid \slbl \pv \mid \slbl \pe~\px \mid \slbl \Let{\px}{\pe_1}{\pe_2} \\
  \\[-0.5em]
  \text{States}        & σ & ∈ & \States        & =   & \Control \times \Environments \times \Heaps \times \Continuations \\
  \text{Controls}      & γ & ∈ & \Control       & ::= & \pe \mid (\pv, v) \\
  \text{Environments}  & ρ & ∈ & \Environments  & =   & \Var \pfun \Addresses \\
  \text{Addresses}    &      \pa & ∈ & \Addresses  & \simeq & ℕ \\
  \text{Heaps}         & μ & ∈ & \Heaps         & =   & \Addresses \pfun \Exp \times \Environments \times \StateD \\
  \text{Continuations} & κ & ∈ & \Continuations & ::= & \StopF \mid \ApplyF(\pa) \pushF κ \mid \UpdateF(\pa) \pushF κ \\
  \text{Parameters}    & \multicolumn{5}{l}{d∈D,~v∈V} \\
 \end{array} \\
  \\[-0.5em]
\end{array}\]

\newcolumntype{L}{>{$}l<{$}} % math-mode version of "l" column type
\newcolumntype{R}{>{$}r<{$}} % math-mode version of "r" column type
\newcolumntype{C}{>{$}c<{$}} % math-mode version of "c" column type
\begin{tabular}{LRCLL}
\toprule
\text{Rule} & σ_1 & \smallstep & σ_2 & \text{where} \\
\midrule
\LetT & (\Let{\px}{\pe_1}{\pe_2},ρ,μ,κ) & \smallstep & (\pe_2,ρ',μ[\pa↦(\pe_1,ρ',d_1)], κ) & \pa \not∈ \dom(μ),\ ρ'\! = ρ[\px↦\pa] \\
\AppIT & (\pe~\px,ρ,μ,κ) & \smallstep & (\pe,ρ,μ,\ApplyF(\pa) \pushF κ) & \pa = ρ(\px) \\
\LookupT & (\px, ρ, μ, κ) & \smallstep & (\pe, ρ', μ, \UpdateF(\pa) \pushF κ) & \pa = ρ(\px),\ (\pe,ρ',\wild) = μ(\pa) \\
\ValueT & (\pv, ρ, μ, κ) & \smallstep & ((\pv, v), ρ, μ, κ) & \\
\AppET & ((\Lam{\px}{\pe},\wild),ρ,μ, \ApplyF(\pa) \pushF κ) & \smallstep & (\pe,ρ[\px ↦ \pa],μ,κ) &  \\
\UpdateT & ((\pv,v), ρ, μ, \UpdateF(\pa) \pushF κ) & \smallstep & ((\pv,v), ρ, μ[\pa ↦ (\pv,ρ,d)], κ) & \\
\bottomrule
\end{tabular}
\caption{Syntax of $Λ$ and Lazy Krivine (LK) transition semantics $\smallstep$}
  \label{fig:lk-syntax}
\end{figure}

% Note [On the role of labels]
% ~~~~~~~~~~~~~~~~~~~~~~~~~~~~
% Simon does not like program labels; he'd much rather just talk about
% the expressions those labels are attached to. Yet it is important for
% the simplicity of analyses *not* to identify structurally equivalent
% sub-expressions, example below.
%
% Our solution is to assume that every expression is implicitly labelled
% (uniquely, so). When we use the expression as an index, we implicitly use the
% label as its identity rather than its structure. When labels are explicitly
% required, we can obtain them via the at() function.
%
% When do we *not* want structural equality on expressions? Example:
%
%   \g -> f a + g (f a)
%
% Now imagine that given a call to the lambda with an unknown `g`, we'd like to
% find out which subexpressions are evaluated strictly. We could annotate our
% AST like this
%
%   \g -> ((f^S a^L)^S + (g (f^L a^L)^L)^S)^S
%
% Note how the two different occurrences of `f a` got different annotations.
%
% For obvious utility (who wants to redefine the entire syntax for *every*
% analysis?), we might want to maintain the analysis information *outside* of
% the syntax tree, as a map `Expr -> Strictness`. But doing so would conflate
% the entries for both occurrences of `f a`! So what we rather do is assume
% that every sub-expression of the syntax tree is labelled with a unique token
% l∈Label and then use that to maintain our external map `Label -> Strictness`.
%
% We write ⌊Expr⌋ to denote the set of expressions where labels are "erased",
% meaning that structural equivalent expressions are identified.
% The mapping `Label -> Expr` is well-defined and injective; the mapping
% `Label -> ⌊Expr⌋` is well-defined and often *not* injective.
% Conversely, `at : Expr -> Label` extracts the label of an expression, whereas
% `⌊Expr⌋ -> Label` is not well-defined.
%
% Note that as long as a function is defined by structural recursion over an
% expression, we'll never see two concrete, structurally equivalent expressions,
% so it's OK to omit labels and use the expression we recurse over (and its immediate
% subexpressions captured as meta variables) as if it contained the omitted labels.

\subsection{Labelled Syntax}

\Cref{fig:lk-syntax} defines syntax and semantics of $Λ$: An untyped,
call-by-name lambda calculus with recursive let bindings in the style of
\citep{Launchbury:93} and \citep{Sestoft:97}.
Any (sub-)expression of $Λ$ has a unique \emph{label} (think of it as the AST node's
pointer identity) that we usually omit. For example, a correct labelling of
$f~(g~f)$ would be
\[
  (\slbln{1} (\slbln{2} f)~(\slbln{3}(\slbln{4} g)~(\slbln{5} f))).
\]
Labels are there so that we do not conflate the (otherwise structurally equal)
sub-terms $(\slbln{2} f)$ and $(\slbln{5} f)$ as equivalent. This is an important
distinction for, \eg, control-flow analysis. Since labels introduce excessive
clutter, we will omit them unless they are distinctively important. If anything,
labels make it so that everything ``works as expected''.

\subsection{Transition System}

An operational semantics of $Λ$ is given in terms of a small-step transition
system closest to the lazy Krivine machine \cite{AgerDanvyMidtgaard:04} for
Launchbury's language.
It is worth having a closer look at the workings of our Gold Standard.
The machine's state comes as a quadruple in the style of a CESK machine
\cite{Felleisen:87}, consisting of the usual \emph{control} component
corresponding to the control-flow node $γ$ under evaluation, the
\emph{environment} $ρ$ mapping lexically-scoped variables to an address bound in
the \emph{heap} $μ$ and a \emph{continuation}, or \emph{stack}, $κ$.
The notation $f ∈ A \pfun B$ used in the definition of $ρ$ and $μ$ denotes a
finite map from $A$ to $B$, a partial function where the domain $\dom(f)$ is
finite.
The literal notation $[a_1↦b_1,...,a_n↦b_n]$ denotes a finite map with domain
$\{a_1,...,a_n\}$ that maps $a_i$ to $b_i$. The extended finite map $f[a ↦ b]$
maps $a$ to $b$ and is otherwise equal to $f$.

The semantics differs from the standard in one way: It is parameterised over
occurrences of $d ∈ D$ and $v ∈ V$ which are meant to be \emph{elaborated} later
on. For now, we can assume the \emph{standard instantiation}, where both
parameters are instantiated to the unit type, $D \triangleq V \triangleq
\mathbb{1} \triangleq \{ () \}$.

The control component $γ$ is either an expression under evaluation $\pe$ or a
value pairing $(\pv,v)$ (remember, $v = ()$).
When the control of a state $σ$ is an expression $\pe$, we call $σ$ an
\emph{evaluation} state and say that $\pe$ drives evaluation, whereas when the
control is $(\pv, v)$ we call it a \emph{return} state in which the continuation
$κ$ drives evaluation.
The entries in the heap are \emph{closures} of the form $(e,ρ,d)$, where the
environment $ρ$ closes over the expression $e$ ($d = ()$).
Finally, the continuation $κ$ lists actions to be taken when the control reaches
a value, such as applying the result to an argument address or updating a heap
entry with its value.

Heap entries are introduced via $\LetT$ transitions under a \emph{fresh} address
$\pa \not∈ \dom(μ)$ that we call an \emph{activation} of the let-bound variable
$\px$. The lexical activation of every variable in scope is maintained
in $ρ$. The $\AppIT$ rule pushes an \emph{application frame} with the address of
the argument variable onto the stack, while the rule $\LookupT$ pushes an
\emph{update frame} with the address of the variable the heap entry of which is
accessed. When a return state is reached, the original heap entry is overwritten
with the value in the control.
An evaluation state transitions to a return state via rule $\ValueT$ when the
control expression is a value.%
\footnote{Operational semantics commonly don't explicate $\ValueT$
transitions, but the original formulation of the CESK machine
did~\cite{Felleisen:87}, as does the lazy Krivine
machine of \citep{AgerDanvyMidtgaard:04}.}

We bake into $σ$ the invariant of \emph{well-addressedness}: Any address $\pa$
occuring in $ρ$, $κ$ or the range of $μ$ must be an element of $\dom(μ)$.
It is easy to see that the transition system maintains this invariant and that
it is still possible to observe scoping errors which are thus confined to
lookup in $ρ$.

Note that whatever parameter sets $D,V$ we pick, the value of parameters $v$ and
$d$ never affect other state components or whether or not a transition can fire.
The transition system is deterministic iff the choice of the particular
$d$ or $v$ in rules $\LetT$, $\ValueT$ and $\UpdateT$ is. That is the case for
the proposed standard instantiation of $D$ and $V$ to the unit type.
Moreover, any deterministic instantiation yields a transition system that is
bisimilar to the standard one.

Let us conclude with an example trace in this transition system, evaluating
$\pe \triangleq \Let{i}{\Lam{x}{x}}{i~i}$ to completion:
\[\begin{array}{c}
  \arraycolsep2pt
  \begin{array}{clclclcl}
             & (\pe, [], [], \StopF)            & \smallstep & (i~i, ρ_1, μ, \StopF)
                & \smallstep & (i, ρ_1, μ, κ_1) & \smallstep & (\Lam{x}{x}, ρ_1, μ, κ_2)
                \\
  \smallstep & ((\Lam{x}{x}, ()), ρ_1, μ, κ_2)  & \smallstep & ((\Lam{x}{x}, ()), ρ_1, μ, κ_1)
                & \smallstep & (x, ρ_2, μ, \StopF) & \smallstep & (\Lam{x}{x}, ρ_1, μ, κ_3)
                \\
  \smallstep & ((\Lam{x}{x}, ()), ρ_1, μ, κ_3)  & \smallstep & ((\Lam{x}{x}, ()), ρ_1, μ, \StopF) \\
  \end{array} \\
  \\[-0.5em]
  \quad \text{where} \quad \begin{array}{ll}
  ρ_1 = [i ↦ \pa_1] & κ_1 = \ApplyF(\pa_1) \pushF \StopF \\
  ρ_2 = [i ↦ \pa_1, x ↦ \pa_1] & κ_2 = \UpdateF(\pa_1) \pushF κ_1 \\
  μ = [\pa_1 ↦ (\Lam{x}{x},ρ_1,())] & κ_3 = \UpdateF(\pa_1) \pushF \StopF \\
  \end{array} \\
\end{array}\]


\begin{figure}
\[\begin{array}{c}
 \begin{array}{rrclcl}
  \text{LK traces}                    & π      & ∈          & \STraces & ::=_{\gfp} & σ\trend \mid σ; π \\
  \text{Domain of LK trace semantics} & D      & \triangleq & \StateD  & = & \States \to \STraces \\
  \text{Values}                       & V      & \triangleq & \Values^\States & ::= & \FunV(f ∈ \Addresses \to \StateD) \\
 \end{array} \\
 \\
 \begin{array}{rcl}
  \multicolumn{3}{c}{ \ruleform{ src_\States(π) = σ \qquad tgt_\States(π) = σ } } \\
  \\[-0.5em]
  src_\States(σ\trend)    & = & σ \\
  src_\States(σ; π) & = & σ \\
  \\[-0.5em]
  tgt_\States(π)    & = & \begin{cases}
    undefined & \text{if $π$ infinite} \\
    σ         & \text{if $π = ...; σ \trend$}
  \end{cases} \\
 \end{array} \quad
 \begin{array}{c}
  \ruleform{ π_1 \sconcat π_2 = π_3 } \\
  \\[-0.5em]
  π_1 \sconcat π_2 = \begin{cases}
    σ; (π_1' \sconcat π_2) & \text{if $π_1 = σ; π_1'$} \\
    π_2                    & \text{if $π_1 = σ\trend$ and $src_\States(π_2) = σ$} \\
    undefined              & \text{if $π_1 = σ\trend$ and $src_\States(π_2) \not= σ$} \\
  \end{cases} \\
 \end{array} \\
 \\
 \begin{array}{c}
  \ruleform{ \validtrace{π} } \\
  \\[-0.5em]
  \inferrule*
    {\quad}
    {\validtrace{σ\trend}}
  \qquad
  \inferrule*
    {σ \smallstep src_\States(π) \quad \validtrace{π}}
    {\validtrace{σ; π}} \\
  \\
 \end{array} \\
\end{array}\]
\caption{LK traces and the domain of LK trace semantics}
  \label{fig:lk-traces}
\end{figure}

\subsection{Characterisation of LK Traces}

A transition system is characterised precisely by the set of \emph{traces} it
generates.
A trace in $(\smallstep)$ is a non-empty and potentially infinite sequence of
states $(σ_i)_{i∈\overline{n}}$ (where $\overline{n} = \{ m \mid m ≤ n \}, n∈ℕ_+
∪ \{ω\}$), such that $σ_i \smallstep σ_{i+1}$ for $i,(i+1)∈\overline{n}$.
\Cref{fig:lk-traces} gives a (coinductive) definition of such a trace type
$\STraces$ which is to be understood as the greatest fixed-point of the
functional $F(X) = \States + (\States \times X)$.
This is isomorphic to the definition as a sequence, hence we will freely
switch between notations for $π$, in particular to index its states.

The well-formedness of a trace $π$ wrt. to $(\smallstep)$ is asserted by
the coinductive predicate $\validtrace{π}$.
When a trace is finite, we often re-associate the right-associative structure
of $;$ to the left and decompose it as $π'; σ\trend$, where $σ$ is the last,
or \emph{target}, state of $π$, and $π'$ is its initial component without the
target state.
The \emph{source} state $src_\States(π)$ can be computed for finite and infinite
traces, while the target state of a trace $tgt_\States(π)$ is only defined for
finite traces.
The expression $π_1 \sconcat π_2$ denotes the concatenation of two traces; it is
simply $π_1$ when $π_1$ is infinite and undefined when the target state of $π_1$
does not coincide with the source state of $π_2$.

An important kind of trace is one that never leaves the evaluation context of its
source state:

\begin{definition}[Convex and balanced traces]
  An LK trace $π = (e_1,ρ_1,μ_1,κ_1); ... (e_i,ρ_i,μ_i,κ_i); ... $ is
  \emph{convex} if $\validtrace{π}$ and every intermediate continuation $κ_i$
  extends $κ_1$ (so $κ_i = κ_1$ or $κ_i = ... \pushF κ_1$).

  Furthermore, a convex trace $π$ is \emph{balanced} \cite{Sestoft:97} if the
  target state is a return state with continuation $κ_1$.
\end{definition}

\begin{example}
  Let $ρ=[x↦\pa_1],μ=[\pa_1↦(\Lam{y}{y},[],())]$ and $κ$ an arbitrary
  continuation. The trace
  \[
     (x, ρ, μ, κ) \smallstep (\Lam{y}{y}, ρ, μ, \UpdateF(\pa_1) \pushF κ) \smallstep ((\Lam{y}{y},()), ρ, μ, \UpdateF(\pa_1) \pushF κ) \smallstep ((\Lam{y}{y},()), ρ, μ, κ)
  \]
  is convex and balanced. Its prefixes are convex but not balanced. The suffix
  \[
     ((\Lam{y}{y},()), ρ, μ, \UpdateF(\pa_1) \pushF κ) \smallstep ((\Lam{y}{y},()), ρ, μ, κ)
  \]
  is neither convex nor balanced.
\end{example}


A balanced trace starting at a focus expression $\pe$ and ending with $(\pv,v)$
loosely corresponds to a derivation of $\pe \Downarrow \pv$ in a natural
big-step semantics~\cite{Sestoft:97} or a non-$⊥$ result in a denotational
semantics.

It is when a derivation in a natural semantics does not exist that a small-step
semantics shows finesse, in that it differentiates two different kinds of
\emph{maximally convex} (or, just \emph{maximal}) traces:

\begin{definition}[Maximal trace]
  An LK trace $π$ is \emph{maximal} if and only if it is convex and there is no $σ$ such
  that $π; σ \trend$ is convex.
  We notate maximal traces as $\maxtrace{π}$.
\end{definition}

We call infinite and convex traces \emph{diverging}.
A maximally finite, but unbalanced trace $π$ is called \emph{stuck}.
\footnote{Note that usually stuckness is associated with a state of a transition
system rather than a trace. That is not possible in our framework; depending on
the trace of which the state is a target, it might either be balanced or stuck.}

\begin{example}[Stuck and diverging traces]
The following trace is stuck because $x$ is not in scope:
\[
  (x~y,[y↦\pa], [\pa↦...], κ); (x,[y↦\pa], [\pa↦...], \ApplyF(\pa) \pushF κ)\trend
\]
An example for a diverging trace where $ρ=[x↦\pa_1]$ and $μ=[\pa_1↦(x,ρ,())]$ is
\[
  (\Let{x}{x}{x}, [], [], κ) \smallstep (x, ρ, μ, κ) \smallstep (x, ρ, μ, \UpdateF(\pa_1) \pushF κ) \smallstep ...
\]
\end{example}

A maximal trace that is not balanced either diverges or is stuck:

\begin{lemma}[Characterisation of maximal traces]
  A trace $π$ is maximal if and only if it is balanced, diverging or stuck.
\end{lemma}
\begin{proof}
  $\Rightarrow$: Let $π$ be maximal.
  If $π$ is infinite, then it is diverging due to convexity, and if $π$ is
  stuck, the goal follows immediately. So we assume that $π$ is maximal, finite
  and not stuck, so it must be balanced by the definition of stuckness.

  $\Leftarrow$: Both balanced and stuck traces are maximal.
  A diverging trace $π$ is infinite and convex.
  Indeed $π$ is maximal, because the expression $π; σ\trend$ is undefined for
  infinite $π$.
\end{proof}

The semantics of an expression $\pe$ is determined completely by the set of
maximal traces which start with $\pe$ as focus, and its kinship with natural
and denotational semantics makes it a nice substrate for our framework.

% Introducing elaborated paramterisation

Finally, \Cref{fig:lk-traces} lists definitions for the parameter space $D$ and
$V$ as function values that we will adopt from now on.
The keen reader may note immediately that $\StateD$ is not well-defined because
of a recursive occurrence in negative position (through $\States \to \Heaps \to
\StateD$); we discuss the necessary domain theory in \Cref{sec:domain-theory}
and focus on its application here.

% The type of S / applying denotations

Let us work out an informal specification of our trace-generating function,
$\semst{\pe}$.
Our goal is to give a structural definition of $\semst{\pe}$ that produces
traces that start out having $\pe$ in control.
Since there are many such states, corresponding to the different evaluation
contexts in which $\pe$ might occur, $\semst{\pe}$ should receive as its first
argument the source state $σ$ for which we wish to generate a trace, hence
$\semst{\pe} ∈ \States → \STraces$.
We will call $\semst{\pe}$ the \emph{denotation} of $\pe$ (a term that is
justified when we flesh out its properties relative to $\pe$) and abbreviate
the set of denotations to $\StateD$, realising that we have at our hands a $d ∈
D$ which our elaborated semantics is parameterised over.

% Motivating convexity of the returned trace

When $d$ is the denotation of $\pe$, then for every LK state $σ$ where the
control is $\pe$, $d(σ)$ should be a well-formed trace (so $\validtrace{d(σ)}$)
with $σ$ as its source state. In itself, that is a trivial specification,
because $d(σ) = σ \trend$ is a valid but hardly useful model. So we refine:
Every such trace $d(σ)$ must follow the evaluation of $\pe$. For example, given
a (sub-)expression $\Lam{x}{x}$ of some program $\pe$, we expect the following
equation:
\[
  \semst{\Lam{x}{x}} (\Lam{x}{x}, ρ, μ, κ) = (\Lam{x}{x}, ρ, μ, κ); ((\Lam{x}{x},\FunV(f)), ρ, μ, κ) \trend
\]
Note that even if $κ$ had an apply frame on top, we require $\semst{\wild}$ to
stop after the value transition. In general, each intermediate continuation
$κ_i$ in the produced trace will be an extension of the first $κ$, so
$κ_i = ... \pushF κ$. We call traces with this property \emph{convex}:

Why produce convex traces rather than letting the individual traces
eat their evaluation contexts to completion? Ultimately, we think it is just
a matter of taste, but certainly it is simpler to identify and talk about
the \emph{semantic value} $v ∈ \Values^\States$ of $\Lam{x}{x}$ this way: In
a trace $\semst{\Lam{x}{x}}(σ)$ it is always the second component of the target
state's control, regardless of the evaluation context encoded in $σ$, and the
semantic value determines how evaluation continues.

% Semantic values

The $\ValueT$ transition pairs up the syntactic lambda value with an associated
\emph{semantic value} $\FunV(f)$, the $f$ of which will involve the semantics of
the sub-program $\semst{x}$. The transition semantics is indifferent to which
$v$ is put in the control, but for $\semst{\wild}$ we specify that $f$ continues
where our balanced trace left off. More precisely, we require the $f$ in a
semantic value $\FunV(f)$ to
closely follow β-reduction of the syntactic value $\Lam{x}{x}$ with the address
of the argument variable $\pa$, as follows
\[
  f(\pa)((\Lam{x}{x},\FunV(f)), ρ, μ, \ApplyF(\pa) \pushF κ) = ((\Lam{x}{x},\FunV(f)), ρ, μ, \ApplyF(\pa) \pushF κ); \semst{x}(x, ρ[x ↦ \pa], μ, κ)
\]
That is, semantic values pop exactly one frame from the stack and then produce
another balanced trace of the redex. This statement extends to other kinds of
values such as data constructors, as we shall see in \cref{ssec:adts}.

Clearly, the syntactic value $\Lam{x}{x}$ determines $f$ one-to-one.
There occurrence of $d$ in a heap entry $(\pe, ρ, d)$ is similarly required to
determine $\pe$ one-to-one, even .

is another occurrence of semantic denotations $d$ in the heap that is
uninstantiated by the transition system. If $(\pa ↦ (\pe, ρ, d)) ∈ μ$, then we
specify that $d = \semst{\pe}$. Since $\UpdateT$ modifies heap entries,
the corresponding denotation $d$ of the heap entry will have to be updated
accordingly to maintain this invariant.

Finally, it is a matter of good hygiene that $\semst{\pe}$ continues \emph{only}
those states that have $\pe$ as control; it should be undefined on any
other expression of the program.
So even if $x~x$ and $y~y$ both structurally are well-scoped application
expressions, $\semst{x~x}(y~y,ρ,μ,κ)$ is undefined.
In other words, if $(\pe_i)_i$ enumerates the (labelled) sub-expressions of a
program $\pe$, then for every state $σ$ that is not stuck in the transition
semantics there is at most one function $\semst{\pe_i}$ that is defined on $σ$.
Take note that a correct labelling is crucial for this point. In an expression
like $f (g f)$ the two, structurally equal occurrences of $f$ are distinct
sub-expressions which would otherwise have the same image under $\semst{\wild}$.

\subsection{Formal Specification}

A stuck trace is one way in which $\semst{\wild}$ does not \emph{always} return
a balanced trace. The other way is when the trace diverges before yielding to
the initial continuation. We call a trace \emph{maximal} if it is a
valid CESK trace and falls in either of the three categories. More precisely:

\begin{definition}[Maximal trace]
  An LK trace $π$ is \emph{maximal} iff it is convex and is either
  infinite or there is no $σ$ such that $π \sconcat (tgt_\States(π); σ \trend)$
  is convex.
  We notate maximal traces as $\maxtrace{π}$.
\end{definition}

\begin{lemma}[Extension relation on continuations is prefix order]
  The binary relation on continuations
  \[
    (\contextendsflipped) = \{ (κ_1,κ_2) ∈ \Continuations \times \Continuations \mid \text{$κ_2 = κ_1$ or $κ_2 = ... \pushF κ_1$} \}
  \]
  is a prefix order.
\end{lemma}
\begin{proof}
  $(\contextendsflipped)$ is isomorphic to the usual prefix order on strings
  over the alphabet of continuation frames.
\end{proof}

\begin{definition}[Return and evaluation states]
  $σ$ is a \emph{return state} if its control is of the form $(\pv, v)$.
  Otherwise, the control is of the form $\pe$ and $σ$ is an \emph{evaluation} state.
\end{definition}

The final states of the CESK semantics are the return states with a $\StopF$
continuation.
A state $σ$ that cannot make a step (so there exists no $σ'$ such that $σ
\smallstep σ'$) and is not a final state is called \emph{stuck}.

A trace is stuck if its destination state is stuck.

In other words, a trace is maximal if it follows the transition
semantics, never leaves the initial evaluation context and is either infinite,
stuck, or successfully returns a value to its initial evaluation context.
The latter case intuitively corresponds to the existence of derivation in a
suitable big-step semantics.

For the initial state with an empty stack, the maximal trace is
exactly the result of running the transition semantics to completion.

Unsurprisingly, infinite traces are never stuck or balanced.
We could also show that no balanced trace is ever stuck, thus proving the three
categories disjoint. Unfortunately, the introduction of algebraic data types
gives rise to balanced but stuck traces, so we won't make use of that fact in
the theory that follows.

\Cref{ex:stuck} gives an example program that is stuck because of an
out-of-scope reference. This is in fact the only way in which our simple lambda
calculus without algebraic data types can get stuck; it is useful to limit such
out-of-scope errors to the lookup of program variables in the environment $ρ$
and require that lookups in the heap $μ$ are always well-defined, captured in
the following predicate:

\begin{definition}[Well-addressed]
  We say that
  \begin{itemize}
    \item An environment $ρ$ is \emph{well-addressed} \wrt
          a heap $μ$ if $\rng(ρ) ⊆ \dom(μ)$.
    \item A continuation $κ$ is \emph{well-addressed} \wrt
          a heap $μ$ if $\fa(κ) ⊆ \dom(μ)$, where
          \[\begin{array}{rcl}
            \fa(\ApplyF(\pa) \pushF κ) & = & \{ \pa \} ∪ \fa(κ) \\
            \fa(\UpdateF(\pa) \pushF κ) & = & \{ \pa \} ∪ \fa(κ) \\
            \fa(\StopF) & = & \{ \} \\
          \end{array}\]
    \item A heap $μ$ is \emph{well-addressed} if, for every entry $(\wild,ρ,\wild) ∈ μ$,
          $ρ$ is well-addressed \wrt $μ$.
    \item A state $(\wild, ρ, μ, κ)$ is \emph{well-addressed} if $ρ$
          and $κ$ are well-addressed with respect to $μ$.
  \end{itemize}
\end{definition}

\begin{lemma}[Transitions preserve well-addressedness]
  \label{lemma:preserve-well-addressed}
  Let $σ$ be a well-addressed state. If there exists $σ'$ such that $σ
  \smallstep σ'$, then $σ'$ is well-addressed.
\end{lemma}
\begin{proof}
  By cases over the transition relation.
\end{proof}

\begin{corollary}
  If $\validtrace{π}$ and $src_\States(π)$ is well-addressed, then every state
  in $π$ is well-addressed.
\end{corollary}

For $\semst{\wild}$ to continue a well-addressed state sensibly, we need to
relate the occurring denotations to their respective syntactic counterparts:

\begin{definition}[Well-elaborated]
  We say that a well-addressed state $σ$ is \emph{well-elaborated}, if
  \begin{itemize}
    \item For every heap entry $(\pe, ρ, d) ∈ \rng(μ)$, where $μ$ is the heap of
          $σ$, we have $d=\semst{\pe}$.
    \item If $σ$ is a return state with control $(\Lam{\px}{\pe}, \FunV(f))$,
          then for any state $σ'$ of the form $((\Lam{\px}{\pe}, \FunV(f)), ρ, μ, \ApplyF(\pa) \pushF κ)$,
          we have $f(\pa)(σ') = σ'; \semst{\pe}(\pe, ρ[\px ↦ \pa], μ, κ)$.
  \end{itemize}
\end{definition}
The initial state of $\smallstep$, $(\pe,[],[],\StopF)$, is well-elaborated,
because it is never a return state and its heap is empty.

Finally, we can give the specification for $\semst{\wild}$:

\begin{definition}[Specification of $\semst{\wild}$]
\label{defn:semst-spec}
Let $σ$ be a well-elaborated state and $\pe$ an arbitrary expression. Then
\begin{itemize}
  \item[(S1)] If the control of $σ$ is not $\pe$, then $\semst{\pe}(σ) = σ \trend$.
  \item[(S2)] $σ$ is the source state of the trace $\semst{\pe}(σ)$.
  \item[(S3)] If the control of $σ$ is $\pe$, then
              $\maxtrace{\semst{\pe}(σ)}$ and all states in the trace
              $\semst{\pe}(σ)$ are well-elaborated.
\end{itemize}
\end{definition}
% Urgh, flesh it out when we know better what to do with S1-S3
%The purpose of $(S1)$ is to guarantee that $\semst{\pe}$ does not continue:
%to make the mapping $\pe \mapsto \semst{\pe}$ injective
%and thus invertible.
%\begin{lemma}
%  Let $E$ denote the set of the expression $\pe$ and its (labelled)
%  sub-expressions. Then the mapping $\fn{\pe}{\semst{\pe}}$ is injective.
%\end{lemma}
%\begin{proof}
%  It is simple to see that for every $\pe$ there exists $σ, σ'$ such that $\pe$
%  is the control of $σ$ and $σ \smallstep σ'$.
%  \sg{Do we need to prove this? I think not.}
%  Let $s(\pe)$ denote this state.
%  Then $\semst{\pe}(s(\pe)) \not= \semst{\pe'}(s(\pe))$ unless $\pe = \pe'$:
%
%\end{proof}
%Clearly, $(S3)$ is the key property for adequacy, but we can't prove it without
%the other 2 properties.

\subsection{Domain Theory}
\label{sec:domain-theory}

\subsubsection{Domain construction}

Note that $\StateD$ in \Cref{fig:lk-syntax} is not a well-formed inductive
definition: It recurses in negative position via $\States \to \Heaps \to \StateD$
and similarly through $\Values^\States$.
Hence we must understand $\StateD$ as a Scott domain and its ``definition'' as
a \emph{domain equation}, of which $\StateD$ is the least solution.

\sg{Some of the ponderings about finite maps should come earlier once we've
identified the right place to put them.}
For that to carry meaning, the RHS $\States \to_c \STraces$ must be the domain
of continuous functions between the hypothetical domains $\States$ and
$\STraces$, indicated by the subscript ${}_c$. The domain of finite maps
$f ∈ A \pfun_c B$ (for flat domain $A$ and arbitrary domain $B$) is the
sub-domain of $f ∈ A \to_c (\{\notfound\} + B)$ of strict functions (so $f(\bot)
= \bot$) where the map's domain $\dom(f) ⊆ A$ is finite. Whenever $a \not∈
\dom(f) \setminus \bot$, we have $f(a) = \notfound$.

Recall that that an element $d$ of some domain $D$ is a \emph{total} element iff
it is maximal, so there exists no $d' ∈ D$ such that $d ⊏ d'$. Otherwise if such
a $d'$ exists, $d$ is \emph{partial}.

Remarkably, we \emph{do not} conflate $\bot ∈ A \pfun_c B$ with $[] ∈ A \pfun_c
B$, the finite map with $\dom(f) = \varnothing$.
The former is a partial element of the domain and assigns $\bot_B$ everywhere;
the latter is totally defined as an element of the domain and assigns
$\notfound$ everywhere.
As a consequence, $[a ↦ b] \triangleq [][a ↦ b] \not\sqsupseteq []$ and the
total elements of the domain are discrete.

The other domain constructors for products $\times$ and inductive EBNF-style
syntax definitions such as the potentially infinite $\STraces$ (indicated by
the $\gfp$ subscript) or finite $\Continuations$ are standard, see for example
\cite{Cartwright:16}. The domains on $\Addresses$ and $\Var$ are the flat ones;
the one on $\Exp$ is the flat one on labels.

This proves that $\StateD$ is well-defined as a domain.

\subsubsection{Continuity}

It is easy to verify (by simple type checking) that $\semst{\wild}$
is indeed a function from elements of $\States$ to something close to the
elements of $\STraces$ (ignoring embedding of recursive calls for a moment).
The missing ingredient to show that $\semst{\wild} ∈ \StateD$ is well-defined is
\emph{continuity}.

As given, the definition of $\semst{\wild}$ does not account for partial
elements at all, although it does a sufficient job provided the elements it
manipulates are total. This is a common kind of formulation in functional
languages like Haskell, ML and other variants of the lambda calculus.

In fact, the supplied artifact gives a model implementation of $\semst{\wild}$
in Haskell. Note that this model must be computable; hence it must be possible
to reflect this model as an element of $\StateD$. Doing so was the entire point
of \citep{ScottStrachey:71} and we don't need to repeat it here.

Ironically, we are at a point where it's easier to reflect a code listing in a
sufficiently well-defined programming language into a continuous function than
to find a language-agnostic, mathematical formulation that we laboriously have
to prove continuous.
Now, we might spend much time and space here clarifying boring technical
details of strictness such as
``Is $step(stay_\StateD)(\wild,\wild,\wild,\ApplyF(\pa) \pushF \bot) = \bot$?''
and ``What is the continuous elaboration of $tgt_\States$?''
that are implicit in the Haskell formulation, only to finally conclude that the
clarified formulation is indeed continuous.

We skip this laborious process and for the remainder of this work assume that
$\semst{\wild}$ was implemented in a strict programming language like OCaml
(both for familiarity with most readers and sanity of evaluation order),
using a lazy field in one key position: The tail $π$ of the cons form $σ; π$ of
$\STraces$. For easy reference of the ``ground truth'', we give such an
implementation in the Appendix.%
\footnote{Note that the artifact is implemented in Haskell because of
familiarity to the authors, but makes abundant use of strictness annotations
for good measure, debuggability and performance. This demonstrates the semantic
leeway a denotational formulation such as $\semst{\wild}$ has while still being
continuous.}

\subsection{Definition}

\Cref{fig:sestoft-semantics} defines $\semst{\wild}$. Before we prove that
$\semst{\wild}$ satisfies the specification in \Cref{defn:semst-spec},
let us understand the function by way of evaluating the example program
$\pe \triangleq \Let{i}{\Lam{x}{x}}{i~i}$:
\[\begin{array}{l}
  \newcommand{\myleftbrace}[4]{\draw[mymatrixbrace] (m-#2-#1.west) -- node[right=2pt] {#4} (m-#3-#1.west);}
  \begin{tikzpicture}[baseline={-0.5ex},mymatrixenv]
      \matrix [mymatrix] (m)
      {
        1  & (\pe, [], [], \StopF); & \hspace{3.5em} & \hspace{3.5em} & \hspace{3.5em} & \hspace{4.5em} & \hspace{7.5em} \\
        2  & (i~i, ρ_1, μ, \StopF); & & & & & \\
        3  & (i, ρ_1, μ, κ_1); & & & & & \\
        4  & (\Lam{x}{x}, ρ_1, μ, κ_2); & & & & & \\
        5  & ((\Lam{x}{x}, \FunV(f)), ρ_1, μ, κ_2); & & & & & \\
        6  & ((\Lam{x}{x}, \FunV(f)), ρ_1, μ, κ_1); & & & & & \\
        7  & (x, ρ_2, μ, \StopF); & & & & & \\
        8  & (\Lam{x}{x}, ρ_1, μ, κ_3); & & & & & \\
        9  & ((\Lam{x}{x}, \FunV(f)), ρ_1, μ, κ_3); & & & & & \\
        10 & ((\Lam{x}{x}, \FunV(f)), ρ_1, μ, \StopF) \trend & & & & & \\
      };
      % Braces, using the node name prev as the state for the previous east
      % anchor. Only the east anchor is relevant
      \foreach \i in {1,...,\the\pgfmatrixcurrentrow}
        \draw[dotted] (m.east|-m-\i-\the\pgfmatrixcurrentcolumn.east) -- (m-\i-2);
      \myleftbrace{3}{1}{10}{$\semst{\pe}$}
      \myleftbrace{4}{1}{2}{$let(\semst{\Lam{x}{x}})$}
      \myleftbrace{4}{2}{10}{$\semst{i~i}$}
      \myleftbrace{5}{2}{3}{$app_1(i~i)$}
      \myleftbrace{5}{3}{6}{$\semst{i}$}
      \myleftbrace{5}{6}{7}{$app_2(\Lam{x}{x},\pa_1)$}
      \myleftbrace{5}{7}{10}{$\semst{x}$}
      \myleftbrace{6}{3}{4}{$look(i)$}
      \myleftbrace{6}{4}{5}{$\semst{\Lam{x}{x}}$}
      \myleftbrace{6}{5}{6}{$upd$}
      \myleftbrace{6}{7}{8}{$look(x)$}
      \myleftbrace{6}{8}{9}{$\semst{\Lam{x}{x}}$}
      \myleftbrace{6}{9}{10}{$upd$}
      \myleftbrace{7}{4}{5}{$ret(\Lam{x}{x},\FunV(f))$}
      \myleftbrace{7}{8}{9}{$ret(\Lam{x}{x},\FunV(f))$}
  \end{tikzpicture} \\
  \quad \text{where} \quad \begin{array}{ll}
  ρ_1 = [i ↦ \pa_1] & κ_1 = \ApplyF(\pa_1) \pushF \StopF \\
  ρ_2 = [i ↦ \pa_1, x ↦ \pa_1] & κ_2 = \UpdateF(\pa_1) \pushF κ_1 \\
  μ = [\pa_1 ↦ (\Lam{x}{x},ρ_1,\semst{\Lam{x}{x}})] & κ_3 = \UpdateF(\pa_1) \pushF \StopF \\
  f = \pa \mapsto step(app_2(\Lam{\px}{\px},\pa)) \sfcomp \semst{\px} \\
  \end{array} \\
\end{array}\]
The annotations to the right of the trace can be understood as denoting the
``call graph'' of the $\semst{\pe}$, with the primitive transition $step$s such
as $let_1$, $app_1$, $look$ \etc as leaves, each \emph{elaborating} the
application of the corresponding transition rule $\LetT$, $\AppIT$, $\LookupT$,
and so on, in the transition semantics with a matching denotation where
necessary.

Evaluation begins by decomposing the let binding and continuing the let body in
the extended environment and heap, transitioning from state 1 to state 2.
Beyond recognising the expected application of the $\LetT$ transition rule,
it is interesting to see that indeed the $d$ in the heap is elaborated to
$\semst{\Lam{x}{x}}$, which ends up in $ρ_1$.

The auxiliary $step$ function and the forward composition operator $\sfcomp$
have been inlined everywhere, as all steps and compositions are well-defined.
One can find that $\sfcomp$ is quite well-behaved: It forms a monoid with
$stay_\StateD$ (which is just $step$ applied to the function that is undefined
everywhere) as its neutral element.

Evaluation of the let binding recurses into $\semst{i~i}$ on state 2,
where an $\AppIT$ transition to state 3, on which $\semst{i}$ is run.
Note that the final state 6 of the call to $\semst{i}$ will later be fed
(via $\sfcomp$) into the auxiliary function $apply$.

$\semst{i}$ guides the trace from state 3 to state 6, during which we
observe a heap update for the first time: Not only does $look$ look up
the binding $\Lam{x}{x}$ of $i$ in the heap and push and update frame,
it also hands over control to the $d = \semst{\Lam{x}{x}}$ stored alongside it.
(And we make a mental note to pick up with $step(upd)$ when $d$ has finished.)
Crucially, that happens without $\semst{\Lam{x}{x}}$ occurring in the definition
of $\semst{i}$, thus the definition remains structural.

By comparison, the value transition governed by $\semst{\Lam{x}{x}}$ from state
4 to 5 is rather familiar from our earlier example, where we specified more or
less the same $\FunV(f)$.

$\semst{\Lam{x}{x}}$ finishes in state 5, where $upd$ picks up to guide the
$\UpdateT$ transition. Crucially, it also updates the $d$ stored in the heap
so that its semantics matches that of the updated value; take note of the
similarity to the lambda clause in the definition of $\semst{\wild}$. Since
$\Lam{x}{x}$ was a value to begin with, there is no observable change to the
heap.

After $\semst{i}$ concludes in state 6, the $apply$ function from $\semst{i~i}$
picks up. Since $apply$ is defined on state 6, it reduces to $f(\pa_1)$%
\footnote{We omitted $apply$ and $f(\pa_1)$ from the call graph for space reasons, but
their activation spans from state 7 to the final state 10.}.
$f(\pa_1)$ will perform an $\AppET$ transition to state 7, binding $x$ to $i$'s
address $\pa_1$ and finally entering the lambda body $\semst{x}$. Since $x$ is
an alias for $i$, steps 7 to 10 just repeat the same same heap update sequence
we observed in steps 3 to 6.

It is useful to review another example involving an observable heap update.
The following trace begins right before the heap update occurs in
$\Let{i}{(\Lam{y}{\Lam{x}{x}})~i}{i~i}$, that is, after reaching the value
in $\semst{(\Lam{y}{\Lam{x}{x}})~i}$:
\[\begin{array}{l}
  \newcommand{\myleftbrace}[4]{\draw[mymatrixbrace] (m-#2-#1.west) -- node[right=2pt] {#4} (m-#3-#1.west);}
  \begin{tikzpicture}[baseline={-0.5ex},mymatrixenv]
      \matrix [mymatrix] (m)
      {
        1  & ((\Lam{x}{x},\FunV(f)), ρ_2, μ_1, κ_2); & \hspace{3.5em} & \hspace{3.5em} & \hspace{4.5em} & \hspace{7.5em} \\
        2  & ((\Lam{x}{x},\FunV(f)), ρ_2, μ_2, κ_1); & & & & \\
        3  & (x, ρ_3, μ, \StopF); & & & & \\
        4  & (\Lam{x}{x}, ρ_2, μ, κ_3); & & & & \\
        5  & ((\Lam{x}{x}, \FunV(f)), ρ_2, μ, κ_3); & & & & \\
        6  & ((\Lam{x}{x}, \FunV(f)), ρ_1, μ, \StopF); & & & & \\
      };
      % Braces, using the node name prev as the state for the previous east
      % anchor. Only the east anchor is relevant
      \foreach \i in {1,...,\the\pgfmatrixcurrentrow}
        \draw[dotted] (m.east|-m-\i-\the\pgfmatrixcurrentcolumn.east) -- (m-\i-2);
      \myleftbrace{3}{1}{6}{$\semst{i~i}$}
      \myleftbrace{4}{1}{2}{$\semst{i}$}
      \myleftbrace{4}{2}{3}{$app_2(\Lam{x}{x},\pa_1)$}
      \myleftbrace{4}{3}{6}{$\semst{x}$}
      \myleftbrace{5}{1}{2}{$upd$}
      \myleftbrace{5}{3}{4}{$look(x)$}
      \myleftbrace{5}{4}{5}{$\semst{\Lam{x}{x}}$}
      \myleftbrace{5}{5}{6}{$upd$}
      \myleftbrace{6}{4}{5}{$ret(\Lam{x}{x},\FunV(f))$}
  \end{tikzpicture} \\
  \quad \text{where} \quad \begin{array}{ll}
  ρ_1 = [i ↦ \pa_1] & κ_1 = \ApplyF(\pa_1) \pushF \StopF \\
  ρ_2 = [i ↦ \pa_1, y ↦ \pa_1] & κ_2 = \UpdateF(\pa_1) \pushF κ_1 \\
  μ_1 = [\pa_1 ↦ ((\Lam{y}{\Lam{x}{x}})~i,ρ_1,\semst{(\Lam{y}{\Lam{x}{x}})~i})] & κ_3 = \UpdateF(\pa_1) \pushF \StopF \\
  μ_2 = [\pa_1 ↦ (\Lam{x}{x},ρ_2,\semst{\Lam{x}{x}})] & f = \pa \mapsto step(app_2(\Lam{\px}{\px},\pa)) \sfcomp \semst{\px} \\
  \end{array} \\
\end{array}\]
Note that both the environment \emph{and} the denotation in the heap is updated
in state 2, and that the new denotation is immediately visible on the next heap
lookup in state 3, so that $\semst{\Lam{x}{x}}$ takes control rather than
$\semst{(\Lam{y}{\Lam{x}{x}})~i}$, just as the transition system requires.

\subsection{Conformance}

Having a good grasp on the workings of $\semst{\wild}$ now, let us show that
$\semst{\wild}$ conforms to its specification in \Cref{defn:semst-spec}.

For the following three lemmas, let $σ$ denote a well-elaborated state and $\pe$
an expression.

\begin{lemma}[(S1)]
  If the control of $σ$ is not $\pe$, then $\semst{\pe}(σ) = σ\trend$.
\end{lemma}
\begin{proof}
  By the first clause of $\semst{\wild}$.
\end{proof}

\begin{lemma}[(S2)]
  $σ$ is the source state of $\semst{\pe}(σ)$.
\end{lemma}
\begin{proof}
  Trivial for the first clause of $\semst{\wild}$.
  Now, realising that $src_\States((l \sfcomp r)(σ)) = src_\States(l(σ))$
  and that $src_\States(step(f)(σ)) = σ$ for any $f$, we can see that the
  proposition follows for other clauses by applying these two rewrites to
  completion.
\end{proof}

\begin{lemma}[(S3)]
  If the control of $σ$ is $\pe$, then $\maxtrace{\semst{\pe}(σ)}$ and all
  states in the trace $\semst{\pe}(σ)$ are well-elaborated.
\end{lemma}
\begin{proof}
  Let us abbreviate the property of interest to
  \[
    OK(d) = ∀σ. σ\text{ well-addressed} \Longrightarrow \validtrace{d(σ)}
  \]
  Let $σ$ be a well-addressed state.
  For the first clause of $\semst{\wild}$, the proposition follows by
  $\validtraceTriv$.

  Every other clause of $\semst{\wild}$ is built from applications of
  $\sfcomp$, $step$ or $apply$ and it suffices to show that these combinators
  yield valid CESK traces when applied to $σ$.

  We can see that $OK(d_1 \sfcomp d_2)$ whenever $OK(d_1)$ and $OK(d_2)$ holds,
  because the destination state of $d_1(σ)$, if it exists, is well-addressed by
  \Cref{lemma:preserve-well-addressed} and $OK(d_2)$ can be usefully applied.
  An interesting case is when $d_1(σ)$ is infinite; then
  $(d_1 \sfcomp d_2)(σ) = d_1(σ)$ and $\validtrace{(d_1 \sfcomp d_2)(σ)}$ follows
  from $\validtrace{d_1(σ)}$.

  We can see that $\validtrace{step(f)(σ)}$ whenever $f(σ)$ is undefined
  (by $\validtraceTriv$) or $\validtrace{σ;f(σ)}$ (by $\validtraceTrans$).
  Enumerating the arguments to $step$, we can see that $val, upd, app_1, app_2$
  and $let$ follow $\ValueT, \UpdateT, \AppIT, \AppET$ and $\LetT$ closely by
  making exactly one transition. That leaves $loop$, which corresponds to doing
  one step with $\LookupT$ and then evaluating the heap-bound expression $\pe$
  as expressed by $d$.

  By (bi-)induction that is always the case.
\end{proof}

Let us begin by defining that a denotation $d$ is \emph{valid everywhere} when
for every initial $σ$, we have $\validtrace{σ}$. This definition lets us express
an important observation:

We can now formulate our correctness theorem as follows:

\begin{theorem}[Adequacy of the stateful trace semantics]
Let $d = \semst{\pe}$. Then for all $ρ,μ$ with $\vdash_\Heaps μ$, we have $μ
\vdash_\StateD \pe \sim_ρ d$.
\end{theorem}

\begin{corollary} Let $d = \semst{\pe}$. Then $π=d(\pe,[],[],\StopF)$ is a
maximal trace for the transition semantics starting at $(\pe,[],[],\StopF)$.
\end{corollary}

\begin{figure}
\[\begin{array}{c}
 \begin{array}{rcl}
  \multicolumn{3}{c}{ \ruleform{ \semst{\wild} \colon \Exp → \StateD } } \\
  \\[-0.5em]
  (d_1 \sfcomp d_2)(σ) & = & d_1(σ) \sconcat d_2(tgt_\States(d_1(σ))) \\
  \\[-0.5em]
  step(f)(σ) & = & \begin{cases}
    σ; f(σ)     & \text{if $f(σ)$ is defined} \\
    σ\trend & \text{otherwise} \\
  \end{cases} \\
  \\[-0.5em]
  val(\pv,v)(\pv,ρ,μ,κ) & = & ((\pv,v),ρ,μ,κ) \trend \\
  \\[-0.5em]
  look(\px)(\px,ρ,μ,κ) & = &
    \begin{letarray}
      \text{let} & \pa = ρ(\px) \\
                 & (\pe,ρ',d) = μ(\pa) \\
      \text{in}  & d(\pe,ρ',μ,\UpdateF(\pa) \pushF κ) \\
    \end{letarray} \\
  \\[-0.5em]
  upd((\pv,v),ρ,μ,\UpdateF(\pa) \pushF κ) & = & ((\pv,v),ρ,μ[\pa ↦ (\pv,ρ,step(val(\pv,v)))], κ)\trend \\
  \\[-0.5em]
  app_1(\pe~\px)(\pe~\px,ρ,μ,κ) & = & (\pe,ρ,μ,\ApplyF(ρ(\px)) \pushF κ)\trend \\
  \\[-0.5em]
  app_2(\Lam{\px}{\pe},\pa)((\Lam{\px}{\pe},\FunV(\wild)),ρ,μ, \ApplyF(\pa) \pushF κ) & = & (\pe,ρ[\px ↦ \pa],μ,κ) \trend \\
  \\[-0.5em]
  let(d_1)(\Let{\px}{\pe_1}{\pe_2},ρ,μ,κ) & = &
    \begin{letarray}
      \text{let} & ρ' = ρ[\px ↦ \pa] \quad \text{where $\pa \not∈ \dom(μ)$} \\
      \text{in}  & (\pe_2,ρ',μ[\pa ↦ (\pe_1, ρ', d_1)],κ)\trend \\
    \end{letarray} \\
  \\[-0.5em]
  apply(σ) & = & \begin{cases}
    f(\pa)(σ) & \text{if $σ=((\wild,\FunV(f)),\wild,\wild,\ApplyF(\pa) \pushF \wild)$} \\
    σ \trend & \text{otherwise} \\
  \end{cases} \\
  \\[-0.5em]
  % We need this case, otherwise we'd continue e.g.
  %   S[x]((sv,v),ρ,μ,upd(a).κ) = ((sv,v),ρ,μ,upd(a).κ); ((sv,v),ρ,μ[a...],κ)
  % Because of how the composition operator works.
  \semst{\pe}(σ) & = & undefined \text{ if the control of $σ$ is not $\pe$} \\
  \\[-0.5em]
  \semst{\px} & = & step(look(\px)) \sfcomp step(upd) \\
  \\[-0.5em]
  \semst{\Lam{\px}{\pe}} & = & \begin{letarray}
    \text{let} & f = \pa \mapsto step(app_2(\Lam{\px}{\pe},\pa)) \sfcomp \semst{\pe} \\
    \text{in}  & step(val(\Lam{\px}{\pe},\FunV(f))) \\
  \end{letarray} \\
  \\[-0.5em]
  \semst{\pe~\px} & = & step(app_1(\pe~\px)) \sfcomp \semst{\pe} \sfcomp apply \\
  \\[-0.5em]
  \semst{\Let{\px}{\pe_1}{\pe_2}} & = & step(let(\semst{\pe_1})) \sfcomp \semst{\pe_2} \\
 \end{array} \\
\end{array}\]
\caption{Structural call-by-need stateful trace semantics $\semst{-}$}
  \label{fig:sestoft-semantics}
\end{figure}

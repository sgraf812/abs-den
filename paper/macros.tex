% Theorems
\theoremstyle{plain} % Theorem/Definition/... in bold serif
\newtheoremrep{theorem}{Theorem}
\newtheoremrep{lemma}[theorem]{Lemma}
\newtheoremrep{corollary}[theorem]{Corollary}
\newtheorem{proposition}[theorem]{Proposition}
\newtheorem{example}[theorem]{Example}
\newtheorem{definition}[theorem]{Definition}
\newtheorem{axiom}[theorem]{Axiom}
\newtheorem{abbreviation}[theorem]{Abbreviation}
\crefname{proof}{Proof}{Proofs}

% Abbrev
\newcommand\ie{i.e.\xspace}
\newcommand\eg{e.g.\xspace}
\newcommand\cf{cf.\xspace}
\newcommand\etc{etc.\xspace}
\newcommand\wrt{wrt.\xspace}

% Comments and notes
\makeatletter
\newcommand\ifnonanon[1]{%
\if@ACM@anonymous
\else
  #1%
\fi
}
\makeatother
\newcommand{\slpj}[1]{\ifnonanon{\emph{SLPJ: #1}}{}}
\newenvironment{slpjenv}{\em SLPJ:}{}
\newcommand{\sg}[1]{\ifnonanon{\emph{SG: #1}}{}}
\newcommand{\todo}[1]{\ifnonanon{\textcolor{red}{TODO: #1}}{}}

% Auxiliary
\newcommand{\many}[1]{\overline{#1}}
\newcommand{\wild}{\ensuremath{\mathunderscore}}
\newcommand{\pfun}{\rightharpoonup}
\newcommand{\ruleform}[1]{\fbox{$#1$}}
\newcommand{\highlight}[1]{\setlength{\fboxsep}{2pt}\colorbox[gray]{0.8}{\ensuremath{#1}}}
\newcommand{\nelems}[1]{\lvert#1\rvert}
\newcommand{\restrict}[1]{{|_{#1}}}
\newcommand{\fn}[2]{\ensuremath{λ#1.\ #2}}
%\newcommand{\fn}[2]{#1 ↦ #2}
\newcommand{\constfn}[1]{\fn{\mathunderscore}{#1}}
%\newcommand{\repeat}[2]{\foreach \n in {1,...,#1}{#2}} % https://tex.stackexchange.com/a/16190/52414
% https://tex.stackexchange.com/a/21647
\makeatletter
\newcommand{\superimpose}[3][\mathord]{#1{\mathpalette\superimpose@{{#2}{#3}}}}
\newcommand{\superimpose@}[2]{\superimpose@@{#1}#2}
\newcommand{\superimpose@@}[3]{%
  \ooalign{%
    \hfil$\m@th#1#2$\hfil\cr
    \hfil$\m@th#1#3$\hfil\cr
  }%
}
\makeatother
% Stacked operators like \lesseq, \lesssim, etc.
% https://tex.stackexchange.com/a/536604/52414
\makeatletter
\newcommand{\impsim}{\lesssim\!\gtrsim}
\newcommand{\lessequiv}{\mathrel{\raisebox{-1pt}{$\gl@over[0.25pt]{\vartriangleleft}{\sim}$}}}
\newcommand{\impequiv}{\mathrel{\raisebox{-1.25pt}{$\gl@over[0.25pt]{\superimpose{\raisebox{1pt}{\clipbox{0pt 3pt 0pt 3pt}{$\mid$}}}{\rotatebox[origin=c]{90}{$\lozenge$}}}{\sim}$}}}
\newcommand{\impapprox}{\lessapprox\!\gtrapprox}
\newcommand{\gl@over}[3][1pt]{%
  \vcenter{\m@th\offinterlineskip\ialign{%
    \hfil$##$\hfil\cr #2\cr \noalign{\vskip#1} #3\cr
  }}%
}
\makeatother
\newenvironment{letarray}{\begin{array}{@{}l@{\hspace{1ex}}l}}{\end{array}}
\newcommand{\noleftdelimiter}{\left.\kern-\nulldelimiterspace}
\newcommand{\later}{{\mathop{\vcenter{\hbox{$\scriptscriptstyle\blacktriangleright$}}}}}
\newcommand{\purelater}{\mathop{\mathsf{next}}}
\newcommand{\aplater}{\mathbin{\circledast}}
\newcommand{\lBrace}{{\{\hspace{-0.22em}\mid}}
\newcommand{\rBrace}{{\mid\hspace{-0.22em}\}}}
\DeclarePairedDelimiter\idiom{\lBrace}{\rBrace}
\newcommand{\formgoal}[2]{%
\vspace{0.5\baselineskip}
\noindent
\begin{minipage}{\linewidth}
\centering\fbox{\textbf{Goal #1}: \begin{varwidth}[t]{\linewidth - 9em}#2\end{varwidth}}
\end{minipage}%
\\[-0.5\baselineskip]
}

% Syntax
%% LC
\newcommand{\Con}{\mathsf{Con}}
\newcommand{\Var}{\mathsf{Var}}
\newcommand{\Lab}{\mathsf{Lab}}
\newcommand{\Exp}{\mathsf{Exp}}
\newcommand{\Ctx}{\mathsf{Ctx}}
\newcommand{\Val}{\mathsf{Val}}
\newcommand{\Usg}{\mathsf{Usg}}
\newcommand{\px}{\mathsf{x}}
\newcommand{\py}{\mathsf{y}}
\newcommand{\pz}{\mathsf{z}}
\newcommand{\pv}{\mathsf{v}}
\newcommand{\pe}{\mathsf{e}}
\newcommand{\pc}{\mathsf{c}}
\newcommand{\pp}{\dot{\mathsf{e}}}
\newcommand{\pC}{\mathsf{C}}
\newcommand{\pE}{\mathsf{E}}
\newcommand{\Lam}[2]{\mathbf{\bar{\lambdaup}} #1. #2}
\newcommand{\LLam}[3]{\lambda_{#1} #2. #3}
\newcommand{\Lit}[1]{\llparenthesis #1 \rrparenthesis}
\newcommand{\Let}[3]{\mathbf{let}~{#1}\nobreak=\nobreak{#2}~\mathbf{in}~{#3}}
\newcommand{\Letn}[3]{\mathbf{let1}~{#1}\nobreak=\nobreak{#2}~\mathbf{in}~{#3}}
\newcommand{\Letsmall}[3]{\mathbf{let}\,{#1}\!\!=\!\!{#2}\,\mathbf{in}\,{#3}}
\newcommand{\Ifz}[3]{\mathbf{ifz}(#1,#2,#3)}
\newcommand{\Ite}[3]{\mathbf{if}~#1~\mathbf{then}~#2~\mathbf{else}~#3)}
\newcommand{\LLet}[4]{\mathbf{let}_{#1}~{#2}={#3}~\mathbf{in}~{#4}}
\newcommand{\Case}[2]{\mathbf{case}~{#1}~\mathbf{of}~{#2}}
\newcommand{\Sel}[1][]{\many{K~\many{\px} \rightarrow \pe_{#1}}}
\newcommand{\SelArity}{\many{K~\many{\px}^{α_K} \rightarrow \pe}}
\newcommand{\hole}{\square}
\newcommand{\adom}{\mathop{\mathsf{adom}}}
\newcommand{\dom}{\mathop{\mathsf{dom}}}
\newcommand{\rng}{\mathop\mathsf{rng}}
\newcommand{\fv}{\mathop\mathsf{fv}}
\newcommand{\fa}{\mathop\mathsf{fa}}
\newcommand{\bv}{\mathop\mathsf{bv}}
\newcommand{\disjoint}{\mathbin{\#}}
\newcommand{\delvar}[1]{{\setminus_{#1}}}
\newcommand{\notfound}{\lightning}
\newcommand{\bool}{\mathtt{bool}}
\newcommand{\ttrue}{\mathtt{tt}}
\newcommand{\ffalse}{\mathtt{ff}}
\newcommand{\tick}{\checkmark}
%% Events
\newcommand{\Events}{\mathbb{Ev}}
\newcommand{\ValueA}{\mathsf{val}}
\newcommand{\BindE}{\mathsf{bind}}
\newcommand{\BindOE}{\mathsf{bind1}}
\newcommand{\LookupE}{\mathsf{look}}
\newcommand{\UpdateE}{\mathsf{upd}}
\newcommand{\AppIE}{\mathsf{app}_1}
\newcommand{\AppEE}{\mathsf{app}_2}
\newcommand{\CaseIE}{\mathsf{case}_1}
\newcommand{\CaseEE}{\mathsf{case}_2}
\newcommand{\TickE}{\mathsf{tick}}
%% Labels
\newcommand{\Labels}{\mathsf{Lab}}
\newcommand{\lbl}{\ensuremath{\ell}}
\newcommand{\lbln}[1]{\ensuremath{\ell_{#1}}}
\newcommand{\slbl}{\raisebox{0.08em}{$\scriptstyle{\ell}$\;}}
\newcommand{\slbln}[1]{\raisebox{0.1em}{$\;\scriptstyle{\ell_{#1}}\;$}}
\newcommand{\atlbl}[1]{at(#1)}
%% Traces
\newcommand{\return}{\rotatebox[origin=c]{180}{$\hspace{0.7pt}\Rsh$}}
\newcommand{\ITraces}{{\mathbb{T}^*}}
\newcommand{\Traces}{{\mathbb{T}^{ε}}}
\newcommand{\act}[1]{\xrightarrow{#1}}
\newcommand{\someend}[1]{\langle #1 \rangle}
\newcommand{\goodend}[1]{\someend{#1}_{\checkmark}}
\newcommand{\stuckend}{\lightning}
\newcommand{\makestuck}[1]{\lightning(#1)}
\newcommand{\cons}{\mathbin{::}}
\newcommand{\lcons}{\mathbin{:>}}
\newcommand{\laterC}{\mathop{\mathsf{L}}}
\newcommand{\trend}{\raisebox{-0.8pt}{\scalebox{0.3}{$\square$}}}
\newcommand{\betastep}{\mathbin{>\hspace{-0.40em}>\hspace{-0.48em}\text{β}\hspace{-0.40em}=}}
\newcommand{\cont}{\mathit{cont}}
\newcommand{\ctrl}{\mathit{ctrl}}
\newcommand{\ret}{\mathit{ret}}
\newcommand{\memo}{\mathit{memo}}
\newcommand{\deref}{\mathit{deref}}
\newcommand{\derefn}{\textit{deref-name}}
\newcommand{\apply}{\mathit{apply}}
\newcommand{\select}{\mathit{select}}
\newcommand{\alts}{\mathit{alts}}
\newcommand{\len}{\mathit{len}}
\newcommand{\obs}{\mathit{obs}}
\newcommand{\fin}{\mathsf{fin}}
\newcommand{\dead}{\mathsf{dead}}
\newcommand{\noalias}{\mathsf{noalias}}
\newcommand{\ok}{\mathsf{ok}}
\newcommand{\diverges}{\mathsf{div}}
\newcommand{\replace}{\mathit{replace}}
\newcommand{\inj}{\mathit{inj}}
\newcommand{\card}{\mathit{card}}
\newcommand{\usg}{\mathit{usg}}
\newcommand{\byname}{\textit{by-name}}
\newcommand{\intra}{\mathit{intra}}
\newcommand{\ctx}{\mathit{ctx}}
% https://latex.org/forum/viewtopic.php?p=62177&sid=26890ac77d076f3338384a47a2ffd4bc#p62177
\makeatletter
\newcommand{\concatop}[1]{%
  \mathbin{\mathop{#1}\limits^{\vbox to -1\ex@{\kern-\tw@\ex@
   \hbox{$\smallfrown$}\vss}}}}
\makeatother
\makeatletter
\newcommand{\compop}[1]{%
  \mathbin{\vec{\ooalign{$\circ$\cr\hidewidth$#1$\hidewidth}}}}
\makeatother
\newcommand{\concat}{\concatop{\cdot}}
\newcommand{\fcomp}{\compop{\cdot}}
\newcommand{\fcons}{\mathbin{\vec{::}}}
\newcommand{\getval}[2]{#1 \Downarrow #2}
\newcommand{\subtrceq}{\superimpose[\mathrel]{\clipbox{0pt 0pt {0.3\width} 0pt}{${\subset}$}}{\hspace{0.6ex}\cdot}}
\newcommand{\rightsubtrceq}{\subtrceq_r}
\newcommand{\leftsubtrceq}{\subtrceq_l}
\newcommand{\stepm}[3]{\lceil #1 \act{#2} #3\trend \rceil}
\newcommand{\step}[2]{\stepm{\wild}{#1}{#2}}
\newcommand{\tell}[1]{\lceil #1 \rceil}

%% Domains
\newcommand{\FunV}{\mathsf{Fun}}
\newcommand{\unFunV}{\mathsf{unFun}}
\newcommand{\ConV}{\mathsf{Con}}
\newcommand{\Environments}{\mathbb{E}}
\newcommand{\Controls}{\mathbb{C}}
\newcommand{\ProgramPoints}{\mathbb{P}}
\newcommand{\Heaps}{\mathbb{H}}
\newcommand{\Continuations}{\mathbb{K}}
\newcommand{\Domain}[1]{{\mathbb{D}^{#1}}}
\newcommand{\Values}[1]{{\mathbb{V}^{#1}}}
\newcommand{\Base}[1]{\mathbf{#1}}
\newcommand{\ApproxMap}[2]{\mathsf{Map}(#1,#2)}
\newcommand{\FinStepMap}[2]{#1 \Rightarrow #2}
\newcommand{\PrefD}{\Domain{*}}
\newcommand{\EventD}{\Domain{ε}}
\newcommand{\EventV}{\Values{ε}}
\newcommand{\ScottD}{\Domain{\bot}}
\newcommand{\AbsD}{\Domain{\raisebox{0.1em}{\scalebox{0.4}{$\square$}}}}
\newcommand{\AbsCD}{\Domain{\raisebox{0.1em}{\scalebox{0.4}{c$\square$}}}}
\newcommand{\StateD}{\Domain{σ}}
\newcommand{\StateV}{\Values{\States}}
\newcommand{\VanD}{\Domain{\laterC}}
\newcommand{\VanV}{\Values{\laterC}}
\newcommand{\UsgD}{\Domain{u}}
\newcommand{\Look}{{\lceil \LookupE \rceil}}
\newcommand{\LookD}{\Domain{\Look}}
\newcommand{\LookTraces}{{\mathbb{T}^{\Look}}}
\newcommand{\LookValues}{\Values{\Look}}
\newcommand{\tr}{{\ensuremath{\tilde{ρ}}}}
\newcommand{\tm}{{\ensuremath{\tilde{μ}}}}
\newcommand{\ttau}{{\ensuremath{\tilde{τ}}}}
\newcommand{\tv}{{\ensuremath{\tilde{v}}}}
\newcommand{\td}{{\ensuremath{\tilde{d}}}}
\newcommand{\bigsteppn}[3]{\ensuremath{#2\!\Downarrow^{#1}\!#3}}
\newcommand{\bigstepp}[2]{\bigsteppn{}{#1}{#2}}
\newcommand{\bigstep}[4]{\ensuremath{{\langle#1,#2\rangle\!\!\Downarrow\!\!\langle#3,#4\rangle}}}
\newcommand{\bigstepn}[5]{\ensuremath{{\langle#2,#3\rangle\!\!\Downarrow^{#1}\!\!\langle#4,#5\rangle}}}
\newcommand{\forcesto}{\ensuremath{\rightsquigarrow}}
\newcommand{\forcessym}{\ensuremath{\leftrightsquigarrow}}
\newcommand{\correct}{\mathcal{C}}
\newcommand{\fincons}{\textsc{Fin-$\cons$}}
\newcommand{\fingood}{\textsc{Fin-$\checkmark$}}
\newcommand{\finstuck}{\textsc{Fin-$\stuckend$}}
\newcommand{\eqlrcons}{\textsc{Eq-LR-$\cons$}}
\newcommand{\eqlcons}{\textsc{Eq-L-$\cons$}}
\newcommand{\eqrcons}{\textsc{Eq-R-$\cons$}}
\newcommand{\eqstuck}{\textsc{Eq-$\stuckend{}$}}
\newcommand{\eqfun}{\textsc{Eq-$\FunV$}}
\newcommand{\eqcon}{\textsc{Eq-$\ConV$}}
\newcommand{\eqheap}{\textsc{Eq-$\Heaps$}}
\newcommand{\eqdenot}{\textsc{Eq-$\EventD$}}
\newcommand{\implrcons}{\textsc{Imp-LR-$\cons$}}
\newcommand{\implcons}{\textsc{Imp-L-$\cons$}}
\newcommand{\imprcons}{\textsc{Imp-R-$\cons$}}
\newcommand{\impstuck}{\textsc{Imp-$\stuckend{}$}}
\newcommand{\impfun}{\textsc{Imp-$\FunV$}}
\newcommand{\impcon}{\textsc{Imp-$\ConV$}}
\newcommand{\impval}{\textsc{Imp-$\Values{}$}}
\newcommand{\impheap}{\textsc{Imp-$\Heaps$}}
\newcommand{\impdenot}{\textsc{Imp-$\Domain{}$}}
\newcommand{\corcons}{\textsc{$\correct$-$\cons$}}
\newcommand{\corstuck}{\textsc{$\correct$-$\stuckend{}$}}
\newcommand{\corfun}{\textsc{$\correct$-$\FunV$}}

% Math
\newcommand{\pow}[1]{\wp(#1)}
\newcommand{\Nat}{\mathbb{N}}
\DeclareMathSymbol{\bbcolon}{\mathpunct}{bbold}{"3A}
\DeclareMathSymbol{\bbquestionmark}{\mathpunct}{bbold}{"3F}
\DeclareMathSymbol{\bblparen}{\mathpunct}{bbold}{"28}
\DeclareMathSymbol{\bbrparen}{\mathpunct}{bbold}{"29}
\DeclareMathSymbol{\bbpipe}{\mathpunct}{bbold}{"7C}
\newcommand{\ternary}[3]{\mathop{\bblparen} #1 \mathrel{\bbquestionmark} #2 \mathrel{\bbcolon} #3 \mathop{\bbrparen}}

% Order theory
\DeclareMathOperator*{\lub}{\sqcup}
\DeclareMathOperator*{\Lub}{\bigsqcup}
\DeclareMathOperator*{\glb}{\sqcap}
\DeclareMathOperator*{\Glb}{\bigsqcap}
\newcommand{\fix}{\mathop\mathsf{fix}}
\newcommand{\lfp}{\mathop\mathsf{lfp}}
\newcommand{\gfp}{\mathop\mathsf{gfp}}

% Semantics
\newcommand{\denot}[1]{\llbracket {#1} \rrbracket}
\newcommand{\semscott}[1]{\mathcal{S}_{\bot}\denot{#1}}
\newcommand{\semsmall}[1]{\mathcal{S}_{\smallstep[\!]}\denot{#1}}
\newcommand{\semevt}[1]{\mathcal{S}_{ε}\denot{#1}}
\newcommand{\semst}[1]{\mathcal{S}_{σ}\denot{#1}}
\newcommand{\semvan}[1]{\mathcal{S}_{\laterC}\denot{#1}}
\newcommand{\semss}[1]{\mathcal{S}_{Σ}\denot{#1}}
\newcommand{\semusg}[1]{\mathcal{S}_{u}\denot{#1}}
\newcommand{\semcoll}[1]{\mathcal{S}_{\mathbb{C}}\denot{#1}}
\newcommand{\sem}[1]{\mathcal{S}\denot{#1}}

% Stateful
\newcommand{\States}{\mathbb{S}}
\newcommand{\VTraces}{{\mathbb{T}^{\laterC}}}
\newcommand{\STraces}{\mathbb{S}^{\infty}}
\newcommand{\sconcat}{\concatop{\hspace{0.04ex}\scalebox{0.5}{$\States$}}}
\newcommand{\sfcomp}{\compop{\scalebox{0.25}{\raisebox{0.7em}{$\States$}}}}
\newcommand{\pushF}{\cdot}
\newcommand{\StopF}{\mathbf{stop}}
\newcommand{\ApplyF}{\mathbf{ap}}
\newcommand{\SelF}{\mathbf{sel}}
\newcommand{\UpdateF}{\mathbf{upd}}

% Small-step
\newcommand{\Addresses}{\mathsf{Addr}}
\newcommand{\pa}{\mathsf{a}}
\newcommand{\smallstep}[1][\hspace{1.5ex}]{\xhookrightarrow{#1}}
\newcommand{\ssconcat}{\concatop{\hspace{0.04ex}\scalebox{0.5}{\Sigma}}}
\newcommand{\AppIT}{\textsc{App}_1}
\newcommand{\AppET}{\textsc{App}_2}
\newcommand{\ValueT}{\textsc{Val}}
\newcommand{\CaseIT}{\textsc{Case}_1}
\newcommand{\CaseET}{\textsc{Case}_2}
\newcommand{\LookupT}{\textsc{Look}}
\newcommand{\UpdateT}{\textsc{Upd}}
\newcommand{\BindT}{\textsc{Bind}}
\newcommand{\LetOT}{\textsc{Let}_0}
\newcommand{\LetIT}{\textsc{Let}_1}
\newcommand{\UnrollT}{\textsc{Unroll}}
\newcommand{\validtrace}[1]{#1\,\mathsf{ok\text{-}}τ}
\newcommand{\validtraceTriv}{\textsc{ok}_1}
\newcommand{\validtraceTrans}{\textsc{ok}_2}
\newcommand{\deep}[2]{#1\,\mathsf{deep}\,#2}
\newcommand{\deepn}[3]{#2\,\mathsf{deep}^{#1}\,#3}
\newcommand{\interior}[1]{#1\,\mathsf{inter}}
\newcommand{\interiorn}[2]{#2\,\mathsf{inter}^{#1}}
\newcommand{\maxtrace}[1]{#1\,\mathsf{max}}
\newcommand{\maxtracen}[2]{#2\,\mathsf{max}^{#1}}
\newcommand{\elabstate}[1]{#1\,\mathsf{elab}_σ}
\newcommand{\elabtrace}[1]{#1\,\mathsf{elab}_τ}
\newcommand{\elabtracen}[2]{#2\,\mathsf{elab}_τ^{#1}}
\newcommand{\contextendsflipped}{⊑_\Continuations}
\newcommand{\contextends}{⊒_\Continuations}
\newcommand{\contextendsstrict}{⊐_\Continuations}

% Liveness
\newcommand{\lStacks}{\Stacks^l}
\newcommand{\lLiveness}{\mathsf{Live}}
\newcommand{\lS}{\pS^l}
\newcommand{\lSBot}{\texttt{Seq}}
\newcommand{\lSAp}[1]{\texttt{\$[$#1$]}}
\newcommand{\lSTop}{\texttt{Deep}}
\newcommand{\lAbs}{\texttt{Abs}}
\newcommand{\lUsed}[1]{\texttt{Used[$#1$]}}
\newcommand{\SVar}{\mathsf{SVar}}
\newcommand{\SCall}{\mathsf{SCall}}

% Types
\newcommand{\TyCon}{\mathsf{TyCon}}
\newcommand{\Type}{\mathsf{Type}}
\newcommand{\ArrowTy}{\Rightarrow}

% Side braces in a matrix, https://tex.stackexchange.com/q/443407
\pgfkeys{tikz/mymatrixenv/.style={decoration={brace},every left delimiter/.style={xshift=8pt},every right delimiter/.style={xshift=-8pt}}}
\pgfkeys{tikz/mymatrix/.style={matrix of math nodes,nodes in empty cells,column sep=2pt,row sep=2pt,nodes={minimum width=10pt,minimum height=10pt,anchor=west,inner sep=0pt,outer sep=0pt}}}
\pgfkeys{tikz/mymatrixbrace/.style={decorate,thick}}
\newcommand*\mymatrixbraceright[4][m]{
    \draw[mymatrixbrace] (#1.west|-#1-#3-1.south west) -- node[left=2pt] {#4} (#1.west|-#1-#2-1.north west);
}
\newcommand*\mymatrixbraceleft[5]{
    \draw[mymatrixbrace] (#1.east|-#2-#3-1.east) -- node[right=2pt] (#1) {#5} (#1.east|-#2-#4-1.east);
}
\newcommand*\mymatrixbracetop[4][m]{
    \draw[mymatrixbrace] (#1.north-|#1-1-#2.north west) -- node[above=2pt] {#4} (#1.north-|#1-1-#3.north east);
}
\newcommand*\mymatrixbracebottom[4][m]{
    \draw[mymatrixbrace] (#1.south-|#1-1-#2.north east) -- node[below=2pt] {#4} (#1.south-|#1-1-#3.north west);
}

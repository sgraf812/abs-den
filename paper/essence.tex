\section{The Imperative Essence of Lazy Evaluation}
\label{sec:essence}

Before we can talk about abstraction, we have to record a few meta-theoretic
properties.
We will prove these properties in terms of the eventful semantics but they apply
just the same for the stateful one, or even the ``bare-bones'' version from
the Appendix.

As we have observed in \Cref{sec:problem}, the semantics of call-by-need is more
complicated than the semantics for call-by-name and call-by-value because it
relies on a heap to model memoisation.
This leads to a more complicated domain model that essentially passes around
heaps as state.

On the other hand, memoisation is a very ``benign'' kind of state and often we
can reason rather naturally about call-by-need programs without thinking too
much about heaps, in contrast to a calculus with arbitrary assignments to ref
cells.

For example, we can observe that heaps in our semantics evolve in a very
specific way:
When $\semevt{\pe}_ρ(μ) = ... \cons \goodend{v,μ'}$, we intuit that $μ'$ must be
``at least as evaluated'' as $μ$, \eg, for all $\pa ∈ \dom(μ)$ either $μ(\pa) =
μ'(\pa)$ or $μ'(\pa)$ is ``the value of'' $μ(\pa)$.
We make this observation precise via the following definitions:

\begin{definition}[Big-step]
  \label{defn:eval-d}
  For all $d ∈ \EventD$, $v ∈ \EventV$, $μ, μ' ∈ \Heaps$ we say that
  $d$ \emph{evaluates} in $μ$ to $(v,μ')$ (written $\bigstep{d}{μ}{v}{μ'}$) if
  and only if $d(μ) = ... \cons \goodend{v,μ'}$.
\end{definition}

Big-step notation deliberately throws away much information about the trace,
retaining only whether it was balanced and what the final value and heap is,
hence the name ``big-step'' is fitting.

\begin{definition}[Heap forcing]
  \label{defn:force-heap}
  For all $μ_1, μ_2 ∈ \Heaps$ we say that $μ_1$ \emph{forces} to $μ_2$
  (written $μ_1 \forcesto μ_2$) if and only if
  \begin{enumerate}
    \item $\dom(μ_1) ⊆ \dom(μ_2)$
    \item For all $\pa$, either
      \begin{itemize}
        \item $μ_1(\pa) = μ_2(\pa)$, or
        \item there exists $v,μ'_1$ such
          that $\bigstep{μ_1(\pa)}{μ_1}{v}{μ'_1}$ and $μ_2(\pa) = \ret(v)$ as well as
          $μ'_1 \forcesto μ_2$ if $μ_1 \not= μ'_1$.
      \end{itemize}
  \end{enumerate}
\end{definition}

Our earlier intuition can now be formulated as follows:
If $\bigstep{\semevt{\pe}_ρ}{μ}{v}{μ'}$, then $μ \forcesto μ'$.

For the remainder of this work, we will identify heaps modulo consistent
readdressing.
Of course, a rigorous treatment would have to carry around readdressing
substitutions and apply them to adjust the domain of a heap, its entries,
as well as the heaps and values in the returned traces.
\sg{Perhaps I should give a figure with the definitions I have in mind, maybe
in the Appendix.}

%We write $μ_1 \forcessym μ_2$ if either $μ_1 \forcesto μ_2$ or
%$μ_2 \forcesto μ_1$ holds.
When a heap entry is evaluated, it stays that way during forcing:

\begin{lemma}
  \label{thm:force-heap-val}
  For all $μ_1,μ_2,\pa,v$ such that $μ_1 \forcesto μ_2$ and $μ_1(\pa) = \ret(v)$,
  then $μ_2(\pa) = \ret(v)$.
\end{lemma}
\begin{proof}
  Immediate proof by contraposition and unfolding of $\ret$.
\end{proof}

So once an entry becomes a value during forcing, it stays that way.

The heap forcing relation is reflexive (it is always $μ(\pa) = μ(\pa)$) and
also antisymmetric:

\begin{lemma}[$(\forcesto)$ is antisymmetric]
  \label{thm:force-heap-trans}
  Let $μ_1,μ_2$ be lazy heaps and $μ_1 \forcesto μ_2, μ_2 \forcesto μ_1$.
  Then $μ_1 = μ_2$.
\end{lemma}
\begin{proof}
  It is easy to see that both $μ_1$ and $μ_2$ must have the same domain.

  Now suppose that there exists $\pa$ such that $μ_1(\pa) \not= μ_2(\pa)$.
  Then there exists $v_1$ such that $\bigstep{μ_1(\pa)}{μ_1}{v_1}{\wild}$ and
  $μ_2(\pa) = \ret(v_1)$.
  By \Cref{thm:force-heap-val} applied to $μ_2 \forcesto μ_1$, we get $μ_1(\pa)
  = v_1 = μ_2(\pa)$, a contradiction.
\end{proof}

(It is worth noting that antisymmetry only holds modulo readdressing.)
However, $(\forcesto)$ is not easily proven transitive without further
characterisation of the domain elements returned by $\semevt{\wild}$.

\begin{definition}[Lazy domain]
  \label{defn:lazy-d}
  A heap $μ$ or environment $ρ$ is \emph{lazy} if and only if its entries are lazy elements.
  A value $v$ is \emph{lazy} if and only if
  \begin{itemize}
    \item When $v = \FunV(f)$ (for some $f$), then $f(d')$ is lazy whenever $d'$ is lazy.
    \item When $v = \ConV(K,\many{d})$ (for some $K,\many{d}$), then all $\many{d}$ are lazy.
  \end{itemize}
  An element $d ∈ \EventD$ is \emph{lazy} if and only if
  \begin{enumerate}
    \item \textup{(Forces)} For all lazy $μ,μ' ∈ \Heaps, v ∈ \EventV$ such that $d(μ) = ... \cons \goodend{v,μ'}$, $v$ is lazy and $μ \forcesto μ'$.
    \item \textup{(Postpone)} For all lazy $μ_1,μ_2 ∈ \Heaps$ such that $μ_1 \forcesto μ_2$ and
     $\bigstep{d}{μ_1}{v_1}{μ'_1}$ (for some $v_1,μ'_1$), we also have
     $\bigstep{d}{μ_2}{v_2}{μ'_2}$ (for some $v_2,μ'_2$) and
     $v_1 = v_2$ as well as $μ'_1 \forcesto μ'_2$.
    \item \textup{(Speculate)} For all lazy $μ_1,μ_2 ∈ \Heaps$ such that $μ_1 \forcesto μ_2$ and
     $\bigstep{d}{μ_2}{v_2}{μ'_2}$ (for some $v_2,μ'_2$), we also have
     $\bigstep{d}{μ_1}{v_1}{μ'_1}$ (for some $v_1,μ'_1$) and
     $v_1 = v_2$ as well as $μ'_1 \forcesto μ'_2$.
  \end{enumerate}
\end{definition}

So when an element $d$ is \emph{lazy}, every input heap forces to the output
heap, it evaluates to the same value whether or not the input heap is more or
less forced, and $\bigstep{d}{\wild}{v}{\wild}$ is natural (in the category
theoretic sense) in that it transfers the forcedness relation from input heaps
to output heaps.
The definitions for lazy heaps, environments are necessary by congruence.

When we restrict ourselves to lazy heaps, we can prove transitivity of the
forcing relation.

\begin{lemma}[$(\forcesto)$ is transitive]
  \label{thm:force-heap-trans}
  Let $μ_1,μ_2,μ_3$ be lazy heaps and $μ_1 \forcesto μ_2, μ_2 \forcesto μ_3$.
  Then $μ_1 \forcesto μ_3$.
\end{lemma}
\begin{proof}
  We need a termination measure on $(\forcesto)$ before we can proceed.
  Note that in successive steps $μ_1 \forcesto ... \forcesto μ_n$ where all
  $μ_i$ are lazy heaps, heap entries progress from being undefined,
  then to a non-value denotation, and finally to a value element $\ret(v)$.

  More formally, we can define a measure $m : \Heaps \to \dom(μ_n) \pfun \{0,1,2\}$
  \[
    m(μ)(\pa) = \begin{cases}
        0 & μ(\pa) = \ret(\wild) \\
        1 & \pa ∈ \dom(μ) \\
        2 & \text{otherwise} \\
      \end{cases}
  \]
  with the pointwise partial order on $\dom(μ_n) \pfun \{0,1,2\}$.
  When $\dom(μ_n)$ is finite, this partial order has finite height, which can
  thus serve as a termination measure.

  We proceed by well-founded induction with the above measure $m$ defined on
  $\dom(μ_3)$.
  More precisely, show that the proposition is satisfied for $μ_1,μ_2,μ_3$
  assuming that the proposition holds for any $μ_1',μ_2',μ_3'$
  such that $m(μ_1') - m(μ_3') < m(μ_1) - m(μ_3)$.

  We can see that $\dom(μ_1) ⊆ \dom(μ_2) ⊆ \dom(μ_3)$, proving the first property
  of \Cref{defn-force-heap}.
  For the second property, fix an arbitrary $\pa$.
  When $μ_1(\pa) = μ_2(\pa)$ and $μ_2(\pa) = μ_3(\pa)$, we have $μ_1(\pa) = μ_3(\pa)$.
  Otherwise, either $μ_1(\pa) \not= μ_2(\pa)$ or $μ_2(\pa) \not= μ_3(\pa)$.

  Suppose that $μ_1(\pa) \not= μ_2(\pa)$.
  Then there exists $v,μ'_1$ such that $\bigstep{μ_1(\pa)}{μ_1}{v}{μ'_1}$ and
  $μ_2(\pa) = \ret(v)$ as well as $μ'_1 \forcesto μ_2$.
  By \Cref{thm:force-heap-val} applied to $μ_2 \forcesto μ_3$, we have
  $μ_2(\pa) = μ_3(\pa) = \ret(v)$.
  If $μ_1' \not= μ_1$, then $m(μ_1') < m(μ_1)$ and by well-founded induction
  we can combine $μ'_1 \forcesto μ_2$ and $μ_2 \forcesto μ_3$ to
  $μ'_1 \forcesto μ_3$.
  Hence we have shown that $μ_1 \forcesto μ_3$.

  Suppose that $μ_1(\pa) = μ_2(\pa)$, but $μ_2(\pa) \not= μ_3(\pa)$.
  Then there exists $v,μ'_2$ such that $\bigstep{μ_2(\pa)}{μ_2}{v}{μ'_2}$ and
  $μ_3(\pa) = \ret(v)$ as well as $μ'_2 \forcesto μ_3$.
  We know that $μ_1(\pa) = μ_2(\pa)$, so $\bigstep{μ_1(\pa)}{μ_2}{v}{μ'_2}$.
  Since $μ_1(\pa)$ is lazy and $μ_1 \forcesto μ_2$, we get
  $\bigstep{μ_1(\pa)}{μ_1}{v}{μ'_1}$ such that $μ_1' \forcesto μ_2'$ by
  definition of laziness.
  By well-founded induction, we can combine $μ'_1 \forcesto μ_2'$ and
  $μ_2' \forcesto μ_3$ to $μ'_1 \forcesto μ_3$.
  If $μ_1' \not= μ_1$, then $m(μ_1') < m(μ_1)$ and by well-founded induction
  we can combine $μ'_1 \forcesto μ_2'$ and $μ_2' \forcesto μ_3$ to
  $μ'_1 \forcesto μ_3$.
  Hence have shown that $μ_1 \forcesto μ_3$.
\end{proof}

\begin{corollary}
  $(\forcesto)$ is a partial order on lazy heaps.
\end{corollary}

Another neat consequence of lazy elements is the following confluence property:

\begin{corollary}[Evaluating lazy elements is confluent]
  \label{thm:force-heap-confluent}
  Let $μ_1,μ_1',μ_2$ be lazy heaps, $d$ a lazy element and $v$ a value.
  If $\bigstep{d}{μ_1}{v}{μ_1'}$ and $μ_1 \forcesto μ_2$,
  then there exists a lazy heap $μ_2'$ such that $μ_1' \forcesto μ_2'$
  and $\bigstep{d}{μ_2}{v}{μ_2'}$. As a commutative diagram:
  \[
  \begin{tikzcd}
    μ_1 \arrow[Rightarrow]{d}[swap]{\langle d, \wild\rangle}{\langle v, \wild\rangle} \arrow[rightsquigarrow]{r}{} & μ_2 \arrow[Rightarrow]{d}{\langle d, \wild\rangle}[swap]{\langle v, \wild\rangle} \\
    μ_1' \arrow[rightsquigarrow]{r}{} & μ_2' \\
  \end{tikzcd}
  \]
\end{corollary}
\begin{proof}
  This is just the Postpone property in \Cref{defn:lazy-d}.
\end{proof}

And now we finally prove that $\semevt{\wild}$ indeed produces a lazy element:

\begin{theorem}
  \label{thm:semevt-lazy}
  For all lazy environments $ρ$ and expressions $\pe$, $\semevt{\pe}_ρ$ is a lazy element.
\end{theorem}
\begin{proof}
  By induction on $\pe$.
  \begin{itemize}
    \item \textbf{Case $\px$}: $ρ(\px)$ is lazy, and
      $\fn{\wild}{\stuckend{}}$ is trivially so as it never produces a balanced
      trace.

    \item \textbf{Case $\Lam{px}{\pe'}, K~\many{\px}$}:
      For any lazy $μ$, we have $\bigstep{\semevt{\pe}_ρ}{μ}{v}{μ}$, where
      $v$ is a $\FunV$ or $\ConV$ value which is independent of $μ$.
      The laziness of function bodies and constructor field denotations follows
      by induction.
      Clearly, $μ \forcesto μ$ by reflexivity.
      The Postpone and Speculate properties follow by assumption.

    \item \textbf{Case $\pe'~\px$}:
      First note that $d \betastep f$ is lazy when $d$ is lazy and $f(v)$ is
      lazy when defined for a lazy $v$.
      This result makes use of transitivity in the first case of $(\betastep)$.

      Both arguments to $\apply$ are lazy:
      $\semevt{\pe}_ρ$ is lazy by the induction hypothesis and $ρ(\px)$ is lazy
      by assumption.
      In the definition of $\apply$, we can see that if the second argument to
      $(\betastep)$ is given a lazy $\FunV(f)$ value, then $f(d)$ is lazy by
      definition of laziness of $\FunV$ values and laziness of $d$.
      Hence the whole $\apply$ form is lazy and prepending the $\AppIA$ event
      does not change that because it does not affect the heap.

    \item \textbf{Case $\pe'~\px$}:
      First note that $d \betastep f$ is lazy when $d$ is lazy and for all
      $μ,μ',v$, such that $\bigstep{d}{μ}{v}{μ'}$, $f(v)$ is lazy on $μ'$ when
      defined.
      This result makes use of transitivity in the first case of $(\betastep)$.

      Both arguments to $\apply$ are lazy:
      $\semevt{\pe}_ρ$ is lazy by the induction hypothesis and $ρ(\px)$ is lazy
      by assumption.
      In the definition of $\apply$, we can see that if the second argument to
      $(\betastep)$ is given a lazy $\FunV(f)$ value, then $f(d)$ is lazy by
      definition of laziness of $\FunV$ values and laziness of $d$.
      Hence the whole $\apply$ form is lazy and prepending the $\AppIA$ event
      does not change that because it does not affect the heap.

    \item \textbf{Case $\Case{\pe_s}{\Sel[r]}$}:
      Similar to the application case.

    \item \textbf{Case $\Let{\px}{\pe_1}{\pe_2}$}:
      Let $μ_l$ be lazy and consider $\semevt{\pe}_ρ(μ_l)$.
      The key is to this theorem is to prove that $\deref(\pa)$ is lazy.

      Well-addressedness means that we can focus on any lazy input heap $μ$ that
      satisfies $\dom(μ) ⊆ (\{ \pa \} ∪ \dom(μ_l))$ to conduct the proof.
      Clearly, prepending the $\LookupA$ event in the definition of $\deref$
      does not affect the heap and thus laziness.
      The first argument to $(\betastep)$, $μ(\pa)$, is lazy because $μ$ is.
      Furthermore, assuming $\bigstep{μ(\pa)}{μ}{v}{μ'}$ for suitable lazy $v,μ'$,
      the trace resulting from calling the second argument is
      \[
        \UpdateA(\pa↦\ret(v)) \cons \goodend{v,μ'[\pa↦\ret(v)]}
      \]
      And crucially, $μ' \forcesto μ'[\pa↦ret(v)]$ due to
      $\bigstep{μ(\pa)}{μ}{v}{μ'}$ for the differing $\pa$ entry, proving
      the Forces property.
      The Postpone and Speculate property for the second argument follow quite
      simply, because the updated heap entry is constant in all cases.
      We can conclude by our earlier observation about $(\betastep)$ that
      $\deref(\pa)$ is indeed lazy.

      Given that fact, it is easy to see that $ρ'$ is lazy in the definition
      of $\semevt{\pe}$ (on all heaps extending $μ_l[\pa↦...]$).
      Then $\semevt{\pe_i}_{ρ'}$ are lazy by induction.
      Hence is $μ_l[\pa↦\semevt{\pe_1}_{ρ'}]$, which we apply to
      $\semevt{\pe_2}_{ρ'}$, in which it is lazy.
      In order to show that the whole clause is lazy in $μ_l$,
      it suffices to show that $μ_l \forcesto μ_l[\pa↦\semevt{\pe_1}_{ρ'}]$
      which is simple to see because $\pa \not∈ \dom(μ_l)$.
      The Postpone and Speculate properties are unaffected.
  \end{itemize}
\end{proof}

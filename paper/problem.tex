\section{Problem Statement}
\label{sec:problem}

By way of the poster child example of a compositional definition of \emph{usage
analysis}, we showcase how the operational detail available in traditional
denotational semantics is too coarse to substantiate a correctness criterion,
although the proof of (a weaker notion of) correctness is simple and direct.
While operational semantics observe sufficient detail to formulate a correctness
relation, we give an excerpt from an exemplary proof attempt that highlights the
complexities involved.

%At the same time,
%we show how operational semantics
%Further consideration of more
%elaborate cardinality analyses~\citep{cardinality} and quantitative type
%systems~\citep{Atkey:18} characterise this weakness not only in terms of
%unobservable divergence but also in terms of evaluation cardinality.

%\subsection{Notation}
%
%Collection of stuff to explain, for now:
%\begin{itemize}
%  \item $\triangleq$ for defining an object (defn eq), rather than $=$
%  \item $\text{letrec}$
%    The (meta-level, math) notation
%    \[
%    \text{letrec}~l.~\many{x = rhs_x} ~ l = rhs_{l} ~ \many{y = rhs_y}~\text{in}~body
%    \]
%    where $l$ might occur freely in any $rhs_{\wild}$ and $body$, is syntactic sugar for
%    \[
%    snd(\lfp(\fn{(l,\wild)}{\text{let}~\many{x = rhs_x} ~ \many{y = rhs_y}~\text{in}~(rhs_{l},body)}))
%    \]
%    Where $\lfp$ is the least fixpoint operator and $snd(a,b) = b$. Clearly, this
%    desugaring's use of $\lfp$ is well-defined for its use on elements of the
%    powerset lattice $\UsgD$.
%
%    But \citep{Shivers:91} uses the similar $\text{whererec}$ and gets by without
%    ever explaining it, so we might as well.
%\end{itemize}

\subsection{Usage Analysis}

\begin{figure}
\begin{minipage}{\textwidth}
\[\begin{array}{c}
 \arraycolsep=3pt
 \begin{array}{rrclcrrclcl}
  \text{Variables}    & \px, \py & ∈ & \Var        &     & \quad \text{Constructors} &        K & ∈ & \Con        &     & \text{with arity $α_K ∈ ℕ$} \\
  \text{Labels}       &     \lbl & ∈ & \Labels     &     & \quad \text{Values}       &      \pv & ∈ & \Val        & ::= & \highlight{\Lam{\px}{\pe}} \mid K~\many{\px}^{α_K} \\
  \text{Expressions}  &      \pe & ∈ & \Exp        & ::= & \multicolumn{6}{l}{\highlight{\slbl \px \mid \slbl \pv \mid \slbl \pe~\px \mid \slbl \Let{\px}{\pe_1}{\pe_2}} \mid \slbl \Case{\pe}{\SelArity}} \\
  \\[-0.5em]
 \end{array} \\
 \\[-0.5em]
 \begin{array}{rrclcl}
  \text{Scott Domain}      &  d & ∈ & \ScottD & =   & [\ScottD \to_c \ScottD]_\bot \\
  \text{Usage cardinality} &  u & ∈ & \Card & =   & \{ 0 ⊏ 1 ⊏ ω \} ⊂ ℕ_ω \\
  \text{Usage Domain}      &  d & ∈ & \UsgD & =   & \Var \to \Card \\
 \end{array} \quad
 \begin{array}{rcl}
   (ρ_1 ⊔ ρ_2)(\px) & = & ρ_1(\px) ⊔ ρ_2(\px) \\
   (ρ_1 + ρ_2)(\px) & = & ρ_1(\px) + ρ_2(\px) \\
   (u * ρ_1)(\px)   & = & u * ρ_1(\px) \\
 \end{array}
 \\[-0.5em]
\end{array}\]
\subcaption{Syntax of expressions and semantic domains}
  \label{fig:scott-syntax}
\newcommand{\scalefactordenot}{0.92}
\scalebox{\scalefactordenot}{%
\begin{minipage}{0.49\textwidth}
\arraycolsep=0pt
\[\begin{array}{rcl}
  \multicolumn{3}{c}{ \ruleform{ \semscott{\wild} \colon \Exp → (\Var \to \ScottD) → \ScottD } } \\
  \\[-0.5em]
  \semscott{\px}_ρ & {}={} & ρ(\px) \\
  \semscott{\Lam{\px}{\pe}}_ρ & {}={} & d ↦ \semscott{\pe}_{ρ[\px ↦ d]} \\
  \semscott{\pe~\px}_ρ & {}={} & \begin{cases}
     f(ρ(x)) & \text{if $\semscott{\pe} = f$}  \\
     \bot   & \text{otherwise}  \\
   \end{cases} \\
  \semscott{\Letsmall{\px}{\pe_1}{\pe_2}}_ρ & {}={} &
    \begin{letarray}
      \text{letrec}~ρ'. & ρ' = ρ \mathord{⊔} [\px \mathord{↦} d_1] \\
                        & d_1 = \semscott{\pe_1}_{ρ'} \\
      \text{in}         & \semscott{\pe_2}_{ρ'}
    \end{letarray} \\
\end{array}\]
\subcaption{\relscale{\fpeval{1/\scalefactordenot}} Denotational semantics after Scott}
  \label{fig:denotational}
\end{minipage}%
\quad
\begin{minipage}{0.56\textwidth}
\arraycolsep=0pt
\[\begin{array}{rcl}
  \multicolumn{3}{c}{ \ruleform{ \semusg{\wild} \colon \Exp → (\Var → \UsgD) → \UsgD } } \\
  \\[-0.5em]
  \semusg{\px}_ρ & {}={} & ρ(\px) \\
  \semusg{\Lam{\px}{\pe}}_ρ & {}={} & ω*\semusg{\pe}_{ρ[\px ↦ \bot]} \\
  \semusg{\pe~\px}_ρ & {}={} & \semusg{\pe} + ω*ρ(\px)
    \phantom{\begin{cases}
       f(ρ(x)) & \text{if $\semscott{\pe} = f$}  \\
       \bot   & \text{otherwise}  \\
     \end{cases}} \\
  \semusg{\Letsmall{\px}{\pe_1}{\pe_2}}_ρ& {}={} & \begin{letarray}
      \text{letrec}~ρ'. & ρ' = ρ \mathord{⊔} [\px \mathord{↦} d_1] \\
                        & d_1 = [\px\mathord{↦}1] \mathord{+} \semscott{\pe_1}_{ρ'} \\
      \text{in}         & \semusg{\pe_2}_{ρ'}
    \end{letarray}
\end{array}\]
\subcaption{\relscale{\fpeval{1/\scalefactordenot}} Naïve usage analysis}
  \label{fig:usage}
\end{minipage}
}
\end{minipage}
  \label{fig:intro}
\caption{Connecting usage analysis to denotational semantics}
\end{figure}

\Cref{fig:scott-syntax} defines the labelled syntax of a lambda calculus with
recursive let bindings and algebraic data types, reminiscent of
\citet{Sestoft:97}. The calculus is factored into \emph{administrative normal
form}, that is, the arguments of applications are restricted to be variables, so
the difference between call-by-name and call-by-value manifests purely in the
semantics of $\mathbf{let}$.
In this section, only the highlighted parts are relevant; we will ignore labels
and data types for brevity.

We give a standard call-by-name denotational semantics $\semscott{\wild}$ in
\Cref{fig:denotational} \citep{ScottStrachey:71}, assigning meaning to our
syntax by means of the infamous Scott domain $\ScottD$.
Squinting a bit, we find that it looks quite similar to the function
to its right in \Cref{fig:usage}, depicting a \emph{usage analysis}
$\semusg{\wild}$, a static analysis for estimating an upper bound on how often
a variable is evaluated. The given analysis is naïve in that its treatment of
function application assumes that every function deeply evaluates its argument.

\subsubsection{Deadness, Continuity and Divergence}

Assuming that all program variables are distinct (a silent assumption from
here on throughout), the result of $\semusg{\wild}$ is an element $d ∈ \UsgD$,
an environment that maps to each variable an upper bound on its \emph{evaluation
cardinality}, that is, how often the variable is evaluated over the cause of any
of its activations.
Whenever $0 ⊏ d(x)$, we say that it is \emph{potentially live} in $d$ and
extend this meaning to a program $\pe$ whenever $\semusg{\pe} = d$.
Likewise, when $d(x) = 0$ we say that $x$ is \emph{dead} in $d$ and the programs
$d$ denotes, \eg, \emph{never evaluated}. In this way, $\semusg{\wild}$ can
be used to infer facts of the form ``$\pe$ never evaluates $x$'' from the
introduction.

The \emph{requirement} (in the sense of informal specification) on an assertion
such as ``$x$ is dead'' in a program like $\Let{x}{\pe_1}{\pe_2}$ is that we
may rewrite to $\pe_2$ (perhaps to reduce heap allocations or code size) without
observing any change in semantics.

This can be made formal in the following definition of deadness in terms of
$\semscott{\wild}$:

\begin{definition}[Deadness]
  \label{defn:deadness}
  A variable $\px$ is \emph{dead} in an expression $\pe$ if and only
  if, for all $ρ ∈ \Var \to \ScottD$ and $d_1, d_2 ∈ \ScottD$, we have
  $\semscott{\pe}_{ρ[\px↦d_1]} = \semscott{\pe}_{ρ[\px↦d_2]}$.
  Otherwise, $\px$ is \emph{potentially live}.
\end{definition}

Indeed, if we know that $x$ is dead, then the following equation justifies our
rewrite above: $\semscott{\Let{x}{\pe_1}{\pe_2}}_ρ = \semscott{\pe_2}_{ρ[x↦d]} =
\semscott{\pe_2}_ρ$ (for all $ρ$ and the suitable $d$).
So our definition of deadness is not only simple to grasp, but also simple to
exploit.

We can now try to prove our usage analysis correct as a liveness analysis in
terms of this notion of deadness. After a bit of trial and error, we could
arrive at the following theorem:

\begin{theorem}[$\semusg{\wild}$ is a correct potential liveness analysis]
  \label{thm:semusg-correct-live}
  Let $\pe$ be an expression and $\px$ a variable.
  Then $\px$ is dead in $\pe$ whenever
  there exists $\tr ∈ \Var \to \UsgD$ such that
  $\tr(\px) \not⊑ \semusg{\pe}_{\tr}$.
\end{theorem}
\begin{proof}
  By induction over $\pe$. The full proof can be found in
  \Cref{prf:semusg-correct-1}.
\end{proof}

Let us stop and reflect about this theorem for a bit.
Whenever thinking about the witnessing $\tr$ giving values to free variables of
$\pe$, it helps to think that $\tr(\px) \triangleq \fn{\py}{\ternary{\px = \py}{1}{0}}$, then the
intuitive notion of deadness translates directly.%
\footnote{In fact, it can be proven that \emph{if} any $\tr$ exists, then the
``diagonal'' $\fn{\px~\py}{\ternary{\px = \py}{1}{0}}$ is also a witness.}

It is surprising that the theorem does not relate $\tr$ with $ρ$; after all,
$ρ(\py)$ (for $\py \not= \px$) might be bound to the \emph{meaning} of an
expression that is potentially live in $\px$, such as $\semscott{\px}_{ρ'}$, and
we have no way to observe the dependency on $\px$ just through $ρ(\py)$.
The key is to realise that our notion of deadness varies $ρ(\px)$ (the meaning
of $\px$), but that does not vary $ρ(\py)$, because that only sees $ρ'(\px)$,
so for all intents and purposes, the proof may assume that $ρ(\py)$ is dead in
$\px$.
The analysis, on the other hand, encodes transitive deadness relationships via
$\tr(\px) ⊑ \tr(\py)$ in case $\px$ occurs in the RHS of a $\mathsf{let}$-bound
$\py$ to encode that deadness of $\py$ is a necessary condition for deadness of
$\px$.

The proof capitalises on the similarities in structure by using induction on the
program expression, hence it is simple and direct, at just under a page of
accessible prose. Often, such a proof needs to assume conformance to some
logical relation to strengthen the induction hypothesis for the application
case or prove admissability to apply fixpoint induction in the let case, but for
deadness and our very simple analysis we do not need to be so crafty.

Nevermind our confidence in the ultimate correctness of $\semusg{\wild}$,
note that our notion of deadness has a blind spot \wrt diverging computations:
A looping program is automatically dead in all its free variables.
This is a central weakness of \emph{any} notion of correctness defined in terms
of traditional denotational semantics because it assigns $\bot$ to any diverging
computation. We could take inspiration in the work of \citet{Milner:78} and
navigate around the issue by introducing a $\mathbf{wrong}$ denotation which is
propagated strictly; then we would notice when we optimise a looping program
into one that has a scoping error (the only kind of stuckness that our calculus
admits without data types).

\begin{figure}
\[\begin{array}{c}
 \begin{array}{rrclcl}
  \text{LK States}     & σ   & ∈ & \States        & =      & \Controls \times \Environments \times \Heaps \times \Continuations \\
  \text{Controls}      & \pc & ∈ & \Controls      & =      & \pe \mid \pv \\
  \text{Environments}  & ρ   & ∈ & \Environments  & =      & \Var \pfun \Addresses \\
  \text{Addresses}     & \pa & ∈ & \Addresses     & \simeq & ℕ \\
  \text{Heaps}         & μ   & ∈ & \Heaps         & =      & \Addresses \pfun \Environments \times \Exp \\
  \text{Continuations} & κ   & ∈ & \Continuations & ::=    & \StopF \mid \ApplyF(\pa) \pushF κ \mid \SelF(ρ,\SelArity) \pushF κ \mid \UpdateF(\pa) \pushF κ \\
 \end{array} \\
  \\[-0.5em]
\end{array}\]

\newcolumntype{L}{>{$}l<{$}} % math-mode version of "l" column type
\newcolumntype{R}{>{$}r<{$}} % math-mode version of "r" column type
\newcolumntype{C}{>{$}c<{$}} % math-mode version of "c" column type
\resizebox{\textwidth}{!}{%
\begin{tabular}{LR@{\hspace{0.4em}}C@{\hspace{0.4em}}LL}
\toprule
\text{Rule} & σ_1 & \smallstep & σ_2 & \text{where} \\
\midrule
\BindT & (\Let{\px}{\pe_1}{\pe_2},ρ,μ,κ) & \smallstep & (\pe_2,ρ',μ[\pa↦(ρ',\pe_1)], κ) & \pa \not∈ \dom(μ),\ ρ'\! = ρ[\px↦\pa] \\
\AppIT & (\pe~\px,ρ,μ,κ) & \smallstep & (\pe,ρ,μ,\ApplyF(\pa) \pushF κ) & \pa = ρ(\px) \\
\CaseIT & (\Case{\pe}{\Sel},ρ,μ,κ) & \smallstep & (\pe,ρ,μ,\SelF(ρ,\Sel) \pushF κ) & \\
\LookupT & (\px, ρ, μ, κ) & \smallstep & (\pe, ρ', μ, \UpdateF(\pa) \pushF κ) & \pa = ρ(\px),\ (ρ',\pe) = μ(\pa) \\
\AppET & (\Lam{\px}{\pe},ρ,μ, \ApplyF(\pa) \pushF κ) & \smallstep & (\pe,ρ[\px ↦ \pa],μ,κ) &  \\
\CaseET & (K'~\many{y},ρ,μ, \SelF(ρ',\Sel) \pushF κ) & \smallstep & (\pe_i,ρ'[\many{\px_i ↦ \pa}],μ,κ) & K_i = K',\ \many{\pa = ρ(\py)} \\
\UpdateT & (\pv, ρ, μ, \UpdateF(\pa) \pushF κ) & \smallstep & (\pv, ρ, μ[\pa ↦ (ρ,\pv)], κ) & \\
\bottomrule
\end{tabular}
} % resizebox
\caption{Lazy Krivine transition semantics $\smallstep$}
  \label{fig:lk-semantics}
\end{figure}

Fortunately, our analysis actually behaves more benevolently than ``necessary'',
so we are hopeful that it adheres to a stricter notion of deadness. Indeed, we
can give one in terms of a small-step operational semantics such as the Mark
II machine of \citet{Sestoft:97} given in \Cref{fig:lk-semantics}, the Gold
Standard for this work. (A close sibling for call-by-value would be a CESK machine
\citep{Felleisen:87} or a simpler derivative thereof.) It is a variant of the
Lazy Krivine (LK) machine implementing call-by-need, so for a meaningful
comparison to $\semscott{\wild}$, we ignore rules $\CaseIT, \CaseET, \UpdateT$
and the pushing of update frames in $\LookupT$ for now to recover a call-by-name
semantics.%
\footnote{Note that discarding update frames makes the heap append-only, which
simplifies our proofs. Of course, for call-by-name we would not need a heap to
begin with, but the point is to get a glimpse at the effort necessary for
call-by-need.}

The configurations in this transition system resemble abstract machine
states, consisting of control expression $\pe$, an environment $ρ$ mapping
lexically-scoped variables to their current heap address, a heap $μ$ listing a
closure for each adress, and a stack of continuation frames $κ$.
Now we are able to define a notion of ``strong deadness'', as follows

\begin{definition}[Deadness, Mark II]
  Let $\pe$ be an expression and $\px$ a variable.
  $\px$ is \emph{dead} in $\pe$ if and only if
  for any $\pe_1,\pe_2$ and evaluation context $(ρ,μ,κ)$
  (where $\px$ does not occur in the context)
  the sequences of transitions $(\Let{\px}{\pe_1}{\pe},ρ,μ,κ) \smallstep^*$
  and $(\Let{\px}{\pe_2}{\pe},ρ,μ,κ) \smallstep^*$ operate in lockstep.
  Otherwise, $\px$ is \emph{potentially live}.
\end{definition}

While this definition captures diverging behaviors correctly and is intuitively
easy to grasp, it is quite a narrow property. Knowing that $x$ is dead in $\pe$ does not give us
any information on whether $x$ is dead in $\pe~y$, because that evaluates
$\pe$ in a non-empty stack! Consequently, it is impossible to prove that
$\semusg{\pe}$ satisfies \Cref{thm:semusg-correct-live} by structural induction
on $\pe$ (in a way that doing so would be useful).

Instead, such proofs are often conducted by induction over the reflexive
transitive closure of $(\pe,[],[],\StopF) \smallstep^* σ$, for any $σ$.
For that it is necessary to give an inductive hypothesis that considers
environments, stacks and heaps.
One way is to extend the analysis function $\semusg{\wild}$ to entire
configurations and then prove that if $σ_1 \smallstep σ_2$ we have $\semusg{σ_2}
⊑ F(\semusg{σ_1})$, where $F$ is the abstraction of the particular transition
rule taken and is often left implicit.
This is a daunting task, for multiple reasons:
First off, $\semusg{\wild}$ might be quite complicated in practice and extending
it to entire configurations multiplies this complexity.
Secondly, $\semusg{\wild}$ makes use of fixpoints in the let case and
undoubtedly needs some more fixpoints in its extension to the heap,
so $\semusg{σ_2} ⊑ F(\semusg{σ_1})$ relates fixpoints that are ``out of sync'',
implying a need for fixpoint induction for every transition that touches
the heap.

(In call-by-need, there will be a fixpoint between the heap and stack due to
update frames acting somewhat like heap bindings, so fixpoint induction needs to
be applied \emph{at every case of the proof}, diminishing confidence in
correctness unless the proof is fully mechanised.)

For an analogy with type systems: What we just tried is like attempting a proof
of preservation by referencing the result of an inference algorithm rather than
the declarative type system. So what is often done instead is to define a
declarative and more permissive \emph{correctness relation} $C(σ)$ to prove
preservation $C(σ_1) \Rightarrow C(σ_2)$ (\eg, that $C$ is \emph{logical} \wrt
$\smallstep$). $C$ is chosen such that
  (1) it is strong enough to imply the property of interest (deadness)
  (2) it is weak enough that it is implied by the analysis result for an initial state ($\tr(\px) \not⊑ \semusg{\pe}_{\tr}$).
In our case, the correctness predicate $D = \gfp(F)$ is the greatest fixpoint of
the functional%
\footnote{Alternatively, we could have stratified the recursion and applied
step-indexing~\citep{DreyerAhmedBirkedal:11}.}
\[\begin{array}{rcl}
  F(D)(\pe,ρ,μ,κ)(A) & \triangleq & \forall \pa ∈ \rng(κ).\ \pa \not∈ A \wedge (\forall μ',κ'.\ D(μ(\pa),μ',κ')(A)) \\
  F(D)(\px,ρ,μ,κ)(A) & \triangleq & ρ(\px) \not∈ A \wedge \forall (ρ',\pe)=μ(ρ(\px)).\ D(\pe,ρ',μ,κ)\\
  F(D)(\pe~\px,ρ,μ,κ)(A) & \triangleq & ρ(\px) \not∈ A \wedge F(D)(\pe,ρ,μ,\ApplyF(ρ(\px)) \pushF κ) \\
  F(D)(\Lam{\px}{\pe},ρ,μ,\StopF)(A) & \triangleq & true \\
  F(D)(\Lam{\px}{\pe},ρ,μ,\ApplyF(\pa) \pushF κ)(A) & \triangleq & D(\pe,ρ[\px↦\pa],μ,κ) \\
  F(D)(\Let{\px}{\pe_1}{\pe_2},ρ,μ,κ)(A) & \triangleq & F(D)(\pe_2,ρ,μ[\pa↦(ρ',\pe_1)],κ)(A) \\
\end{array}\]

task,
or define a notion of
``well-annotatedness'' that
It is possible to define the action
an inductive hypothesis and first had to be strengthened into a \emph{logical
relation}.
Then the next question is: Under which conditions can this statement be
generalised to executions of the form $(\pe,ρ_1,μ_1,κ_1) \smallstep^*
(\px,ρ_2,μ_2,κ_2)$ and still be simple to prove? It's a bit tricky: For any
address $\pa$ in the range of $ρ$ or pushed as an apply frame onto the stack, we
have to assume that the closure $μ(\pa)$ satisfies this generalised condition,
too.

No, not immediately:
First off, an argument pushed onto the stack $κ$ might evaluate $\px$ although
$\pe$ itself is dead
Thus, we have to generalise the
inductive hypothesis

While more accurate a definition, the corresponding correctness proof
requires an elaborate setup, extending the liveness analysis (or the correctness
criterion) to every component of the machine states, \eg, the heap $μ$,
the environment $ρ$ and the stack $κ$. Doing so usually culminates in
an extensive logical relation such as $\sim_V$ in \citep[Theorem
2.21]{Nielson:99} or the ``well-typed'' relation for elaborated configurations
in \citep[Lemma 4.3]{cardinality-ext} (only the extended version has the proof
and full typing relations for heaps and stacks in the Appendix).

\subsubsection{Evaluation Cardinality and Call-by-need}
Our usage analysis can be used to justify optimisations beyond dead code
elimination. Consider $\Let{id}{\Lam{x}{x}}{\Let{t}{id}{t~t}}$



The situation is a bit less dire for call-by-name and call-by-value calculi
because their heap is immutable. Furthermore, if evaluation cardinality is
unimportant, lambda-bound variables can be substituted immediately for their
arguments during $β$-reduction, hence no need for a heap whatsoever. The
stack can be reflected back into the premises of the judgment rules, so that
the system distinguishes \emph{instruction transitions} from \emph{search
transitions}, a distinction which is made explicit in a \emph{contextual
semantics}. Happily, such a contextual semantics also exists for call-by-need
\citep{Ariola:95}, although its complexity is comparatively mind-boggling.
All these considerations are with the ultimate goal of simplifying proofs:
Instead of defining an inductive well-formedness predicate on the heap, we prove
a substitution lemma. Instead of a well-formedness predicate for the stack, we
appeal to the well-formedness of the search transition rule.
\sg{Urgh, more rambling. This was an important realisation to me, although I
fear it doesn't really belong here. Perhaps it belongs in the call-by-need
subsection?}

To re-iterate: The mismatch in structure between analysis and semantics leads to
a drastic increase in proof complexity to bridge said gap. It would be desirable
to find a proof framework that allows us to re-use this bridge by means of
abstract interpretation, in the spirit of \citet{Cousot:21}.

Another way to work around the structure gap is to adopt the structure of the
semantics in the analysis; this is done in the Abstracting Abstract
Machines work \citep{aam}. To our knowledge, its exclusive application seems
to be control-flow analysis \citep{Shivers:91}, so that the analyses and
optimisations that follow do not need to reason about arbitrary function call
structure and can apply traditional intraprocedural analysis techniques that are
well-explored in the imperative world. \sg{Move this babbling to Related Work?}

\subsection{Type Analysis}

Perhaps it's best if we simply give the analysis at the end with the adequate
correctness criterion.

\subsection{Control-Flow Analysis}

Dito

\subsection{Call-by-need and Evaluation Cardinality}

While the traditional denotational semantics distinguishes call-by-name from
call-by-value by termination, it has no means to distinguish call-by-need from
call-by-name as there is no explicit notion of state or ``observable evaluation''.

Furthermore, apart from the denotational property of deadness and strictness,
which corresponds to the operational interpretation of ``never evaluated'' and
``evaluated at least once or diverging'', there is no general way to observe
properties of \emph{evaluation cardinality} such as ``evaluated at most once''
(affine) or even ``called at most once'', such as inferred by the Glasgow
Haskell Compiler \citep{cardinality}.

You might still think that the symptoms described in this subsection should
hardly affect the semantics of call-by-value languages, but the applications
of evaluation cardinality go beyond lazy languages such as Haskell.
For example, it affects linear, uniqueness and quantitative type
theory.
%, although~\citet{Atkey:18} is able to give a categorical semantics that
%postulates observability of cardinality in a suitable \emph{$R$-Quantitative Category with Families}.
%\sg{I don't really understand what Atkey does there?! If anything, our
%semantics is simpler... But this is quickly drifting towards Related Work}
Another example is recent work on re-use analysis in reference-counted language
implementations~\citep{Ullrich:19,perceus}.
\emph{Even if} we were uncaring towards infinite behaviors%
\footnote{Such as strongly-normalising type theories},
or had a denotational semantics that allowed us to observe them (but not
cardinality) as suggested in the introduction, the soundness proof for these
type systems would still need to resort to an operational semantics for the lack
of observable evaluation cardinality or postulate observability as is the case
for the categorical semantics of \citet{Atkey:18}.

\begin{theorem}[Correctness of $\semusg{\wild}$]
  \label{thm:semusg-correct-1}
  Let $\pe$ be an expression and $\px$ a variable.
  Then $\px$ is evaluated at most $u$ times whenever
  there exists $\tr ∈ \Var \to \UsgD$ such that
  $(u+1)*\tr(\px) \not⊑ \semusg{\pe}_{\tr}$.
\end{theorem}
% Curiosly, this is sound for call-by-name. Call-by-need would be a bit more complicated.

\section{Introduction}
\label{sec:introduction}

As an implementor of a programming language, it is often useful to automatically
glean facts about a program such as ``this program is well-typed'', ``this
higher-order function is always called with argument $\Lam{x}{x+1}$'' or ``this
program never evaluates $x$'' by way of \emph{static (program) analysis}.

If the implementation language is a functional one, then usually such static
analyses are formulated as a function defined by \emph{structural recursion} on
the input term.
For example, given an application expression $(\pe_1~\pe_2)$,
the property ``$(\pe_1~\pe_2)$ never evaluates its free variable $x$'' can be
\emph{conservatively approximated} (here: It's OK to say ``No'' more often) by
the property ``$\pe_1$ and $\pe_2$ never evaluate $x$''.

It is quite convenient to formulate \emph{potential liveness analysis}
structurally:
(1) Structural recursion gives an immediate proof of termination and can
    further be exploited in other inductive proofs, all within decidable
    territory.
(2) A structurally-defined function $f$ is often \emph{compositional}, meaning that
    the result of $f\denot{\pe_1~\pe_2}$ is a function of $f\denot{\pe_1}$ and
    $f\denot{\pe_2}$, but not of the structure of either $\pe_1$ or $\pe_2$.
    Compositionality makes it easy for humans to understand and reason about the
    function, as well as allowing to abstract away program syntax in the first
    place.

For static analyses, especially more complicated ones, it is good practice to
provide a proof of correctness of some sort. If the correctness statement can
be expressed in terms of a \emph{denotational
semantics}~\citep{ScottStrachey:71}, then the recursion structure of analysis
function and semantics function line up nicely. As a result, the proof of
correctness can be conducted by simple induction over the expression and
abstract interpretation~\citep{Cousot:21} provides a neat framework to derive
static analyses that are proven correct by simple calculation.

\subsection{Domain Theory is leaky and too abstract}
Alas, there are several shortcomings to traditional denotational semantics:
\begin{itemize}
  \item Infinite program behaviors are denoted by $\bot$, the least element of
        the approximation order on elements of an algebraic domain.
        This encoding necessitates semantic properties to be proven
        \emph{admissable} before they can be used. This means that although
        the denotational semantics might be ``standard'', the semantic domain
        can't be treated like a black box --- the abstraction is leaky and
        $\bot$ is lurking everywhere to make life complicated.
  \item Traditional domain theory proved non-compositional~\citep{WrightFelleisen:94},
        in that language features such as exceptions, state and nondeterminism
        require individual, complicated adjustments where it is far from clear
        how to combine these adjustments.
  \item There have been denotational semantics for call-by-name and
        call-by-value, but none for call-by-need. That is a symptom of the fact
        that traditional domain theory eschews any account of operational
        detail, such as evaluation cardinality.
\end{itemize}

The third point is important in the context of quantitative type
systems~\citep{Atkey:18} and optimising compilers for call-by-need languages such
as the Glasgow Haskell Compiler.
The first two points have been noted by \citet{WrightFelleisen:94}, establishing
the now standard approach of operational semantics as the Gold Standard.
The past twenty years have proven that the methodology scales to
big and complicate type systems, building on techniques such as
\emph{step-indexing}~\citep{AppelMcAllester:01,DreyerAhmedBirkedal:11}.

\subsection{Rewrite systems are syntactic and not structural}
However, transition systems don't easily admit a compositional definition as a
function of the input expression \sg{Cousot has derived trace-based fixpoint
semantics from a transition system~\citep{Cousot:02}, but the resulting function
is not structurally-recursive in the expression}. In giving in to operational
semantics, semanticists have chosen predictable step-wise reasoning over elegant
equational reasoning backed by compositionality.

In preservation proofs of type systems, step-wise reasoning necessitates
substitution lemmas to cope with the non-structural recursion during variable
lookup and beta reduction, an inconvenience that is readily accepted when the
alternative would be to tame domain theory.

It is not so simple in lattice-based compiler analyses, because recursive let
bindings invite fixpoint iteration. If you think about ``well-annotatedness''
of a machine configuration according to an analysis as a type system, the
corresponding ``preservation lemma'' needs to relate fixpoints before and after
\emph{every} kind of machine transition! This necessitates a dreadful application of fixpoint
induction, unfolding the whole analysis function body before and after and thus
culminating in beaurocratic nightmare.
Of course, the definition of ``well-annotatedness'' needs to encompass whole
machine configurations, including the mutually recursive stack and heap, see the
proofs in~\citep{cardinality-ext} for a taste. It is hard to raise confidence in
such a proof without full mechanisation.
A denotational semantics would arguably simplify the proof, unfortunately one
allowing to observe evaluation cardinality of call-by-need is lacking to this
day.

When there is no semantics that structurally matches our analysis, we could
try to rearrange our analysis. That is the idea of \emph{Abstracting Abstract
Machines}~\citep{aam} which abandons compositional analysis in favour of a
convenient proof by abstract interpretation of the transition semantics.
Not only is it deeply unsatisfying to be pushed into such rearrangements, it
is also impossible to perform in a production-grade compiler because of the
inevitable regressions it introduces.

\subsection{Summary of Contributions}

Our ambitious goal is to resolve the aforementioned tensions via the following
contributions:
\begin{itemize}
  \item In \Cref{sec:problem}, we highlight the main points raised in the
    Introduction by way of example analyses.
  \item In \Cref{sec:semantics}, we give a compositional semantics
    for call-by-need lambda calculus that generates potentially infinite
    small-step traces. Unlike traditional formulations of denotational
    semantics, our semantics is defined by \emph{guarded recursion} and thus
    total as a mathematical function. Our semantics is also distinct from
    similar ones for call-by-name or call-by-value and allows to observe
    evaluation cardinality as needed.
    We believe that our semantics is the first with the aforementioned two
    qualities and prove it correct \wrt a standard operational semantics. The
    idea borrows heavily from the idea of a maximal prefix trace semantics
    advocated by \citet{Cousot:21}.
  \item The semantics in \Cref{sec:semantics} is one generating \emph{stateful}
    traces in a standard operational semantics, and serves mostly as a
    convenient bridge for proving bisimulation \sg{No longer}. In
    \Cref{sec:stateless} we will define an equivalent, but more convenient
    \emph{stateless} semantics and we will see how to recover necessary state
    from program history as needed.
  \item We employ the stateless semantics as a collecting semantics and derive
    $\semlive{\wild}$ by calculational design \citep{Cousot:21}.
    Similar derivations will be made for a simple type system as well as for
    control flow analysis. \sg{Hopefully :)}
  \item Talk about prototype in Haskell?
  \item Related Work \Cref{sec:related-work}
\end{itemize}

% TO RECYCLE:
%
%\paragraph{Analysis follows structure}
%Alas, even when the denotational semantics is ``standard'', the hard part is in
%coming up with a suitable correctness predicate!
%
%Traditionally, denotational semantics denotes diverging and stuck programs with
%$\bot$, corresponding to the function that is undefined everywhere.
%There is rich and complicated literature on defining an algebraic
%domain~\citep{Scott:71} that is suitable to denote untyped lambda calculus.
%The key is embedding the subclass of \emph{continuous} functions between domain
%elements into the domain itself. All computable functions are continuous,
%so an algebraic domain is a sufficient substrate for giving meaning to all
%computable functions.
%
%Now, to prove a predicate $P(d)$ of some denotation $d$, such as ``$d$ has type
%$τ$'' by structural induction, one has to show first that $P$ is compatible
%with continuity; perhaps by proving that $P$ characterises a sub-domain or an
%ideal in the domain~\citep{Milner:78}.
%
%Note that this is all \emph{before} even attempting the proof! Often, the
%denotational semantics (for example to distinguish \textbf{wrong} behaviors from
%diverging ones~\citep{Milner:78}) or even the domain itself needs adjustments.
%The former case needs a proof of continuity, while the latter can be quite
%involved and non-compositional for effects such as exceptions, concurrency and
%state, as \citet{WrightFelleisen:94} noted.
%
%We think that most of the troubles in the application of domain theory are
%ultimately caused by the non-commital nature of the approximation order, in that
%any predicate on total elements also needs to accept its partial approximations;
%hence it is desirable to strive for \emph{total} descriptions of the potentially
%program infinite behaviors.
%
%\citet{WrightFelleisen:94} give a detailed account of the shortcomings of
%partial denotational semantics and propose to use small-step operational
%semantics paired with the now standard syntactic approach to type soundness.
%They demonstrate that their approach scales well to composition of different
%language features such as state, exceptions and nondeterminism. The past
%twenty years have scaled the approach to big and complicated type systems,
%
%What's more, small-step operational semantics distinguish infinite
%and stuck behaviors quite naturally; a distinction that is crucial
%in (a correctness property of) control-flow and liveness analysis.
%
%the crucial property that any
%, and it was
%a great achievement of \citet{Scott:71} turns out that Domain Theory is the result of equipping Doing so invites
%and stuck programs. Consider the following program using recursive let that is
%infinitely-looping
%\begin{equation}
%  \label{eqn:loop}
%  \pe_{loop} \triangleq \Let{id}{\Lam{x}{x}}{\Let{loop}{id~loop}{loop}}
%\end{equation}
%The traditional denotational semantics after Scott and Strachey would equate all
%of the following programs:
%$\semscott{\pe_{loop}}_ρ = \semscott{\Let{loop}{id~loop}{loop}}_ρ =
%\semscott{\mathsf{segfault}}_ρ = \bot$.
%Note that the first program has an infinite loop, the second one is not
%well-scoped (thus stuck at some point) and the last one is a straight out crash
%in the style of an imprecise exception \citep{imprecise-exceptions}.
%To a compiler developer, this conflation is both a reason for joy (more
%optimisation opportunities) and a reason for ... reflection (users didn't expect
%their infinite loop to be optimised into a crash). Such issues come up in
%practice \sg{cite GHC issues}.
%
%More seriously, it is impossible to prove by way of a denotational semantics
%that a static analysis does not misoptimise infinite behaviors.
%(a) The \emph{potential liveness} analysis that says ``$\pe_{loop}$ never
%evaluates $id$'' could be proven ``correct''.
%(b) A type analysis that says ``$\Let{loop}{id~loop}{loop}$ is closed and
%well-typed'' can still be proven progressing as long as $\pe_{loop}$ can
%be well-typed (which would not be too surprising), because the latter is
%denotationally equivalent to the former.
%(c) Imagine that $id$ was supplied as a parameter to $loop$ instead, \eg
%$...\ \Let{loop}{\Lam{f}{f~(loop~f)}}{loop~id}$. Then a control-flow analysis
%\citep{Shivers:91} that says ``$f$ is never bound to $id$'' can be proven correct
%in terms of the denotational semantics.
%
%Furthermore, although it is sensible (in the terminating case) to ask whether or
%not $x$ is \emph{never} evaluated in terms of the denotational semantics, asking
%whether $x$ is evaluated \emph{at most once} is not, for the same reason that
%traditional denotational semantics is not able to discern call-by-name from
%call-by-need. Yet, to the Glasgow Haskell Compiler, this distinction is very
%much of concern!
%
%When denotational semantics fails the compiler developer, they turn to a
%correctness criterion in terms of a \emph{structural operational semantics}.
%This was the approach taken by \citep{cardinality} to prove evaluation
%cardinality properties such as potential liveness. The drawback of this proof
%framework is the immense complexity arising from the disconnect between a
%structural definition and a transition system; matters such as substitution,
%multiple heap activations of the let binding, non-determinism and fixpoint
%induction abound.
%One could adopt the approach of \emph{Abstracting Abstract Machines} \citep{aam}
%and let the structure of the semantics dictate the structure of the
%analysis for a re-usable proof of correctness via abstract interpretation
%\citep{Cousot:21}.
%However, that is not how the static analyses work that the authors are familiar
%with.
%For example, it would be quite an effort to rewrite the neat,
%structurally-defined analyses of the Glasgow Haskell Compiler into a fixpoint
%iteration on the approximated states of an abstract transition system.

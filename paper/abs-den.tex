% -*- mode: LaTeX -*-
%% For double-blind review submission, w/o CCS and ACM Reference (max submission space)
%\documentclass[acmsmall,review,anonymous,natbib=false]{acmart}\settopmatter{printfolios=true,printccs=false,printacmref=false}
%% For double-blind review submission, w/ CCS and ACM Reference
%\documentclass[acmsmall,review,anonymous]{acmart}\settopmatter{printfolios=true}
%% For single-blind review submission, w/o CCS and ACM Reference (max submission space)
\documentclass[acmsmall,review,anonymous]{acmart}\settopmatter{printfolios=true,printccs=false,printacmref=false}
%% For single-blind review submission, w/ CCS and ACM Reference
%\documentclass[acmsmall,review]{acmart}\settopmatter{printfolios=true}
%% For final camera-ready submission, w/ required CCS and ACM Reference
%\documentclass[acmsmall,screen]{acmart}\settopmatter{}

%\documentclass[acmsmall,review,anonymous]{acmart}\settopmatter{printfolios=true,printccs=false,printacmref=false}

%% Journal information
%%
\setcopyright{rightsretained}
\acmPrice{}
\acmDOI{10.1145/1111111}
\acmYear{2024}
\copyrightyear{2024}
%\acmSubmissionID{popl24main-p11-p}
\acmJournal{PACMPL}
\acmVolume{1}
\acmNumber{POPL}
\acmArticle{1}
\acmMonth{1}

%% Bibliography style
\bibliographystyle{ACM-Reference-Format}
%% Citation style
%% Note: author/year citations are required for papers published as an
%% issue of PACMPL.
\citestyle{acmauthoryear}   %% For author/year citations

% Some conditional build stuff for handling the Appendix

\newif\ifmain
\newif\ifappendix

% Builds only the main paper by default.
\maintrue
\appendixfalse
% But we provide a switch to build the Appendix only.
\def\appendixonly{\mainfalse{}\appendixtrue{}}

% .. so that you can comment out the following line to build the Appendix only
% This is done by the `make appendix.pdf` target.
%\appendixonly

% Same thing for an extended version that includes the Appendix
\def\extended{\maintrue{}\appendixtrue{}}
%\extended

%%%%%%%
\ifappendix
\usepackage[appendix=inline]{apxproof}
\else
\usepackage[appendix=strip]{apxproof}
\fi
\usepackage{array} % \newcolumntype
\usepackage{color}
\usepackage{ifdraft}
%\usepackage[svgnames]{xcolor}
\usepackage{cleveref}
\usepackage{xspace}
\usepackage{url}
\usepackage{varwidth}
\usepackage{galois}
\usepackage{hsyl-listing} % SML listing style, hsyl-style
%\usepackage{scalerel}
%\usepackage[all]{xy}
\usepackage{relsize} % relscale
\usepackage{xfp} % fpeval
%\usepackage{stackengine}
\usepackage{mathtools} % xhookrightarrow
\usepackage{trimclip} % clipbox
\usepackage{mathpartir} % inference rules
\usepackage{subcaption}
\usepackage{mathbbol} % \bbcolon and \bbquestionmark
\usepackage{stmaryrd} % \lightning
\usepackage{tikz}
\usepackage{witharrows}
\usetikzlibrary{cd} % commutative diagrams
\usetikzlibrary{calc}
\usetikzlibrary{fit}
\usetikzlibrary{patterns}
\usetikzlibrary{matrix}
\usetikzlibrary{decorations.pathreplacing}
\usetikzlibrary{decorations.pathmorphing}
\usepackage{placeins} % flush floats with \FloatBarrier
\usepackage[T1]{fontenc} % https://tex.stackexchange.com/a/181119

\usepackage{utf8-symbols}
% Theorems
\theoremstyle{plain} % default
\newtheorem{theorem}{Theorem}
\newtheorem{lemma}[theorem]{Lemma}
\newtheorem{proposition}[theorem]{Proposition}
\crefname{proof}{Proof}{Proofs}

\theoremstyle{definition}
\newtheorem{example}[theorem]{Example}
\newtheorem{definition}[theorem]{Definition}

% Abbrev
\newcommand\eg{\emph{e.g.}\ }
\newcommand\etc{\emph{etc.}}

% Comments and notes
\newcommand{\slpj}[1]{\emph{SLPJ: #1}}
\newenvironment{slpjenv}{\em SLPJ:}{}
\newcommand{\sg}[1]{\emph{SG: #1}}
\newcommand{\todo}[1]{\textcolor{red}{TODO: #1}}

% Auxiliary
\newcommand{\many}[1]{\overline{#1}}
\newcommand{\wild}{\ensuremath{\mathunderscore}}
\newcommand{\pfun}{\rightharpoonup}
\newcommand{\ruleform}[1]{\fbox{$#1$}}
\newcommand{\highlight}[1]{\setlength{\fboxsep}{2pt}\colorbox[gray]{0.8}{\ensuremath{#1}}}
\newcommand{\nelems}[1]{\lvert#1\rvert}
\newcommand{\fn}[2]{\ensuremath{λ#1.\ #2}}
\newcommand{\constfn}[1]{\fn{\mathunderscore}{#1}}
%\newcommand{\repeat}[2]{\foreach \n in {1,...,#1}{#2}} % https://tex.stackexchange.com/a/16190/52414
% https://tex.stackexchange.com/a/21647
\makeatletter
\newcommand{\superimpose}[3][\mathord]{#1{\mathpalette\superimpose@{{#2}{#3}}}}
\newcommand{\superimpose@}[2]{\superimpose@@{#1}#2}
\newcommand{\superimpose@@}[3]{%
  \ooalign{%
    \hfil$\m@th#1#2$\hfil\cr
    \hfil$\m@th#1#3$\hfil\cr
  }%
}
\newenvironment{letarray}{\begin{array}{@{}l@{\hspace{1ex}}l}}{\end{array}}

% Syntax
%% LC
\newcommand{\Con}{\mathsf{Con}}
\newcommand{\Var}{\mathsf{Var}}
\newcommand{\Lab}{\mathsf{Lab}}
\newcommand{\Exp}{\mathsf{Exp}}
\newcommand{\Val}{\mathsf{Val}}
\newcommand{\px}{\mathsf{x}}
\newcommand{\py}{\mathsf{y}}
\newcommand{\pz}{\mathsf{z}}
\newcommand{\pv}{\mathsf{v}}
\newcommand{\pe}{\mathsf{e}}
\newcommand{\Lam}[2]{\lambda #1. #2}
\newcommand{\LLam}[3]{\lambda_{#1} #2. #3}
\newcommand{\Let}[3]{\mathbf{let}~{#1}={#2}~\mathbf{in}~{#3}}
\newcommand{\LLet}[4]{\mathbf{let}_{#1}~{#2}={#3}~\mathbf{in}~{#4}}
\newcommand{\Case}[2]{\mathbf{case}~{#1}~\mathbf{of}~{#2}}
\newcommand{\Sel}[1][]{\many{K~\many{x} \rightarrow e_{#1}}}
\newcommand{\dom}{\mathop\mathsf{dom}}
\newcommand{\rng}{\mathop\mathsf{rng}}
\newcommand{\fv}{\mathop\mathsf{fv}}
\newcommand{\bv}{\mathop\mathsf{bv}}
\newcommand{\fresh}[2]{#1\mathbin{\#}#2}
\newcommand{\delvar}[1]{{\setminus_{#1}}}
%% Actions
\newcommand{\Actions}{\mathbb{A}}
\newcommand{\ValA}{\mathsf{val}}
\newcommand{\BindA}{\mathsf{bind}}
\newcommand{\LookupA}{\mathsf{look}}
\newcommand{\UpdateA}{\mathsf{upd}}
\newcommand{\AppIA}{\mathsf{app}_1}
\newcommand{\AppEA}{\mathsf{app}_2}
\newcommand{\CaseIA}{\mathsf{case}_1}
\newcommand{\CaseEA}{\mathsf{case}_2}
%% Labels
\newcommand{\Labels}{\mathbb{L}}
\newcommand{\lbl}{\ensuremath{\ell}}
\newcommand{\lbln}[1]{\ensuremath{\ell_{#1}}}
\newcommand{\slbl}{\raisebox{0.08em}{$\;\scriptstyle{\ell}\;$}}
\newcommand{\slbln}[1]{\raisebox{0.1em}{$\;\scriptstyle{\ell_{#1}}\;$}}
\newcommand{\atlbl}[1]{at(#1)}
\newcommand{\deref}[1]{*#1}
%% Traces
\newcommand{\Traces}{\mathbb{T}}
\newcommand{\act}[1]{\xrightarrow{#1}}
% https://latex.org/forum/viewtopic.php?p=62177&sid=26890ac77d076f3338384a47a2ffd4bc#p62177
\makeatletter
\newcommand{\concatop}[1]{%
  \mathbin{\mathop{#1}\limits^{\vbox to 1\ex@{\kern-\tw@\ex@
   \hbox{\scriptsize$\smallfrown$}\vss}}}}
\makeatother
\makeatletter
\newcommand{\compop}[1]{%
  \mathbin{\vec{\ooalign{$\circ$\cr\hidewidth$#1$\hidewidth}}}}
\makeatother
\newcommand{\concat}{\concatop{\cdot}}
\newcommand{\fcomp}{\compop{\cdot}}
\newcommand{\balanced}[1]{#1\;\mathsf{bal}}
\newcommand{\getval}[2]{#1 \Downarrow #2}
\newcommand{\lookup}[2]{#1\;\mathsf{lookup}}
\newcommand{\subtrceq}{\superimpose[\mathrel]{\clipbox{0pt 0pt {0.3\width} 0pt}{${\subset}$}}{\hspace{0.6ex}\cdot}}
\newcommand{\rightsubtrceq}{\subtrceq_r}
\newcommand{\leftsubtrceq}{\subtrceq_l}
\newcommand{\stepm}[3]{\lceil #1 \act{#2} #3 \rceil}
\newcommand{\step}[2]{\stepm{\wild}{#1}{#2}}

%% Values
\newcommand{\Values}{\mathbb{V}}
\newcommand{\FunV}{\mathsf{fun}}
\newcommand{\ConV}{\mathsf{con}}
%% Domains
\newcommand{\Environments}{\mathbb{E}}
\newcommand{\Heaps}{\mathbb{H}}
\newcommand{\Continuations}{\mathbb{K}}
\newcommand{\Domain}[1]{{\mathbb{D}^{#1}}}
\newcommand{\PrefD}{\Domain{*}}
\newcommand{\MaxD}{\Domain{+\infty}}
\newcommand{\AbsD}{\Domain{\raisebox{0.1em}{\scalebox{0.4}{$\square$}}}}
\newcommand{\AbsCD}{\Domain{\raisebox{0.1em}{\scalebox{0.4}{c$\square$}}}}
\newcommand{\StateD}{\Domain{\States}}
\newcommand{\SSD}{\Domain{Σ}}
\newcommand{\SSCD}{\Domain{cΣ}}
\newcommand{\LiveD}{\Domain{\exists l}}
\newcommand{\LiveSD}{\Domain{sl}}
\newcommand{\LiveCD}{\Domain{cl}}
\newcommand{\LiveSCD}{\Domain{scl}}
\newcommand{\testD}{\mathbin{?\!!}}

% Math
\newcommand{\poset}[1]{\wp(#1)}
\newcommand{\Nat}{\mathbb{N}}
\DeclareMathSymbol{\bbcolon}{\mathpunct}{bbold}{"3A}
\DeclareMathSymbol{\bbquestionmark}{\mathpunct}{bbold}{"3F}
\DeclareMathSymbol{\bblparen}{\mathpunct}{bbold}{"28}
\DeclareMathSymbol{\bbrparen}{\mathpunct}{bbold}{"29}
\DeclareMathSymbol{\bbpipe}{\mathpunct}{bbold}{"7C}
\newcommand{\ternary}[3]{\bblparen\, #1 \,\mathrel{\bbquestionmark}\, #2 \,\mathrel{\bbcolon}\, #3 \,\bbrparen}

% Order theory
\DeclareMathOperator*{\lub}{\sqcup}
\DeclareMathOperator*{\Lub}{\bigsqcup}
\DeclareMathOperator*{\glb}{\sqcap}
\DeclareMathOperator*{\Glb}{\bigsqcap}
\newcommand{\lfp}{\mathsf{lfp}}
\newcommand{\gfp}{\mathsf{gfp}}

% Semantics
\newcommand{\denot}[1]{\llbracket {#1} \rrbracket}
\newcommand{\sempref}[1]{\mathcal{S}^*\denot{#1}}
\newcommand{\seminf}[1]{\mathcal{S}^{+\infty}\denot{#1}}
\newcommand{\semst}[1]{\mathcal{S}^{\States}\denot{#1}}
\newcommand{\semss}[1]{\mathcal{S}^{Σ}\denot{#1}}
\newcommand{\semlive}[1]{\mathcal{S}^{∃l}\denot{#1}}
\newcommand{\semclive}[1]{\mathcal{S}^{cl}\denot{#1}}
\newcommand{\semslive}[1]{\mathcal{S}^{sl}\denot{#1}}
\newcommand{\semsclive}[1]{\mathcal{S}^{scl}\denot{#1}}
\newcommand{\semty}[1]{\mathcal{S}^{τ}\denot{#1}}

% Stateful
\newcommand{\States}{\mathbb{S}}
\newcommand{\STraces}{\mathbb{S}^{+\infty}}
\newcommand{\sconcat}{\concatop{\hspace{0.04ex}\scalebox{0.5}{$\States$}}}
\newcommand{\sfcomp}{\compop{\scalebox{0.25}{\raisebox{0.7em}{$\States$}}}}
\newcommand{\pushF}{\cdot}
\newcommand{\StopF}{\mathbf{stop}}
\newcommand{\ReturnF}{\mathbf{ret}}
\newcommand{\ApplyF}{\mathbf{ap}}
\newcommand{\UpdateF}{\mathbf{upd}}

% Small-step
\newcommand{\Configurations}{\mathsf{Config}}
\newcommand{\Addresses}{\mathsf{Addr}}
%\newcommand{\Heaps}{\mathsf{Heap}}
\newcommand{\Envs}{\mathsf{Env}}
\newcommand{\Stacks}{\mathsf{Stack}}
\newcommand{\Frames}{\mathsf{Frame}}
\newcommand{\pH}{\mathsf{H}}
\newcommand{\pa}{\mathsf{a}}
\newcommand{\pE}{\mathsf{E}}
\newcommand{\pS}{\mathsf{S}}
\newcommand{\SSTraces}{\mathsf{SSTrace}}
\newcommand{\smallstep}{\hookrightarrow}
\newcommand{\straceend}{\scalebox{0.5}{$\square$}}
\newcommand{\ssconcat}{\concatop{\hspace{0.04ex}\scalebox{0.5}{\Sigma}}}
\newcommand{\AppIT}{\textsc{app}_1}
\newcommand{\AppET}{\textsc{app}_2}
\newcommand{\ValueT}{\textsc{val}}
\newcommand{\LookupT}{\textsc{look}}
\newcommand{\UpdateT}{\textsc{upd}}
\newcommand{\LetT}{\textsc{let}}
\newcommand{\funnyComp}{\mathop{\mathord{>}\!\mathord{\circ}\!\mathord{>}}}
\newcommand{\validtrace}[1]{#1\;\mathsf{ok\text{-}}π}
\newcommand{\maxbaltrace}[1]{#1\;\mathsf{maxbal\text{-}}π}
\newcommand{\validtracefun}[1]{#1\;\mathsf{ok\text{-}}d}

% Liveness
\newcommand{\lStacks}{\Stacks^l}
\newcommand{\lLiveness}{\mathsf{Live}}
\newcommand{\lS}{\pS^l}
\newcommand{\lSBot}{\texttt{Seq}}
\newcommand{\lSAp}[1]{\texttt{\$[$#1$]}}
\newcommand{\lSTop}{\texttt{Deep}}
\newcommand{\lAbs}{\texttt{Abs}}
\newcommand{\lUsed}[1]{\texttt{Used[$#1$]}}
\newcommand{\SVar}{\mathsf{SVar}}
\newcommand{\SCall}{\mathsf{SCall}}

% Types
\newcommand{\TyCon}{\mathsf{TyCon}}
\newcommand{\Type}{\mathsf{Type}}
\newcommand{\ArrowTy}{\Rightarrow}


\ifnonanon{\usepackage[mark]{gitinfo2}}

% Tables should have the caption above
\floatstyle{plaintop}
\restylefloat{table}

\begin{document}

%\special{papersize=8.5in,11in}
\setlength{\pdfpageheight}{\paperheight}
\setlength{\pdfpagewidth}{\paperwidth}

\title{Compositional Trace Semantics for Lambda Calculus}
%\subtitle{Or: Denotational semantics for Call-by-need Lambda Calculus}

\author{Sebastian Graf}
\affiliation{%
  \institution{Karlsruhe Institute of Technology}
  \city{Karlsruhe}
  \country{Germany}
}
\email{sgraf1337@gmail.com}

\author{Simon Peyton Jones}
\affiliation{%
  \institution{Epic Games}
  \city{Cambridge}
  \country{UK}
}
\email{simon.peytonjones@gmail.com}

\ifmain

\begin{abstract}
  We present a class of denotational semantics for lambda calculus that
  generates coinductive traces of a corresponding small-step operational
  semantics.
  The appeal of our method is that the generated traces can be elaborated with
  as much operational detail as needed, and that the structural definition
  enables favorable induction principles.
  Use of Guarded Domain Theory enables mutable state and thus the first
  denotational semantics for call-by-need lambda calculus.
  We demonstrate the usefulness of our trace semantics by proving correct a
  static analysis for estimating the number of variable lookups by abstract
  interpretation.
\end{abstract}

%% 2012 ACM Computing Classification System (CSS) concepts
%% Generate at 'http://dl.acm.org/ccs/ccs.cfm'.
\begin{CCSXML}
<ccs2012>
   <concept>
       <concept_id>10011007.10011006.10011041</concept_id>
       <concept_desc>Software and its engineering~Compilers</concept_desc>
       <concept_significance>500</concept_significance>
       </concept>
   <concept>
       <concept_id>10011007.10011006.10011073</concept_id>
       <concept_desc>Software and its engineering~Software maintenance tools</concept_desc>
       <concept_significance>300</concept_significance>
       </concept>
   <concept>
       <concept_id>10011007.10011006.10011008.10011024.10011035</concept_id>
       <concept_desc>Software and its engineering~Procedures, functions and subroutines</concept_desc>
       <concept_significance>100</concept_significance>
       </concept>
   <concept>
       <concept_id>10011007.10011006.10011008.10011024.10011032</concept_id>
       <concept_desc>Software and its engineering~Constraints</concept_desc>
       <concept_significance>300</concept_significance>
       </concept>
   <concept>
       <concept_id>10011007.10011006.10011008.10011009.10011012</concept_id>
       <concept_desc>Software and its engineering~Functional languages</concept_desc>
       <concept_significance>300</concept_significance>
       </concept>
   <concept>
       <concept_id>10011007.10011006.10011008.10011009.10011021</concept_id>
       <concept_desc>Software and its engineering~Multiparadigm languages</concept_desc>
       <concept_significance>300</concept_significance>
       </concept>
 </ccs2012>
\end{CCSXML}

\ccsdesc[500]{Software and its engineering~Compilers}
\ccsdesc[300]{Software and its engineering~Software maintenance tools}
\ccsdesc[100]{Software and its engineering~Procedures, functions and subroutines}
\ccsdesc[300]{Software and its engineering~Constraints}
\ccsdesc[300]{Software and its engineering~Functional languages}
\ccsdesc[300]{Software and its engineering~Multiparadigm languages}
%% End of generated code

%% Keywords
%% comma separated list
\keywords{Programming language semantics}  %% \keywords are mandatory in final camera-ready submission

\maketitle

\section{Introduction}
\label{sec:introduction}

As an implementor of a programming language, it is often useful to automatically
glean facts about a program such as ``this program is well-typed'', ``this
higher-order function is always called with argument $\Lam{x}{x+1}$'' or ``this
program never evaluates $x$'' by way of \emph{static (program) analysis}.

\paragraph{Analysis follows structure}
If the implementation language is a functional one, then usually such static
analyses are formulated as a function defined by \emph{structural recursion} on
the input term.
For example, given an application expression $(\pe_1~\pe_2)$,
the property ``$(\pe_1~\pe_2)$ never evaluates its free variable $x$'' can be
\emph{conservatively approximated} (here: It's OK to say ``No'' more often) by
the property ``$\pe_1$ and $\pe_2$ never evaluate $x$''.

Such a structural formulation is quite convenient:
(1) Structural recursion gives an immediate proof of termination and can
    further be exploited in other inductive proofs, all within decidable
    territory.
(2) A structurally-defined function $f$ is often \emph{compositional}, meaning that
    if you replace a sub-expression $e$ in a bigger expression $C[e]$ by another
    expression $e'$ with $f(e) = f(e')$, the overall result
    of the containing expression $f(C[e]) = f(C[e'])$ does not change.
    This makes it easy for humans to understand and reason about the function,
    because the result of a big expression depends on the results of its parts
    (and not on the shape of the sub-expressions themselves).

For static analyses, especially more complicated ones, it is good practice to
provide a proof of correctness of some sort. If the correctness statement can
be expressed in terms of a \emph{denotational
semantics}~\cite{ScottStrachey:71}, then the recursion structure of analysis
function and semantics function line up nicely. As a result, the proof of
correctness can be conducted by simple induction over the expression.

\paragraph{Domain Theory is a leaky abstraction}
Alas, even when the denotational semantics is ``standard'', the hard part is in
coming up with a suitable correctness predicate!
Traditionally, the semantic domain of denotational semantics models diverging
and stuck programs with $\bot$, the function that is undefined everywhere.
There is rich and complicated literature on defining an algebraic
domain~\cite{Scott:71} that is suitable to denote untyped lambda calculus.
The key is embedding the subclass of \emph{continuous} functions between domain
elements into the domain itself. All computable functions can be proven
continuous, so this is a sufficient substrate.
Now, to prove a predicate $P(d)$ of some denotation $d$, such as ``$d$ has type
$τ$'' by structural induction, one has to show first that $P$ is compatible
with continuity; perhaps by proving that $P$ characterises a sub-domain or an
ideal in the domain~\cite{Milner:78}.

Note that this is all \emph{before} even attempting the proof! Often, the
denotational semantics or even the domain itself needs adjustments.
The former case needs a proof of continuity, while the latter can be quite
involved and non-compositional for effects such as exceptions, concurrency and
state, as \citep{WrightFelleisen:87} noted.

We think that most of the troubles in the application of Domain Theory are
caused by the non-commital nature of the approximation order, in that any
predicate on total elements also needs to accept its partial approximation;
hence it is desirable to strive for \emph{total} descriptions of the
potentially program infinite behaviors.

the crucial property that any
, and it was
a great achievement of \citep{Scott:71} turns out that Domain Theory is the result of equipping Doing so invites
and stuck programs. Consider the following program using recursive let that is
infinitely-looping
\begin{equation}
  \label{eqn:loop}
  \pe_{loop} \triangleq \Let{id}{\Lam{x}{x}}{\Let{loop}{id~loop}{loop}}
\end{equation}
The traditional denotational semantics after Scott and Strachey would equate all
of the following programs:
$\semscott{\pe_{loop}}_ρ = \semscott{\Let{loop}{id~loop}{loop}}_ρ =
\semscott{\mathsf{segfault}}_ρ = \bot$.
Note that the first program has an infinite loop, the second one is not
well-scoped (thus stuck at some point) and the last one is a straight out crash
in the style of an imprecise exception \cite{imprecise-exceptions}.
To a compiler developer, this conflation is both a reason for joy (more
optimisation opportunities) and a reason for ... reflection (users didn't expect
their infinite loop to be optimised into a crash). Such issues come up in
practice \sg{cite GHC issues}.

More seriously, it is impossible to prove by way of a denotational semantics
that a static analysis does not misoptimise infinite behaviors.
(a) The \emph{potential liveness} analysis that says ``$\pe_{loop}$ never
evaluates $id$'' could be proven ``correct''.
(b) A type analysis that says ``$\Let{loop}{id~loop}{loop}$ is closed and
well-typed'' can still be proven progressing as long as $\pe_{loop}$ can
be well-typed (which would not be too surprising), because the latter is
denotationally equivalent to the former.
(c) Imagine that $id$ was supplied as a parameter to $loop$ instead, \eg
$...\ \Let{loop}{\Lam{f}{f~(loop~f)}}{loop~id}$. Then a control-flow analysis
\cite{Shivers:91} that says ``$f$ is never bound to $id$'' can be proven correct
in terms of the denotational semantics.

Furthermore, although it is sensible (in the terminating case) to ask whether or
not $x$ is \emph{never} evaluated in terms of the denotational semantics, asking
whether $x$ is evaluated \emph{at most once} is not, for the same reason that
traditional denotational semantics is not able to discern call-by-name from
call-by-need. Yet, to the Glasgow Haskell Compiler, this distinction is very
much of concern!

When denotational semantics fails the compiler developer, they turn to a
correctness criterion in terms of a \emph{structural operational semantics}.
This was the approach taken by \cite{cardinality} to prove evaluation
cardinality properties such as potential liveness. The drawback of this proof
framework is the immense complexity arising from the disconnect between a
structural definition and a transition system; matters such as substitution,
multiple heap activations of the let binding, non-determinism and fixpoint
induction abound. It is hard to raise confidence in such a proof without full
mechanisation.

One could adopt the approach of \emph{Abstracting Abstract Machines} \cite{aam}
and let the structure of the semantics dictate the structure of the
analysis for a re-usable proof of correctness via abstract interpretation
\cite{Cousot:21}.
However, that is not how the static analyses work that the authors are familiar
with.
For example, it would be quite an effort to rewrite the neat,
structurally-defined analyses of the Glasgow Haskell Compiler into a fixpoint
iteration on the approximated states of an abstract transition system.

The contributions of this work are as follows:
\begin{itemize}
  \item In \Cref{sec:problem}, we give a more formal exposition to the
    problems we just introduced and are inclined to solve.
  \item In \Cref{sec:semantics}, we give a \emph{structurally-defined} semantics
    for lambda calculus that is \emph{able to discern stuck, diverging and
    converging programs}. Furthermore, it is a semantics for \emph{call-by-need}
    lambda calculus that is distinct from similar ones for call-by-name or
    call-by-value and allows to observe evaluation cardinality as needed.
    We believe that our semantics is the first with the aforementioned two
    qualities and prove it correct \wrt a standard operational semantics. The
    idea borrows heavily from the idea of a maximal prefix trace semantics
    advocated by \citep{Cousot:21}.
  \item The semantics in \Cref{sec:semantics} is one generating \emph{stateful}
    traces in a standard operational semantics, and serves mostly as a
    convenient bridge for proving bisimulation. In \Cref{sec:stateless} we will
    define an equivalent, but more convenient \emph{stateless} semantics and we
    will see how to recover necessary state from program history as needed.
  \item We employ the stateless semantics as a collecting semantics and derive
    $\semlive{\wild}$ by calculational design \cite{Cousot:21}.
    Similar derivations will be made for a simple type system as well as for
    control flow analysis. \sg{Hopefully :)}
  \item Talk about prototype in Haskell?
  \item Related Work \Cref{sec:related-work}
\end{itemize}

\section{Problem Statement}
\label{sec:problem}

By way of the poster child example of a compositional definition of \emph{usage
analysis}, we showcase how the operational detail available in traditional
denotational semantics is too coarse to substantiate a correctness criterion,
although the proof of (a weaker notion of) correctness is simple and direct.
While operational semantics observe sufficient detail to formulate a correctness
criterion, it is quite complicated to come up with a suitable inductive
hypothesis for the correctness proof.

%\subsection{Notation}
%
%Collection of stuff to explain, for now:
%\begin{itemize}
%  \item $\triangleq$ for defining an object (defn eq), rather than $=$
%  \item $\text{letrec}$
%    The (meta-level, math) notation
%    \[
%    \text{letrec}~l.~\many{x = rhs_x} ~ l = rhs_{l} ~ \many{y = rhs_y}~\text{in}~body
%    \]
%    where $l$ might occur freely in any $rhs_{\wild}$ and $body$, is syntactic sugar for
%    \[
%    snd(\lfp(\fn{(l,\wild)}{\text{let}~\many{x = rhs_x} ~ \many{y = rhs_y}~\text{in}~(rhs_{l},body)}))
%    \]
%    Where $\lfp$ is the least fixpoint operator and $snd(a,b) = b$. Clearly, this
%    desugaring's use of $\lfp$ is well-defined for its use on elements of the
%    powerset lattice $\UsgD$.
%
%    But \citep{Shivers:91} uses the similar $\text{whererec}$ and gets by without
%    ever explaining it, so we might as well.
%\end{itemize}

\subsection{Usage Analysis and Deadness, Intuitively}

\begin{figure}
\begin{minipage}{\textwidth}
\[\begin{array}{c}
 \arraycolsep=3pt
 \begin{array}{rrclcrrclcl}
  \text{Variables}    & \px, \py & ∈ & \Var        &     & \quad \text{Constructors} &        K & ∈ & \Con        &     & \text{with arity $α_K ∈ ℕ$} \\
  \text{Labels}       &     \lbl & ∈ & \Labels     &     & \quad \text{Values}       &      \pv & ∈ & \Val        & ::= & \highlight{\Lam{\px}{\pe}} \mid K~\many{\px}^{α_K} \\
  \text{Expressions}  &      \pe & ∈ & \Exp        & ::= & \multicolumn{6}{l}{\highlight{\slbl \px \mid \slbl \pv \mid \slbl \pe~\px \mid \slbl \Let{\px}{\pe_1}{\pe_2}} \mid \slbl \Case{\pe}{\SelArity}} \\
  \\[-0.5em]
 \end{array} \\
 \\[-0.5em]
 \begin{array}{rrclcl}
  \text{Scott Domain}      &  d & ∈ & \ScottD & =   & [\ScottD \to_c \ScottD]_\bot \\
  \text{Usage cardinality} &  u & ∈ & \Card & =   & \{ 0 ⊏ 1 ⊏ ω \} ⊂ ℕ_ω \\
  \text{Usage Domain}      &  d & ∈ & \UsgD & =   & \Var \to \Card \\
 \end{array} \quad
 \begin{array}{rcl}
   (ρ_1 ⊔ ρ_2)(\px) & = & ρ_1(\px) ⊔ ρ_2(\px) \\
   (ρ_1 + ρ_2)(\px) & = & ρ_1(\px) + ρ_2(\px) \\
   (u * ρ_1)(\px)   & = & u * ρ_1(\px) \\
 \end{array}
 \\[-0.5em]
\end{array}\]
\subcaption{Syntax of expressions and semantic domains}
  \label{fig:syntax}
\newcommand{\scalefactordenot}{0.92}
\scalebox{\scalefactordenot}{%
\begin{minipage}{0.49\textwidth}
\arraycolsep=0pt
\[\begin{array}{rcl}
  \multicolumn{3}{c}{ \ruleform{ \semscott{\wild} \colon \Exp → (\Var \to \ScottD) → \ScottD } } \\
  \\[-0.5em]
  \semscott{\px}_ρ & {}={} & ρ(\px) \\
  \semscott{\Lam{\px}{\pe}}_ρ & {}={} & d ↦ \semscott{\pe}_{ρ[\px ↦ d]} \\
  \semscott{\pe~\px}_ρ & {}={} & \begin{cases}
     f(ρ(x)) & \text{if $\semscott{\pe} = f$}  \\
     \bot   & \text{otherwise}  \\
   \end{cases} \\
  \semscott{\Letsmall{\px}{\pe_1}{\pe_2}}_ρ & {}={} &
    \begin{letarray}
      \text{letrec}~ρ'. & ρ' = ρ \mathord{⊔} [\px \mathord{↦} d_1] \\
                        & d_1 = \semscott{\pe_1}_{ρ'} \\
      \text{in}         & \semscott{\pe_2}_{ρ'}
    \end{letarray} \\
\end{array}\]
\subcaption{\relscale{\fpeval{1/\scalefactordenot}} Denotational semantics after Scott}
  \label{fig:denotational}
\end{minipage}%
\quad
\begin{minipage}{0.56\textwidth}
\arraycolsep=0pt
\[\begin{array}{rcl}
  \multicolumn{3}{c}{ \ruleform{ \semusg{\wild} \colon \Exp → (\Var → \UsgD) → \UsgD } } \\
  \\[-0.5em]
  \semusg{\px}_ρ & {}={} & ρ(\px) \\
  \semusg{\Lam{\px}{\pe}}_ρ & {}={} & ω*\semusg{\pe}_{ρ[\px ↦ \bot]} \\
  \semusg{\pe~\px}_ρ & {}={} & \semusg{\pe} + ω*ρ(\px)
    \phantom{\begin{cases}
       f(ρ(x)) & \text{if $\semscott{\pe} = f$}  \\
       \bot   & \text{otherwise}  \\
     \end{cases}} \\
  \semusg{\Letsmall{\px}{\pe_1}{\pe_2}}_ρ& {}={} & \begin{letarray}
      \text{letrec}~ρ'. & ρ' = ρ \mathord{⊔} [\px \mathord{↦} d_1] \\
                        & d_1 = [\px\mathord{↦}1] \mathord{+} \semscott{\pe_1}_{ρ'} \\
      \text{in}         & \semusg{\pe_2}_{ρ'}
    \end{letarray}
\end{array}\]
\subcaption{\relscale{\fpeval{1/\scalefactordenot}} Naïve usage analysis}
  \label{fig:usage}
\end{minipage}
}
\end{minipage}
  \label{fig:intro}
\caption{Connecting usage analysis to denotational semantics}
\end{figure}

\Cref{fig:syntax} defines the labelled syntax of a lambda calculus with
recursive let bindings and algebraic data types, reminiscent of
\citet{Sestoft:97}. The calculus is factored into \emph{administrative normal
form}, that is, the arguments of applications are restricted to be variables, so
the difference between call-by-name and call-by-value manifests purely in the
semantics of $\mathbf{let}$.
In this section, only the highlighted parts are relevant; we will ignore labels
and data types for brevity.

We give a standard call-by-name denotational semantics $\semscott{\wild}$ in
\Cref{fig:denotational} \citep{ScottStrachey:71}, assigning meaning to our
syntax by means of the infamous Scott domain $\ScottD$.
Squinting a bit, we find that it looks quite similar to the function
to its right in \Cref{fig:usage}, depicting a \emph{usage analysis}
$\semusg{\wild}$, a static analysis for estimating an upper bound on how often
a variable is evaluated. The given analysis is naïve in that its treatment of
function application assumes that every function deeply evaluates its argument.

Assuming that all program variables are distinct (a silent assumption from
here on throughout), the result of $\semusg{\wild}$ is an element $d ∈ \UsgD$,
an environment that maps to each variable an upper bound on its \emph{evaluation
cardinality}, that is, how often the variable is evaluated over the cause of any
of its activations.
Whenever $0 ⊏ d(x)$, we say that it is \emph{potentially live} in $d$ and
extend this meaning to a program $\pe$ whenever $\semusg{\pe} = d$.
Likewise, when $d(x) = 0$ we say that $x$ is \emph{dead} in $d$ and the programs
$d$ denotes, \eg, \emph{never evaluated}. In this way, $\semusg{\wild}$ can
be used to infer facts of the form ``$\pe$ never evaluates $x$'' from the
introduction.

\subsection{Denotational Deadness, Continuity and Divergence}

The \emph{requirement} (in the sense of informal specification) on an assertion
such as ``$x$ is dead'' in a program like $\Let{x}{\pe_1}{\pe_2}$ is that we
may rewrite to $\Let{x}{panic}{\pe_2}$ and perhaps even to $\pe_2$ without
observing any change in semantics. Doing so reduces code size and heap
allocation.

This can be made formal in the following definition of deadness in terms of
$\semscott{\wild}$:

\begin{definition}[Deadness]
  \label{defn:deadness}
  A variable $\px$ is \emph{dead} in an expression $\pe$ if and only
  if, for all $ρ ∈ \Var \to \ScottD$ and $d_1, d_2 ∈ \ScottD$, we have
  $\semscott{\pe}_{ρ[\px↦d_1]} = \semscott{\pe}_{ρ[\px↦d_2]}$.
  Otherwise, $\px$ is \emph{live}.
\end{definition}

Indeed, if we know that $x$ is dead, then the following equation justifies our
rewrite above: $\semscott{\Let{x}{\pe_1}{\pe_2}}_ρ = \semscott{\pe_2}_{ρ[x↦d]} =
\semscott{\pe_2}_ρ$ (for all $ρ$ and the suitable $d$).
So our definition of deadness is not only simple to grasp, but also simple to
exploit.

We can now try to prove our usage analysis correct as a liveness analysis in
terms of this notion of deadness. After a bit of trial and error, we could
arrive at the following theorem:

\begin{theorem}[$\semusg{\wild}$ is a correct potential liveness analysis]
  \label{thm:semusg-correct-live}
  Let $\pe$ be an expression and $\px$ a variable.
  Then $\px$ is dead in $\pe$ whenever
  there exists $\tr ∈ \Var \to \UsgD$ such that
  $\tr(\px) \not⊑ \semusg{\pe}_{\tr}$.
\end{theorem}
\begin{proof}
  By induction over $\pe$. The full proof can be found in
  \Cref{prf:semusg-correct-1}.
\end{proof}

Let us stop and reflect about this theorem for a bit.
Deadness is witnessed by a particular $\tr$ and it helps to think of this
witness as the ``diagonal'' $\tr(\px) \triangleq \fn{\py}{\ternary{\px =
\py}{1}{0}}$, because then the intuitive notion of deadness applies.%
\footnote{In fact, it can be proven that if \emph{any} $\tr$ exists, then the
diagonal is also a witness.}

It is surprising that the theorem does not relate $\tr$ with $ρ$; after all,
$ρ(\py)$ (for $\py \not= \px$) might be bound to the \emph{meaning} of an
expression that is potentially live in $\px$, such as $\semscott{\px}_{ρ'}$, and
we have no way to observe the dependency on $\px$ just through $ρ(\py)$.
The key is to realise that our notion of deadness varies $ρ(\px)$ (the meaning
of $\px$), but that does not vary $ρ(\py)$, because that only sees $ρ'(\px)$,
so for all intents and purposes, the proof may assume that $ρ(\py)$ is dead in
$\px$.
The analysis, on the other hand, encodes transitive deadness relationships via
$\tr(\px) ⊑ \tr(\py)$ in case $\px$ occurs in the RHS of a $\mathsf{let}$-bound
$\py$ to encode that deadness of $\py$ is a necessary condition for deadness of
$\px$.

The proof capitalises on the similarities in structure by using induction on the
program expression, hence it is simple and direct, at just under a page of
accessible prose. Often, such a proof needs to strengthen the induction
hypothesis for the application case, or prove admissability of a predicate to
apply fixpoint induction in the let case, but for deadness and our very simple
analysis we do not need to be so crafty.

Nevermind our confidence in the ultimate correctness of $\semusg{\wild}$,
note that our notion of deadness has a blind spot \wrt diverging computations:
A looping program is automatically dead in all its free variables, even though
any of them might influence which particular endless loop is taken.

This is not a curiosity of $\semusg{\wild}$; it also applies to the original
control-flow analysis work~\citep[p. 23]{Shivers:91} where it is remedied
by the introduction of a \emph{non-standard semantic interpretation} that
assigns meaning to diverging programs where the denotational semantics only
says $\bot$. Credibility of this approach solely rests on the structural
similarity to the standard denotational semantics.

So the issue is not with $\semusg{\wild}$ but with traditional denotational
semantics because it (necessarily) assigns $\bot$ to any diverging computation.
Furthermore, as is often done, $\semscott{\wild}$ abuses $\bot$ as a collecting
pool for error cases.
This shows in the following example:
$x$ is dead in $(\Lam{y}{\Lam{z}{z}})~x$, but rewriting
$\Let{x}{\pe_1}{(\Lam{y}{\Lam{z}{z}})~x}$ to $(\Lam{y}{\Lam{z}{z}})~x$
introduces a scoping error, a change that is not observable under
$\semscott{\wild}$.
We could take inspiration in the work of \citet{Milner:78}
and navigate around the issue by introducing a $\mathbf{wrong}$ denotation for
errors which is propagated strictly; then we would notice when we optimise a
looping program into one that has a scoping error (the only kind of stuckness
that our calculus admits without data types).

However, $\bot$ is still there as the denotation of diverging computations;
hence a predicate such as ``Denotation $d$ will get stuck and not diverge'' is
not an admissable one, because an admissable predicate would be true for $\bot$.

\subsection{Evaluation Cardinality and Call-by-need}
Blind spots notwithstanding, the notion of deadness above is quite reasonable.
But our usage analysis infers more detailed cardinality information; for
example, it can infer whether a binding is evaluated at most once.
This information can be useful under call-by-need to omit pushing of update
frames~\citep{cardinality-ext}.%
\footnote{A more useful application of the ``at most once'' cardinality is the
notion of a \emph{one-shot} lambda~\citep{cardinality-ext}, a function which is
called at most once for every activation, because it allows floating of heap
allocations from a hot code path into cold function bodies.
Simplicity prohibits $\semusg{\wild}$ from inferring such properties.}
Thus, our usage analysis should satisfy the following generalisation of
\Cref{thm:semusg-correct-live}:

\begin{theorem}[Correctness of $\semusg{\wild}$]
  \label{thm:semusg-correct-2}
  Let $\pe$ be an expression and $\px$ a variable.
  Then $\pe$ evaluates $\px$ at most $u$ times whenever
  there exists $\tr ∈ \Var \to \UsgD$ such that
  $(u+1)*\tr(\px) \not⊑ \semusg{\pe}_{\tr}$.
\end{theorem}

Unfortunately, our denotational semantics does not allow us to express the
operational property ``$\pe$ evaluates $\px$ at most $u$ times'', so
this theorem cannot be proven correct.

% We should probably mvoe this RElated Work? Don't want to discuss it here
%The problem of observable cardinality also comes up in Quantitative Type
%Theory~\citep{Atkey:18}, where the solution is to give a categorical
%semantics that postulates observability of cardinality in a suitable
%\emph{$R$-Quantitative Category with Families} without giving a concrete
%model.

\begin{figure}
\[\begin{array}{c}
 \begin{array}{rrclcl}
  \text{LK States}     & σ   & ∈ & \States        & =      & \Controls \times \Environments \times \Heaps \times \Continuations \\
  \text{Controls}      & \pc & ∈ & \Controls      & =      & \Exp \\
  \text{Environments}  & ρ   & ∈ & \Environments  & =      & \Var \pfun \Addresses \\
  \text{Addresses}     & \pa & ∈ & \Addresses     & \simeq & ℕ \\
  \text{Heaps}         & μ   & ∈ & \Heaps         & =      & \Addresses \pfun \Environments \times \Exp \\
  \text{Continuations} & κ   & ∈ & \Continuations & ::=    & \StopF \mid \ApplyF(\pa) \pushF κ \mid \SelF(ρ,\SelArity) \pushF κ \mid \UpdateF(\pa) \pushF κ \\
 \end{array} \\
  \\[-0.5em]
\end{array}\]

\newcolumntype{L}{>{$}l<{$}} % math-mode version of "l" column type
\newcolumntype{R}{>{$}r<{$}} % math-mode version of "r" column type
\newcolumntype{C}{>{$}c<{$}} % math-mode version of "c" column type
\resizebox{\textwidth}{!}{%
\begin{tabular}{LR@{\hspace{0.4em}}C@{\hspace{0.4em}}LL}
\toprule
\text{Rule} & σ_1 & \smallstep & σ_2 & \text{where} \\
\midrule
\BindT & (\Let{\px}{\pe_1}{\pe_2},ρ,μ,κ) & \smallstep & (\pe_2,ρ',μ[\pa↦(ρ',\pe_1)], κ) & \pa \not∈ \dom(μ),\ ρ'\! = ρ[\px↦\pa] \\
\AppIT & (\pe~\px,ρ,μ,κ) & \smallstep & (\pe,ρ,μ,\ApplyF(\pa) \pushF κ) & \pa = ρ(\px) \\
\CaseIT & (\Case{\pe}{\Sel},ρ,μ,κ) & \smallstep & (\pe,ρ,μ,\SelF(ρ,\Sel) \pushF κ) & \\
\LookupT & (\px, ρ, μ, κ) & \smallstep & (\pe, ρ', μ, \UpdateF(\pa) \pushF κ) & \pa = ρ(\px),\ (ρ',\pe) = μ(\pa) \\
\AppET & (\Lam{\px}{\pe},ρ,μ, \ApplyF(\pa) \pushF κ) & \smallstep & (\pe,ρ[\px ↦ \pa],μ,κ) &  \\
\CaseET & (K'~\many{y},ρ,μ, \SelF(ρ',\Sel) \pushF κ) & \smallstep & (\pe_i,ρ'[\many{\px_i ↦ \pa}],μ,κ) & K_i = K',\ \many{\pa = ρ(\py)} \\
\UpdateT & (\pv, ρ, μ, \UpdateF(\pa) \pushF κ) & \smallstep & (\pv, ρ, μ[\pa ↦ (ρ,\pv)], κ) & \\
\bottomrule
\end{tabular}
} % resizebox
\caption{Lazy Krivine transition semantics $\smallstep$}
  \label{fig:lk-semantics}
\end{figure}

Let us try a different approach then and define a stronger notion of deadness
in terms of a small-step operational semantics such as the Mark II machine of
\citet{Sestoft:97} given in \Cref{fig:lk-semantics}, the semantic ground truth
for this work. (A close sibling for call-by-value would be a CESK machine
\citep{Felleisen:87} or a simpler derivative thereof.) It is a variant of
the Lazy Krivine (LK) machine implementing call-by-need, so for a meaningful
comparison to $\semscott{\wild}$, we ignore rules $\CaseIT, \CaseET, \UpdateT$
and the pushing of update frames in $\LookupT$ for now to recover a call-by-name
Krivine machine with explicit heap addresses.%
\footnote{Note that discarding update frames makes the heap entries immutable,
which makes the explicit heap unnecessary. Of course, for call-by-name we would
not need a heap to begin with, but the point is to get a glimpse at the effort
necessary for call-by-need.}

The configurations $σ$ in this transition system resemble abstract machine
states, consisting of control expression $\pe$, an environment $ρ$ mapping
lexically-scoped variables to their current heap address, a heap $μ$ listing a
closure for each address, and a stack of continuation frames $κ$.

The notation $f ∈ A \pfun B$ used in the definition of $ρ$ and $μ$ denotes a
finite map from $A$ to $B$, a partial function where the domain $\dom(f)$ is
finite and $\rng(f)$ denotes its range.
The literal notation $[a_1↦b_1,...,a_n↦b_n]$ denotes a finite map with domain
$\{a_1,...,a_n\}$ that maps $a_i$ to $b_i$. Function update $f[a ↦ b]$
maps $a$ to $b$ and is otherwise equal to $f$.

The initial machine state for a closed expression $\pe$ is given by the
injection function $inj(\pe) = (\pe,[],[],\StopF)$ and
the final machine states are of the form $(\pv,\wild,\wild,\StopF)$.
We bake into $σ$ the simplifying invariant of \emph{well-addressedness}: Any
address $\pa$ occuring in $ρ$, $κ$ or the range of $μ$ must be an element of
$\dom(μ)$. It is easy to see that the transition system maintains this invariant
and that it is still possible to observe scoping errors which are thus confined
to lookup in $ρ$.

Now we are able to define a notion of ``strong deadness'':

\begin{definition}[Deadness, Mark II]
  \label{defn:deadness2}
  Let $\pe$ be an expression and $\px$ a variable.
  $\px$ is \emph{dead} in $\pe$ if and only if
  for any evaluation context $(ρ,μ,κ)$ and expressions $\pe_1,\pe_2$
  (where $\px$ does not occur in the context)
  the sequences of transitions $(\Let{\px}{\pe_1}{\pe},ρ,μ,κ) \smallstep^*$
  and $(\Let{\px}{\pe_2}{\pe},ρ,μ,κ) \smallstep^*$ operate in lockstep.
  Otherwise, $\px$ is \emph{live}.
\end{definition}

This definition captures diverging behaviors correctly and straightforwardly
legitimises the transformation we want to perform, without any mention of
addresses. It is however unwieldy in a correctness proof due to its use of
bisimulation, so a bit of rejigging is in order:

\begin{lemma}[Without proof]
  $\px$ is dead in $\pe$ if and only if for any evaluation context $(ρ,μ,κ)$
  and $\pa \not∈ \dom(μ)$ there exists no sequence of transitions
  $(\pe,ρ[\px↦\pa],μ[\pa↦([],\Lam{z}{z})],κ) \smallstep^* (\py,ρ',μ',κ')$ such
  that $ρ'(\py) = \pa$.
\end{lemma}

This property is a bit easier to handle in a proof.
Unfortunately, it is still not compositional in $\pe$: Consider a variable
occurrence $y$; is $x$ dead in $y$? That depends on which expression $y$ is
bound to in the heap, but our deadness predicate has no notion of making
assumptions about free variables.
Consequently, it is impossible to prove that $\semusg{\pe}$ satisfies
\Cref{thm:semusg-correct-live} by direct structural induction on $\pe$ (in a way
that would be useful to the proof).

Instead, such proofs are often conducted by induction over the reflexive
transitive closure of the transition relation.
For that it is necessary to give an inductive hypothesis that considers
environments, stacks and heaps.
One way is to extend the analysis function $\semusg{\wild}$ to entire
configurations and then prove that if $σ_1 \smallstep σ_2$ we have $\semusg{σ_2}
⊑ F(\semusg{σ_1})$, where $F$ is the abstraction of the particular transition
rule taken and is often left implicit.
This is a daunting task, for multiple reasons:
First off, $\semusg{\wild}$ might be quite complicated in practice and extending
it to entire configurations multiplies this complexity.
Secondly, $\semusg{\wild}$ makes use of fixpoints in the let case and
undoubtedly needs some more fixpoints in its extension to the heap,
so $\semusg{σ_2} ⊑ F(\semusg{σ_1})$ relates fixpoints that are ``out of sync'',
implying a need for fixpoint induction for every transition that touches
the heap.

In call-by-need, there will be a fixpoint between the heap and stack due to
update frames acting like heap bindings whose right-hand side is under
evaluation (a point that is very apparent in the contextual semantics of
\citet{Ariola:95}), so fixpoint induction needs to be applied \emph{at every
case of the proof}, diminishing confidence in correctness unless the proof is
fully mechanised.

For an analogy with type systems: What we just tried is like attempting a proof
of preservation by referencing the result of an inference algorithm rather than
the declarative type system. So what is often done instead is to define a declarative and
more permissive \emph{correctness relation} $C(σ)$ to prove preservation $C(σ_1)
\Longrightarrow C(σ_2)$ (\eg, that $C$ is \emph{logical} \wrt $\smallstep$).
$C$ is chosen such that
  (1) it is strong enough to imply the property of interest (deadness)
  (2) it is weak enough that it is implied by the analysis result for an initial state ($\tr(\px) \not⊑ \semusg{\pe}_{\tr}$).

Examples of this approach are the
``well-annotatedness'' relation in \citep[Lemma 4.3]{cardinality-ext} or
$\sim_V$ in \citep[Theorem 2.21]{Nielson:99}).
We found it quite hard to come up with a suitable ad-hoc correctness relation
and postpone futher discussion to \Cref{sec:abstractions}, where the full
correctness relation in \Cref{thm:semusg-correct-2} and its proof is derived by
abstract interpretation~\citep{Cousot:21}.

Often, correctness proofs do not need to keep track about which address
is an activation of which let-bound program variable, in which case the
distinction between addresses and variables is unnecessary and the
environment component vanishes.
The stack can often be reflected back into the premises of the judgment
rules, so that the system distinguishes \emph{instruction transitions}
from \emph{search transitions}, a distinction which is made explicit in a
\emph{contextual semantics}.
Applying both these translations yields a CS machine~\citep{Felleisen:87}.
For call-by-need, the evaluation context corresponding to an update frame
is neither obvious nor simple~\citep{Ariola:95}.
For effect-free call-by-value and call-by-name calculi, the heap becomes
immutable and variables can be substituted immediately for their right-hand
sides/arguments rather than delaying lookup to the variable case, abolishing the
need for the heap altogether and yielding a contextual semantics where
states are a simple expression.

These refactorings are with the ultimate goal of simplifying proofs:
Instead of defining an coinductive well-formedness predicate on the heap, we
prove a substitution lemma.
Instead of a well-formedness predicate for the stack, we appeal to the
well-formedness of the search transition rule.

\subsection{Abstracting Abstract Machines}

Another way to work around the structure gap is to adopt the structure of the
semantics in the analysis; this is done in the Abstracting Abstract
Machines work \citep{aam}.
To our knowledge, its exclusive application seems to be control-flow analysis
\citep{Shivers:91}, so that the analyses and optimisations that follow do not
need to reason about arbitrary function call structure and can apply traditional
intraprocedural analysis techniques that are well-explored in the imperative
world.
Unfortunately, control-flow information is often invalidated during compiler
passes and we would expect re-running or interleaving the analysis after or
during each pass to be quite costly.

\section{A Denotational Semantics for Call-by-need}
\label{sec:stateful}

\subsection{Labelled Syntax}

Recall the syntax definition of our object language in
\Cref{sec:usage-intuition} in the style of \citet{Launchbury:93} and
\citet{Sestoft:97}.
Any (sub-)expression has a unique \emph{label} (think of it as the AST node's
pointer identity) that we usually omit. For example, a correct labelling of
$\Let{x}{f~y}{f~x}$ would be
\[
  (\slbln{1} \Let{x}{(\slbln{2} (\slbln{3} f)~y)}{(\slbln{4} (\slbln{5} f)~x)}).
\]
Labels are there so that we do not conflate the (otherwise structurally equal)
sub-terms $(\slbln{3} f)$ and $(\slbln{5} f)$ as equivalent. This is an important
distinction for, \eg, control-flow analysis. Since labels introduce excessive
clutter, we will omit them unless they are distinctively important. If anything,
labels make it so that everything ``works as expected''.

\subsection{Transition System}

\Cref{fig:lk-semantics} gave an operational semantics in terms of
a small-step transition system closest to the lazy Krivine machine
\citep{AgerDanvyMidtgaard:04} for Launchbury's language as presented
in \citet{Sestoft:97}.
It is worth having a second look at the workings of our Gold Standard.

When the control expression $\ctrl(σ)$ of a state $σ$ is a value $\pv$, we
call $σ$ a \emph{return} state and say that the continuation $\cont(σ)$ drives
evaluation.
Otherwise, $σ$ is an \emph{evaluation} state and $\ctrl(σ)$ drives evaluation.
The entries in the heap $μ$ are \emph{closures} of the form $(ρ,e)$, where the
environment $ρ$ closes over the expression $e$.
Finally, the $\cont(σ)$ lists actions to be taken in a return state, such as
applying the result to an argument address or updating a heap entry with its
value.

Heap entries are introduced via $\BindT$ transitions under a \emph{fresh} address
$\pa \not∈ \dom(μ)$ that we call an \emph{activation} of the let-bound variable
$\px$. The lexical activation of every variable in scope is maintained
in $ρ$. The $\AppIT$ rule pushes an \emph{application frame} with the address of
the argument variable onto the stack, while the rule $\LookupT$ pushes an
\emph{update frame} with the address of the variable the heap entry of which is
accessed. When a return state is reached, the original heap entry is overwritten
with the value in the control.

Let us conclude with an example trace in this transition system, evaluating
$\pe \triangleq \Let{i}{\Lam{x}{x}}{i~i}$ to completion:
\[\begin{array}{c}
  \arraycolsep2pt
  \begin{array}{clclclcl}
             & (\pe, [], [], \StopF)         & \smallstep & (i~i, ρ_1, μ, \StopF)
             & \smallstep & (i, ρ_1, μ, κ_1) & \smallstep & (\Lam{x}{x}, ρ_1, μ, κ_2)
             \\
  \smallstep & (\Lam{x}{x}, ρ_1, μ, κ_1)     & \smallstep & (x, ρ_2, μ, \StopF) & \smallstep & (\Lam{x}{x}, ρ_1, μ, κ_3)
             & \smallstep & (\Lam{x}{x}, ρ_1, μ, \StopF) \\
  \end{array} \\
  \\[-0.5em]
  \quad \text{where} \quad \begin{array}{lll}
  ρ_1 = [i ↦ \pa_1] & ρ_2 = [i ↦ \pa_1, x ↦ \pa_1] & μ = [\pa_1 ↦ (ρ_1,\Lam{x}{x})] \\
  κ_1 = \ApplyF(\pa_1) \pushF \StopF & κ_2 = \UpdateF(\pa_1) \pushF κ_1 & κ_3 = \UpdateF(\pa_1) \pushF \StopF
  \end{array}
\end{array}\]

\subsection{Domain Theory}
\label{sec:domain-theory}

The challenge that domain theory sets out to solve is that the ``inductive
datatype''
\[
  D ::= \FunV(f ∈ D \to D) \mid \bot
\]
is ill-defined:
The usual interpretation of such a declaration as the least fixed-point of
the implied set-valued functional
$F(X) \triangleq \{ f \mid ∀a∈X.\ ∃b∈X.\ f(a) = b \} ∪ \{ \bot \}$
does not exist.

To see that, suppose $μF$ was that set.
Then there exists an injection (``data constructor'') $\FunV$ from $μF \to μF =
μF^{μF}$ into $μF$.
We can see that $\{\bot, (\fn{\wild}{\bot}) \} \subseteq μF$, so there are at
least two elements in $μF$.
Then to accomodate $μF^{μF}$, $μF$ must be at least as large as $2^{μF}$, the
set of two-valued functions on $μF$.
But this latter set is one-to-one with $\poset{μF}$ and it is a known result by
Cantor that the $\poset{μF}$ has greater cardinality than $μF$, in contradiction
to the existence of the injection $\FunV$.

Domain theory, on the other hand, interprets the implied recursion equation
in terms of topology and continuous functions, where the fixed-point exists
when restricted to \emph{algebraic domains}.
At the same time, algebraic domains are expressive enough to encode any
computable function as a continuous function.

\subsection{Guarded Domain Theory}

As we have discussed in \Cref{sec:continuity}, there are a few strings attached
to working with continuity and partiality in the context of denotational
semantics.

The key to getting rid of partiality and thus denoting infinite computations
with total elements is to avoid working with algebraic domains altogether and
instead work in a total type theory with \emph{guarded recursive types}, such
as Guarded Dependent Type Theory (GDTT)~\citep{gdtt} or Ticked Cubical Type
Theory~\citep{tctt}.%
\footnote{Of course, in reality we are just using GDTT as a meta
language~\citep{Moggi:07} with a known domain-theoretic model in terms
of the topos of trees~\citep{gdtt}.
This meta language is sufficiently expressive as a logic to
express proofs, though, justifying the view that we are extending ``math''
with the ability to conveniently reason about computable functions on infinite
data without needing to think about topology and approximation directly.}
The fundamental innovation of these theories is the integration of the
``later'' modality $\later$ which allows to define coinductive data types
with negative recursive occurrences such in our ``data type'' $D$ from
\Cref{sec:domain-theory}, as first realised by \citet{Nakano:00}.

GDTT walks a fine line:
The theory guarantees that all well-typed functions (naturally) correspond
to continuous functions in the underlying model, while it also allows for
embedding of a sufficiently expressive restriction to give meaning to $D$.

Whereas previous theories of coinduction require syntactic productivity
checks~\citep{Coquand:94}, requiring tiresome constraints on the form of guarded
recursive functions, the appeal of GDTT is that productivity is instead proven
semantically, in the type system.

The way that GDTT achieves this is roughly as follows: The type $\later T$
represents data of type $T$ that will become available after a finite amount
of computation, such as unrolling one layer of a fixpoint definition.
It comes with a general fixpoint combinator $\fix : \forall A.\ (\later A \to
A) \to A$ that can be used to define both coinductive \emph{types} (via guarded
recursive functions on the universe of types~\citep{BirkedalMogelbergEjlers:13})
as well as guarded recursive \emph{terms} inhabiting said types.
The classic example is that of coinductive streams:
\[
  Str = ℕ \times \later Str \qquad ones = \fix (r : \later Str).\ (1,r),
\]
where $ones : Str$ is the constant stream of $1$.
In particular, $Str$ is the fixpoint of a locally contractive functor $F(X) =
ℕ \times \later X$.
According to \citet{BirkedalMogelbergEjlers:13}, any type expression in simply
typed lambda calculus defines a locally contractive functor as long as any
occurrence of $X$ is under a $\later$, so we take that as the well-formedness
criterion of coinductive types in this work.
The most exciting consequence is that
$D ::= \FunV(f ∈ \later D \to D) \mid \bot$ (where $\bot$ is interpreted as a
plain nullary data constructor rather than as the least element of some partial
order) is a sound coinductive encoding of the data type in
\Cref{sec:domain-theory}.

As a type constructor, $\later$ is an applicative
functor~\citep{McBridePaterson:08} via functions
\[
  \purelater : \forall A.\ A \to \later A \qquad \wild \aplater \wild : \forall A,B.\ \later (A \to B) \to \later A \to \later B,
\]
allowing us to apply a familiar framework of reasoning around $\later$.
In order not to obscure our work with pointless symbol pushing
in, \eg, \Cref{fig:semvan}, we will often omit the idiom
brackets~\citep{McBridePaterson:08} $\idiom{\wild}$
to indicate where the $\later$ ``effects'' happen.
Rest assured, they are still there in the Guarded Cubical Agda development in
the Supplement.

\sg{The Guarded Cubical Agda prototype type-checks and is available at
\url{https://github.com/sgraf812/comp-trace/blob/main/agda}}

\begin{figure}
\[\begin{array}{c}
 \begin{array}{rrclclrrclcl}
 \end{array} \\
 \begin{array}{rrclclrrcrclcl}
  \text{Environment}  & ρ   & ∈ & \Environments  & =      & \Var \pfun \Addresses
  &
  \text{Heap}         & μ   & ∈ & \Heaps         & =      & \Addresses \pfun \later\VanD
  \\
  \\[-0.5em]
  \text{Trace} & τ      & ∈          & \VTraces & ::= & \goodend{v,μ} \mid \stuckend \mid \laterC~τ^{\later}
  &
  \multicolumn{3}{r}{\text{Delayed trace}} & τ^{\later} & ∈ & \later\VTraces &   &
  \\
  \text{Domain} & d & ∈ & \VanD & = & \Heaps \to \VTraces
  &
  \multicolumn{3}{r}{\text{Delayed element}} & d^{\later} & ∈ & \later\VanD &   &
  \\
  \\[-0.5em]
 \end{array} \\
 \begin{array}{rrclcl}
  \text{Value} & v & ∈ & \VanV & ::= & \FunV(f ∈ (\Addresses \to \VanD)) \mid \ConV(K,\many{\pa}^{α_K}) \\
 \end{array} \\
  \\[-0.5em]
\end{array}\]
\[\begin{array}{c}
 \begin{array}{rcl}
  \multicolumn{3}{c}{ \ruleform{
    \begin{array}{c}
      (\betastep) : \VanD \to (\VanV \pfun \VanD) \to \VanD \quad \ret : \VanV \to \VanD \quad \apply : \VanD \times \Addresses \to \VanD \\
      \memo : \Addresses \times \VanD \to \VanD \quad \select : \VanD \times ((K:\Con) \times \pa^{α_K} \pfun \VanD) \to \VanD \\
    \end{array}
  }} \\
  \\[-0.5em]
  (d \betastep f)(μ) & = & \begin{cases}
      \laterC^n~f(v,μ')  & \text{$d(μ) = \laterC^n~\goodend{v,μ'}$ and $(v,μ') ∈ \dom(f)$} \\
      \laterC^n~\stuckend  & \text{$d(μ) = \laterC^n~\goodend{v,μ'}$ and $(v,μ') \not∈ \dom(f)$} \\
      d(μ) & \text{otherwise} \\
    \end{cases} \\
  \\[-0.5em]
  \ret(v)(μ) & = & \goodend{v,μ} \\
  \apply(d,\pa) & = & d \betastep \fn{(\FunV(f))}{f(\pa)} \\
  \select(d,\alts) & = & d \betastep \fn{(\ConV(K_s,\many{\pa}))}{\alts(K_s, \many{\pa})} \quad \text{where } (K_s, \many{\pa}) ∈ \dom(\alts) \\
  \memo(\pa,d) & = & d \betastep \fn{v}{(\fn{μ}{\laterC~\goodend{v,μ[\pa ↦ \memo(\pa,\ret(v))]}})} \\
 \end{array} \\
 \\[-0.5em]
 \begin{array}{rcl}
  \multicolumn{3}{c}{ \ruleform{ \semvan{\wild} \colon \Exp → (\Var \pfun \Addresses) → \VanD } } \\
  \\[-0.5em]
  \semvan{\px}_ρ & = & \begin{cases}
    \laterC~μ(ρ(\px))(μ) & \px ∈ \dom(ρ) \\
    \stuckend & \text{otherwise}
    \end{cases} \\
  \\[-0.5em]
  \semvan{\Lam{\px}{\pe}}_ρ & = & \ret(\FunV(\fn{\pa}{\laterC~\semvan{\pe}_{ρ[\px↦\pa]}})) \\
  \\[-0.5em]
  \semvan{\pe~\px}_ρ(μ) & = & \begin{cases}
      \laterC~\apply(\semvan{\pe}_ρ(μ),\pa) & \px ∈ \dom(ρ) \\
      \stuckend & \text{otherwise} \\
    \end{cases} \\
  \\[-0.5em]
  \semvan{\Let{\px}{\pe_1}{\pe_2}}_ρ(μ) & = & \begin{letarray}
    \text{let} & ρ' = ρ[\px↦\pa] \quad \text{where $\pa \not∈ \dom(μ)$} \\
    \text{in}  & \laterC~\semvan{\pe_2}_{ρ'}(μ[\pa ↦ \memo(\pa,\semvan{\pe_1}_{ρ'})]) \\
  \end{letarray} \\
  \\[-0.5em]
  \semvan{K~\many{\px}}_ρ & = & \ret(\ConV(K,\many{ρ(\px)})) \\
  \\[-0.5em]
  \semvan{\Case{\pe_s}{\Sel[r]}}_ρ(μ) & = &
    \begin{letarray}
      \text{let} & \alts = \fn{(K_i,\many{\pa})}{\laterC~\semevt{\pe_{r_i}}_{ρ[\many{\px_i↦\pa}]}} \\
      \text{in} & \laterC~\select(\semevt{\pe_s}_ρ, \alts)  \\
    \end{letarray}
 \end{array}
  \\[-0.5em]
\end{array}\]
\caption{Call-by-need Denotational Semantics $\semvan{-}$}
  \label{fig:semvan}
\end{figure}

\subsection{Definition}

\Cref{fig:semvan} finally gives the definition for $\semvan{\wild}$, a function
defined by structural recursion on an input expression $\pe$. Given a free
variable environment $ρ$, $\semvan{\pe}_ρ$ assigns $\pe$ a denotation $d$ in
terms of the semantic domain $\VanD$ of stateful call-by-need trace functions.
If such a trace function is supplied the heap $μ$ just before $\pe$ takes
control, then $d(μ)$ is a trace $τ$ starting at $\pe$ in that heap $μ$.
As with $(\smallstep)$, the job of the environment $ρ$ is to assign meaning to
free variables of $\pe$ via address tokens $\pa$ bound in the heap.
If evaluation of $\pe$ terminates, then $τ$ will be a finite list of $\laterC$
wrappers ending with $\goodend{v,μ'}$ for some return semantic value $v$ and
heap $μ'$.
Otherwise, it might be finite but stuck ($\stuckend$), or diverge without
ever leaving $\pe$, in which case $τ$ will be an infinite layering of $\laterC$s.

One can think of $\laterC$ (read as ``later'' or ``delayed'') as introducing
a finite portion of latency into the trace -- an atomic computation step of a
trace.
It makes crucial use of $\later$ to achieve that and naturally encode divergence
in a infinite sequence of productive steps.

Compared to the LK transition semantics, the most striking difference is that
the returned trace does not contain any intermediate information such as an
environment or control stack component.
The entire state is internal to the definition of $\semvan{\wild}$:
The stack in particular is implicitly encoded in call structure while the
environment follows lexical structure.
As we have seen in \Cref{sec:problem} at the example of $\semscott{\wild}$,
this reflection of machine state into ``math'' bears great potential for program
analysis, one we will exploit in \Cref{sec:abstractions}.
We will see that for every LK transition, the trace will have one $\laterC$
step.

The second difference is that the heap $μ$ does not map to syntactic closures
but to delayed semantic values $\later \VanD$, offering further abstraction
possibilities compared to the rigid and indirect syntactic domain.

The choice to have environments map to addresses naturally leads to function
values $\FunV(f)$ that take said addresses as parameter rather than a
(necessarily guarded) $\VanD$ as in $\semscott{\wild}$.%
\footnote{An earlier version of this paper had
$\Environments = \Var \pfun \VanD$ instead, where every entry would simply look
up a particular address in the heap.
Though it felt more ``semantic'', it was ultimately more taxing than useful,
because many later proofs will have to pose additional preconditions on $ρ$.}
Crucially, this allows the embedding of function values $\FunV(f)$ in the
lambda case $\semvan{\Lam{\px}{\pe}}$, enabling a compositional definition of the
application case $\semvan{\pe~\px}$, just as in $\semscott{\wild}$.

It is worth noting that without guarded recursive types, the definition of
$\VanD$ would not be well-founded because of the negative cycle arising through
$\VanD \to \Heaps \to \VanD$ (\cf \Cref{sec:domain-theory}).
Interpreted as a traditional algebraic domain, we would now have to do a lot of
complicated justification.
However, the valiant use of a guarded recursive type $\later \VanD$ in the
definition of heaps breaks that cycle, thus the definition is well-founded.

All definitions of semantics in this work have been type-checked using Guarded
Cubical Agda and are available in the Supplement.
Hence we leave out the necessary $\purelater$, $\aplater$ and $\idiom{\wild}$
combinators in the presentation to avoid distraction from the payload.

Let us now understand $\semvan{\wild}$ by way of evaluating the example program
from earlier, $\pe \triangleq \Let{i}{\Lam{x}{x}}{i~i}$:
\[\begin{array}{ll}
  \newcommand{\myleftbrace}[4]{\draw[mymatrixbrace] (m-#2-#1.west) -- node[right=2pt] {#4} (m-#3-#1.west);}
  \vcenter{\hbox{$
    \begin{tikzpicture}[mymatrixenv,anchor=center]
      \matrix [mymatrix] (m)
      {
        1 & \laterC & \hspace{3.7em} & \hspace{4.2em} & \hspace{3.9em} & \hspace{2.5em} \\
        2 & \laterC & & & & \\
        3 & \laterC & & & & \\
        4 & \laterC & & & & \\
        5 & \laterC & & & & \\
        6 & \laterC & & & & \\
        7 & \laterC & & & & \\
        8 & \goodend{\FunV(f), μ} & & & & \\
      };
      % Braces, using the node name prev as the state for the previous east
      % anchor. Only the east anchor is relevant
      \foreach \i in {1,...,\the\pgfmatrixcurrentrow}
        \draw[dotted] (m.east|-m-\i-\the\pgfmatrixcurrentcolumn.east) -- (m-\i-2);
      \myleftbrace{3}{1}{8}{$\semvan{\pe}_{[]}$}
      \myleftbrace{4}{1}{2}{$\BindT$}
      \myleftbrace{4}{2}{8}{$\semvan{i~i}_{ρ_1}$}
      \myleftbrace{5}{2}{3}{$\AppIT$}
      \myleftbrace{5}{3}{5}{$\semvan{i}_{ρ_1}$}
      \myleftbrace{5}{5}{6}{$\AppET$}
      \myleftbrace{5}{6}{8}{$\semvan{x}_{ρ_2}$}
      \myleftbrace{6}{3}{4}{$\LookupT$}
      \myleftbrace{6}{4}{5}{$\UpdateT$}
      \myleftbrace{6}{6}{7}{$\LookupT$}
      \myleftbrace{6}{7}{8}{$\UpdateT$}
  \end{tikzpicture}
  $}} &
  \!\!\!\!\text{where}  \begin{array}{ll}
  ρ_1 = [i ↦ \pa_1] & \\
  ρ_2 = ρ_1[x ↦ \pa_1] &  \\
  μ = [\pa_1 ↦ \memo(\pa_1,\semvan{\Lam{x}{x}}_{ρ_1})] & \\
  f = \pa \mapsto \semvan{\px}_{ρ_1[\px↦\pa]}
  \end{array}
\end{array}\]
The annotations to the right of the trace can be understood as denoting the
``call graph'' of $\semvan{\pe}_{[]}$, with the corresponding LK transitions
as leaves.
Evaluation begins at timestamp 1 with a $\BindT$ transition to timestamp 2,
which $\semvan{\pe}_{[]}$ acknowledges by emitting a $\laterC$.
A fresh address $\pa_1$ is allocated for variable $i$ and the heap is extended
with $\memo(\pa,\semvan{\Lam{x}{x}}_{ρ_1})$.
It is interesting to realise that this process does not involve a fixpoint
despite the recursive semantics of $\mathbf{let}$.
Of course, the self-application in $\semvan{\px}$ does the job just as well, as
we will see in due course.

Evaluation recurses into the body $\semvan{i~i}_{ρ_1}$, yielding another
$\AppIT$ transition (corresponding to a bland $\laterC$) into $\semvan{i}_{ρ_1}$.
Note that the target state $\semvan{i}_{ρ_1}$ will later be fed (via
$\betastep$) into the anonymous function in $\apply$.
This scheme is quite common: Continuation items of the transition semantics
(``data'') are reflected into the call stack of the trace semantics (``code'').
The reverse process an be recognised as defunctionalisation~\citep{Reynolds:72}.

$\semvan{i}_{ρ_1}$ guides the trace through a heap lookup:
After emitting a $\laterC$ corresponding to the $\LookupT$ transition,
$μ$ is dereferenced at $ρ(i)$ where we find $\memo(\pa,\semvan{\Lam{x}{x}}_{ρ_1})$.
This heap entry is ``self-applied'' to $μ$, effectively ``tying the knot''.
The memoisation action will run $\semvan{\Lam{x}{x}}_{ρ_1}$ to completeness
through $(\betastep)$ on heap $μ$ at timestamp 4.
Since that is already a value, $(\betastep)$ calls its second argument with
$\goodend{\FunV(f),μ}$ and we witness for the first time a reduction operation,
making an $\UpdateT$ transition to update $μ(\pa)$.
Note that this update has no effect on the heap $μ$ because
$\ret(\FunV(f))$ is precisely the same as $\semvan{\Lam{x}{x}}_{ρ_1}$.

After the heap update, we leave $\semvan{i}$ in timestamp 5,
where $\betastep$ yields to the anonymous function in $\apply$ (from the earlier
call to $\semvan{i~i}_{ρ_1}$), yielding another $\AppET$ reduction and giving
control to $f(ρ_1(i)) = \semvan{x}_{ρ_1[x ↦ ρ_1(i)]}$.
Since $x$ is an alias for $i$, steps 6 to 8 just repeat the same same heap
update sequence we observed in steps 3 to 5, concluding the example.

It is useful to review another example involving an observable heap
update. The following trace begins right before the heap update occurs in
$\Let{i}{(\Lam{y}{\Lam{x}{x}})~i}{i~i}$, that is, after reaching the value
in $\semvan{(\Lam{y}{\Lam{x}{x}})~i}_{ρ_1}$.
We will indicate the heap before a transition takes place by
writing it before $\laterC$:
\[\begin{array}{ll}
  \newcommand{\myleftbrace}[4]{\draw[mymatrixbrace] (m-#2-#1.west) -- node[right=2pt] {#4} (m-#3-#1.west);}
  \vcenter{\hbox{$
    \begin{tikzpicture}[baseline={-0.5ex},mymatrixenv]
      \matrix [mymatrix] (m)
      {
        1 & (μ_1)~\laterC & \hspace{4em} & \hspace{4em} & \hspace{2.5em} \\
        2 & (μ_2)~\laterC & & & \\
        3 & (μ_2)~\laterC & & & \\
        4 & (μ_2)~\laterC & & & \\
        5 & \goodend{\FunV(f), μ_2} & & & \\
      };
      % Braces, using the node name prev as the state for the previous east
      % anchor. Only the east anchor is relevant
      \foreach \i in {1,...,\the\pgfmatrixcurrentrow}
        \draw[dotted] (m.east|-m-\i-\the\pgfmatrixcurrentcolumn.east) -- (m-\i-2);
      \myleftbrace{3}{1}{5}{$\semvan{i~i}_{ρ_1}$}
      \myleftbrace{4}{1}{2}{$\semvan{i}_{ρ_1}$}
      \myleftbrace{4}{2}{3}{$\AppET$}
      \myleftbrace{4}{3}{5}{$\semvan{x}_{ρ_3}$}
      \myleftbrace{5}{1}{2}{$\UpdateT$}
      \myleftbrace{5}{3}{4}{$\LookupT$}
      \myleftbrace{5}{4}{5}{$\UpdateT$}
    \end{tikzpicture}
  $}} &
  \!\!\!\text{where} \begin{array}{l}
  ρ_1 = [i ↦ \pa_1] \\
  ρ_2 = [i ↦ \pa_1, y ↦ \pa_1] \\
  ρ_3 = [i ↦ \pa_1, y ↦ \pa_1, x ↦ \pa_1] \\
  μ_1 = [\pa_1 ↦ \memo(\pa_1,\semvan{(\Lam{y}{\Lam{x}{x}})~i}_{ρ_1})] \\
  μ_2 = [\pa_1 ↦ \memo(\pa_1,\semvan{\Lam{x}{x}}_{ρ_2})] \\
  f = \fn{d}{\semvan{\px}_{ρ_2[\px↦d]}} \\
  \end{array} \\
\end{array}\]
Note that both the denotation in the heap $μ_1$ \emph{and} its environment are updated
in timestamp 2, and that the new denotation is immediately visible on the next heap
lookup in timestamp 3, so that $\semvan{\Lam{x}{x}}_{ρ_2}$ takes control rather than
$\semvan{(\Lam{y}{\Lam{x}{x}})~i}_{ρ_1}$, just as the transition system requires.

The handling of data types and case expressions is routine (if a bit
syntactically heavy) and not so different to denotational semantics in
call-by-name or call-by-value.
It is a useful inclusion because it allows us to observe type errors other than
scoping errors, as well as enable more interesting examples.
Let us consider evaluation of the closed expression
$\pe \triangleq \Let{x}{\ttrue}{\ttrue~x}$
(where $\ttrue$ is one of two nullary data constructors of the data type $\bool
::= \ttrue \mid \ffalse$).
$\semvan{\wild}$ makes it is easy to observe that the trace gets stuck:
\[\begin{array}{ll}
  \newcommand{\myleftbrace}[4]{\draw[mymatrixbrace] (m-#2-#1.west) -- node[right=2pt] {#4} (m-#3-#1.west);}
  \vcenter{\hbox{$
    \begin{tikzpicture}[baseline={-0.5ex},mymatrixenv]
      \matrix [mymatrix] (m)
      {
        1 & (\pe, []) \cons {} & \hspace{4em} & \hspace{5em} & \hspace{2.5em} \\
        2 & (\ttrue~x, μ) \cons {} & & & \\
        3 & \stuckend & & & \\
      };
      % Braces, using the node name prev as the state for the previous east
      % anchor. Only the east anchor is relevant
      \foreach \i in {1,...,\the\pgfmatrixcurrentrow}
        \draw[dotted] (m.east|-m-\i-\the\pgfmatrixcurrentcolumn.east) -- (m-\i-2);
      \myleftbrace{3}{1}{3}{$\semvan{\pe}_{[]}$}
      \myleftbrace{4}{1}{2}{$\BindT$}
      \myleftbrace{4}{2}{3}{$\semvan{\ttrue~x}_{ρ}$}
      \myleftbrace{5}{2}{3}{$\AppIT$}
    \end{tikzpicture}
  $}} &
  \!\!\!\text{where} \begin{array}{l}
  ρ = [x ↦ \pa_1] \\
  μ = [\pa_1 ↦ \semvan{\ttrue}_ρ] \\
  \end{array} \\
\end{array}\]
Crucially, $\betastep$ is equipped to propagate $\stuckend$ up the call
stack (through potential $\UpdateT$ transitions, in particular), similar to
\citeauthor{Milner:78}'s $\mathbf{wrong}$.

Diverging traces hold no new surprises, other than they are observably different
to stuck traces.

\subsection{Maximal LK Traces}
\label{sec:maximal-traces}

It turns out that the traces $\semvan{\pe}$ generates correspond to
\emph{maximal} traces in the LK transition system.
Let us make precise what that means.

A transition system is characterised by the set of \emph{traces} it generates.
An \emph{LK trace} is a trace in $(\smallstep)$, \eg, a non-empty and
potentially infinite sequence of LK states $(σ_i)_{i∈\overline{n}}$
(where $\overline{n} = \{ m ∈ ℕ_+ \mid m ≤ n \}$ when $n∈ℕ$, $\overline{ω} = ℕ$),
such that $σ_i \smallstep σ_{i+1}$ for $i,(i+1)∈\overline{n}$.

The \emph{source} state $σ_0$ exists for finite and infinite traces, while the
\emph{target} state $σ_n$ is only defined when $n \not= ω$ is finite.

For proofs, we will often regard $(σ_i)_{i∈\overline{n}}$ as an object of type
$\STraces \triangleq ∃n∈ℕ_ω.\ \overline{n} \to \States$, where $ℕ_ω$ is defined by guarded recursion
as $ℕ_ω = \{\mathit{Z}\} + \later ℕ_ω$.
The constructor for the right sum alternative is written $\mathit{succ}$.
Now $ℕ_ω$ contains all natural numbers (where $n$ is encoded as
$(\mathit{succ})^{n}(\mathit{Z})$) and the transfinite limit ordinal
$ω = \mathit{succ}(\mathit{succ}(...))$.
We will write $1+m$ to denote $\mathit{succ}(m)$ (a different kind of $+$ than
in the recursive equation for $ℕ_ω$), just as we are used to from $ℕ$.
Hence, when $(σ_i)_{i∈\overline{n}} ∈ \STraces$ is an LK trace and $n > 0$, then
$(σ_{i+1})_{i∈\overline{n-1}} ∈ \later \STraces$ is the guarded tail of the
trace with an associated induction principle.

An important kind of trace is one that never leaves the evaluation context of
its source state:

\begin{definition}[Deep, interior and balanced traces]
  An LK trace $(σ_i)_{i∈\overline{n}}$ is
  \emph{$κ$-deep} if every intermediate continuation
  $κ_i = \cont(σ_i)$ extends $κ$ (so $κ_i = κ$ or $κ_i = ... \pushF κ$,
  abbreviated $κ_i = ...κ$).

  A trace $(σ_i)_{i∈\overline{n}}$ is called \emph{interior} if it is
  $\cont(σ_0)$-deep.
  Furthermore, an interior trace $(σ_i)_{i∈\overline{n}}$ is
  \emph{balanced}~\citep{Sestoft:97} if the target state exists and is a return
  state with continuation $κ_1$.

  We notate $κ$-deep, interior and balanced traces as
  $\deep{κ}{(σ_i)_{i∈\overline{n}}}$, $\interior{(σ_i)_{i∈\overline{n}}}$ and
  $\balanced{(σ_i)_{i∈\overline{n}}}$, respectively.
\end{definition}

\begin{example}
  Let $ρ=[x↦\pa_1],μ=[\pa_1↦([], \Lam{y}{y})]$ and $κ$ an arbitrary
  continuation. The trace
  \[
     (x, ρ, μ, κ) \smallstep (\Lam{y}{y}, ρ, μ, \UpdateF(\pa_1) \pushF κ) \smallstep (\Lam{y}{y}, ρ, μ, \UpdateF(\pa_1) \pushF κ) \smallstep (\Lam{y}{y}, ρ, μ, κ)
  \]
  is interior and balanced. Its prefixes are interior but not balanced.
  The trace suffix
  \[
     (\Lam{y}{y}, ρ, μ, \UpdateF(\pa_1) \pushF κ) \smallstep (\Lam{y}{y}, ρ, μ, κ)
  \]
  is neither interior nor balanced.
\end{example}

We will say that the transition rules $\LookupT$, $\AppIT$, $\CaseIT$ and $\BindT$
are interior, because the lifting into a trace is, whereas the returning
transitions $\UpdateT$, $\AppET$ and $\CaseET$ are not.

A balanced trace starting at a focus expression $\pe$ and ending with $\pv$
loosely corresponds to a derivation of $\pe \Downarrow \pv$ in a natural
big-step semantics~\citep{Sestoft:97} or a non-$⊥$ result in a denotational
semantics.

It is when a derivation in a natural semantics does not exist that a small-step
semantics shows finesse, in that it differentiates two different kinds of
\emph{maximally interior} (or, just \emph{maximal}) traces:

\begin{definition}[Maximal trace]
  An LK trace $(σ_i)_{i∈\overline{n}}$ is \emph{maximal} if and only if it is
  interior and there is no $σ_{n+1}$ such that $(σ_i)_{i∈\overline{n+1}}$ is
  interior.
  More formally (and without a negative occurrence of ``interior''),
  \[
    \maxtrace{(σ_i)_{i∈\overline{n}}} \triangleq \interior{(σ_i)_{i∈\overline{n}}} \wedge (\not\exists σ_{n+1}.\ σ_n \smallstep σ_{n+1} \wedge \cont(σ_{n+1}) = ...\cont(σ_0))
  \]
  We notate maximal traces as $\maxtrace{(σ_i)_{i∈\overline{n}}}$.
\end{definition}

We call infinite and interior traces \emph{diverging}.
A maximally finite, but unbalanced trace is called \emph{stuck}.
Note that usually stuckness is associated with a state of a transition
system rather than a trace.
That is not possible in our framework; the following example clarifies.

\begin{example}[Stuck and diverging traces]
Consider the interior trace
\[
             (\ttrue~x, [x↦\pa_1], [\pa_1↦...], κ)
  \smallstep (\ttrue, [x↦\pa_1], [\pa_1↦...], \ApplyF(\pa_1) \pushF κ)
\]
It is stuck, but its singleton suffix is balanced.
An example for a diverging trace where $ρ=[x↦\pa_1]$ and $μ=[\pa_1↦(ρ,x)]$ is
\[
  (\Let{x}{x}{x}, [], [], κ) \smallstep (x, ρ, μ, κ) \smallstep (x, ρ, μ, \UpdateF(\pa_1) \pushF κ) \smallstep ...
\]
\end{example}

A maximal trace that is not balanced either diverges or is stuck:

\begin{lemma}[Characterisation of maximal traces]
  An LK trace $(σ_i)_{i∈\overline{n}}$ is maximal if and only if it is balanced,
  diverging or stuck.
\end{lemma}
\begin{proof}
  $\Rightarrow$: Let $(σ_i)_{i∈\overline{n}}$ be maximal.
  If $n=ω$ is infinite, then it is diverging due to interiority, and if
  $(σ_i)_{i∈\overline{n}}$ is stuck, the goal follows immediately.
  So we assume that $(σ_i)_{i∈\overline{n}}$ is maximal, finite and not stuck,
  so it must be balanced by the definition of stuckness.

  $\Leftarrow$: Both balanced and stuck traces are maximal.
  A diverging trace $(σ_i)_{i∈\overline{n}}$ is interior and infinite,
  hence $n=ω$.
  Indeed $(σ_i)_{i∈\overline{ω}}$ is maximal, because the expression $σ_ω$
  is undefined and hence does not exist.
\end{proof}

Interiority guarantees that the particular initial stack $κ$ of a maximal trace
is irrelevant to execution, so maximal traces that differ only in the initial
stack are bisimilar.

One class of maximal traces is of particular interest:
The maximal trace starting in $\inj(\pe)$!
Whether it is inifinite, stuck or balanced is the defining operational
characteristic of $\pe$.
If we can show that $\semvan{\pe}$ distinguishes these behaviors of $\pe$, we
have proven it an adequate replacement for the LK transition system.

\subsection{Adequacy}

\Cref{fig:semvan-correctness} shows the correctness predicate $\mathcal{C}$ in
our endeavour to prove $\semvan{\wild}$ adequate.
It builds on the framework of maximal LK traces established in the last section;
specifically it encodes that an \emph{abstraction} of every maximal LK trace can be
recovered by running $\semvan{\wild}$ starting from the abstraction of an initial
state.

\begin{figure}
\[\begin{array}{rcl}
  α_\Heaps([\many{\pa ↦ (ρ,\pe)}]) & = & [\many{\pa ↦ \memo(\pa,\semvan{\pe}_{ρ})}] \\
  α_\VanV(\Lam{\px}{\pe},ρ,μ,κ) & = & \goodend{\FunV(\fn{\pa}{\laterC~\semvan{\pe}_{ρ[\px↦\pa]}}),α_\Heaps(μ)} \\
  α_\VanV(K~\overline{\px},ρ,μ,κ) & = & \goodend{\ConV(K,\overline{ρ(\pa)}),α_\Heaps(μ)} \\
  α_{\STraces}((σ_i)_{i∈\overline{n}},κ) & = & \begin{cases}
    \laterC~\idiom{α_{\STraces}((σ_{i+1})_{i∈\overline{n-1}},κ)} & n > 0 \\
    α_\VanV(σ_0) & \ctrl(σ_0) \text{ value } \wedge \cont(σ_0) = κ \\
    \stuckend & \text{otherwise} \\
  \end{cases} \\
  \mathcal{C}((σ_i)_{i∈\overline{n}}) & = & \maxtrace{(σ_i)_{i∈\overline{n}}} \Longrightarrow ∀((\pe,ρ,μ,κ) = σ_0).\ α_{\STraces}((σ_i)_{i∈\overline{n}},κ) = \semvan{\pe}_{ρ}(α_\Heaps(μ)) \\
\end{array}\]
\caption{Correctness predicate for $\semvan{\wild}$}
  \label{fig:semvan-correctness}
\end{figure}

The family of abstraction functions makes precise the intuitive connection
between the semantic objects in $\semvan{\wild}$ and the syntactic objects in
the transition system.

We will sometimes need to disambiguate the clashing definitions from
\Cref{sec:stateful} and \Cref{sec:problem}.
We do so by adorning semantic objects with a tilde, so $\tm \triangleq
α_\Heaps(μ)$ denotes a semantic heap which in this instance is
defined to be the abstraction of a syntactic heap $μ$.

Note first that $α_\STraces$ is defined by guarded recursion over
the LK trace, in the sense defined in \Cref{sec:maximal-traces}.
As such, the expression $\idiom{α_{\STraces}((σ_{i+1})_{i∈\overline{n-1}},κ)}$ has type
$\later \VTraces$ (the $\later$ in the type of $(σ_{i+1})_{i∈\overline{n-1}}$
maps through $α_\STraces$ via the idiom brackets).
Likewise, $=$ on $\VTraces$ is defined in the obvious structural way by guarded
recursion (as it would be if it was a finite, inductive type).

Our first goal is to establish a few auxiliary lemmas showing what kind of
properties of LK traces are preserved by $α_\STraces$ and in which way.

Let us warm up by defining a length function on traces:
\begin{definition}[Length of a trace]
  \label{defn:length}
  The \emph{length} $\len : \VTraces \to ℕ_ω$ of a trace is defined by
  guarded recursion
  \[
    \len(τ) = \begin{cases}
      1 + \purelater \len \aplater τ^{\later} & τ = \laterC~τ^{\later} \\
      0 & \text{otherwise} \\
    \end{cases}
  \]
\end{definition}

\begin{lemma}[Preservation of length]
  \label{thm:abs-length}
  Let $(σ_i)_{i∈\overline{n}}$ be an arbitrary trace.
  Then $\len(α_\STraces((σ_i)_{i∈\overline{n}},\cont(σ_0))) = n$.
\end{lemma}
\begin{proof}
  This is quite simple to see and hence a good opportunity to familiarise
  ourselves with the concept of \emph{Löb induction}, the induction principle of
  guarded recursion.
  Löb induction arises simply from applying the guarded recursive fixpoint
  combinator to a proposition:
  \[
    \textsf{löb} = \fix : \forall P.\ (\later P \Longrightarrow P) \Longrightarrow P
  \]
  That is, we assume that our proposition holds \emph{later}, \eg
  \[
    IH ∈ (\later P \triangleq \later (
        \forall n ∈ ℕ_ω.\
        \forall (σ_i)_{i∈\overline{n}}.\
        \len(α_\STraces((σ_i)_{i∈\overline{n}},\cont(σ_0))) = n
      ))
  \]
  and use $IH$ to prove $P$.
  Let us assume $n$ and $(σ_i)_{i∈\overline{n}}$ are given, define
  $τ \triangleq α_\STraces((σ_i)_{i∈\overline{n}},\cont(σ_0))$ and proceed by case analysis
  over $n$:
  \begin{itemize}
    \item \textbf{Case $n=0$}: Then we have $j = 0$ and either $τ = \goodend{α_\States(σ_0)}$
      or $τ = \stuckend$, both of which map to $0$ under
      $\len$.
    \item \textbf{Case $n>0$}: Then $n = 1+m$ and $m ∈ \later ℕ_ω$ and the
      first case of $α_\States$ applies, hence $τ = σ \cons τ^{\later}$ for some
      $σ∈\States, τ^{\later}∈\later \VTraces$.
      Now we apply the inductive hypothesis, as follows:
      Let $(σ_{i+1})_{i∈\overline{m}} ∈ \later \STraces$ be the guarded
      tail of the LK trace $(σ_i)_{i∈\overline{n}}$.
      Then we can apply $IH \aplater m \aplater (σ_{i+1})_{i∈\overline{m}}$ and
      get a proof for $\later (\len(τ^{\later}) = m)$.
      Now we can prove
      \[
        n = 1 + m = 1 + \len(τ^{\later}) = \len(τ).
      \]
  \end{itemize}
\end{proof}

\begin{lemma}[Preservation of characteristic]
  \label{thm:abs-max-trace}
  Let $(σ_i)_{i∈\overline{n}}$ be a maximal trace.
  Then $α_\STraces((σ_i)_{i∈\overline{n}}, cont(σ_0))$ is ...
  \begin{itemize}
    \item ... infinite if and only if $(σ_i)_{i∈\overline{n}}$ is diverging
    \item ... ending with $\goodend{\wild,\wild}$ if and only if $(σ_i)_{i∈\overline{n}}$ is balanced
    \item ... ending with $\stuckend$ if and only if $(σ_i)_{i∈\overline{n}}$ is stuck
  \end{itemize}
\end{lemma}
\begin{proof}
  The first point follows by a similar inductive argument as in \Cref{thm:abs-length}.

  In the other cases, we may assume that $n$ is finite.
  If $(σ_i)_{i∈\overline{n}}$ is balanced, then $σ_n$ is a return state with
  continuation $\cont(σ_0)$, so its control expression is a value.
  Then $α_\STraces$ will conclude with $\goodend{\wild,\wild}$.
  Conversely, if the trace ended with $\goodend{\wild,\wild}$, then $\cont(σ_n) = \cont(σ_0)$
  and $\ctrl(σ_n)$ is a value, so $(σ_i)_{i∈\overline{n}}$ forms a
  balanced trace.
  The stuck case is similar.
\end{proof}

The previous lemma is interesting as it allows us to apply the classifying
terminology of interior traces to a $τ$ that is an abstraction of a
\emph{maximal} LK trace.
For such a maximal $τ$ we will say that it is balanced when it ends with
$\goodend{\wild,\wild}$, stuck if ending in $\stuckend$ and diverging if
infinite.

With increased clarity, we go on to prove the correctness predicate:

\begin{theorem}[Correctness of $\semvan{\wild}$]
  \label{thm:semvan-correct}
  $\mathcal{C}$ from \Cref{fig:semvan-correctness} holds.
  That is, whenever $(σ_i)_{i∈\overline{n}}$ is a maximal LK trace with source
  state $(\pe,ρ,μ,κ)$, we have
  $α_{\STraces}((σ_i)_{i∈\overline{n}},κ) = \semvan{\pe}_{ρ}(α_\Heaps(μ))$.
\end{theorem}
\begin{proof}
By Löb induction, with $IH ∈ \later C$ as the hypothesis.

We will say that an LK state $σ$ is stuck if there is no applicable rule in the
transition system (\ie, the singleton LK trace $σ$ is maximal and stuck).

Now let $(σ_i)_{i∈\overline{n}}$ be a maximal LK trace with source state
$σ_0=(\pe,ρ,μ,κ)$ and let $τ = \semvan{\pe}_{α_\Environments(ρ)}(α_\Heaps(μ))$.
Then the goal is to show $α_{\STraces}((σ_i)_{i∈\overline{n}},κ) = τ$.
We do so by cases over $\pe$, abbreviating $\tm \triangleq α_\Heaps(μ)$:
\begin{itemize}
  \item \textbf{Case $\px$}:
    Let us assume first that $σ_0$ is stuck. Then $\px \not∈ \dom(ρ)$ (because
    $\LookupT$ is the only transition that could apply), so
    $τ = \stuckend$ and the goal follows from
    \Cref{thm:abs-max-trace}.

    Otherwise, $σ_1 \triangleq (\pe', ρ', μ, \UpdateF(\pa) \pushF κ), σ_0 \smallstep σ_1$
    via $\LookupT$, and $ρ(\px) = \pa, μ(\pa) = (ρ', \pe')$.
    To show that the tails equate, it suffices to show that they equate \emph{later}.

    We can infer that $\tm(\pa) = \memo(\pa,\semvan{\pe'}_{ρ'})$ from the
    definition of $α_\Heaps$, so
    \[
      \tm(ρ(\px))(\tm) = (\semvan{\pe'}_{ρ'} \betastep \fn{v}{(\fn{μ}{\laterC~\goodend{v,μ[\pa ↦ \memo(\pa,\ret(v))]}})})(\tm)
    \]

    Let us define $τ^\later \triangleq \idiom{\semvan{\pe'}_{ρ'}(\tm)}$ and
    apply the induction hypothesis $IH$ to the maximal trace starting at $σ_1$.
    This yields an equality
    \[
      IH \aplater (σ_{i+1})_{i∈\overline{m}} ∈ \idiom{α_\STraces((σ_{i+1})_{i∈\overline{m}},\UpdateF(\pa) \pushF κ) = τ^{\later}}
    \]
    When $τ^{\later}$ is infinite, we are done. Similarly, if $τ^{\later}$ ends
    in $\stuckend$ then $(\betastep)$ will return $τ^{\later}$, indicating
    by \Cref{thm:abs-length} and \Cref{thm:abs-max-trace} that
    $(σ_{i+1})_{i∈\overline{n-1}}$ is stuck and hence $(σ_i)_{i∈\overline{n}}$
    is, too.

    Otherwise $τ^{\later}$ ends after $m-1$ $\laterC$s with $\goodend{v,\tm_m}$ and
    by \Cref{thm:abs-max-trace} $(σ_{i+1})_{i∈\overline{m}}$ is balanced; hence
    $\cont(σ_m) = \UpdateF(\pa) \pushF κ$ and $\ctrl(σ_m)$ is a value.
    So $σ_m = (\pv,ρ_m,μ_m,\UpdateF(\pa) \pushF κ)$ and the
    $\UpdateT$ transition fires, reaching $(\pv,ρ_m,μ_m[\pa ↦ (ρ_m, \pv)],κ)$
    and this must be the target state $σ_n$ (so $m = n-2$), because it remains
    a return state and has continuation $κ$, so $(σ_i)_{i∈\overline{n}}$ is
    balanced.
    Likewise, by the second argument of $(\betastep)$, we call do another
    memoisation step on $\goodend{v,\tr_m}$, updating the heap to
    \[
      \goodend{v,\tm_m[\pa↦\memo(\pa,\ret(v))]} = \goodend{v,\tm_m[\pa↦\memo(\pa,\semvan{\pv}_{ρ_m})]} = α_\VanV(σ_n),
    \]
    and this equality concludes the proof.

  \item \textbf{Case $\pe~\px$}:
    The cases where $τ$ gets stuck or diverges before finishing evaluation of
    $\pe$ are similar to the variable case.

    So let us focus on the situation when $τ^{\later} \triangleq
    \idiom{\semvan{\pe}_{ρ}(\tm)}$ returns and let $σ_m$ be LK state at the
    end of the finite maximal trace $(σ_{i+1})_{i∈\overline{m-1}}$ through $\pe$
    starting in stack $\ApplyF(\pa) \pushF κ$.
    Since it is maximal, any transition $σ_m \smallstep σ_{m+1}$ must leave
    the stack $\ApplyF(\pa) \pushF κ$, necessarily by an $\AppET$ transition.
    That in turn means that the value in $\ctrl(σ_m)$ must be a lambda
    $\Lam{\py}{\pe'}$, hence $τ^{\later}$ ends in $\goodend{\FunV(\fn{\pa}{\idiom{\semvan{\pe'}_{ρ_m[\py ↦ \pa]}}}), \tm_m} = σ_\VanV(σ_m)$
    (where $\tm_m$ corresponds to the heap in $σ_m$ in the usual way).

    Now let $(σ_{i+m+1})_{i∈\overline{k}}$ be the maximal trace starting at
    $σ_{m+1}=(\pe',ρ_m[\py↦\pa], μ_m,κ)$.
    We can apply the induction hypothesis to this LK trace and
    $τ_2^{\later} \triangleq \idiom{\semvan{\pe'}_{ρ_m[\py↦\pa]}(\tm_m)}$
    to get the equality
    $\idiom{α_\STraces((σ_{i+m+1})_{i∈\overline{k}},κ) = τ_2^{\later}}$.
    From this and our earlier equalities, we get
    $α_\STraces((σ_i)_{i∈\overline{n}},κ) = τ$, concluding the proof.

  \item \textbf{Case $\Case{\pe_s}{\Sel[r]}$}:
    Similar to the application and lookup case.

  \item \textbf{Cases $\Lam{\px}{\pe}$, $K~\many{\px}$}:
    The length of both traces is $n = 0$ and the goal follows by simple calculation.

  \item \textbf{Case $\Let{\px}{\pe_1}{\pe_2}$}:
    Let $σ_0 = (\Let{\px}{\pe_1}{\pe_2},ρ,μ,κ)$.
    Then $σ_1 = (\pe_2, ρ', μ',κ)$ by $\BindT$, where $ρ' = ρ[\px↦\pa], μ'
    = μ[\pa↦(ρ',\pe_1)]$.
    Since the stack does not grow, maximality from the tail $(σ_{i+1})_{i∈\overline{n-1}}$
    transfers to $(σ_{i})_{i∈\overline{n}}$.
    Straightforward application of the induction hypothesis to
    $(σ_{i+1})_{i∈\overline{n-1}}$ yields the equality for the tail (after a bit
    of calculation for the updated environment and heap), which concludes the
    proof.
\end{itemize}
\end{proof}

\Cref{thm:semvan-correct} and \Cref{thm:abs-max-trace} are the key to proving a
strong version of adequacy for $\semvan{\wild}$, where $σ$ is defined to be a
\emph{final} state if $\ctrl(σ)$ is a value and $\cont(σ) = \StopF$.

\begin{theorem}[Adequacy of $\semvan{\wild}$]
  \label{thm:semvan-adequate}
  Let $τ = \semvan{\pe}_{[]}([])$.
  \begin{itemize}
    \item
      $τ$ ends with $\goodend{\wild,\wild}$ (is balanced) iff there exists a final
      state $σ$ such that $\inj(\pe) \smallstep^* σ$.
    \item
      $τ$ ends with $\stuckend$ (is stuck) iff there exists a non-final
      state $σ$ such that $\inj(\pe) \smallstep^* σ$ and there exists no $σ'$
      such that $σ \smallstep σ'$.
    \item
      $τ$ is infinite (is diverging) iff for all $σ$ with $\inj(\pe)
      \smallstep^* σ$ there exists $σ'$ with $σ \smallstep σ'$.
  \end{itemize}
\end{theorem}
\begin{proof}
  There exists a maximal trace $(σ_i)_{i∈\overline{n}}$ starting
  from $σ_0 = \inj(\pe)$, and by \Cref{thm:semvan-correct} we have
  $α_\STraces((σ_i)_{i∈\overline{n}},\StopF) = τ$.
  \begin{itemize}
    \item[$\Rightarrow$]
      \begin{itemize}
        \item
          If $(σ_i)_{i∈\overline{n}}$ is balanced, its target state $σ_n$
          is a return state that must also have the empty continuation, hence it
          is a final state.
        \item
          If $(σ_i)_{i∈\overline{n}}$ is stuck, it is finite and maximal, but not balanced, so its
          target state $σ_n$ cannot be a return state;
          otherwise maximality implies $σ_n$ has an (initial) empty continuation
          and the trace would be balanced. On the other hand, the only returning
          transitions apply to return states, so maximality implies there is no
          $σ'$ such that $σ \smallstep σ'$ whatsoever.
        \item
          If $(σ_i)_{i∈\overline{n}}$ is diverging, $n=ω$ and for every $σ$ with
          $\inj(\pe) \smallstep^* σ$ there exists an $i$ such that $σ = σ_i$ by
          determinism.
      \end{itemize}

    \item[$\Leftarrow$]
      \begin{itemize}
        \item
          If $σ_n$ is a final state, it has $\cont(σ) = \cont(\inj(\pe)) = []$,
          so the trace is balanced.
        \item
          If $σ$ is not a final state, $τ'$ is not balanced. Since there is no
          $σ'$ such that $σ \smallstep^* σ'$, it is still maximal; hence it must
          be stuck.
        \item
          Suppose that $n∈ℕ_ω$ was finite.
          Then, if for every choice of $σ$ there exists $σ'$ such that $σ
          \smallstep σ'$, then there must be $σ_{n+1}$ with $σ_n \smallstep
          σ_{n+1}$, violating maximality of the trace.
          Hence it must be infinite.
          It is also interior, because every stack extends the empty stack,
          hence it is diverging.
      \end{itemize}
  \end{itemize}
\end{proof}

\subsection{Discussion}

We can already give perspective on two of the goals we set ourselves in
\Cref{sec:problem}:

As the domain $\VanD$ of our semantics $\semvan{\wild}$ is one defined by
guarded recursion, the approximation order between elements of the domain is
discrete and all elements are total.
$\semvan{\wild}$ is formulated entirely within this internal language and as
such, looping programs are denoted by total elements as well,
fullfilling Goal 2.

\section{Eventful Semantics}
\label{sec:eventful}

In the previous section, we gave a trace-based call-by-need semantics and proved
it adequate \wrt the LK transition semantics.
With compositionality and structural induction we recover strong advantages of
denotational semantics.

It seems that \Cref{thm:semvan-adequate} is at odds with the fact that
the information encoded in the generated traces
is by far not enough to recover the LK transition system in the sense of
\citet[Chapter 43]{Cousot:21}.
It is however easily possible to ``elaborate'' $\semvan{\wild}$ to include the
proper intermediate state.
We think it is rather uninteresting to give the closed, elaborated definition of
$\semvan{\wild}$.
For one, such a definition would necessarily give up on simplicity.
Furthermore, for our usage analysis in \Cref{sec:problem}, we do not care so
much about the \emph{state} in which variable lookup happens, but rather about
at which \emph{address} the $\LookupT$ transition happened.

In this section, we will embellish the generated traces
to track the instantiations of the transition rules taken, calling those
instantiations \emph{events}.
The resulting \emph{eventful trace semantics} $\semevt{\wild}$ is (closer to)
the preferred framework of \citet{Cousot:21}.%
\footnote{Cousot's \emph{stateless} semantics goes one step further
and rematerialises the heap as needed from the history of the trace.}
%In essence, we shift attention from the \emph{nodes} of the control-flow graph
%to the \emph{edges} modelling the transitions.

\begin{figure}
\[\begin{array}{c}
 \arraycolsep=3pt
 \begin{array}{rrclcl}
  \text{Event} & ε & ∈ & \Events & ::=  & \BindE(\px,\pa↦d) \mid \LookupE(\pa) \mid \UpdateE(\pa↦v) \\
               &   &   &          & \mid & \AppIE(\pa) \mid \AppEE(\px↦\pa) \mid \CaseIE(d) \mid \CaseEE(K,\many{\px↦\pa}) \\
 \end{array} \\
 \\[-1em]
 \arraycolsep=3pt
 \begin{array}{rrclcl@{\quad}rrclcl}
  \text{Heap}         & μ   & ∈ & \Heaps         & =      & \Addresses \pfun \later\EventD
  &
  \text{Delayed event}        & ε^{\later} & ∈ & \later \Events         &     &
  \\
  \text{Program trace}        & τ          & ∈ & \Traces        & ::= & \goodend{v,μ} \mid \stuckend{} \mid ε^{\later} \cons τ^{\later}
  &
  \text{Delayed trace}         & τ^{\later} & ∈ & \later \Traces &     &
  \\
  \text{Eventful domain}      & d          & ∈ & \EventD                   & =   & \Heaps \to \Traces
  &
  \text{Delayed element}       & d^{\later} & ∈ & \later \EventD            &     &
 \end{array} \\
 \\[-1em]
 \arraycolsep=3pt
 \begin{array}{rrclcl}
  \text{Eventful value} & v & ∈ & \EventV & ::= & \FunV(f ∈ \Addresses \to \EventD) \mid \ConV(K,\many{\pa}^{α_K}) \\
 \end{array} \\
 \\[-0.75em]
 \begin{array}{rcl}
  \multicolumn{3}{c}{ \ruleform{
    \begin{array}{c}
      (\betastep) : \EventD \to (\EventV \pfun \EventD) \to \EventD \quad \ret : \EventV \to \EventD \quad \apply : \EventD \times \Addresses \to \EventD \\
      \memo : \Addresses \times \EventD \to \EventD \quad \select : \EventD \times ((K:\Con) \times \pa^{α_K} \pfun \EventD) \to \EventD \\
    \end{array}
  }} \\
  \\[-0.75em]
  (d \betastep f)(μ) & = & \begin{cases}
    \many{ε \cons {}} f(v)(μ') & \text{$d(μ) = \many{ε \cons {}} \goodend{v,μ'}$ and $v ∈ \dom(f)$} \\
    \many{ε \cons {}} \stuckend{} & \text{$d(μ) = \many{ε \cons {}} \goodend{v,μ'}$ and $v \not∈ \dom(f)$} \\
    d(μ) & \text{otherwise} \\
  \end{cases} \\
  \\[-0.75em]
  \ret(v)(μ) & = & \goodend{v,μ} \\
  \apply(d,\pa) & = & d \betastep \fn{(\FunV(f))}{f(\pa)} \\
  \select(d,\alts) & = & d \betastep \fn{(\ConV(K_s,\many{d_s^\later}))}{\alts(K_s, \many{d_s^\later})} \quad \text{where } (K_s, \many{d_s^\later}) ∈ \dom(\alts) \\
  \memo(\pa,d)   & = & d \betastep \fn{v}{(\fn{μ'}{\UpdateE(\pa↦v) \cons \goodend{v,μ'[\pa↦\memo(\pa,\ret(v))]}})} \\
 \end{array} \\
 \\[-0.75em]
 \begin{array}{rcl}
  \multicolumn{3}{c}{ \ruleform{ \semevt{\wild} \colon \Exp → (\Var \pfun \Addresses) → \EventD } } \\
  \\[-0.75em]
  \semevt{\px}_ρ(μ)       & = & \begin{cases}
    \LookupE(ρ(\px)) \cons μ(ρ(\px))(μ) & \px ∈ \dom(ρ) \\
    \stuckend{}  & \text{otherwise} \\
  \end{cases} \\
  \\[-0.75em]
  \semevt{\Lam{\px}{\pe}}_ρ & = & \ret(\FunV(\fn{\pa}{\AppEE(\px↦\pa) \cons \semevt{\pe}_{ρ[\px↦\pa]}})) \\
  \\[-0.75em]
  \semevt{\pe~\px}_ρ(μ)   & = & \begin{cases}
    \AppIE(ρ(x)) \cons \apply(\semevt{\pe}_ρ, ρ(\px))(μ) & \px ∈ \dom(ρ) \\
    \stuckend{}  & \text{otherwise} \\
  \end{cases} \\
  \\[-0.75em]
  \semevt{\Let{\px}{\pe_1}{\pe_2}}_ρ(μ) & = &
    \begin{letarray}
      \text{let} & ρ' = ρ[\px ↦ \pa] \quad \text{where $\pa \not∈ \dom(μ)$} \\
                 & d_1^\later = \semevt{\pe_1}_{ρ'} \\
      \text{in}  & \BindE(\px,\pa↦d_1^\later) \cons \semevt{\pe_2}_{ρ'}(μ[\pa ↦ \memo(\pa,d_1^\later)])
    \end{letarray} \\
  \\[-0.75em]
  \semevt{K~\many{\px}}_ρ(μ) & = & \begin{cases}
    \ret(\ConV(K,\many{ρ(\px)}))(μ) & \many{\px ∈ \dom(ρ)} \\
    \stuckend & \text{otherwise} \\
  \end{cases} \\
  \\[-0.75em]
  \semevt{\Case{\pe_s}{\Sel[r]}}_ρ(μ) & = &
    \begin{letarray}
      \text{let} & \alts = \fn{(K_i, \many{\pa})}{\CaseEE(K_i,\many{\px_i↦\pa}) \cons \semevt{\pe_{r_i}}_{ρ[\many{\px_i↦\pa}]}} \\
      \text{in} & \CaseIE(\semevt{\pe_s}_ρ) \cons \select(\semevt{\pe_s}_ρ, \alts)(μ)  \\
    \end{letarray}
 \end{array} \\
\end{array}\]
\caption{Eventful Trace Semantics}
  \label{fig:semevt}
\end{figure}

\subsection{Favouring Events over State}

\Cref{fig:semevt} gives the definition for the eventful trace semantics
$\semevt{\wild}$.
The domain of eventful traces $\EventD$ maps a heap
to a possibly infinite or stuck program trace $τ$, just as the vanilla
semantics $\semvan{\wild}$.

Upon a transition of the internal machine state, our new semantics emits
an \emph{event} $ε$ with the cons $\cons$ constructor, rather than just
delaying with an uninformative $\laterC$.
In fact, each event carries with it part of the ``closure'' of the
corresponding LK transition rule, which is readily available for analysis.
For example, the $\LookupE$ event carries the heap address which is accessed,
$\BindE$ events carry the $\mathsf{let}$-bound variable, its fresh address and
the denotation of the right-hand side.
We thus open up our semantics for instrumentation.
Certainly we could have omitted some or added more information than what
the events carry; in this work we mostly care about $\LookupE$ events.

\begin{toappendix}
\begin{figure}
\[\begin{array}{rcl}
  α_\Heaps([\many{\pa ↦ (ρ,\pe)}]) & = & [\many{\pa ↦ \memo(\pa,\semvan{\pe}_{ρ})}] \\
  α_\EventV(ρ,\Lam{\px}{\pe}) & = & \FunV(\fn{\pa}{\AppEE(\px↦\pa) \cons \semevt{\pe}_{ρ[\px↦\pa]}}) \\
  α_\EventV(ρ,K~\many{\px}) & = & \ConV(K,\many{ρ(\px)}) \\
  α_\Events(σ) & = & \begin{cases}
    \BindE(\px,\pa↦\semevt{\pe_1}_{ρ[\px↦\pa])}) & σ = (\Let{\px}{\pe_1}{\wild},ρ,μ,\wild) \wedge \pa \not∈ \dom(μ) \\
    \AppIE(\pa) & σ = (\wild~\px,ρ,\wild,\ApplyF(\pa) \pushF \wild) \\
    \CaseIE(\semevt{\pe_s}_{ρ}) & σ = (\Case{\pe_s}{\wild},ρ,\wild, \wild)\\
    \LookupE(ρ(\px)) & σ = (\px,ρ,\wild,\UpdateF(\pa) \pushF \wild) \\
    \AppEE(\px↦\pa) & σ = (\Lam{\px}{\wild},\wild,\wild,\ApplyF(\pa) \pushF \wild) \\
    \CaseEE(K',\many{\px_i ↦ ρ(\py)}) & σ = (K'~\many{\py}, ρ, \wild, \SelF(\wild,\Sel) \pushF \wild) \wedge K' = K_i \\
    \UpdateE(\pa↦α_\EventV(ρ,\pv)) & σ = (\pv,ρ,\wild,\UpdateF(\pa) \pushF \wild) \\
  \end{cases} \\
  α_{\Traces}((σ_i)_{i∈\overline{n}},κ) & = & \begin{cases}
    α_\Events(σ_1) \cons α_{\Traces}((σ_{i+1})_{i∈\overline{n-1}},κ) & n > 1 \\
    \goodend{α_\EventV(ρ,\pv)} & n = 1 \wedge σ_1 = (\pv, ρ, \wild, κ) \\
    \stuckend{} & \text{otherwise} \\
  \end{cases} \\
  \correct & = & ∀(σ_i)_{i∈\overline{n}}.\ \maxtrace{(σ_i)_{i∈\overline{n}}} \Longrightarrow ∀((\pe,ρ,μ,κ) = σ_1).\ α_{\Traces}((σ_i)_{i∈\overline{n}},κ) = \semvan{\pe}_{ρ}(α_\Heaps(μ)) \\
\end{array}\]
\caption{Correctness predicate for $\semevt{\wild}$}
  \label{fig:semevt-correctness}
\end{figure}
\end{toappendix}

There is one minor structural difference to $\semvan{\wild}$:
$\FunV$ values now return an unguarded $\EventD$ so that the $\AppEE$ event
can carry the lambda-bound variable of the lambda expression from whence it
came.
This unguarded (positive) occurrence is not prohibited by the type construction
rules of guarded domain theory, but \citet[Section 5.2]{tctt} suggests that we
could work around this issue by defining $\EventD$ as a simultaneous guarded
recursive and inductive type, \ie,
$\EventD = \fix X.\ \lfp Y.\ ... \FunV(f ∈ X \to Y) ...$, where $\fix$ is the
guarded fixpoint operator and $\lfp$ is the least fixed-point operator.

All results connecting $\semvan{\wild}$ to the LK transition semantics also
hold for $\semevt{\wild}$, since $\semvan{\wild}$ can be recovered by throwing
away all events and replacing $\cons$ with $\laterC$.

Thus the only thing worth verifying is that the events actually correspond to
meaningful LK concepts.
One can read off this correpondence in the abstraction function for the
extended correctness predicate for $\semevt{\wild}$ which we give in
\Cref{fig:semevt-correctness} in the Appendix.

\begin{theorem}[Correctness of $\semevt{\wild}$]
  \label{thm:semvan-correct}
  $\correct$ from \Cref{fig:semevt-correctness} holds.
  That is, whenever $(σ_i)_{i∈\overline{n}}$ is a maximal LK trace with source
  state $(\pe,ρ,μ,κ)$, we have
  $α_{\Traces}((σ_i)_{i∈\overline{n}},κ) = \semevt{\pe}_{ρ}(α_\Heaps(μ))$.
\end{theorem}

It is clear that this style of semantics allows to observe arbitrary operational
detail on an as-needed basis, thus we fulfill our Goal 3.
Let us conclude this section by giving the embellished version of
the earlier example from \Cref{sec:vanilla}, $\Let{i}{\Lam{x}{x}}{i~i}$:
\[\begin{array}{ll}
  \newcommand{\myleftbrace}[4]{\draw[mymatrixbrace] (m-#2-#1.west) -- node[right=2pt] {#4} (m-#3-#1.west);}
  \vcenter{\hbox{$
    \begin{tikzpicture}[mymatrixenv,anchor=center]
      \matrix [mymatrix] (m)
      {
        1 & \BindE(i,\pa_1↦d) \cons {} & \hspace{3.7em} & \hspace{4.2em} & \hspace{3.5em} \\
        2 & \AppIE(\pa_1) \cons {} & & & \\
        3 & \LookupE(\pa_1) \cons {} & & & \\
        4 & \UpdateE(\pa_1↦\FunV(f)) \cons {} & & & \\
        5 & \AppEE(x↦\pa_1) \cons {} & & & \\
        6 & \LookupE(\pa_1) \cons {} & & & \\
        7 & \UpdateE(\pa_1↦\FunV(f)) \cons {} & & & \\
        8 & \goodend{\FunV(f), μ} & & & \\
      };
      % Braces, using the node name prev as the state for the previous east
      % anchor. Only the east anchor is relevant
      \foreach \i in {1,...,\the\pgfmatrixcurrentrow}
        \draw[dotted] (m.east|-m-\i-\the\pgfmatrixcurrentcolumn.east) -- (m-\i-2);
      \myleftbrace{3}{1}{8}{$\semevt{\pe}_{[]}$}
      \myleftbrace{4}{2}{8}{$\semevt{i~i}_{ρ_1}$}
      \myleftbrace{5}{3}{5}{$\semevt{i}_{ρ_1}$}
      \myleftbrace{5}{6}{8}{$\semevt{x}_{ρ_2}$}
  \end{tikzpicture}
  $}} &
  \!\!\!\!\text{where}  \begin{array}{ll}
  ρ_1 = [i ↦ \pa_1] & \\
  ρ_2 = ρ_1[x ↦ \pa_1] &  \\
  d = \semevt{\Lam{x}{x}}_{ρ_1} \\
  μ = [\pa_1 ↦ \memo(\pa_1,d)] & \\
  f = \pa \mapsto \semevt{\px}_{ρ_1[\px↦\pa]}
  \end{array}
\end{array}\]

%And then also for the heap update in $\Let{i}{(\Lam{y}{\Lam{x}{x}})~i}{i~i}$:
%
%\[\begin{array}{ll}
%  \newcommand{\myleftbrace}[4]{\draw[mymatrixbrace] (m-#2-#1.west) -- node[right=2pt] {#4} (m-#3-#1.west);}
%  \vcenter{\hbox{$
%    \begin{tikzpicture}[baseline={-0.5ex},mymatrixenv]
%      \matrix [mymatrix] (m)
%      {
%        1 & (μ_1)~\UpdateE(\pa_1↦v) \cons {} & \hspace{4em} & \hspace{3.5em} \\
%        2 & (μ_2)~\AppEE(x↦\pa_1) \cons {} & & \\
%        3 & (μ_2)~\LookupE(\pa_1) \cons {} & & \\
%        4 & (μ_2)~\UpdateE(\pa_1↦v) \cons {} & & \\
%        5 & \goodend{v, μ_2} & & \\
%      };
%      % Braces, using the node name prev as the state for the previous east
%      % anchor. Only the east anchor is relevant
%      \foreach \i in {1,...,\the\pgfmatrixcurrentrow}
%        \draw[dotted] (m.east|-m-\i-\the\pgfmatrixcurrentcolumn.east) -- (m-\i-2);
%      \myleftbrace{3}{1}{5}{$\semevt{i~i}_{ρ_1}$}
%      \myleftbrace{4}{1}{2}{$\semevt{i}_{ρ_1}$}
%      \myleftbrace{4}{3}{5}{$\semevt{x}_{ρ_3}$}
%    \end{tikzpicture}
%  $}} &
%  \!\!\!\text{where} \begin{array}{l}
%  ρ_1 = [i ↦ \pa_1] \\
%  ρ_2 = [i ↦ \pa_1, y ↦ \pa_1] \\
%  ρ_3 = [i ↦ \pa_1, y ↦ \pa_1, x ↦ \pa_1] \\
%  v = \FunV(\fn{\pa}{\semevt{\px}_{ρ_2[\px↦\pa]}}) \\
%  μ_1 = [\pa_1 ↦ \memo(\pa_1,\semevt{(\Lam{y}{\Lam{x}{x}})~i}_{ρ_1})] \\
%  μ_2 = [\pa_1 ↦ \memo(\pa_1,\semevt{\Lam{x}{x}}_{ρ_2})] \\
%  \end{array} \\
%\end{array}\]

\section{The Imperative Essence of Memoisation}
\label{sec:essence}

Before we can talk about abstraction, we have to record a few meta-theoretic
properties.
We will prove these properties in terms of the eventful semantics
$\semevt{\wild}$ but they apply just the same to the vanilla $\semvan{\wild}$.

\subsection{Heap Forcing and Laziness}

As we have observed in \Cref{sec:problem}, the semantics of call-by-need is more
complicated than the semantics for call-by-name and call-by-value because it
relies on a heap to implement thunk memoisation.
This leads to a tricky domain model that passes around
heaps as state and exposes tiresome operational detail such as heap addresses.

On the other hand, memoisation is a very ``benign'' use of state and often we
can reason rather naturally about call-by-need programs without thinking too
much about heaps, in contrast to a call-by-name calculus with arbitrary
assignments to ref cells.

For example, we can observe that heaps in our semantics evolve in a very
specific way:
When $\semevt{\pe}_ρ(μ) = ... \cons \goodend{v,μ'}$, we intuit that $μ'$ must be
``at least as evaluated'' as $μ$, \eg, for all $\pa ∈ \dom(μ)$ either $μ(\pa) =
μ'(\pa)$ or $μ'(\pa)$ is ``the value of'' $μ(\pa)$.
We make this observation precise via the following definitions:

\begin{definition}[Big-step]
  \label{defn:eval-d}
  For all $d ∈ \EventD$, $v ∈ \EventV$, $μ, μ' ∈ \Heaps$ we say that
  $d$ \emph{evaluates} in $μ$ to $(v,μ')$ (written $\bigstep{d}{μ}{v}{μ'}$) if
  and only if $d(μ) = ... \cons \goodend{v,μ'}$.
\end{definition}

Big-step notation deliberately throws away much information about the trace,
retaining only whether it was balanced and what the final value and heap is,
hence the name ``big-step'' is fitting.

\begin{definition}[Heap forcing]
  \label{defn:force-heap}
  For all $μ_1, μ_2 ∈ \Heaps$ we say that $μ_1$ \emph{forces} to $μ_2$
  (written $μ_1 \forcesto μ_2$) if and only if
  \begin{enumerate}
    \item $\dom(μ_1) ⊆ \dom(μ_2)$, and
    \item For all $\pa ∈ \dom(μ_1)$, either
      \begin{itemize}
        \item $μ_1(\pa) = μ_2(\pa)$, or
        \item there exists $v,μ'_1$ such that
          $μ_2(\pa) = \memo(\pa,\ret(v))$ and $\bigstep{μ_1(\pa)}{μ_1}{v}{μ'_1}$.
          Additionally, if $μ_1 \not= μ'_1$ then  $μ'_1 \forcesto μ_2$.
      \end{itemize}
  \end{enumerate}
\end{definition}

(This is a well-formed inductive definition because $μ_1' \forcesto μ_2$
is strictly decreasing in the termination measure we give in the proof of
\Cref{defn:lazy-d}.)

Our earlier intuition can now be formulated as follows:
If $\bigstep{\semevt{\pe}_ρ}{μ}{v}{μ'}$, then $μ \forcesto μ'$.

For the remainder of this work, we will identify heaps modulo consistent
readdressing.
Of course, a rigorous treatment would have to carry around readdressing
substitutions and apply them to adjust the domain of a heap, its entries,
as well as the heaps, domain elements and values in the returned traces.
\sg{Perhaps I should give a figure with the definitions I have in mind, maybe
in the Appendix.}

Once a heap entry is evaluated, it stays that way during forcing:

\begin{lemma}
  \label{thm:force-heap-val}
  For all $μ_1,μ_2,\pa,v$ such that $μ_1 \forcesto μ_2$ and $μ_1(\pa) = \memo(\pa,\ret(v))$,
  then $μ_2(\pa) = \memo(\pa,\ret(v))$.
\end{lemma}
\begin{proof}
  Immediate proof by contraposition and unfolding of $\ret$.
\end{proof}

The heap forcing relation is reflexive (it is always $μ(\pa) = μ(\pa)$) and
also antisymmetric:

\begin{lemmarep}[$(\forcesto)$ is antisymmetric]
  \label{thm:force-heap-trans}
  Let $μ_1,μ_2$ be heaps and $μ_1 \forcesto μ_2, μ_2 \forcesto μ_1$.
  Then $μ_1 = μ_2$.
\end{lemmarep}
\begin{proof}
  It is easy to see that both $μ_1$ and $μ_2$ must have the same domain.

  Now suppose that there exists $\pa$ such that $μ_1(\pa) \not= μ_2(\pa)$.
  Then there exists $v_1$ such that $\bigstep{μ_1(\pa)}{μ_1}{v_1}{\wild}$ and
  $μ_2(\pa) = \ret(v_1)$.
  By \Cref{thm:force-heap-val} applied to $μ_2 \forcesto μ_1$, we get $μ_1(\pa)
  = v_1 = μ_2(\pa)$, a contradiction.
\end{proof}

(It is worth stressing that antisymmetry only holds on equivalence classes modulo readdressing.)
However, $(\forcesto)$ is not easily proven transitive without further
characterisation of the domain elements returned by $\semevt{\wild}$.

\begin{definition}[Lazy domain]
  \label{defn:lazy-d}
  An element $d ∈ \EventD$ is \emph{lazy} if and only if
  \begin{itemize}
    \item \textup{(Forces)} For all lazy $μ,μ' ∈ \Heaps, v ∈ \EventV$ such that
     $\bigstep{d}{μ}{v}{μ'}$, $v$ is lazy and $μ \forcesto μ'$.
    \item \textup{(Postpone)} For all lazy $μ_1,μ_2 ∈ \Heaps$ such that $μ_1 \forcesto μ_2$,
     \begin{itemize}
       \item If $\bigstep{d}{μ_1}{v_1}{μ'_1}$ (for some $v_1,μ'_1$) then
         $\bigstep{d}{μ_2}{v_2}{μ'_2}$ (for some $v_2,μ'_2$) and
         $v_1 = v_2$ as well as $μ'_1 \forcesto μ'_2$
       \item If $d(μ_1)$ diverges then so does $d(μ_2)$.
     \end{itemize}
    \item \textup{(Speculate)} For all lazy $μ_1,μ_2 ∈ \Heaps$ such that $μ_1 \forcesto μ_2$ and $d(μ_1)$ is defined,
     \begin{itemize}
       \item If $\bigstep{d}{μ_2}{v_2}{μ'_2}$ (for some $v_2,μ'_2$) then
         $\bigstep{d}{μ_1}{v_1}{μ'_1}$ (for some $v_1,μ'_1$) and
         $v_1 = v_2$ as well as $μ'_1 \forcesto μ'_2$
       \item If $d(μ_2)$ diverges then so does $d(μ_1)$.
     \end{itemize}
  \end{itemize}
  A heap $μ$ is \emph{lazy} if and only if it is of the form $[\many{\pa↦\memo(\pa,d)}]$ and all $\many{d}$ are lazy and defined on $μ$.\\
  A value $v$ is \emph{lazy} if and only if $v = \FunV(f)$ (for some $f$) implies that $f(\pa)$ is lazy.
\end{definition}

So when an element $d$ is \emph{lazy}, every input heap forces to the output
heap, it evaluates to the same value whether or not the input heap is more or
less forced, and $\bigstep{d}{\wild}{v}{\wild}$ is natural (in the category
theoretic sense) in that it transfers the forcedness relation from input heaps
to output heaps.
Diagramatically:
\vspace{-\baselineskip}
\[
\begin{tikzcd}
  μ_1 \arrow[Rightarrow]{d}[swap]{\langle d, \wild\rangle}{\langle v, \wild\rangle} \arrow[rightsquigarrow]{r}{} & μ_2 \arrow[Rightarrow]{d}{\langle v, \wild\rangle}[swap]{\langle d, \wild\rangle} \\
  μ_1' \arrow[rightsquigarrow]{r}{} & μ_2'
\end{tikzcd}
\]
\noindent
Well-addressedness (the so far implicit assumption that $d(μ)$ is defined)
in conjunction with the Speculate property means that a lazy element may branch
on the contents of the heap, but not on whether or not an entry in the heap is
present.

The definitions for lazy heaps, environments and values are necessary by congruence.
When we restrict ourselves to lazy heaps, we can prove transitivity of the
forcing relation.

\begin{lemmarep}[$(\forcesto)$ is transitive]
  \label{thm:force-heap-trans}
  Let $μ_1,μ_2,μ_3$ be lazy heaps and $μ_1 \forcesto μ_2, μ_2 \forcesto μ_3$.
  Then $μ_1 \forcesto μ_3$.
\end{lemmarep}
\begin{proof}
  We need a termination measure on $(\forcesto)$ before we can proceed.
  Note that in successive steps $μ_1 \forcesto ... \forcesto μ_n$ where all
  $μ_i$ are lazy heaps, heap entries progress from being undefined,
  then to a non-value denotation, and finally to a value element $\ret(v)$.

  More formally, we can define a measure $m : \Heaps \to \dom(μ_n) \pfun \{0,1,2\}$
  \[
    m(μ)(\pa) = \begin{cases}
        0 & μ(\pa) = \ret(\wild) \\
        1 & \pa ∈ \dom(μ) \\
        2 & \text{otherwise} \\
      \end{cases}
  \]
  with the pointwise partial order on $\dom(μ_n) \pfun \{0,1,2\}$.
  When $\dom(μ_n)$ is finite, this partial order has finite height, which can
  thus serve as a termination measure.

  We proceed by well-founded induction with the above measure $m$ defined on
  $\dom(μ_3)$.
  More precisely, show that the proposition is satisfied for $μ_1,μ_2,μ_3$
  assuming that the proposition holds for any $μ_1',μ_2',μ_3'$
  such that $m(μ_1') - m(μ_3') < m(μ_1) - m(μ_3)$.

  We can see that $\dom(μ_1) ⊆ \dom(μ_2) ⊆ \dom(μ_3)$, proving the first property
  of \Cref{defn:force-heap}.
  For the second property, fix an arbitrary $\pa$.
  When $μ_1(\pa) = μ_2(\pa)$ and $μ_2(\pa) = μ_3(\pa)$, we have $μ_1(\pa) = μ_3(\pa)$.
  Otherwise, either $μ_1(\pa) \not= μ_2(\pa)$ or $μ_2(\pa) \not= μ_3(\pa)$.

  Suppose that $μ_1(\pa) \not= μ_2(\pa)$.
  Then there exists $v,μ'_1$ such that $\bigstep{μ_1(\pa)}{μ_1}{v}{μ'_1}$ and
  $μ_2(\pa) = \ret(v)$ as well as $μ'_1 \forcesto μ_2$.
  By \Cref{thm:force-heap-val} applied to $μ_2 \forcesto μ_3$, we have
  $μ_2(\pa) = μ_3(\pa) = \ret(v)$.
  If $μ_1' \not= μ_1$, then $m(μ_1') < m(μ_1)$ and by well-founded induction
  we can combine $μ'_1 \forcesto μ_2$ and $μ_2 \forcesto μ_3$ to
  $μ'_1 \forcesto μ_3$.
  Hence we have shown that $μ_1 \forcesto μ_3$.

  Suppose that $μ_1(\pa) = μ_2(\pa)$, but $μ_2(\pa) \not= μ_3(\pa)$.
  Then there exists $v,μ'_2$ such that $\bigstep{μ_2(\pa)}{μ_2}{v}{μ'_2}$ and
  $μ_3(\pa) = \ret(v)$ as well as $μ'_2 \forcesto μ_3$.
  We know that $μ_1(\pa) = μ_2(\pa)$, so $\bigstep{μ_1(\pa)}{μ_2}{v}{μ'_2}$.
  Since $μ_1(\pa)$ is lazy and well-addressed \wrt $μ_1$ and $μ_1 \forcesto
  μ_2$, we get $\bigstep{μ_1(\pa)}{μ_1}{v}{μ'_1}$ such that $μ_1' \forcesto
  μ_2'$ by definition of laziness.
  By well-founded induction, we can combine $μ'_1 \forcesto μ_2'$ and
  $μ_2' \forcesto μ_3$ to $μ'_1 \forcesto μ_3$.
  If $μ_1' \not= μ_1$, then $m(μ_1') < m(μ_1)$ and by well-founded induction
  we can combine $μ'_1 \forcesto μ_2'$ and $μ_2' \forcesto μ_3$ to
  $μ'_1 \forcesto μ_3$.
  Hence have shown that $μ_1 \forcesto μ_3$.
\end{proof}

\begin{corollary}
  $(\forcesto)$ is a partial order on lazy heaps.
\end{corollary}

And now we finally prove that $\semevt{\wild}$ indeed produces a lazy element:

\begin{theoremrep}
  \label{thm:semevt-lazy}
  For all environments $ρ$ and expressions $\pe$, $\semevt{\pe}_ρ$ is a lazy element.
\end{theoremrep}
\begin{proof}
  By induction on $\pe$.
  \begin{itemize}
    \item \textbf{Case $\px$}:
      Then $\fn{\wild}{\stuckend{}}$ case is easy to see, so let $\pa = ρ(\px)$.

      To show that $\fn{μ}{\LookupE(\pa) \cons μ(\pa)(μ)}$ is lazy,
      it suffices to show that $\fn{μ}{μ(\pa)(μ)}$ is.

      We begin with the Forces property.
      So let $μ$ be a lazy input heap with $\pa ∈ \dom(μ)$.
      (The undefined case is boring, because the Forces property holds trivially.)
      By laziness, we know that $μ(\pa) = \memo(\pa,d)$ for some lazy $d$
      and that $d$ is defined on $μ$.
      Furthermore, in the interesting case $\bigstep{μ(\pa)}{μ}{v}{μ'}$
      for suitable lazy $v,μ'$, the trace resulting from calling the second
      argument to $(\betastep)$ is
      \[
        \UpdateE(\pa↦\ret(v)) \cons \goodend{v,μ'[\pa↦\memo(\pa,\ret(v))]}.
      \]
      And crucially, $μ' \forcesto μ'[\pa↦\memo(\pa,\ret(v))]$ due to
      $\bigstep{μ(\pa)}{μ}{v}{μ'}$ for the differing $\pa$ entry, proving
      the Forces property.
      The Postpone and Speculate property for the second argument follow quite
      simply, because the updated heap entry is constant in all cases.

    \item \textbf{Case $\Lam{px}{\pe'}$}:
      For the Forces property, we assume a lazy $μ$ such that
      $\bigstep{\semevt{\pe}_ρ}{μ}{\FunV(f)}{μ}$ holds for
      $f \triangleq \fn{d}{\AppEE(\px↦d) \cons \semevt{\pe}_{ρ[\px↦d]}}$.
      We have $μ \forcesto μ$ by reflexivity.
      To show that $\FunV(f)$ is indeed a lazy value, we assume a lazy $d$ and
      show that $f(d)$ is lazy by induction,
      so $\FunV(f)$ is a lazy value.
      The Postpone and Speculate properties follow by assumption.

    \item \textbf{Case $K~\many{\px}$}:
      Similar to the lambda case.

    \item \textbf{Case $\pe'~\px$}:
      The first argument to $\apply$,
      $d \triangleq \semevt{\pe}_ρ$, is lazy by the induction hypothesis.

      The interesting case is when $\bigstep{d}{μ}{\FunV(f)}{μ'}$.
      $\FunV(f)$ is lazy by the Forces property of $d$ and then
      so is $f(ρ(\px))$.

      Hence the whole $\apply$ form is lazy and prepending the $\AppIE$ event
      does not change that because it does not affect the heap.

    \item \textbf{Case $\Case{\pe_s}{\Sel[r]}$}:
      Similar to the application case.

    \item \textbf{Case $\Let{\px}{\pe_1}{\pe_2}$}:
      $d_{\pe_i} \triangleq \semevt{\pe_i}_{ρ'}$ are lazy by induction.
      Now let $μ$ be lazy and consider $\semevt{\pe}_ρ(μ)$.
      Then $μ[\pa↦\memo(\pa,d_{\pe_1})]$ is lazy and we see that
      $μ \forcesto μ[\pa↦d_{\pe_1}]$ because $\pa \not∈ \dom(μ)$.
      Now we can apply transitivity to the Forces property of the lazy
      $d_{\pe_2}$.
      The Postpone and Speculate properties follow by transitivity as well.
  \end{itemize}
\end{proof}

This result justifies our tacit assumption from now on that all elements of the
semantic domain are lazy.

\subsection{A Useful Program Equivalence}

\begin{figure}
\[\begin{array}{c}
 \ruleform{ \fin(τ) \qquad A ⊦ d \qquad A ⊦ τ_1 \sim τ_2 \qquad μ_1 \approx μ_2 \qquad d_1 \equiv d_2}
 \\
 \\[-0.5em]
 \inferrule*[right=\textsc{Fin}]
    {τ = \many{...{}\cons {}} τ' \quad τ' \not= \wild \cons \wild}
    {\fin(τ)}
 \quad
 \inferrule*[right=\textsc{Dom}]
    {∀μ.\ A ⊆ \dom(μ) \Longrightarrow d(μ) \text{ is defined}}
    {A ⊦ d}
 \\
 \\[-0.5em]
 \inferrule*[right=\eqlrcons]
    {\later (A ⊦ τ_1 \sim τ_2)}
    {A ⊦ \wild \cons τ_1 \sim \wild \cons τ_2}
 \quad
 \inferrule*[right=\eqrcons]
    {A ⊦ τ_1 \sim τ_2 \quad \fin(τ_2)}
    {A ⊦ τ_1 \sim \wild \cons τ_2}
 \quad
 \inferrule*[right=\eqlcons]
    {A ⊦ τ_1 \sim τ_2 \quad \fin(τ_1)}
    {A ⊦ \wild \cons τ_1 \sim τ_2}
 \\
 \\[-0.5em]
 \inferrule*[right=\eqstuck]
    {\quad}
    {A ⊦ \stuckend{} \sim \stuckend{}}
 \quad
 \inferrule*[right=\eqcon]
    {\many{A ⊦ μ_1(\pa_1) \sim μ_2(\pa_2)}}
    {A ⊦ \goodend{\ConV(K, \many{\pa_1}), μ_1} \sim \goodend{\ConV(K, \many{\pa_2}), μ_2}}
 \\
 \\[-0.5em]
 \inferrule*[right=\eqfun]
    {\forall \pa.\ \pa ∈ A ⟹  A ⊦ f_1(\pa)(μ_1) \sim f_2(\pa)(μ_2)}
    {A ⊦ \goodend{\FunV(f_1), μ_1} \sim \goodend{\FunV(f_2), μ_2}}
 \\
 \\[-0.5em]
 \inferrule*[right=\eqheap]
    {\dom(μ_1) = \dom(μ_2) \quad \forall \pa.\ \later(\dom(μ_1) ⊦ μ_1(\pa)(μ_1) \sim μ_2(\pa)(μ_2))}
    {μ_1 \approx μ_2}
 \\
 \\[-0.5em]
 \inferrule*[right=\eqdenot]
    {\forall μ_1,μ_2.\ μ_1 \approx μ_2 \wedge  \dom(μ_1) ⊦ d_1,d_2 ⟹  \dom(μ_1) ⊦ d_1(μ_1) \sim d_2(μ_2)}
    {d_1 \equiv d_2}
\end{array}\]
\caption{Semantic equivalence relation on lazy elements}
  \label{fig:sem-equiv}
\end{figure}

Consider the two expressions $\pe_1 \triangleq \Let{x}{\ttrue}{\ttrue}$ and
$\pe_2 \triangleq \Let{x}{\ffalse}{\ttrue}$.
For all intents and purposes, these two expressions are equivalent.
Indeed, they equate under $\semscott{\wild}$, suggesting that $x$ is dead
according to \Cref{defn:deadness}.

However, the definitional equality on the lazy domain $\EventD$ is \emph{too
fine}:
\[\begin{array}{rclcl}
  \semevt{\pe_1}_{[]}([])
  & = & \BindE(x,\pa_1↦\semevt{\ttrue}_{...}) \cons \goodend{\ConV(\ttrue),[\pa_1↦\semevt{\ttrue}_{...}]} \\
  & \not= & \BindE(x,\pa_1↦\semevt{\ffalse}_{...}) \cons \goodend{\ConV(\ttrue),[\pa_1↦\semevt{\ffalse}_{...}]}
  & = & \semevt{\pe_2}_{[]}([])
\end{array}\]
It would be reasonable to abstract away the contents of the trace to get a
coarser equivalence, thus eliminating the difference in $\BindE$ actions.
However, that still leaves the difference in heap entries such as $\pa_1$ which
will never be looked at!
Since the heap entry for $\pa_1$ was introduced while executing $\pe_1$/$\pe_2$,
we will call them \emph{local} to the trace $\semevt{\pe_i}_{[]}([])$.
It is prudent for the observer to ignore local entries.

That is the essence of what the \emph{semantic equivalence relation} $(\equiv)$
defined on lazy elements in \Cref{fig:sem-equiv} does.
It is coarse enough to equate the example above, but still finer than contextual
equivalence, as we shall see.

The key is its observer model for function values, which necessitates the
address space parameter $A$ in $\sim$ to track non-local (and thus observable)
addresses.
Consider a successful evaluation $\bigstep{d_i}{μ_i}{\FunV(f_i)}{μ_i'}$, as in
\[\begin{array}{rclcl}
  \semevt{\Let{t}{\ttrue}{\Lam{z}{t}}}_{[]}([])
  & = & ... \cons \goodend{\FunV(\fn{\wild}{\fn{μ}{...μ(\pa_1)...}}),[\pa_1↦d_t]} \\
  & \equiv & ... \cons \goodend{\FunV(\fn{\wild}{\fn{μ}{...μ(\pa_2)...}}),[\pa_1↦d_f,\pa_2↦d_t]} \\
  & = & \semevt{\Let{f}{\ffalse}{\Let{t}{\ttrue}{\Lam{z}{t}}}}_{[]}([])
\end{array}\]
The denotations of the two expressions are not definitionally equivalent in the final heap,
yet they are semantically equivalent.
To see that, we have to relate the heap entries in $\pa_1$ and $\pa_2$, which
only succeeds when the local entries in $μ_i'$ are retained.
So $\wild ⊦ \wild \sim \wild$ may not simply discard final heaps and carry
on with the initial one.

On the other hand, some detail must be hidden from the observer who may probe
functions for equality by supplying (nearly) any argument denotation they wish
via the heap.
The following example illustrates how exactly such argument denotations should be
restricted:
\[\begin{array}{rclcl}
  \semevt{\Let{x}{\ttrue}{\Lam{y}{y}}}_{[]}([])
  & = & ... \cons \goodend{\FunV(\fn{\pa}{\fn{μ}{...μ(\pa)...}}),[\pa_1↦\semevt{\ttrue}_{...}]} \\
  & \equiv & ... \cons \goodend{\FunV(\fn{\pa}{\fn{μ}{...μ(\pa)...}}),[\pa_1↦\semevt{\ffalse}_{...}]} \\
  & = & \semevt{\Let{x}{\ffalse}{\Lam{y}{y}}}_{[]}([]),
\end{array}\]
In both cases, the function value $\FunV(\fn{\pa}{\fn{μ}{...μ(\pa)...}})$ is to
be probed by the observer with an address of their choosing.
If the observer was given free reign, they could choose $\pa_1$ and thus be able
to observe the different heaps.
Since the parameter $A$ tracks non-local addresses (as can be asserted in
$\equiv$), we can prohibit such exploits by requiring $\pa∈A$.
This is not too grave a restriction on the observer, because they may
pre-allocate any denotation into the heap they want (respecting
well-addressedness, of course) due to laziness.

\begin{lemmarep}[Weakening of address domain]
  \label{thm:weaken-address-domain}
  Let $A_1,A_2$ be address domains such that $A_1 ⊆ A_2$. \\
  For all $d$, if $A_1 ⊦ d$ then $A_2 ⊦ d$.
  For all traces $τ_1,τ_2$, if $A_2 ⊦ τ_1 \sim τ_2$ then $A_1 ⊦ τ_1 \sim τ_2$.
\end{lemmarep}
\begin{proof}
  The first property follows by unpacking $\textsc{Dom}$.

  For the second property, take note that the check $\pa ∈ A$ occurs in negative
  position in $\wild ⊦ \wild \sim \wild$ and $A$ is never used anywhere else.
  Intuitively, the smaller $A$, the fewer $d$s the caller is allowed to
  ``probe'' $\FunV$ values with and the easier it is to prove bisimulation.
\end{proof}

\begin{lemma}[Address domain of $\semevt{\wild}$]
  \label{thm:addr-dom-sem}
  For all address spaces $A$, environments $ρ$ and expressions $\pe$,
  if $\rng(ρ) ⊆ A$ then $A ⊦ \semevt{\pe}_ρ$.
\end{lemma}
\begin{proofsketch}
  Simple proof by induction.
\end{proofsketch}

Our implicit assumption on well-addressedness can be encoded as an axiom as
follows:

\begin{axiom}[Well-addressedness]
  \label{thm:well-addressedness}
  For all address spaces $A$, environments $ρ$ and expressions $\pe$,
  if $A ⊦ \semevt{\pe}_ρ$ then $\rng(ρ) ⊆ A$.
\end{axiom}

One could tweak the definition of $\semevt{\wild}$ such that
it is undefined on an input heap $μ$ when $\rng(ρ) \not⊆ \dom(μ)$,
thus making this axiom a provable lemma.

Evaluating lazy elements on heaps that force to each other is semantically equivalent:

\begin{lemmarep}
  \label{thm:lazy-force-bisimilar}
  Let $d$ be lazy. Then for all $μ_1,μ_2$ with $\dom(μ_1) ⊦ d$ and
  $μ_1 \forcesto μ_2$, we have $\dom(μ_1) ⊦ d(μ_1) \sim d(μ_2)$.
\end{lemmarep}
\begin{proof}
  This lemma makes essential use of the Postpone and Speculate properties.
  Note that we are free to make use of the latter because $d$ is defined
  at $μ_1$.

  If either $d(μ_1)$ or $d(μ_2)$ is ultimately stuck, then so is the other,
  because it neither diverges nor evaluates.
  In that case, we see that $\dom(μ_1) ⊦ d(μ_1) \sim d(μ_2)$.

  If either $d(μ_1)$ or $d(μ_2)$ diverges, then so does the other, in which case
  $\dom(μ_1) ⊦ d(μ_1) \sim d(μ_2)$.
  If either $\bigstep{d}{μ_1}{v}{μ_1'}$ or $\bigstep{d}{μ_2}{v}{μ_2'}$ then both
  judgments hold and $μ_1' \forcesto μ_2'$, so it suffices to show
  ${\dom(μ_1) ⊦ \goodend{v,μ_1'} \sim \goodend{v,μ_2'}}$.

  For the $\ConV$ case, we can show $\many{\dom(μ_1) ⊦ μ_1'(\pa_1) \sim μ_2'(\pa_2)})$
  via Löb induction, because $μ_1' \forcesto μ_2'$ and we may relax the address
  space from $\dom(μ_1')$ to $\dom(μ_1)$ via \Cref{thm:weaken-address-domain}.

  For the $\FunV(f)$ case, we assume that $\dom(μ_1) ⊦ d$ holds for some
  $d$ and have to show that $\dom(μ_1) ⊦ f(d)(μ_1') \sim f(d)(μ_2')$.
  Again, that follows via Löb induction, because $μ_1' \forcesto μ_2'$ and we
  may relax the address space from $\dom(μ_1')$ to $\dom(μ_1)$ via
  \Cref{thm:weaken-address-domain}.
\end{proof}

\begin{theoremrep}
  For any fixed address space $A$, $A ⊦ \wild \sim \wild$ is an equivalence relation on lazy traces.
\end{theoremrep}
\begin{proof}
Reflexivity follows by straightforwardly by Löb induction.
Symmetry does so as well, with the twist that the rule applications of
$\eqlcons$ and $\eqrcons$ need to be swapped.

For transitivity, we assume
$A ⊦ τ_1 \sim τ_2$, $A ⊦ τ_2 \sim τ_3$ to show $A ⊦ τ_1 \sim τ_3$ by Löb
induction, \eg, assuming that
\[
  \later(A ⊦ τ_1 \sim τ_2 \wedge A ⊦ τ_2 \sim τ_3 \Longrightarrow A ⊦ τ_1 \sim τ_3)
\]
for all $τ_1,τ_2,τ_3$.

By cases on the last rule applied to either $A ⊦ τ_1 \sim τ_2$ or $A ⊦ τ_2 \sim τ_3$.
\begin{itemize}
  \item \textbf{Case }$\eqlcons$ to $A ⊦ τ_1 \sim τ_2$:
    Then $τ_1 = \wild \cons τ_1'$ and $A ⊦ τ_1' \sim τ_2$.
    We can apply the induction hypothesis to get $A ⊦ τ_1' \sim τ_3$
    and re-apply $\eqlcons$.
  \item \textbf{Case }$\eqrcons$ to $A ⊦ τ_2 \sim τ_3$: Similar.
  \item \textbf{Case }$\eqrcons$ to $A ⊦ τ_1 \sim τ_2$:
    Then $τ_2 = \wild \cons τ_2'$, $\fin(τ_2')$ and $A ⊦ τ_1 \sim τ_2'$.
    $\fin(τ_2')$ means that the entire derivation is of finite height.
    By rule inversion on $A ⊦ τ_2 \sim τ_3$, either $\eqlcons$ or $\eqlrcons$
    applies.
    If we unpack $\eqlcons$, then $A ⊦ τ_2' \sim τ_3$ and we can apply the
    induction hypothesis to get $A ⊦ τ_1 \sim τ_3$.
    Otherwise we unpack $\eqlrcons$ and get $A ⊦ τ_2' \sim τ_3'$ for
    some $τ_3'$ to which we apply the induction hypothesis to get
    $A ⊦ τ_1 \sim τ_3'$ which we wrap up with $\eqrcons$.
  \item \textbf{Case }$\eqlrcons$ to $A ⊦ τ_1 \sim τ_2$: Similar.
  \item \textbf{Case }$\eqfun$ to $A ⊦ τ_1 \sim τ_2$:
    By rule inversion on $A ⊦ τ_2 \sim τ_3$, $\eqfun$ must apply.
    The rest is a simple matter of applying the inductive hypothesis.
  \item \textbf{Case }$\eqcon$ to $A ⊦ τ_1 \sim τ_2$: Similar.
  \item \textbf{Case }$\eqstuck$ to $A ⊦ τ_1 \sim τ_2$: Similar.
\end{itemize}
\end{proof}

It follows that $(\approx)$ is an equivalence relation on lazy heaps as well.

Unfortunately, the notion of a lazy element we have is too weak to prove
reflexivity of $(\equiv)$.
We fix that by introducing the following extensionality requirement on lazy
elements $d$:

\begin{definition}[Extensionality]
  A lazy element $d$ is \emph{extensional} if and only if $d \equiv d$.
\end{definition}

\begin{theoremrep}
  $(\equiv)$ is an equivalence relation on lazy and extensional elements in
  $\EventD$.
\end{theoremrep}
\begin{proof}
Reflexivity follows by extensionality.
Symmetry follows straightforwardly by Löb induction, noting that the $\dom(μ_1)$
requirement in $(\equiv)$ can be replaced with $\dom(μ_2)$ due to the
$μ_1 \approx μ_2$ premise.

Now we show transitivity.
Assume that $d_1 \equiv d_2$ and $d_2 \equiv d_3$.
We have to show that $d_1 \equiv d_3$.

For that we assume $μ_1 \approx μ_3$ for some $μ_1,μ_3$ such that
$\dom(μ_1) ⊦ d_1, d_3$ and have to show that
\[
  \dom(μ_1) ⊦ d_1(μ_1) \sim d_3(μ_3).
\]
We can construct extended heaps $μ_1' \approx μ_3'$ such that
$μ_1 \forcesto μ_1'$, $μ_3 \forcesto μ_3'$ and $\dom(μ_i') ⊦ d_2$.

Clearly, $μ_1' \approx μ_1'$ by reflexivity, so we can invoke $d_1 \equiv d_2$
to get
\[
  \dom(μ_1') ⊦ d_1(μ_1') \sim d_2(μ_1').
\]
We invoke $d_2 \equiv d_3$ on $μ_1' \approx μ_3'$ to get
\[
  \dom(μ_1') ⊦ d_2(μ_1') \sim d_3(μ_3')
\]
and by transitivity we get
\[
  \dom(μ_1') ⊦ d_1(μ_1') \sim d_3(μ_3')
\]
which can be weakened via \Cref{thm:weaken-address-domain} to $\dom(μ_1)$.
\end{proof}

Note that we only make use of extensionality to prove reflexivity; in fact,
$(\equiv)$ is still a (perhaps incomparable) partial equivalence relation on lazy elements.

What remains is a proof that $\semevt{\pe}_ρ$ is indeed extensional.

\begin{theoremrep}
  For all environments $ρ$ and expressions $\pe$,
  $\semevt{\pe}_ρ$ is extensional.
\end{theoremrep}
\begin{proof}
By induction on $\pe$.
\begin{itemize}
  \item \textbf{Case $\px$}:
    Let $μ_1 \approx μ_2$ such that $\pa ∈ \dom(μ_i)$ and abbreviate
    $d_i \triangleq μ_i(\pa)$.
    Then the goal is to show that
    \[
      \dom(μ_1) ⊦ (d_1 \betastep f)(μ_1) \sim (d_2 \betastep f)(μ_2)
    \]
    where $f \triangleq  \fn{v\,μ'}{\UpdateE(\pa↦v) \cons \goodend{v,μ'[\pa↦\memo(\pa,\ret(v))]}}$.

    The interesting case is when $\bigstep{d_i}{μ_i}{v}{μ'_i}$, otherwise
    $\dom(μ_1) ⊦ d_1(μ_1) \sim d_2(μ_2)$ as per $μ_1 \approx μ_2$.
    So we assume $\dom(μ_1) ⊦ \goodend{v_1,μ_1'} \sim \goodend{v_2,μ_2'}$ and have to show that
    \[
       \dom(μ_1) ⊦ \goodend{v_1,μ_1'[\pa↦\memo(\pa,\ret(v_1))]} \sim \goodend{v_2,μ_2'[\pa↦\memo(\pa,\ret(v_2))]}
    \]
    But since $μ_i' \forcesto μ_i'[\pa↦\memo(\pa,\ret(v_i))]$, we can apply
    \Cref{thm:lazy-force-bisimilar} twice and transitivity of $\sim$ to show the
    goal.

  \item \textbf{Case $\Lam{\px}{\pe'}$}:
    Fix a particular $μ_1 \approx μ_2$. The goal is to show
    \[
      \dom(μ_1) ⊦ \goodend{v,μ_1} \sim \goodend{v,μ_2}
    \]
    where $v \triangleq \FunV(\fn{d}{\AppEE(\px↦d) \cons \semevt{\pe'}_{ρ[\px↦d]}})$.
    So let us fix (extensional!) $d$ such that $\dom(μ_1) ⊦ d$ and show
    \[
      \dom(μ_1) ⊦ \semevt{\pe'}_{ρ[\px↦d]}(μ_1) \sim \semevt{\pe'}_{ρ[\px↦d]}(μ_2)
    \]
    which follows by the extensionality of $\semevt{\pe'}_{ρ[\px↦d]}$.

  \item \textbf{Case $K~\many{\px}$}:
    Similar to the lambda case.

  \item \textbf{Case $\pe'~\px$}:
    $\fn{\wild}{\stuckend{}}$ is extensional in case $\px \not∈ \dom(ρ)$.

    Otherwise, by induction it suffices to show that for extensional
    $d \triangleq \semevt{\pe'}_ρ$, the $\apply$ expression
    $d \betastep \fn{(\FunV(f))}{f(ρ(\px))}$ is extensional.

    Now fix particular $μ_1 \approx μ_2$ such that $\dom(μ_1) ⊦ \semevt{\pe}_ρ$.
    The interesting case is when $\bigstep{d}{μ_i}{\FunV(f_i)}{μ'_i}$.
    We have to show that $\dom(μ_1) ⊦ f_1(ρ(\px))(μ_1') \sim f_2(ρ(\px))(μ_2')$,
    the rest follows by congruence.
    From extensionality of $d$, we know by $\eqfun$ that
    $\dom(μ_1) ⊦ f_1(\pa)(μ_1') \sim f_2(\pa)(μ_2')$ for any $\pa$ such that
    $\pa ∈ \dom(μ_1)$.
    Since $ρ(\px) ∈ \dom(μ_1)$ by \Cref{thm:well-addressedness}, we can show the goal.

  \item \textbf{Case $\Case{\pe_s}{\Sel[r]}$}:
    Similar to the application case.

  \item \textbf{Case $\Let{\px}{\pe_1}{\pe_2}$}:
    For extensionality of $\semevt{\Let{\px}{\pe_1}{\pe_2}}_ρ$, note
    that the equivalence $μ_1 \approx μ_2$ can be extended to
    $μ_1[\pa ↦ \memo(\pa,d)] \approx μ_2[\pa ↦ \memo(\pa,d)]$ for any $d$ by
    reflexivity of $\sim$.
    Abbreviating
    $ρ' \triangleq ρ[\px↦\pa]$, the above holds for
    $d \triangleq \semevt{\pe_1}_{ρ'}$ in particular.
    Then the goal follows by the inductive hypothesis applied to
    extensionality of $\semevt{\pe_2}_{ρ'}$ and
    \Cref{thm:weaken-address-domain}.
\end{itemize}
\end{proof}

As before, we will assume that all elements $d$ are lazy and extensional from
now on.

\subsection{Compositionality}

Compositionality is an important property of a semantics (or, rather of a
congruence relation on the semantics).
Roughly, when a semantics is compositional, the denotation of an expression
is a function of the meaning of its subexpressions.
The ``is a function of'' qualifier is most often understood syntactically
in terms of a property of \emph{contexts}.

We follow \citet{MoranSands:99} and define \emph{evaluation contexts} $\pE$%
\footnote{In \Cref{defn:deadness2} we already used the term ``evaluation
context'' to denote small-step machine components $(ρ,μ,κ)$ rather than to
$\pE$. \citet{MoranSands:99} prove that both notions are equivalent, so this
overload is justified.}
with holes $\hole$ in order to formalise compositionality.
Then $\pE[\pe]$ denotes an expression where the hole in $\pE$ has been replaced
with $\pe$, expressing that the bigger expression $\pE[\pe]$ is a function of
$\pe$.
\[\begin{array}{rcl}
  \pE & ::=  & \hole \mid \pE~\px \mid \Let{\px}{\pe}{\pE} \mid \Let{\px}{\pE_1}{\pE_2[\px]} \mid \Case{\pE}{\Sel[r]} \\
\end{array}\]

\begin{definition}[Compositionality of an equivalence relation]
  Let $\cong$ be an equivalence relation on expressions.
  $\cong$ is \emph{compositional} if and only if for all $\pe_1,\pe_2$,
  $\pe_1 \cong \pe_2$ implies $\pE[\pe_1] \cong \pE[\pe_2]$ for all evaluation
  contexts $\pE$.
\end{definition}

It is well-known that \emph{contextual equivalence} is compositional:

\begin{definition}[Contextual equivalence, as in \citep{MoranSands:99}]
  Two expressions $\pe_1,\pe_2$ are \emph{contextually equivalent}, written
  $\pe_1 \equiv_{\mathit{ctx}} \pe_2$, if their termination behavior coincides
  in all contexts, \eg, for all closed evaluation contexts $\pE$ it is
  \[
    \len(\semevt{\pE[\pe_1]}_{[]}([])) = ω \Longleftrightarrow \len(\semevt{\pE[\pe_2]}_{[]}([])) = ω
  \]
  where $\len$ is the length function from \Cref{defn:length}.
\end{definition}

The definition of \citet{MoranSands:99} is in terms of a transition system like
$(\smallstep)$, but adequacy (\Cref{thm:semvan-adequate}) guarantees that the two
notions coincide.

\begin{lemma}[Contextual equivalence is compositional]
  Let $\pe_1, \pe_2$ be expressions such that
  $\pe_1 \equiv_{\mathit{ctx}} \pe_2$.
  Then for all evaluation contexts $\pE$, we have
  $\pE[\pe_1] \equiv_{\mathit{ctx}} \pE[\pe_2]$.
\end{lemma}
\begin{proof}
  Immediate, because $\pE[\pE_2[\hole]]$ is just another context in which
  contextually equivalent expressions are contextually equivalent.
\end{proof}

$\semevt{\wild}$ is compositional \wrt definitional equality $=$ on
$\EventD$:

\begin{lemma}[Definitional equivalence is compositional]
  Let $\pe_1, \pe_2$ be expressions such that
  for all environments $ρ$, $\semevt{\pe_1}_ρ = \semevt{\pe_2}_ρ$.
  Then for all evaluation contexts $\pE$ and environments $ρ$, we have
  $\semevt{\pE[\pe_1]}_{ρ} = \semevt{\pE[\pe_2]}_{ρ}$.
\end{lemma}
\begin{proofsketch}
  By induction on $\pE$.
\end{proofsketch}

Most interestingly, $\semevt{\wild}$ is also compositional \wrt to $(\equiv)$!

\begin{theoremrep}[Semantic equivalence is compositional]
  Let $\pe_1, \pe_2$ be expressions such that
  for all environments $ρ$, $\semevt{\pe_1}_ρ \equiv \semevt{\pe_2}_ρ$.
  Then for all evaluation contexts $\pE$ and environments $ρ$, we have
  $\semevt{\pE[\pe_1]}_{ρ} \equiv \semevt{\pE[\pe_2]}_{ρ}$.
\end{theoremrep}
\begin{proof}
  By induction on $\pE$.
  \begin{itemize}
    \item \textbf{Case $\hole$}: By assumption, $\semevt{\pe_1}_ρ \equiv \semevt{\pe_2}_ρ$.

    \item \textbf{Case $\pE~\px$}:
      It is $\fn{\wild}{\stuckend{}} \equiv \fn{\wild}{\stuckend{}}$ in case $\px \not∈ \dom(ρ)$.

      Otherwise, by induction it suffices to show that for $d_1 \equiv d_2$
      (where $d_i \triangleq \semevt{\pE[\pe_i]}_ρ$), we have
      \[
        d_1 \betastep \fn{(\FunV(f))}{f(ρ(\px))} \equiv d_2 \betastep \fn{(\FunV(f))}{f(ρ(\px))}.
      \]
      Now fix particular $μ_1 \approx μ_2$ such that $\dom(μ_1) ⊦ \semevt{\pE[\pe_i]~\px}_ρ$.
      The interesting case is when
      $\bigstep{d_i}{μ_i}{\FunV(f_i)}{μ'_i}$.
      We have to show that $\dom(μ_1) ⊦ f_1(ρ(\px))(μ_1') \sim f_2(ρ(\px))(μ_2')$, the
      rest follows by congruence.
      From $d_1 \equiv d_2$, we know that $\dom(μ_1) ⊦ f_1(d)(μ_1') \sim f_2(d)(μ_2')$ for any lazy $d$ such that $\dom(μ_1) ⊦ d$.
      Since $\dom(μ_1) ⊦ ρ(\px)$ by \Cref{thm:well-addressedness} we can show the goal.

    \item \textbf{Case $\Case{\pE}{\Sel[r]}$}:
      Similar to the application case.

    \item \textbf{Case $\Let{\px}{\pe}{\pE}$}:
      Note that the equivalence $μ_1 \approx μ_2$ can be extended to
      $μ_1[\pa ↦ d] \approx μ_2[\pa ↦ d]$ for any $d$ by reflexivity of
      $\equiv$, and for $d \triangleq \semevt{\pe}_{ρ'}$ in particular.
      Then the goal follows by the inductive hypothesis applied to
      $\semevt{\pE[\pe_1]}_{ρ'} \equiv \semevt{\pE[\pe_2]}_{ρ'}$
      and \Cref{thm:weaken-address-domain}.

    \item \textbf{Case $\Let{\px}{\pE_1}{\pE_2[\px]}$}:
      Fix a particular $μ_1 \approx μ_2$.
      Abbreviating $d_{\pe_i} \triangleq \semevt{\pE_1[\pe_1]}_{ρ'}$, the induction hypothesis gives us
      $d_{\pe_1} \equiv d_{\pe_2}$,
      so $μ_1[\pa ↦ d_{\pe_1}] \approx μ_2[\pa ↦ d_{\pe_2}]$
      and we get
      \[
        \dom(μ_1) ∪ \{ \pa \} ⊦ \semevt{\pE_2[\px]}_{ρ'}(μ_1[\pa ↦ d_{\pe_1}]) \sim \semevt{\pE_2[\px]}_{ρ'}(μ_2[\pa ↦ d_{\pe_2}])
      \]
      by extensionality of $\semevt{\pE_2[\px]}_{ρ'}$, which we can weaken to
      show the goal.
  \end{itemize}
\end{proof}

And finally, we can use this result to prove that $(\equiv)$ is indeed coarser than contextual equivalence:

\begin{corollaryrep}[Semantic equivalence implies contextual equivalence]
  Let $\pe_1, \pe_2$ be expressions such that $\semevt{\pe_1}_ρ \equiv \semevt{\pe_2}_ρ$.
  Then for all closed evaluation contexts $\pE$, we have
  \[
    \len(\semevt{\pE[\pe_1]}_{[]}([])) = ω \Leftrightarrow \len(\semevt{\pE[\pe_1]}_{[]}([])) = ω.
  \]
\end{corollaryrep}
\begin{proof}
  It suffices to show that for all $d_1,d_2$ with $\varnothing ⊦ d_1,d_2$,
  if $d_1 \equiv d_2$ then $\len(d_1([])) = ω \Leftrightarrow \len(d_2([])) = ω$.

  Since $[] \approx []$, we have $\varnothing ⊦ d_1([]) \sim d_2([])$.
  Then either trace diverges if and only if the other does, if and only if the
  length of the trace is $ω$.
\end{proof}

This corollary naturally leads to the converse question:
Does contextual equivalence imply semantic equivalence?
If that were the case, then $\semevt{\wild}$ would be \emph{fully abstract} via
$(\equiv)$~\citep{Plotkin:77}.
\citeauthor{Plotkin:77} established that a definition like $\semscott{\wild}$ is
not fully abstract via definitional equality on $\ScottD$.
Guarded types prevent probing for diverging computations without executing them,
but it is still possible for lazy elements to probe traces for stuckness without
propagating it or evaluating multiple traces concurrently, so we have reason to
doubt that $\semevt{\wild}$ is fully abstract.
We do not intend to find a final answer in this work.

\section{Abstract Interpretation}
\label{sec:abstractions}

\subsection{Lazy Denotational Deadness}

Let us first try to reformulate semantic deadness in terms of $\semevt{\wild}$:

\begin{definition}[Denotational deadness, lazily and unusable]
  \label{defn:deadness3}
  A variable $\px$ is \emph{dead} in an expression $\pe$ if and only
  if, for all $ρ ∈ \Var \to \EventD, d_1, d_2 ∈ \EventD, μ ∈ \Heaps$ and $\pa ∈ \Addresses$ such that $ρ(\px) = \deref(\pa)$, we have
  \[\semevt{\pe}_{ρ}(μ[\pa ↦ d_1]) = \semevt{\pe}_{ρ}(μ[\pa ↦ d_2]).\]
  Otherwise, $\px$ is \emph{live}.
\end{definition}

Alas, this definition is vacuously true!
We can observe distinct $d_1,d_2$ in the entry $μ(\pa)$ listed in the final heap
$μ$ of a trace ending in $\goodend{v,μ}$.
Furthermore, there is no guarantee that some other entry in $ρ$ besides $ρ(\px)$
might dereference $\pa$.

It is reasonable to \emph{restrict} what we can observe then in order to get
a coarser equivalence relation than definitional equality on traces.
For example, we could compare traces modulo the kernel of the length function
such as defined in \Cref{thm:abs-length}.

Alas, the equivalence relation induced in this way would be \emph{too} coarse:
Consider $(\Lam{x}{y})$; this expression would be dead in $y$ according to that
definition because all its maximal traces will have length 1.
But clearly we should not be able to rewrite $\Let{y}{\pe}{\Lam{x}{y}}$ to
$\Let{y}{\mathit{crash}}{\Lam{x}{y}}$, the latter of which would crash when
applied to an argument.
So our equivalence relation must be finer on values, just like observational
equivalence in operational semantics.

The key is to introduce an equivalence relation on heaps with a ``blind spot''
concerning the contents of address $\pa$:%
\footnote{This is nearly the correctness predicate $\sim_V$ of
\citet[Theorem 2.21]{Nielson:99}.
The only conceptual difference is that $V$ tracks the set of \emph{live} variables,
whereas we track a single address assumed to be \emph{dead}.}
\[
 \inferrule*
    {\forall \pa' ∈ \Var \setminus \{ \pa \}.\ \Longrightarrow μ_1(\pa') \overset{\pa}{\sim}_{\EventD} μ_2(\pa') }
    {μ_1 \overset{\pa}{\sim}_{\Heaps} μ_2}
\]
Now we lift $\overset{\pa}{\sim}_{\EventD}$ congruently over $\Traces{}$,
$\Events$, $\EventD$ and $\EventV$ which is entirely mechanical.
For $\Events$ we only give the rule concerning the $\AppEA$ event as an example:%
\footnote{These definitions are to be understood as coinductive, although we omit distracting $\later$.}
\[\begin{array}{cc}
 \begin{array}{c}
 \inferrule*
    {ε_1 \overset{\pa}{\sim}_{\Events} ε_2 \quad τ_1 \overset{\pa}{\sim}_{\Traces{}} τ_2}
    {ε \cons τ_1 \overset{\pa}{\sim}_{\Traces{}} ε \cons τ_2}
 \qquad
 \inferrule*
    {\quad}
    {\stuckend{} \overset{\pa}{\sim}_{\Traces{}} \stuckend{}}
 \qquad
 \inferrule*
    {v_1 \overset{\pa}{\sim}_{\EventV} v_2 \quad μ_1 \overset{\pa}{\sim}_{\Heaps} μ_2}
    {\goodend{v_1,μ_1} \overset{\pa}{\sim}_{\Traces{}} \goodend{v_2,μ_2}}
 \\
 \\[-0.5em]
 \inferrule*[right=\ldots]
    {d_1 \overset{\pa}{\sim}_{\EventD} d_2}
    {\AppEA(\px ↦ d_1) \overset{\pa}{\sim}_{\Events} \AppEA(\px ↦ d_2)}
 \qquad
 \inferrule*
    {\forall μ_1, μ_2.\ μ_1 \overset{\pa}{\sim}_{\Heaps} μ_2 \Longrightarrow d_1(μ_1) \overset{\pa}{\sim}_{\Traces{}} d_2(μ_2) }
    {d_1 \overset{\pa}{\sim}_{\EventD} d_2}
 \\
 \\[-0.5em]
 \inferrule*
    {\many{d_1 \overset{\pa}{\sim}_{\EventD} d_2} }
    {\ConV(K, \many{d_1}) \overset{\pa}{\sim}_{\EventV} \ConV(K, \many{d_2})}
 \qquad
 \inferrule*
    {\forall d_1, d_2.\ d_1 \overset{\pa}{\sim}_{\EventD} d_2 \Longrightarrow f_1(d_1) \overset{\pa}{\sim}_{\EventD} f_2(d_2) }
    {\FunV(f_1) \overset{\pa}{\sim}_{\EventV} \FunV(f_2)}
 \end{array}
\end{array}\]
The congruence rule for $\overset{\pa}{\sim}_{\EventD}$ is particularly interesting:
It expresses both an \emph{assumption} about the kind of heaps $μ_i$ passed
to $d_i$ as well as a \emph{requirement} on the trace resulting from applying
said heaps to $d_i$.
We can now formulate a useful notion of deadness in terms of this equivalence
relation:
\begin{definition}[Denotational deadness, lazily and useful]
  \label{defn:deadness4}
  An address $\pa$ is \emph{dead} in an element $d ∈ \EventD$
  if and only if $d \overset{\pa}{\sim}_{\EventD} d$.
  Otherwise, $\pa$ is \emph{live}.

  Furthermore, a variable $\px$ is \emph{dead} in an expression $\pe$
  if and only if, for all $ρ ∈ \Var \to \EventD$ and addresses $\pa$ such that $\pa$ is dead in $\rng(ρ)$,
  $\pa$ is dead in $\semevt{\pe}_{ρ[\px ↦ \deref(\pa)]}$.
  Otherwise, $\px$ is \emph{live}.
\end{definition}
If $x$ is dead in $\pe_2$ according to this definition, then we can justify the
following rewrite if we choose not to observe heaps:
\[\arraycolsep=3pt
\begin{array}{lclcl}
\semevt{\Let{x}{\pe_1}{\pe_2}}_ρ(μ)
& = & \semevt{\pe_2}_{ρ[x↦\deref(\pa)]}(μ[\pa↦d])
\\
& \overset{\pa}{\sim}_{\Traces{}} & \semevt{\pe_2}_{ρ[x↦\deref(\pa)]}(μ[\pa↦\fn{\wild}{\stuckend{}}])
& = & \semevt{\Let{x}{\mathit{crash}}{\pe_2}}_ρ(μ)
\end{array}
\]
A syntactic occurrence analysis could subsequently figure out whether the binding for $x$
can be dropped without introducing scoping errors.

We can now prove \Cref{thm:semusg-correct-live} in terms of this new
characterisation of deadness by induction:

\begin{theorem}[$\semusg{\wild}$ is a correct deadness analysis]
  \label{thm:semusg-correct-live-3}
  Let $\pe$ be an expression, $\px$ a variable and $\tr$ a usage environment.
  If $\tr(\px) \not⊑ \semusg{\pe}_{\tr}$
  then $\px$ is dead in $\pe$.
\end{theorem}
\begin{proof}
  By induction over $\pe$. The full proof can be found in
  \Cref{prf:semusg-correct-live-3}.
\end{proof}

So that is nice.
Unfortunately, we have made no progress towards

\begin{figure}
\[\begin{array}{c}
  \forall \tr.\ \tr(x) ⊑ \semusg{\pe}_{\tr} \\
  \Uparrow \\
  \exists \tr,\td_1,\td_2.\ \semusg{\pe}_{\tr[x ↦ \td_1]} \not= \semusg{\pe}_{\tr[x ↦ \td_2]} \\
  \Uparrow \text{ This one is non-trivial, but we can take any $α(d_1),α(d_2)$ and add $[x_1↦1],[x_2↦1]$ for some "fresh" vars}\\
  \exists ρ,d_1,d_2,ι_1,ι_2.\ \semevt{\pe}_{ρ[x ↦ d_1]}(ι_1) \not= \semevt{\pe}_{ρ[x ↦ d_2]}(ι_2) \\
  \Uparrow \\
  \exists \pe_1,\pe_2.\ \semevt{\Let{x}{\pe_1}{\pe}} \not= \semevt{\Let{x}{\pe_2}{\pe}} \\
  \Updownarrow \\
  x \text{ live in } \pe \\
\end{array}\]

\[\begin{array}{c}
  \exists \tr.\ \tr(x) \not⊑ \semusg{\pe}_{\tr} \\
  \Downarrow \\
  \forall \tr,\td_1,\td_2.\ \semusg{\pe}_{\tr[x ↦ \td_1]} = \semusg{\pe}_{\tr[x ↦ \td_2]} \\
  \Downarrow \text{ This one is non-trivial, but we can take any $α(d_1),α(d_2)$ and add $[x_1↦1],[x_2↦1]$ for some "fresh" vars}\\
  \forall ρ,d_1,d_2,ι_1,ι_2.\ \semevt{\pe}_{ρ[x ↦ d_1]}(ι_1) = \semevt{\pe}_{ρ[x ↦ d_2]}(ι_2) \\
  \Downarrow \\
  \forall \pe_1,\pe_2.\ \semevt{\Let{x}{\pe_1}{\pe}} = \semevt{\Let{x}{\pe_2}{\pe}} \\
  \Updownarrow \\
  x \text{ dead in } \pe \\
\end{array}\]

\[\begin{array}{c}
 \begin{array}{rcl}
  \multicolumn{3}{c}{ \ruleform{ defn : \Traces \to (\Addresses \pfun \Var) \qquad blub : \Traces \to (\Var \to \UsgD) \to (\Addresses \to \UsgD) } } \\
  \\[-0.5em]
  defn_{\Traces}(a \cons τ) & = & defn(a) \uplus defn(τ) \\
  defn_{\Events}(\BindA(\px,\pa↦\pe,d)) & = & [\pa↦\px] \\
  blub(τ)(\tr)(\pa) & = & \tr(defn_{\Traces}(τ)(\pa)) \text{ or $\bot$} \\
  \\[-0.5em]
  \multicolumn{3}{c}{ \ruleform{ usg : (\Addresses \to \UsgD) \to \Traces \to \UsgD \times \Values{\UsgD} } } \\
  \\[-0.5em]
  usg_{\Traces}(\tm)(\goodend{\FunV(f)}) & = & (\fn{\wild}{0}, α_{\Values{}}(\FunV(f))) \\
  usg_{\Traces}(\tm)(\stuckend{}) & = & (\fn{\wild}{0}, \bot_{\Values{\UsgD}}) \\
  usg_{\Traces}(\tm)(\pe \act{a} τ) & = & usg_{\Events}(\tm)(a) +_1 usg_{\Traces}(\tm)(τ) \\
  usg_{\Events}(\tm)(a) & = & \begin{cases}
      \tm(\pa) & a = \LookupT(\pa) \\
      0 & \text{otherwise}
    \end{cases} \\
  usg^{-1}_{\Traces}(\tm)(\goodend{(\lbl, \FunV(f))}) & = & (0, α_{\Values{}}(\FunV(f))) \\
  usg_{\EventD}(\tm)(d) & = & usg_{\Traces} \circ d \circ usg^{-1}_{\Traces} \\
  usg^{-1}_{\EventD}(\tm)(d) & = & usg^{-1}_{\Traces} \circ d \circ usg_{\Traces} \\
  usg_{\Values{}}(\tm)(\FunV(f)) & = & \FunV(usg_{\EventD} \circ f \circ usg^{-1}_{\EventD}) \\
  \\[-0.5em]
  \multicolumn{3}{c}{ \ruleform{ intra : \UsgD \times \Values{\UsgD} \to \UsgD \times \Values{\UsgD} } } \\
  \\[-0.5em]
  intra(d, \FunV(f)) & = & (d + ω*f(\bot), \FunV(\fn{d}{ω*d})) \\
  \\[-0.5em]
  \multicolumn{3}{c}{ \ruleform{ α : \Traces \to \UsgD } } \\
  \\[-0.5em]
  α(τ) & = & fst(intra(usg_\Traces(blub(τ)(\fn{\px}{[\px↦1]}))(τ))) \\
 \end{array}
\end{array}\]
\end{figure}

\begin{itemize}
  \item Correctness predicate simpler to come up than for liveness analysis directly. Later
\end{itemize}

\section{Related Work}
\label{sec:related-work}

%\sg{Move to related work.}
%There have been attempts to discern crashes from other kinds of loops, such as
%\cite{imprecise-exceptions}. Unfortunately, in Section 5.3 they find it
%impossible give non-terminating programs a denotation other than $\bot$, which
%still encompasses all possible exception behaviors.
%
% eval/apply or push/enter?
% Given an expr like $f x$, we first push $ρ(x)$ onto the stack and then
% evaluate $f$, which will look it up (pushing an udpate frame) and evaluate its
% RHS. Since we will never return to the "eval site" of $f$, IMO this qualifies
% as push/enter rather than eval/apply. Which is in contrast to what the Krivine
% paper says, which dubs return states as "apply" transitions


\begin{acks}
We would like to thank the anonymous POPL reviewers for their feedback.
%, as well as Bohdan Liesnikov and Sebastian Ullrich.
\end{acks}

\clearpage
\bibliography{references}

\fi % \ifmain

\end{document}

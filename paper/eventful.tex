\section{Eventful Semantics}
\label{sec:eventful}

In the previous section, we gave a trace semantics and proved it adequate \wrt
the LK transition semantics.
With compositionality and structural induction we recover strong advantages of
denotational semantics.

It is striking that although $\semst{\wild}$ is as expressive as the LK
transition system, the information encoded in the states of the generated traces
is not enough to recover the LK transition system in the sense of
\citet[Chapter 43]{Cousot:21}.
For example, every return state $((\lbl,v),\tm)$ admits at least two possible
transitions:
Either a heap update or a $β$-reduction.
In the LK transition system, the particular transition is governed by the
continuation stack, of which $\semst{\wild}$ maintains no reification in its states.
It is however easily possible to ``elaborate'' $\semst{\wild}$ to include the
proper stack manipulations and emit the current lexical environment as part of
every state.
We think it is rather uninteresting to give the closed, elaborated definition
of $\semst{\wild}$ which necessarily gives up simplicity \sg{Give it in the Appendix}.
It is more interesting to see what other parts of the traces we can \emph{omit}
before compromising on expressivity and retain just as much information as
needed by the particular analysis we want to prove correct.

For example, in Control-Flow Analysis~\citep{Shivers:91}, we are ultimately
interested in observing the different lambda labels a variable occurrence might
evaluate to and thus construct a conservative but useful approximation of
interprocedural control-flow.
We do not need the lexical environment or the continuation stack, but the
call-by-need nature of our semantics needs the state in $\betastep$.
That makes $\semst{\wild}$ a suitable concrete semantics for abstraction.

Yet for our usage analysis in \Cref{sec:problem}, we do not care so much
about the \emph{state} in which variable lookup happens, but rather about at
which \emph{address} the $\LookupT$ transition happened, as well as the $\BindT$
transition that tells us what $\mathbf{let}$ binding that particular address
refers to.

In this section, we will omit states in favor of tracing the instantiations of
transition rules, calling those instantiations \emph{events}.
The resulting \emph{eventful trace semantics} $\semevt{\wild}$ is (closer to)
the preferred framework of \citet{Cousot:21}.%
\footnote{Cousot's \emph{stateless} semantics even goes one step further
and stops passing around a materialised mapping for the heap, instead
rematerialising the heap as needed from the history of previous events in
an effort to model weak memory models.}
In essence, we shift attention from the \emph{nodes} of the control-flow graph
to the \emph{edges} modelling the transitions.

\begin{figure}
\[\begin{array}{c}
 \arraycolsep=3pt
 \begin{array}{rrclcl}
  \text{Event} & ε & ∈ & \Events & ::=  & \BindA(\px,\pa↦d) \mid \LookupA(\pa) \mid \UpdateA(\pa↦d) \\
               &   &   &          & \mid & \AppIA(d) \mid \AppEA(\px↦d) \mid \CaseIA(d) \mid \CaseEA(K,\many{\px↦d}) \\
 \end{array} \\
 \\[-0.5em]
 \arraycolsep=3pt
 \begin{array}{rrclcl@{\quad}rrclcl}
  \text{Heap}         & μ   & ∈ & \Heaps         & =      & \Addresses \pfun \later\EventD
  &
  \text{Delayed event}        & ε^{\later} & ∈ & \later \Events         &     &
  \\
  \text{Program trace}        & τ          & ∈ & \Traces        & ::= & \goodend{v,μ} \mid \stuckend{} \mid ε^{\later} \cons τ^{\later}
  &
  \text{Delayed trace}         & τ^{\later} & ∈ & \later \Traces &     &
  \\
  \text{Eventful domain}      & d          & ∈ & \EventD                   & =   & \Heaps \to \Traces
  &
  \text{Delayed element}       & d^{\later} & ∈ & \later \EventD            &     &
 \end{array} \\
 \\[-0.5em]
 \arraycolsep=3pt
 \begin{array}{rrclcl}
  \text{Eventful value} & v & ∈ & \EventV & ::= & \FunV(f ∈ \later\EventD \to \EventD) \mid \ConV(K,\many{d^\later}^{α_K}) \\
 \end{array} \\
 \\[-0.5em]
 \begin{array}{rcl}
  \multicolumn{3}{c}{ \ruleform{
    \begin{array}{c}
      (\betastep) : \EventD \to (\EventV \pfun \EventD) \to \EventD \quad \ret : \EventV \to \EventD \quad \apply : \EventD \to \EventD \to \EventD \\
      \deref : \Addresses \to \EventD \quad \select : \EventD \to ((K:\Con) \to (\later\EventD)^{α_K} \pfun \EventD) \to \EventD \\
    \end{array}
  }} \\
  \\[-0.5em]
  (d \betastep f)(μ) & = & \begin{cases}
    \many{ε \cons {}} f(v)(μ') & \text{$d(μ) = \many{ε \cons {}} \goodend{v,μ'}$ and $v ∈ \dom(f)$} \\
    \many{ε \cons {}} \stuckend{} & \text{$d(μ) = \many{ε \cons {}} \goodend{v,μ'}$ and $v \not∈ \dom(f)$} \\
    d(μ) & \text{otherwise} \\
  \end{cases} \\
  \\[-0.5em]
  \ret(v)(μ) & = & \goodend{v,μ} \\
  \deref(\pa)(μ)   & = & \LookupA(\pa) \cons (μ(\pa) \betastep \fn{v~μ'}{\UpdateA(\pa↦\ret(v)) \cons \goodend{v,μ'[\pa↦\ret(v)]}})(μ) \\
  \apply(d_\pe,d_\px) & = & d_\pe \betastep \fn{(\FunV(f))}{f(d_\px)} \\
  \select(d,\alts) & = & d \betastep \fn{(\ConV(K_s,\many{d_s^\later}))}{\alts(K_s, \many{d_s^\later})} \quad \text{where } (K_s, \many{d_s^\later}) ∈ \dom(\alts) \\
 \end{array} \\
 \\[-0.5em]
 \begin{array}{rcl}
  \multicolumn{3}{c}{ \ruleform{ \semevt{\wild} \colon \Exp → (\Var \pfun \EventD) → \EventD } } \\
  \\[-0.5em]
  \semevt{\px}_ρ       & = & \begin{cases}
    ρ(\px) & \px ∈ \dom(ρ) \\
    \fn{\wild}{\stuckend{}}  & \text{otherwise} \\
  \end{cases} \\
  \\[-0.5em]
  \semevt{\Lam{\px}{\pe}}_ρ & = & \ret(\FunV(\fn{d^\later}{\AppEA(\px↦d^\later) \cons \semevt{\pe}_{ρ[\px↦d^\later]}})) \\
  \\[-0.5em]
  \semevt{\pe~\px}_ρ   & = & \begin{cases}
    \AppIA(ρ(x)) \cons \apply(\semevt{\pe}_ρ, ρ(\px)) & \px ∈ \dom(ρ) \\
    \fn{\wild}{\stuckend{}}  & \text{otherwise} \\
  \end{cases} \\
  \\[-0.5em]
  \semevt{\Let{\px}{\pe_1}{\pe_2}}_ρ(μ) & = &
    \begin{letarray}
      \text{let} & ρ' = ρ[\px ↦ \deref(\pa)] \quad \text{where $\pa \not∈ \dom(μ)$} \\
                 & d_1^\later = \semevt{\pe_1}_{ρ'} \\
      \text{in}  & \BindA(\px,\pa↦d_1^\later) \cons \semevt{\pe_2}_{ρ'}(μ[\pa ↦ d_1^\later])
    \end{letarray} \\
  \\[-0.5em]
  \semevt{K~\many{\px}}_ρ & = & \ret(\ConV(K,\many{\semevt{\px}_ρ})) \\
  \\[-0.5em]
  \semevt{\Case{\pe_s}{\Sel[r]}}_ρ & = &
    \begin{letarray}
      \text{let} & \alts = \fn{K}{\fn{\many{d^\later}}{\CaseEA(K,\many{\px↦d^\later}) \cons \semevt{\pe_r}_{ρ[\many{\px↦d^\later}]}}} \\
      \text{in} & \CaseIA(\semevt{\pe_s}_ρ) \cons \select(\semevt{\pe_s}_ρ, \alts)  \\
    \end{letarray}
 \end{array} \\
\end{array}\]
\caption{Eventful Trace Semantics}
  \label{fig:semevt}
\end{figure}

\subsection{Reading Between the Nodes}

\Cref{fig:semevt} gives the definition for the eventful trace semantics
$\semevt{\wild}$.
The domain of eventful maximal traces $\EventD$ maps a heap
to a possibly infinite or stuck program trace $τ$, just as the stateful
semantics $\semst{\wild}$.

The fundamental difference to the stateful semantics manifests in the
information encoded in a trace $τ$:
Instead of a sequence of states $σ$ there is a (possibly empty, if stuck)
sequence of \emph{events} $ε$.
With the exception of $\ValueA$ events which we will explain in due course,
each kind of event corresponds to a transition rule in the LK semantics.
In fact, each event carries with it part of the ``closure'' of the
particular transition rule which is readily available for analysis.
For example, the $\LookupA$ event carries the heap address which is accessed,
$\BindA$ events carry the $\mathsf{let}$-bound variable, its fresh address and
the denotation of the right-hand side.

We will occassionally write $ι$ to denote a finite list of events contained in
a finite program trace such as in $\betastep$.
We re-use the same greek letter as in \Cref{sec:stateful} and disambiguate as
needed, as before using tilde $\tilde{τ}$.

\begin{figure}
\[\begin{array}{rcl}
  α_\Environments(ρ) & = & \deref \circ ρ \\
  α_\Heaps([\many{\pa ↦ (ρ,\pe)}]) & = & [\many{\UpdateA(\pa ↦ \semst{\pe}_{α_\Environments(ρ)}}] \\
  α_\EventV(ρ,\Lam{\px}{\pe}) & = & \FunV(\fn{d}{\idiom{\AppEA(\px↦d)} \cons \idiom{\semevt{\pe}_{α_\Environments(ρ)[\px↦d]}}}) \\
  α_\EventV(ρ,K~\many{\px}) & = & \ConV(K,\idiom{\many{\semevt{\px}_{α_\Environments(ρ)}}}) \\
  α_\Events(σ) & = & \begin{cases}
    \BindA(\px,\pa↦\semevt{\pe_1}_{α_\Environments(ρ[\px↦\pa])}) & σ = (\Let{\px}{\pe_1}{\wild},ρ,μ,\wild) \wedge \pa \not∈ \dom(μ) \\
    \AppIA(\deref(\pa)) & σ = (\wild~\px,ρ,\wild,\ApplyF(\pa) \pushF \wild) \\
    \CaseIA(\semevt{\pe_s}_{α_\Environments(ρ)}) & σ = (\Case{\pe_s}{\wild},ρ,\wild, \wild)\\
    \LookupA(ρ(\px)) & σ = (\px,ρ,\wild,\UpdateF(\pa) \pushF \wild) \\
    \AppEA(\px↦\deref(\pa)) & σ = (\Lam{\px}{\wild},\wild,\wild,\ApplyF(\pa) \pushF \wild) \\
    \CaseEA(K',\many{\px_i ↦ \deref(ρ(\py))}) & σ = (K'~\many{\py}, ρ, \wild, \SelF(\wild,\Sel) \pushF \wild) \wedge K' = K_i \\
    \UpdateA(\pa↦\ret(α_\EventV(ρ,\pv))) & σ = (\pv,ρ,\wild,\UpdateF(\pa) \pushF \wild) \\
  \end{cases} \\
  α_{\Traces}((σ_i)_{i∈\overline{n}},κ) & = & \begin{cases}
    \idiom{α_\Events(σ_1)} \cons \idiom{α_{\Traces}((σ_{i+1})_{i∈\overline{n-1}},κ)} & n > 1 \\
    \goodend{α_\EventV(ρ,\pv)} & n = 1 \wedge σ_1 = (\pv, ρ, \wild, κ) \\
    \stuckend{} & \text{otherwise} \\
  \end{cases} \\
  \mathcal{C} & = & ∀(σ_i)_{i∈\overline{n}}.\ \maxtrace{(σ_i)_{i∈\overline{n}}} \Longrightarrow ∀((\pe,ρ,μ,κ) = σ_1).\ α_{\Traces}((σ_i)_{i∈\overline{n}},κ) = \semst{\pe}_{α_\Environments(ρ)}(α_\Heaps(μ)) \\
\end{array}\]
\caption{Correctness predicate for $\semevt{\wild}$}
  \label{fig:semevt-correctness}
\end{figure}

\sg{Justify why we can't abstract $\semst{\wild}$ (where to take the address in the lookup case from?).
Although, perhaps we could; by zipping up the original heaps and stacks. But a direct proof is simpler.}

Similarly to before, we can define preservation lemmas:

\begin{lemma}[Preservation of length]
  \label{thm:abs-length-less}
  Let us define the length $\mathit{len} : \Traces \to ℕ_ω$ of a trace by
  guarded recursion
  \[
    \mathit{len}(τ) = \begin{cases}
      1 + \idiom{\mathit{len}(τ^{\later})} & τ = a^\later \cons τ^{\later} \\
      1 & \text{otherwise} \\
    \end{cases}
  \]
  Now let $(σ_i)_{i∈\overline{n}}$ be an arbitrary trace.
  Then $\mathit{len}(α_\STraces((σ_i)_{i∈\overline{n}},\cont(σ_1))) = n$.
\end{lemma}
\begin{proof}
  Similar to \Cref{thm:abs-length}.
\end{proof}

\begin{lemma}[Preservation of components]
  \label{thm:abs-components}
  Let $(σ_i)_{i∈\overline{n}}$ be a trace and let $τ = α_\Traces((σ_i)_{i∈\overline{n}},\cont(σ_1))$.
  Then for all $j∈\overline{n-1}$, $τ_j = α_\Events(σ_j)$
  (where $τ_j$ denotes the $j$th event in $τ$)
  and if $τ$ ends in $\goodend{v}$, then $v = α_\EventV(ρ,\pv)$, where $σ_n = (\pv, ρ, \wild, \wild)$.
\end{lemma}
\begin{proof}
  Similar to \Cref{thm:abs-states}.
\end{proof}

\begin{lemma}[Preservation of characteristic]
  \label{thm:abs-max-trace-less}
  Let $(σ_i)_{i∈\overline{n}}$ be a maximal trace.
  Then $α_\Traces((σ_i)_{i∈\overline{n}}, cont(σ_1))$ is ...
  \begin{itemize}
    \item ... infinite if and only if $(σ_i)_{i∈\overline{n}})$ is diverging
    \item ... ending with $\goodend{\wild}$ if and only if $(σ_i)_{i∈\overline{n}}$ is balanced
    \item ... ending with $\stuckend{\wild}$ if and only if $(σ_i)_{i∈\overline{n}}$ is stuck
  \end{itemize}
\end{lemma}
\begin{proof}
  Similar to \Cref{thm:abs-max-trace}.
\end{proof}

\begin{theorem}[Correctness of $\semevt{\wild}$]
  \label{thm:semevt-correct}
  $\mathcal{C}$ from \Cref{fig:semevt-correctness} holds.
  That is, whenever $(σ_i)_{i∈\overline{n}}$ is a maximal LK trace with source
  state $(\pe,ρ,μ,κ)$, we have
  $α_{\Traces}((σ_i)_{i∈\overline{n}},κ) = \semst{\pe}_{α_\Environments(ρ)}(α_\Heaps(μ))$.
\end{theorem}
\begin{proof}
  Similar to \Cref{thm:semst-correct}. \sg{We probably want to carry out the proof explicitly.}
\end{proof}

Again, this allows us to prove a strong version of adequacy:

\begin{lemma}[Adequacy of $\semevt{\wild}$]
  \label{thm:semevt-adequate}
  Let $τ = \semevt{\pe}_{[]}([])$.
  \begin{itemize}
    \item
      $τ$ ends with $\goodend{\wild}$ (is balanced) iff there exists a final
      state $σ$ such that $\inj(\pe) \smallstep^* σ$.
    \item
      $τ$ ends with $\stuckend{\wild}$ (is stuck) iff there exists a non-final
      state $σ$ such that $\inj(\pe) \smallstep^* σ$ and there exists no $σ'$
      such that $σ \smallstep σ'$.
    \item
      $τ$ is infinite (is diverging) iff for all $σ$ with $\inj(\pe)
      \smallstep^* σ$ there exists $σ'$ with $σ \smallstep σ'$.
  \end{itemize}
\end{lemma}
\begin{proof}
  Same as for \Cref{thm:semst-adequate}.
\end{proof}

It should be clear that $α_\Events$ could elaborate the returned events with
more information from the machine states without a noticable effect on the
correctness proof.
For example it could interleave the list of events with program labels.
The only drawback of doing so would be that $\semevt{\wild}$ becomes more
complicated.
The same applies to $α_\States$ from \Cref{sec:stateful} and $\semst{\wild}$, of
course, where this increase in complexity can be witnessed by having a look at
the generalisation in the Appendix \sg{Write that}.

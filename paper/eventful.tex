\section{Eventful Semantics}
\label{sec:eventful}

In the previous section, we gave a trace-based call-by-need semantics and proved
it adequate \wrt the LK transition semantics.
With compositionality and structural induction we recover strong advantages of
denotational semantics.

It seems that \Cref{thm:semvan-adequate} is at odds with the fact that
the information encoded in the generated traces
is by far not enough to recover the LK transition system in the sense of
\citet[Chapter 43]{Cousot:21}.
It is however easily possible to ``elaborate'' $\semvan{\wild}$ to include the
proper intermediate state.
We think it is rather uninteresting to give the closed, elaborated definition of
$\semvan{\wild}$.
For one, such a definition would necessarily give up on simplicity.
Furthermore, for our usage analysis in \Cref{sec:problem}, we do not care so
much about the \emph{state} in which variable lookup happens, but rather about
at which \emph{address} the $\LookupT$ transition happened.

In this section, we will embellish the generated traces
to track the instantiations of the transition rules taken, calling those
instantiations \emph{events}.
The resulting \emph{eventful trace semantics} $\semevt{\wild}$ is (closer to)
the preferred framework of \citet{Cousot:21}.%
\footnote{Cousot's \emph{stateless} semantics goes one step further
and rematerialises the heap as needed from the history of the trace.}
%In essence, we shift attention from the \emph{nodes} of the control-flow graph
%to the \emph{edges} modelling the transitions.

\begin{figure}
\[\begin{array}{c}
 \arraycolsep=3pt
 \begin{array}{rrclcl}
  \text{Event} & ε & ∈ & \Events & ::=  & \BindE(\px,\pa↦d) \mid \LookupE(\pa) \mid \UpdateE(\pa↦v) \\
               &   &   &          & \mid & \AppIE(\pa) \mid \AppEE(\px↦\pa) \mid \CaseIE(d) \mid \CaseEE(K,\many{\px↦\pa}) \\
 \end{array} \\
 \\[-1em]
 \arraycolsep=3pt
 \begin{array}{rrclcl@{\quad}rrclcl}
  \text{Heap}         & μ   & ∈ & \Heaps         & =      & \Addresses \pfun \later\EventD
  &
  \text{Delayed event}        & ε^{\later} & ∈ & \later \Events         &     &
  \\
  \text{Program trace}        & τ          & ∈ & \Traces        & ::= & \goodend{v,μ} \mid \stuckend{} \mid ε^{\later} \cons τ^{\later}
  &
  \text{Delayed trace}         & τ^{\later} & ∈ & \later \Traces &     &
  \\
  \text{Eventful domain}      & d          & ∈ & \EventD                   & =   & \Heaps \to \Traces
  &
  \text{Delayed element}       & d^{\later} & ∈ & \later \EventD            &     &
 \end{array} \\
 \\[-1em]
 \arraycolsep=3pt
 \begin{array}{rrclcl}
  \text{Eventful value} & v & ∈ & \EventV & ::= & \FunV(f ∈ \Addresses \to \EventD) \mid \ConV(K,\many{\pa}^{α_K}) \\
 \end{array} \\
 \\[-0.75em]
 \begin{array}{rcl}
  \multicolumn{3}{c}{ \ruleform{
    \begin{array}{c}
      (\betastep) : \EventD \to (\EventV \pfun \EventD) \to \EventD \quad \ret : \EventV \to \EventD \quad \apply : \EventD \times \Addresses \to \EventD \\
      \memo : \Addresses \times \EventD \to \EventD \quad \select : \EventD \times ((K:\Con) \times \pa^{α_K} \pfun \EventD) \to \EventD \\
    \end{array}
  }} \\
  \\[-0.75em]
  (d \betastep f)(μ) & = & \begin{cases}
    \many{ε \cons {}} f(v)(μ') & \text{$d(μ) = \many{ε \cons {}} \goodend{v,μ'}$ and $v ∈ \dom(f)$} \\
    \many{ε \cons {}} \stuckend{} & \text{$d(μ) = \many{ε \cons {}} \goodend{v,μ'}$ and $v \not∈ \dom(f)$} \\
    d(μ) & \text{otherwise} \\
  \end{cases} \\
  \\[-0.75em]
  \ret(v)(μ) & = & \goodend{v,μ} \\
  \apply(d,\pa) & = & d \betastep \fn{(\FunV(f))}{f(\pa)} \\
  \select(d,\alts) & = & d \betastep \fn{(\ConV(K_s,\many{\pa}))}{\alts(K_s, \many{\pa})} \quad \text{where } (K_s, \many{\pa}) ∈ \dom(\alts) \\
  \memo(\pa,d)   & = & d \betastep \fn{v}{(\fn{μ}{\UpdateE(\pa↦v) \cons \goodend{v,μ[\pa↦\memo(\pa,\ret(v))]}})} \\
 \end{array} \\
 \\[-0.75em]
 \begin{array}{rcl}
  \multicolumn{3}{c}{ \ruleform{ \semevt{\wild} \colon \Exp → (\Var \pfun \Addresses) → \EventD } } \\
  \\[-0.75em]
  \semevt{\px}_ρ(μ)       & = & \begin{cases}
    \LookupE(ρ(\px)) \cons μ(ρ(\px))(μ) & \px ∈ \dom(ρ) \\
    \stuckend{}  & \text{otherwise} \\
  \end{cases} \\
  \\[-0.75em]
  \semevt{\Lam{\px}{\pe}}_ρ & = & \ret(\FunV(\fn{\pa}{\AppEE(\px↦\pa) \cons \semevt{\pe}_{ρ[\px↦\pa]}})) \\
  \\[-0.75em]
  \semevt{\pe~\px}_ρ(μ)   & = & \begin{cases}
    \AppIE(ρ(x)) \cons \apply(\semevt{\pe}_ρ, ρ(\px))(μ) & \px ∈ \dom(ρ) \\
    \stuckend{}  & \text{otherwise} \\
  \end{cases} \\
  \\[-0.75em]
  \semevt{\Let{\px}{\pe_1}{\pe_2}}_ρ(μ) & = &
    \begin{letarray}
      \text{let} & ρ' = ρ[\px ↦ \pa] \quad \text{where $\pa \not∈ \dom(μ)$} \\
                 & d_1^\later = \semevt{\pe_1}_{ρ'} \\
      \text{in}  & \BindE(\px,\pa↦d_1^\later) \cons \semevt{\pe_2}_{ρ'}(μ[\pa ↦ \memo(\pa,d_1^\later)])
    \end{letarray} \\
  \\[-0.75em]
  \semevt{K~\many{\px}}_ρ(μ) & = & \begin{cases}
    \ret(\ConV(K,\many{ρ(\px)}))(μ) & \many{\px ∈ \dom(ρ)} \\
    \stuckend & \text{otherwise} \\
  \end{cases} \\
  \\[-0.75em]
  \semevt{\Case{\pe_s}{\Sel[r]}}_ρ(μ) & = &
    \begin{letarray}
      \text{let} & \alts = \fn{(K_i, \many{\pa})}{\CaseEE(K_i,\many{\px_i↦\pa}) \cons \semevt{\pe_{r_i}}_{ρ[\many{\px_i↦\pa}]}} \\
      \text{in} & \CaseIE(\semevt{\pe_s}_ρ) \cons \select(\semevt{\pe_s}_ρ, \alts)(μ)  \\
    \end{letarray}
 \end{array} \\
\end{array}\]
\caption{Eventful Trace Semantics}
  \label{fig:semevt}
\end{figure}

\subsection{Favouring Events over State}

\Cref{fig:semevt} gives the definition for the eventful trace semantics
$\semevt{\wild}$.
The domain of eventful traces $\EventD$ maps a heap
to a possibly infinite or stuck program trace $τ$, just as the vanilla
semantics $\semvan{\wild}$.

Upon a transition of the internal machine state, our new semantics emits
an \emph{event} $ε$ with the cons $\cons$ constructor, rather than just
delaying with an uninformative $\laterC$.
In fact, each event carries with it part of the ``closure'' of the
corresponding LK transition rule, which is readily available for analysis.
For example, the $\LookupE$ event carries the heap address which is accessed,
$\BindE$ events carry the $\mathsf{let}$-bound variable, its fresh address and
the denotation of the right-hand side.
We thus open up our semantics for instrumentation.
Certainly we could have omitted some or added more information than what
the events carry; in this work we mostly care about $\LookupE$ events.

\begin{toappendix}
\begin{figure}
\[\begin{array}{rcl}
  α_\Heaps([\many{\pa ↦ (ρ,\pe)}]) & = & [\many{\pa ↦ \memo(\pa,\semvan{\pe}_{ρ})}] \\
  α_\EventV(ρ,\Lam{\px}{\pe}) & = & \FunV(\fn{\pa}{\AppEE(\px↦\pa) \cons \semevt{\pe}_{ρ[\px↦\pa]}}) \\
  α_\EventV(ρ,K~\many{\px}) & = & \ConV(K,\many{ρ(\px)}) \\
  α_\Events(σ) & = & \begin{cases}
    \BindE(\px,\pa↦\semevt{\pe_1}_{ρ[\px↦\pa])}) & σ = (\Let{\px}{\pe_1}{\wild},ρ,μ,\wild) \wedge \pa \not∈ \dom(μ) \\
    \AppIE(\pa) & σ = (\wild~\px,ρ,\wild,\ApplyF(\pa) \pushF \wild) \\
    \CaseIE(\semevt{\pe_s}_{ρ}) & σ = (\Case{\pe_s}{\wild},ρ,\wild, \wild)\\
    \LookupE(ρ(\px)) & σ = (\px,ρ,\wild,\UpdateF(\pa) \pushF \wild) \\
    \AppEE(\px↦\pa) & σ = (\Lam{\px}{\wild},\wild,\wild,\ApplyF(\pa) \pushF \wild) \\
    \CaseEE(K',\many{\px_i ↦ ρ(\py)}) & σ = (K'~\many{\py}, ρ, \wild, \SelF(\wild,\Sel) \pushF \wild) \wedge K' = K_i \\
    \UpdateE(\pa↦α_\EventV(ρ,\pv)) & σ = (\pv,ρ,\wild,\UpdateF(\pa) \pushF \wild) \\
  \end{cases} \\
  α_{\Traces}((σ_i)_{i∈\overline{n}},κ) & = & \begin{cases}
    α_\Events(σ_1) \cons α_{\Traces}((σ_{i+1})_{i∈\overline{n-1}},κ) & n > 1 \\
    \goodend{α_\EventV(ρ,\pv)} & n = 1 \wedge σ_1 = (\pv, ρ, \wild, κ) \\
    \stuckend{} & \text{otherwise} \\
  \end{cases} \\
  \correct & = & ∀(σ_i)_{i∈\overline{n}}.\ \maxtrace{(σ_i)_{i∈\overline{n}}} \Longrightarrow ∀((\pe,ρ,μ,κ) = σ_1).\ α_{\Traces}((σ_i)_{i∈\overline{n}},κ) = \semvan{\pe}_{ρ}(α_\Heaps(μ)) \\
\end{array}\]
\caption{Correctness predicate for $\semevt{\wild}$}
  \label{fig:semevt-correctness}
\end{figure}
\end{toappendix}

There is one minor structural difference to $\semvan{\wild}$:
$\FunV$ values now return an unguarded $\EventD$ so that the $\AppEE$ event
can carry the lambda-bound variable of the lambda expression from whence it
came.
This unguarded (positive) occurrence is not prohibited by the type construction
rules of guarded domain theory, but \citet[Section 5.2]{tctt} suggests that we
could work around this issue by defining $\EventD$ as a simultaneous guarded
recursive and inductive type, \ie,
$\EventD = \fix X.\ \lfp Y.\ ... \FunV(f ∈ X \to Y) ...$, where $\fix$ is the
guarded fixpoint operator and $\lfp$ is the least fixed-point operator.

All results connecting $\semvan{\wild}$ to the LK transition semantics also
hold for $\semevt{\wild}$, since $\semvan{\wild}$ can be recovered by throwing
away all events and replacing $\cons$ with $\laterC$.

Thus the only thing worth verifying is that the events actually correspond to
meaningful LK concepts.
One can read off this correpondence in the abstraction function for the
extended correctness predicate for $\semevt{\wild}$ which we give in
\Cref{fig:semevt-correctness} in the Appendix.

\begin{theorem}[Correctness of $\semevt{\wild}$]
  \label{thm:semvan-correct}
  $\correct$ from \Cref{fig:semevt-correctness} holds.
  That is, whenever $(σ_i)_{i∈\overline{n}}$ is a maximal LK trace with source
  state $(\pe,ρ,μ,κ)$, we have
  $α_{\Traces}((σ_i)_{i∈\overline{n}},κ) = \semevt{\pe}_{ρ}(α_\Heaps(μ))$.
\end{theorem}

It is clear that this style of semantics allows to observe arbitrary operational
detail on an as-needed basis, thus we fulfill our Goal 3.
Let us conclude this section by giving the embellished version of
the earlier example from \Cref{sec:vanilla}, $\Let{i}{\Lam{x}{x}}{i~i}$:
\[\begin{array}{ll}
  \newcommand{\myleftbrace}[4]{\draw[mymatrixbrace] (m-#2-#1.west) -- node[right=2pt] {#4} (m-#3-#1.west);}
  \vcenter{\hbox{$
    \begin{tikzpicture}[mymatrixenv,anchor=center]
      \matrix [mymatrix] (m)
      {
        1 & \BindE(i,\pa_1↦d) \cons {} & \hspace{3.7em} & \hspace{4.2em} & \hspace{3.5em} \\
        2 & \AppIE(\pa_1) \cons {} & & & \\
        3 & \LookupE(\pa_1) \cons {} & & & \\
        4 & \UpdateE(\pa_1↦\FunV(f)) \cons {} & & & \\
        5 & \AppEE(x↦\pa_1) \cons {} & & & \\
        6 & \LookupE(\pa_1) \cons {} & & & \\
        7 & \UpdateE(\pa_1↦\FunV(f)) \cons {} & & & \\
        8 & \goodend{\FunV(f), μ} & & & \\
      };
      % Braces, using the node name prev as the state for the previous east
      % anchor. Only the east anchor is relevant
      \foreach \i in {1,...,\the\pgfmatrixcurrentrow}
        \draw[dotted] (m.east|-m-\i-\the\pgfmatrixcurrentcolumn.east) -- (m-\i-2);
      \myleftbrace{3}{1}{8}{$\semevt{\pe}_{[]}$}
      \myleftbrace{4}{2}{8}{$\semevt{i~i}_{ρ_1}$}
      \myleftbrace{5}{3}{5}{$\semevt{i}_{ρ_1}$}
      \myleftbrace{5}{6}{8}{$\semevt{x}_{ρ_2}$}
  \end{tikzpicture}
  $}} &
  \!\!\!\!\text{where}  \begin{array}{ll}
  ρ_1 = [i ↦ \pa_1] & \\
  ρ_2 = ρ_1[x ↦ \pa_1] &  \\
  d = \semevt{\Lam{x}{x}}_{ρ_1} \\
  μ = [\pa_1 ↦ \memo(\pa_1,d)] & \\
  f = \pa \mapsto \semevt{\px}_{ρ_1[\px↦\pa]}
  \end{array}
\end{array}\]

%And then also for the heap update in $\Let{i}{(\Lam{y}{\Lam{x}{x}})~i}{i~i}$:
%
%\[\begin{array}{ll}
%  \newcommand{\myleftbrace}[4]{\draw[mymatrixbrace] (m-#2-#1.west) -- node[right=2pt] {#4} (m-#3-#1.west);}
%  \vcenter{\hbox{$
%    \begin{tikzpicture}[baseline={-0.5ex},mymatrixenv]
%      \matrix [mymatrix] (m)
%      {
%        1 & (μ_1)~\UpdateE(\pa_1↦v) \cons {} & \hspace{4em} & \hspace{3.5em} \\
%        2 & (μ_2)~\AppEE(x↦\pa_1) \cons {} & & \\
%        3 & (μ_2)~\LookupE(\pa_1) \cons {} & & \\
%        4 & (μ_2)~\UpdateE(\pa_1↦v) \cons {} & & \\
%        5 & \goodend{v, μ_2} & & \\
%      };
%      % Braces, using the node name prev as the state for the previous east
%      % anchor. Only the east anchor is relevant
%      \foreach \i in {1,...,\the\pgfmatrixcurrentrow}
%        \draw[dotted] (m.east|-m-\i-\the\pgfmatrixcurrentcolumn.east) -- (m-\i-2);
%      \myleftbrace{3}{1}{5}{$\semevt{i~i}_{ρ_1}$}
%      \myleftbrace{4}{1}{2}{$\semevt{i}_{ρ_1}$}
%      \myleftbrace{4}{3}{5}{$\semevt{x}_{ρ_3}$}
%    \end{tikzpicture}
%  $}} &
%  \!\!\!\text{where} \begin{array}{l}
%  ρ_1 = [i ↦ \pa_1] \\
%  ρ_2 = [i ↦ \pa_1, y ↦ \pa_1] \\
%  ρ_3 = [i ↦ \pa_1, y ↦ \pa_1, x ↦ \pa_1] \\
%  v = \FunV(\fn{\pa}{\semevt{\px}_{ρ_2[\px↦\pa]}}) \\
%  μ_1 = [\pa_1 ↦ \memo(\pa_1,\semevt{(\Lam{y}{\Lam{x}{x}})~i}_{ρ_1})] \\
%  μ_2 = [\pa_1 ↦ \memo(\pa_1,\semevt{\Lam{x}{x}}_{ρ_2})] \\
%  \end{array} \\
%\end{array}\]

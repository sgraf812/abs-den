% -*- mode: LaTeX -*-
%% For double-blind review submission, w/o CCS and ACM Reference (max submission space)
%\documentclass[acmsmall,review,anonymous,natbib=false]{acmart}\settopmatter{printfolios=true,printccs=false,printacmref=false}
%% For double-blind review submission, w/ CCS and ACM Reference
%\documentclass[acmsmall,review,anonymous]{acmart}\settopmatter{printfolios=true}
%% For single-blind review submission, w/o CCS and ACM Reference (max submission space)
\documentclass[acmsmall,review]{acmart}\settopmatter{printfolios=true,printccs=false,printacmref=false}
%% For single-blind review submission, w/ CCS and ACM Reference
%\documentclass[acmsmall,review]{acmart}\settopmatter{printfolios=true}
%% For final camera-ready submission, w/ required CCS and ACM Reference
%\documentclass[acmsmall,screen]{acmart}\settopmatter{}

%\documentclass[acmsmall,review,anonymous]{acmart}\settopmatter{printfolios=true,printccs=false,printacmref=false}

%% Journal information
%%
\setcopyright{rightsretained}
\acmPrice{}
\acmDOI{10.1145/1111111}
\acmYear{2024}
\copyrightyear{2024}
\acmSubmissionID{popl24main-p11-p}
\acmJournal{PACMPL}
\acmVolume{1}
\acmNumber{POPL}
\acmArticle{1}
\acmMonth{1}

%% Bibliography style
\bibliographystyle{ACM-Reference-Format}
%% Citation style
%% Note: author/year citations are required for papers published as an
%% issue of PACMPL.
\citestyle{acmauthoryear}   %% For author/year citations

%%%%%%%
\usepackage{array} % \newcolumntype
\usepackage{color}
%\usepackage[svgnames]{xcolor}
\usepackage{cleveref}
\usepackage{xspace}
\usepackage{url}
\usepackage{galois}
\usepackage{hsyl-listing} % SML listing style, hsyl-style
%\usepackage{scalerel}
%\usepackage[all]{xy}
%\usepackage{graphicx}
%\usepackage{stackengine}
\usepackage{mathtools} % xhookrightarrow
\usepackage{trimclip} % clipbox
\usepackage{mathpartir} % inference rules
\usepackage{subcaption}
\usepackage{mathbbol} % \bbcolon and \bbquestionmark
\usepackage{stmaryrd} % \lightning
\usepackage{tikz}
%\usetikzlibrary{cd} % commutative diagrams
\usetikzlibrary{calc}
\usetikzlibrary{fit}
\usetikzlibrary{patterns}
\usetikzlibrary{matrix}
\usetikzlibrary{decorations.pathreplacing}
\usepackage{placeins} % flush floats with \FloatBarrier
\usepackage[T1]{fontenc} % https://tex.stackexchange.com/a/181119

\usepackage{utf8-symbols}
% Theorems
\theoremstyle{plain} % default
\newtheorem{theorem}{Theorem}
\newtheorem{lemma}[theorem]{Lemma}
\newtheorem{proposition}[theorem]{Proposition}
\crefname{proof}{Proof}{Proofs}

\theoremstyle{definition}
\newtheorem{example}[theorem]{Example}
\newtheorem{definition}[theorem]{Definition}

% Abbrev
\newcommand\eg{\emph{e.g.}\ }
\newcommand\etc{\emph{etc.}}

% Comments and notes
\newcommand{\slpj}[1]{\emph{SLPJ: #1}}
\newenvironment{slpjenv}{\em SLPJ:}{}
\newcommand{\sg}[1]{\emph{SG: #1}}
\newcommand{\todo}[1]{\textcolor{red}{TODO: #1}}

% Auxiliary
\newcommand{\many}[1]{\overline{#1}}
\newcommand{\wild}{\ensuremath{\mathunderscore}}
\newcommand{\pfun}{\rightharpoonup}
\newcommand{\ruleform}[1]{\fbox{$#1$}}
\newcommand{\highlight}[1]{\setlength{\fboxsep}{2pt}\colorbox[gray]{0.8}{\ensuremath{#1}}}
\newcommand{\nelems}[1]{\lvert#1\rvert}
\newcommand{\fn}[2]{\ensuremath{λ#1.\ #2}}
\newcommand{\constfn}[1]{\fn{\mathunderscore}{#1}}
%\newcommand{\repeat}[2]{\foreach \n in {1,...,#1}{#2}} % https://tex.stackexchange.com/a/16190/52414
% https://tex.stackexchange.com/a/21647
\makeatletter
\newcommand{\superimpose}[3][\mathord]{#1{\mathpalette\superimpose@{{#2}{#3}}}}
\newcommand{\superimpose@}[2]{\superimpose@@{#1}#2}
\newcommand{\superimpose@@}[3]{%
  \ooalign{%
    \hfil$\m@th#1#2$\hfil\cr
    \hfil$\m@th#1#3$\hfil\cr
  }%
}
\newenvironment{letarray}{\begin{array}{@{}l@{\hspace{1ex}}l}}{\end{array}}

% Syntax
%% LC
\newcommand{\Con}{\mathsf{Con}}
\newcommand{\Var}{\mathsf{Var}}
\newcommand{\Lab}{\mathsf{Lab}}
\newcommand{\Exp}{\mathsf{Exp}}
\newcommand{\Val}{\mathsf{Val}}
\newcommand{\px}{\mathsf{x}}
\newcommand{\py}{\mathsf{y}}
\newcommand{\pz}{\mathsf{z}}
\newcommand{\pv}{\mathsf{v}}
\newcommand{\pe}{\mathsf{e}}
\newcommand{\Lam}[2]{\lambda #1. #2}
\newcommand{\LLam}[3]{\lambda_{#1} #2. #3}
\newcommand{\Let}[3]{\mathbf{let}~{#1}={#2}~\mathbf{in}~{#3}}
\newcommand{\LLet}[4]{\mathbf{let}_{#1}~{#2}={#3}~\mathbf{in}~{#4}}
\newcommand{\Case}[2]{\mathbf{case}~{#1}~\mathbf{of}~{#2}}
\newcommand{\Sel}[1][]{\many{K~\many{x} \rightarrow e_{#1}}}
\newcommand{\dom}{\mathop\mathsf{dom}}
\newcommand{\rng}{\mathop\mathsf{rng}}
\newcommand{\fv}{\mathop\mathsf{fv}}
\newcommand{\bv}{\mathop\mathsf{bv}}
\newcommand{\fresh}[2]{#1\mathbin{\#}#2}
\newcommand{\delvar}[1]{{\setminus_{#1}}}
%% Actions
\newcommand{\Actions}{\mathbb{A}}
\newcommand{\ValA}{\mathsf{val}}
\newcommand{\BindA}{\mathsf{bind}}
\newcommand{\LookupA}{\mathsf{look}}
\newcommand{\UpdateA}{\mathsf{upd}}
\newcommand{\AppIA}{\mathsf{app}_1}
\newcommand{\AppEA}{\mathsf{app}_2}
\newcommand{\CaseIA}{\mathsf{case}_1}
\newcommand{\CaseEA}{\mathsf{case}_2}
%% Labels
\newcommand{\Labels}{\mathbb{L}}
\newcommand{\lbl}{\ensuremath{\ell}}
\newcommand{\lbln}[1]{\ensuremath{\ell_{#1}}}
\newcommand{\slbl}{\raisebox{0.08em}{$\;\scriptstyle{\ell}\;$}}
\newcommand{\slbln}[1]{\raisebox{0.1em}{$\;\scriptstyle{\ell_{#1}}\;$}}
\newcommand{\atlbl}[1]{at(#1)}
\newcommand{\deref}[1]{*#1}
%% Traces
\newcommand{\Traces}{\mathbb{T}}
\newcommand{\act}[1]{\xrightarrow{#1}}
% https://latex.org/forum/viewtopic.php?p=62177&sid=26890ac77d076f3338384a47a2ffd4bc#p62177
\makeatletter
\newcommand{\concatop}[1]{%
  \mathbin{\mathop{#1}\limits^{\vbox to 1\ex@{\kern-\tw@\ex@
   \hbox{\scriptsize$\smallfrown$}\vss}}}}
\makeatother
\makeatletter
\newcommand{\compop}[1]{%
  \mathbin{\vec{\ooalign{$\circ$\cr\hidewidth$#1$\hidewidth}}}}
\makeatother
\newcommand{\concat}{\concatop{\cdot}}
\newcommand{\fcomp}{\compop{\cdot}}
\newcommand{\balanced}[1]{#1\;\mathsf{bal}}
\newcommand{\getval}[2]{#1 \Downarrow #2}
\newcommand{\lookup}[2]{#1\;\mathsf{lookup}}
\newcommand{\subtrceq}{\superimpose[\mathrel]{\clipbox{0pt 0pt {0.3\width} 0pt}{${\subset}$}}{\hspace{0.6ex}\cdot}}
\newcommand{\rightsubtrceq}{\subtrceq_r}
\newcommand{\leftsubtrceq}{\subtrceq_l}
\newcommand{\stepm}[3]{\lceil #1 \act{#2} #3 \rceil}
\newcommand{\step}[2]{\stepm{\wild}{#1}{#2}}

%% Values
\newcommand{\Values}{\mathbb{V}}
\newcommand{\FunV}{\mathsf{fun}}
\newcommand{\ConV}{\mathsf{con}}
%% Domains
\newcommand{\Environments}{\mathbb{E}}
\newcommand{\Heaps}{\mathbb{H}}
\newcommand{\Continuations}{\mathbb{K}}
\newcommand{\Domain}[1]{{\mathbb{D}^{#1}}}
\newcommand{\PrefD}{\Domain{*}}
\newcommand{\MaxD}{\Domain{+\infty}}
\newcommand{\AbsD}{\Domain{\raisebox{0.1em}{\scalebox{0.4}{$\square$}}}}
\newcommand{\AbsCD}{\Domain{\raisebox{0.1em}{\scalebox{0.4}{c$\square$}}}}
\newcommand{\StateD}{\Domain{\States}}
\newcommand{\SSD}{\Domain{Σ}}
\newcommand{\SSCD}{\Domain{cΣ}}
\newcommand{\LiveD}{\Domain{\exists l}}
\newcommand{\LiveSD}{\Domain{sl}}
\newcommand{\LiveCD}{\Domain{cl}}
\newcommand{\LiveSCD}{\Domain{scl}}
\newcommand{\testD}{\mathbin{?\!!}}

% Math
\newcommand{\poset}[1]{\wp(#1)}
\newcommand{\Nat}{\mathbb{N}}
\DeclareMathSymbol{\bbcolon}{\mathpunct}{bbold}{"3A}
\DeclareMathSymbol{\bbquestionmark}{\mathpunct}{bbold}{"3F}
\DeclareMathSymbol{\bblparen}{\mathpunct}{bbold}{"28}
\DeclareMathSymbol{\bbrparen}{\mathpunct}{bbold}{"29}
\DeclareMathSymbol{\bbpipe}{\mathpunct}{bbold}{"7C}
\newcommand{\ternary}[3]{\bblparen\, #1 \,\mathrel{\bbquestionmark}\, #2 \,\mathrel{\bbcolon}\, #3 \,\bbrparen}

% Order theory
\DeclareMathOperator*{\lub}{\sqcup}
\DeclareMathOperator*{\Lub}{\bigsqcup}
\DeclareMathOperator*{\glb}{\sqcap}
\DeclareMathOperator*{\Glb}{\bigsqcap}
\newcommand{\lfp}{\mathsf{lfp}}
\newcommand{\gfp}{\mathsf{gfp}}

% Semantics
\newcommand{\denot}[1]{\llbracket {#1} \rrbracket}
\newcommand{\sempref}[1]{\mathcal{S}^*\denot{#1}}
\newcommand{\seminf}[1]{\mathcal{S}^{+\infty}\denot{#1}}
\newcommand{\semst}[1]{\mathcal{S}^{\States}\denot{#1}}
\newcommand{\semss}[1]{\mathcal{S}^{Σ}\denot{#1}}
\newcommand{\semlive}[1]{\mathcal{S}^{∃l}\denot{#1}}
\newcommand{\semclive}[1]{\mathcal{S}^{cl}\denot{#1}}
\newcommand{\semslive}[1]{\mathcal{S}^{sl}\denot{#1}}
\newcommand{\semsclive}[1]{\mathcal{S}^{scl}\denot{#1}}
\newcommand{\semty}[1]{\mathcal{S}^{τ}\denot{#1}}

% Stateful
\newcommand{\States}{\mathbb{S}}
\newcommand{\STraces}{\mathbb{S}^{+\infty}}
\newcommand{\sconcat}{\concatop{\hspace{0.04ex}\scalebox{0.5}{$\States$}}}
\newcommand{\sfcomp}{\compop{\scalebox{0.25}{\raisebox{0.7em}{$\States$}}}}
\newcommand{\pushF}{\cdot}
\newcommand{\StopF}{\mathbf{stop}}
\newcommand{\ReturnF}{\mathbf{ret}}
\newcommand{\ApplyF}{\mathbf{ap}}
\newcommand{\UpdateF}{\mathbf{upd}}

% Small-step
\newcommand{\Configurations}{\mathsf{Config}}
\newcommand{\Addresses}{\mathsf{Addr}}
%\newcommand{\Heaps}{\mathsf{Heap}}
\newcommand{\Envs}{\mathsf{Env}}
\newcommand{\Stacks}{\mathsf{Stack}}
\newcommand{\Frames}{\mathsf{Frame}}
\newcommand{\pH}{\mathsf{H}}
\newcommand{\pa}{\mathsf{a}}
\newcommand{\pE}{\mathsf{E}}
\newcommand{\pS}{\mathsf{S}}
\newcommand{\SSTraces}{\mathsf{SSTrace}}
\newcommand{\smallstep}{\hookrightarrow}
\newcommand{\straceend}{\scalebox{0.5}{$\square$}}
\newcommand{\ssconcat}{\concatop{\hspace{0.04ex}\scalebox{0.5}{\Sigma}}}
\newcommand{\AppIT}{\textsc{app}_1}
\newcommand{\AppET}{\textsc{app}_2}
\newcommand{\ValueT}{\textsc{val}}
\newcommand{\LookupT}{\textsc{look}}
\newcommand{\UpdateT}{\textsc{upd}}
\newcommand{\LetT}{\textsc{let}}
\newcommand{\funnyComp}{\mathop{\mathord{>}\!\mathord{\circ}\!\mathord{>}}}
\newcommand{\validtrace}[1]{#1\;\mathsf{ok\text{-}}π}
\newcommand{\maxbaltrace}[1]{#1\;\mathsf{maxbal\text{-}}π}
\newcommand{\validtracefun}[1]{#1\;\mathsf{ok\text{-}}d}

% Liveness
\newcommand{\lStacks}{\Stacks^l}
\newcommand{\lLiveness}{\mathsf{Live}}
\newcommand{\lS}{\pS^l}
\newcommand{\lSBot}{\texttt{Seq}}
\newcommand{\lSAp}[1]{\texttt{\$[$#1$]}}
\newcommand{\lSTop}{\texttt{Deep}}
\newcommand{\lAbs}{\texttt{Abs}}
\newcommand{\lUsed}[1]{\texttt{Used[$#1$]}}
\newcommand{\SVar}{\mathsf{SVar}}
\newcommand{\SCall}{\mathsf{SCall}}

% Types
\newcommand{\TyCon}{\mathsf{TyCon}}
\newcommand{\Type}{\mathsf{Type}}
\newcommand{\ArrowTy}{\Rightarrow}


% Tables should have the caption above
\floatstyle{plaintop}
\restylefloat{table}

\begin{document}

\special{papersize=8.5in,11in}
\setlength{\pdfpageheight}{\paperheight}
\setlength{\pdfpagewidth}{\paperwidth}

\title{A Compositional Trace Semantics for Lambda Calculus}
\subtitle{Or: Denotational semantics for Call-by-need Lambda Calculus}

\author{Sebastian Graf}
\affiliation{%
  \institution{Karlsruhe Institute of Technology}
  \city{Karlsruhe}
  \country{Germany}
}
\email{sgraf1337@gmail.com}

\author{Simon Peyton Jones}
\affiliation{%
  \institution{Epic Games}
  \city{Cambridge}
  \country{UK}
}
\email{simon.peytonjones@gmail.com}

% Some conditional build stuff for handling the Appendix

\newif\ifmain
\newif\ifappendix

% Builds only the main paper by default.
\maintrue
\appendixfalse
% But we provide a switch to build the Appendix only.
\def\appendixonly{\mainfalse{}\appendixtrue{}}

% .. so that you can comment out the following line to build the Appendix only
% This is done by the `make appendix` target.
%\appendixonly

% Same thing for an extended version that includes the Appendix
\def\extended{\maintrue{}\appendixtrue{}}
%\extended

\ifmain

\begin{abstract}
\end{abstract}

%% 2012 ACM Computing Classification System (CSS) concepts
%% Generate at 'http://dl.acm.org/ccs/ccs.cfm'.
\begin{CCSXML}
<ccs2012>
   <concept>
       <concept_id>10011007.10011006.10011041</concept_id>
       <concept_desc>Software and its engineering~Compilers</concept_desc>
       <concept_significance>500</concept_significance>
       </concept>
   <concept>
       <concept_id>10011007.10011006.10011073</concept_id>
       <concept_desc>Software and its engineering~Software maintenance tools</concept_desc>
       <concept_significance>300</concept_significance>
       </concept>
   <concept>
       <concept_id>10011007.10011006.10011008.10011024.10011035</concept_id>
       <concept_desc>Software and its engineering~Procedures, functions and subroutines</concept_desc>
       <concept_significance>100</concept_significance>
       </concept>
   <concept>
       <concept_id>10011007.10011006.10011008.10011024.10011032</concept_id>
       <concept_desc>Software and its engineering~Constraints</concept_desc>
       <concept_significance>300</concept_significance>
       </concept>
   <concept>
       <concept_id>10011007.10011006.10011008.10011009.10011012</concept_id>
       <concept_desc>Software and its engineering~Functional languages</concept_desc>
       <concept_significance>300</concept_significance>
       </concept>
   <concept>
       <concept_id>10011007.10011006.10011008.10011009.10011021</concept_id>
       <concept_desc>Software and its engineering~Multiparadigm languages</concept_desc>
       <concept_significance>300</concept_significance>
       </concept>
 </ccs2012>
\end{CCSXML}

\ccsdesc[500]{Software and its engineering~Compilers}
\ccsdesc[300]{Software and its engineering~Software maintenance tools}
\ccsdesc[100]{Software and its engineering~Procedures, functions and subroutines}
\ccsdesc[300]{Software and its engineering~Constraints}
\ccsdesc[300]{Software and its engineering~Functional languages}
\ccsdesc[300]{Software and its engineering~Multiparadigm languages}
%% End of generated code

%% Keywords
%% comma separated list
\keywords{Programming language semantics}  %% \keywords are mandatory in final camera-ready submission

\maketitle

%\section{Introduction}
\label{sec:introduction}

As an implementor of a programming language, it is often useful to automatically
glean facts about a program such as ``this program is well-typed'', ``this
higher-order function is always called with argument $\Lam{x}{x+1}$'' or ``this
program never evaluates $x$'' by way of \emph{static (program) analysis}.

\paragraph{Analysis follows structure}
If the implementation language is a functional one, then usually such static
analyses are formulated as a function defined by \emph{structural recursion} on
the input term.
For example, given an application expression $(\pe_1~\pe_2)$,
the property ``$(\pe_1~\pe_2)$ never evaluates its free variable $x$'' can be
\emph{conservatively approximated} (here: It's OK to say ``No'' more often) by
the property ``$\pe_1$ and $\pe_2$ never evaluate $x$''.

Such a structural formulation is quite convenient:
(1) Structural recursion gives an immediate proof of termination and can
    further be exploited in other inductive proofs, all within decidable
    territory.
(2) A structurally-defined function $f$ is often \emph{compositional}, meaning that
    if you replace a sub-expression $e$ in a bigger expression $C[e]$ by another
    expression $e'$ with $f(e) = f(e')$, the overall result
    of the containing expression $f(C[e]) = f(C[e'])$ does not change.
    This makes it easy for humans to understand and reason about the function,
    because the result of a big expression depends on the results of its parts
    (and not on the shape of the sub-expressions themselves).

For static analyses, especially more complicated ones, it is good practice to
provide a proof of correctness of some sort. If the correctness statement can
be expressed in terms of a \emph{denotational
semantics}~\cite{ScottStrachey:71}, then the recursion structure of analysis
function and semantics function line up nicely. As a result, the proof of
correctness can be conducted by simple induction over the expression.

\paragraph{Domain Theory is a leaky abstraction}
Alas, even when the denotational semantics is ``standard'', the hard part is in
coming up with a suitable correctness predicate!
Traditionally, the semantic domain of denotational semantics models diverging
and stuck programs with $\bot$, the function that is undefined everywhere.
There is rich and complicated literature on defining an algebraic
domain~\cite{Scott:71} that is suitable to denote untyped lambda calculus.
The key is embedding the subclass of \emph{continuous} functions between domain
elements into the domain itself. All computable functions can be proven
continuous, so this is a sufficient substrate.
Now, to prove a predicate $P(d)$ of some denotation $d$, such as ``$d$ has type
$τ$'' by structural induction, one has to show first that $P$ is compatible
with continuity; perhaps by proving that $P$ characterises a sub-domain or an
ideal in the domain~\cite{Milner:78}.

Note that this is all \emph{before} even attempting the proof! Often, the
denotational semantics or even the domain itself needs adjustments.
The former case needs a proof of continuity, while the latter can be quite
involved and non-compositional for effects such as exceptions, concurrency and
state, as \citep{WrightFelleisen:87} noted.

We think that most of the troubles in the application of Domain Theory are
caused by the non-commital nature of the approximation order, in that any
predicate on total elements also needs to accept its partial approximation;
hence it is desirable to strive for \emph{total} descriptions of the
potentially program infinite behaviors.

the crucial property that any
, and it was
a great achievement of \citep{Scott:71} turns out that Domain Theory is the result of equipping Doing so invites
and stuck programs. Consider the following program using recursive let that is
infinitely-looping
\begin{equation}
  \label{eqn:loop}
  \pe_{loop} \triangleq \Let{id}{\Lam{x}{x}}{\Let{loop}{id~loop}{loop}}
\end{equation}
The traditional denotational semantics after Scott and Strachey would equate all
of the following programs:
$\semscott{\pe_{loop}}_ρ = \semscott{\Let{loop}{id~loop}{loop}}_ρ =
\semscott{\mathsf{segfault}}_ρ = \bot$.
Note that the first program has an infinite loop, the second one is not
well-scoped (thus stuck at some point) and the last one is a straight out crash
in the style of an imprecise exception \cite{imprecise-exceptions}.
To a compiler developer, this conflation is both a reason for joy (more
optimisation opportunities) and a reason for ... reflection (users didn't expect
their infinite loop to be optimised into a crash). Such issues come up in
practice \sg{cite GHC issues}.

More seriously, it is impossible to prove by way of a denotational semantics
that a static analysis does not misoptimise infinite behaviors.
(a) The \emph{potential liveness} analysis that says ``$\pe_{loop}$ never
evaluates $id$'' could be proven ``correct''.
(b) A type analysis that says ``$\Let{loop}{id~loop}{loop}$ is closed and
well-typed'' can still be proven progressing as long as $\pe_{loop}$ can
be well-typed (which would not be too surprising), because the latter is
denotationally equivalent to the former.
(c) Imagine that $id$ was supplied as a parameter to $loop$ instead, \eg
$...\ \Let{loop}{\Lam{f}{f~(loop~f)}}{loop~id}$. Then a control-flow analysis
\cite{Shivers:91} that says ``$f$ is never bound to $id$'' can be proven correct
in terms of the denotational semantics.

Furthermore, although it is sensible (in the terminating case) to ask whether or
not $x$ is \emph{never} evaluated in terms of the denotational semantics, asking
whether $x$ is evaluated \emph{at most once} is not, for the same reason that
traditional denotational semantics is not able to discern call-by-name from
call-by-need. Yet, to the Glasgow Haskell Compiler, this distinction is very
much of concern!

When denotational semantics fails the compiler developer, they turn to a
correctness criterion in terms of a \emph{structural operational semantics}.
This was the approach taken by \cite{cardinality} to prove evaluation
cardinality properties such as potential liveness. The drawback of this proof
framework is the immense complexity arising from the disconnect between a
structural definition and a transition system; matters such as substitution,
multiple heap activations of the let binding, non-determinism and fixpoint
induction abound. It is hard to raise confidence in such a proof without full
mechanisation.

One could adopt the approach of \emph{Abstracting Abstract Machines} \cite{aam}
and let the structure of the semantics dictate the structure of the
analysis for a re-usable proof of correctness via abstract interpretation
\cite{Cousot:21}.
However, that is not how the static analyses work that the authors are familiar
with.
For example, it would be quite an effort to rewrite the neat,
structurally-defined analyses of the Glasgow Haskell Compiler into a fixpoint
iteration on the approximated states of an abstract transition system.

The contributions of this work are as follows:
\begin{itemize}
  \item In \Cref{sec:problem}, we give a more formal exposition to the
    problems we just introduced and are inclined to solve.
  \item In \Cref{sec:semantics}, we give a \emph{structurally-defined} semantics
    for lambda calculus that is \emph{able to discern stuck, diverging and
    converging programs}. Furthermore, it is a semantics for \emph{call-by-need}
    lambda calculus that is distinct from similar ones for call-by-name or
    call-by-value and allows to observe evaluation cardinality as needed.
    We believe that our semantics is the first with the aforementioned two
    qualities and prove it correct \wrt a standard operational semantics. The
    idea borrows heavily from the idea of a maximal prefix trace semantics
    advocated by \citep{Cousot:21}.
  \item The semantics in \Cref{sec:semantics} is one generating \emph{stateful}
    traces in a standard operational semantics, and serves mostly as a
    convenient bridge for proving bisimulation. In \Cref{sec:stateless} we will
    define an equivalent, but more convenient \emph{stateless} semantics and we
    will see how to recover necessary state from program history as needed.
  \item We employ the stateless semantics as a collecting semantics and derive
    $\semlive{\wild}$ by calculational design \cite{Cousot:21}.
    Similar derivations will be made for a simple type system as well as for
    control flow analysis. \sg{Hopefully :)}
  \item Talk about prototype in Haskell?
  \item Related Work \Cref{sec:related-work}
\end{itemize}

%\section{Problem Statement}
\label{sec:problem}

By way of the poster child example of a compositional definition of \emph{usage
analysis}, we showcase how the operational detail available in traditional
denotational semantics is too coarse to substantiate a correctness criterion,
although the proof of (a weaker notion of) correctness is simple and direct.
While operational semantics observe sufficient detail to formulate a correctness
criterion, it is quite complicated to come up with a suitable inductive
hypothesis for the correctness proof.

%\subsection{Notation}
%
%Collection of stuff to explain, for now:
%\begin{itemize}
%  \item $\triangleq$ for defining an object (defn eq), rather than $=$
%  \item $\text{letrec}$
%    The (meta-level, math) notation
%    \[
%    \text{letrec}~l.~\many{x = rhs_x} ~ l = rhs_{l} ~ \many{y = rhs_y}~\text{in}~body
%    \]
%    where $l$ might occur freely in any $rhs_{\wild}$ and $body$, is syntactic sugar for
%    \[
%    snd(\lfp(\fn{(l,\wild)}{\text{let}~\many{x = rhs_x} ~ \many{y = rhs_y}~\text{in}~(rhs_{l},body)}))
%    \]
%    Where $\lfp$ is the least fixpoint operator and $snd(a,b) = b$. Clearly, this
%    desugaring's use of $\lfp$ is well-defined for its use on elements of the
%    powerset lattice $\UsgD$.
%
%    But \citep{Shivers:91} uses the similar $\text{whererec}$ and gets by without
%    ever explaining it, so we might as well.
%\end{itemize}

\subsection{Usage Analysis and Deadness, Intuitively}

\begin{figure}
\begin{minipage}{\textwidth}
\[\begin{array}{c}
 \arraycolsep=3pt
 \begin{array}{rrclcrrclcl}
  \text{Variables}    & \px, \py & ∈ & \Var        &     & \quad \text{Constructors} &        K & ∈ & \Con        &     & \text{with arity $α_K ∈ ℕ$} \\
  \text{Labels}       &     \lbl & ∈ & \Labels     &     & \quad \text{Values}       &      \pv & ∈ & \Val        & ::= & \highlight{\Lam{\px}{\pe}} \mid K~\many{\px}^{α_K} \\
  \text{Expressions}  &      \pe & ∈ & \Exp        & ::= & \multicolumn{6}{l}{\highlight{\slbl \px \mid \slbl \pv \mid \slbl \pe~\px \mid \slbl \Let{\px}{\pe_1}{\pe_2}} \mid \slbl \Case{\pe}{\SelArity}} \\
  \\[-0.5em]
 \end{array} \\
 \\[-0.5em]
 \begin{array}{rrclcl}
  \text{Scott Domain}      &  d & ∈ & \ScottD & =   & [\ScottD \to_c \ScottD]_\bot \\
  \text{Usage cardinality} &  u & ∈ & \Card & =   & \{ 0 ⊏ 1 ⊏ ω \} ⊂ ℕ_ω \\
  \text{Usage Domain}      &  d & ∈ & \UsgD & =   & \Var \to \Card \\
 \end{array} \quad
 \begin{array}{rcl}
   (ρ_1 ⊔ ρ_2)(\px) & = & ρ_1(\px) ⊔ ρ_2(\px) \\
   (ρ_1 + ρ_2)(\px) & = & ρ_1(\px) + ρ_2(\px) \\
   (u * ρ_1)(\px)   & = & u * ρ_1(\px) \\
 \end{array}
 \\[-0.5em]
\end{array}\]
\subcaption{Syntax of expressions and semantic domains}
  \label{fig:syntax}
\newcommand{\scalefactordenot}{0.92}
\scalebox{\scalefactordenot}{%
\begin{minipage}{0.49\textwidth}
\arraycolsep=0pt
\[\begin{array}{rcl}
  \multicolumn{3}{c}{ \ruleform{ \semscott{\wild} \colon \Exp → (\Var \to \ScottD) → \ScottD } } \\
  \\[-0.5em]
  \semscott{\px}_ρ & {}={} & ρ(\px) \\
  \semscott{\Lam{\px}{\pe}}_ρ & {}={} & d ↦ \semscott{\pe}_{ρ[\px ↦ d]} \\
  \semscott{\pe~\px}_ρ & {}={} & \begin{cases}
     f(ρ(x)) & \text{if $\semscott{\pe} = f$}  \\
     \bot   & \text{otherwise}  \\
   \end{cases} \\
  \semscott{\Letsmall{\px}{\pe_1}{\pe_2}}_ρ & {}={} &
    \begin{letarray}
      \text{letrec}~ρ'. & ρ' = ρ \mathord{⊔} [\px \mathord{↦} d_1] \\
                        & d_1 = \semscott{\pe_1}_{ρ'} \\
      \text{in}         & \semscott{\pe_2}_{ρ'}
    \end{letarray} \\
\end{array}\]
\subcaption{\relscale{\fpeval{1/\scalefactordenot}} Denotational semantics after Scott}
  \label{fig:denotational}
\end{minipage}%
\quad
\begin{minipage}{0.56\textwidth}
\arraycolsep=0pt
\[\begin{array}{rcl}
  \multicolumn{3}{c}{ \ruleform{ \semusg{\wild} \colon \Exp → (\Var → \UsgD) → \UsgD } } \\
  \\[-0.5em]
  \semusg{\px}_ρ & {}={} & ρ(\px) \\
  \semusg{\Lam{\px}{\pe}}_ρ & {}={} & ω*\semusg{\pe}_{ρ[\px ↦ \bot]} \\
  \semusg{\pe~\px}_ρ & {}={} & \semusg{\pe} + ω*ρ(\px)
    \phantom{\begin{cases}
       f(ρ(x)) & \text{if $\semscott{\pe} = f$}  \\
       \bot   & \text{otherwise}  \\
     \end{cases}} \\
  \semusg{\Letsmall{\px}{\pe_1}{\pe_2}}_ρ& {}={} & \begin{letarray}
      \text{letrec}~ρ'. & ρ' = ρ \mathord{⊔} [\px \mathord{↦} d_1] \\
                        & d_1 = [\px\mathord{↦}1] \mathord{+} \semscott{\pe_1}_{ρ'} \\
      \text{in}         & \semusg{\pe_2}_{ρ'}
    \end{letarray}
\end{array}\]
\subcaption{\relscale{\fpeval{1/\scalefactordenot}} Naïve usage analysis}
  \label{fig:usage}
\end{minipage}
}
\end{minipage}
  \label{fig:intro}
\caption{Connecting usage analysis to denotational semantics}
\end{figure}

\Cref{fig:syntax} defines the labelled syntax of a lambda calculus with
recursive let bindings and algebraic data types, reminiscent of
\citet{Sestoft:97}. The calculus is factored into \emph{administrative normal
form}, that is, the arguments of applications are restricted to be variables, so
the difference between call-by-name and call-by-value manifests purely in the
semantics of $\mathbf{let}$.
In this section, only the highlighted parts are relevant; we will ignore labels
and data types for brevity.

We give a standard call-by-name denotational semantics $\semscott{\wild}$ in
\Cref{fig:denotational} \citep{ScottStrachey:71}, assigning meaning to our
syntax by means of the infamous Scott domain $\ScottD$.
Squinting a bit, we find that it looks quite similar to the function
to its right in \Cref{fig:usage}, depicting a \emph{usage analysis}
$\semusg{\wild}$, a static analysis for estimating an upper bound on how often
a variable is evaluated. The given analysis is naïve in that its treatment of
function application assumes that every function deeply evaluates its argument.

Assuming that all program variables are distinct (a silent assumption from
here on throughout), the result of $\semusg{\wild}$ is an element $d ∈ \UsgD$,
an environment that maps to each variable an upper bound on its \emph{evaluation
cardinality}, that is, how often the variable is evaluated over the cause of any
of its activations.
Whenever $0 ⊏ d(x)$, we say that it is \emph{potentially live} in $d$ and
extend this meaning to a program $\pe$ whenever $\semusg{\pe} = d$.
Likewise, when $d(x) = 0$ we say that $x$ is \emph{dead} in $d$ and the programs
$d$ denotes, \eg, \emph{never evaluated}. In this way, $\semusg{\wild}$ can
be used to infer facts of the form ``$\pe$ never evaluates $x$'' from the
introduction.

\subsection{Denotational Deadness, Continuity and Divergence}

The \emph{requirement} (in the sense of informal specification) on an assertion
such as ``$x$ is dead'' in a program like $\Let{x}{\pe_1}{\pe_2}$ is that we
may rewrite to $\Let{x}{panic}{\pe_2}$ and perhaps even to $\pe_2$ without
observing any change in semantics. Doing so reduces code size and heap
allocation.

This can be made formal in the following definition of deadness in terms of
$\semscott{\wild}$:

\begin{definition}[Deadness]
  \label{defn:deadness}
  A variable $\px$ is \emph{dead} in an expression $\pe$ if and only
  if, for all $ρ ∈ \Var \to \ScottD$ and $d_1, d_2 ∈ \ScottD$, we have
  $\semscott{\pe}_{ρ[\px↦d_1]} = \semscott{\pe}_{ρ[\px↦d_2]}$.
  Otherwise, $\px$ is \emph{live}.
\end{definition}

Indeed, if we know that $x$ is dead, then the following equation justifies our
rewrite above: $\semscott{\Let{x}{\pe_1}{\pe_2}}_ρ = \semscott{\pe_2}_{ρ[x↦d]} =
\semscott{\pe_2}_ρ$ (for all $ρ$ and the suitable $d$).
So our definition of deadness is not only simple to grasp, but also simple to
exploit.

We can now try to prove our usage analysis correct as a liveness analysis in
terms of this notion of deadness. After a bit of trial and error, we could
arrive at the following theorem:

\begin{theorem}[$\semusg{\wild}$ is a correct potential liveness analysis]
  \label{thm:semusg-correct-live}
  Let $\pe$ be an expression and $\px$ a variable.
  Then $\px$ is dead in $\pe$ whenever
  there exists $\tr ∈ \Var \to \UsgD$ such that
  $\tr(\px) \not⊑ \semusg{\pe}_{\tr}$.
\end{theorem}
\begin{proof}
  By induction over $\pe$. The full proof can be found in
  \Cref{prf:semusg-correct-1}.
\end{proof}

Let us stop and reflect about this theorem for a bit.
Deadness is witnessed by a particular $\tr$ and it helps to think of this
witness as the ``diagonal'' $\tr(\px) \triangleq \fn{\py}{\ternary{\px =
\py}{1}{0}}$, because then the intuitive notion of deadness applies.%
\footnote{In fact, it can be proven that if \emph{any} $\tr$ exists, then the
diagonal is also a witness.}

It is surprising that the theorem does not relate $\tr$ with $ρ$; after all,
$ρ(\py)$ (for $\py \not= \px$) might be bound to the \emph{meaning} of an
expression that is potentially live in $\px$, such as $\semscott{\px}_{ρ'}$, and
we have no way to observe the dependency on $\px$ just through $ρ(\py)$.
The key is to realise that our notion of deadness varies $ρ(\px)$ (the meaning
of $\px$), but that does not vary $ρ(\py)$, because that only sees $ρ'(\px)$,
so for all intents and purposes, the proof may assume that $ρ(\py)$ is dead in
$\px$.
The analysis, on the other hand, encodes transitive deadness relationships via
$\tr(\px) ⊑ \tr(\py)$ in case $\px$ occurs in the RHS of a $\mathsf{let}$-bound
$\py$ to encode that deadness of $\py$ is a necessary condition for deadness of
$\px$.

The proof capitalises on the similarities in structure by using induction on the
program expression, hence it is simple and direct, at just under a page of
accessible prose. Often, such a proof needs to strengthen the induction
hypothesis for the application case, or prove admissability of a predicate to
apply fixpoint induction in the let case, but for deadness and our very simple
analysis we do not need to be so crafty.

Nevermind our confidence in the ultimate correctness of $\semusg{\wild}$,
note that our notion of deadness has a blind spot \wrt diverging computations:
A looping program is automatically dead in all its free variables, even though
any of them might influence which particular endless loop is taken.

This is not a curiosity of $\semusg{\wild}$; it also applies to the original
control-flow analysis work~\citep[p. 23]{Shivers:91} where it is remedied
by the introduction of a \emph{non-standard semantic interpretation} that
assigns meaning to diverging programs where the denotational semantics only
says $\bot$. Credibility of this approach solely rests on the structural
similarity to the standard denotational semantics.

So the issue is not with $\semusg{\wild}$ but with traditional denotational
semantics because it (necessarily) assigns $\bot$ to any diverging computation.
Furthermore, as is often done, $\semscott{\wild}$ abuses $\bot$ as a collecting
pool for error cases.
This shows in the following example:
$x$ is dead in $(\Lam{y}{\Lam{z}{z}})~x$, but rewriting
$\Let{x}{\pe_1}{(\Lam{y}{\Lam{z}{z}})~x}$ to $(\Lam{y}{\Lam{z}{z}})~x$
introduces a scoping error, a change that is not observable under
$\semscott{\wild}$.
We could take inspiration in the work of \citet{Milner:78}
and navigate around the issue by introducing a $\mathbf{wrong}$ denotation for
errors which is propagated strictly; then we would notice when we optimise a
looping program into one that has a scoping error (the only kind of stuckness
that our calculus admits without data types).

However, $\bot$ is still there as the denotation of diverging computations;
hence a predicate such as ``Denotation $d$ will get stuck and not diverge'' is
not an admissable one, because an admissable predicate would be true for $\bot$.

\subsection{Evaluation Cardinality and Call-by-need}
Blind spots notwithstanding, the notion of deadness above is quite reasonable.
But our usage analysis infers more detailed cardinality information; for
example, it can infer whether a binding is evaluated at most once.
This information can be useful under call-by-need to omit pushing of update
frames~\citep{cardinality-ext}.%
\footnote{A more useful application of the ``at most once'' cardinality is the
notion of a \emph{one-shot} lambda~\citep{cardinality-ext}, a function which is
called at most once for every activation, because it allows floating of heap
allocations from a hot code path into cold function bodies.
Simplicity prohibits $\semusg{\wild}$ from inferring such properties.}
Thus, our usage analysis should satisfy the following generalisation of
\Cref{thm:semusg-correct-live}:

\begin{theorem}[Correctness of $\semusg{\wild}$]
  \label{thm:semusg-correct-2}
  Let $\pe$ be an expression and $\px$ a variable.
  Then $\pe$ evaluates $\px$ at most $u$ times whenever
  there exists $\tr ∈ \Var \to \UsgD$ such that
  $(u+1)*\tr(\px) \not⊑ \semusg{\pe}_{\tr}$.
\end{theorem}

Unfortunately, our denotational semantics does not allow us to express the
operational property ``$\pe$ evaluates $\px$ at most $u$ times'', so
this theorem cannot be proven correct.

% We should probably mvoe this RElated Work? Don't want to discuss it here
%The problem of observable cardinality also comes up in Quantitative Type
%Theory~\citep{Atkey:18}, where the solution is to give a categorical
%semantics that postulates observability of cardinality in a suitable
%\emph{$R$-Quantitative Category with Families} without giving a concrete
%model.

\begin{figure}
\[\begin{array}{c}
 \begin{array}{rrclcl}
  \text{LK States}     & σ   & ∈ & \States        & =      & \Controls \times \Environments \times \Heaps \times \Continuations \\
  \text{Controls}      & \pc & ∈ & \Controls      & =      & \Exp \\
  \text{Environments}  & ρ   & ∈ & \Environments  & =      & \Var \pfun \Addresses \\
  \text{Addresses}     & \pa & ∈ & \Addresses     & \simeq & ℕ \\
  \text{Heaps}         & μ   & ∈ & \Heaps         & =      & \Addresses \pfun \Environments \times \Exp \\
  \text{Continuations} & κ   & ∈ & \Continuations & ::=    & \StopF \mid \ApplyF(\pa) \pushF κ \mid \SelF(ρ,\SelArity) \pushF κ \mid \UpdateF(\pa) \pushF κ \\
 \end{array} \\
  \\[-0.5em]
\end{array}\]

\newcolumntype{L}{>{$}l<{$}} % math-mode version of "l" column type
\newcolumntype{R}{>{$}r<{$}} % math-mode version of "r" column type
\newcolumntype{C}{>{$}c<{$}} % math-mode version of "c" column type
\resizebox{\textwidth}{!}{%
\begin{tabular}{LR@{\hspace{0.4em}}C@{\hspace{0.4em}}LL}
\toprule
\text{Rule} & σ_1 & \smallstep & σ_2 & \text{where} \\
\midrule
\BindT & (\Let{\px}{\pe_1}{\pe_2},ρ,μ,κ) & \smallstep & (\pe_2,ρ',μ[\pa↦(ρ',\pe_1)], κ) & \pa \not∈ \dom(μ),\ ρ'\! = ρ[\px↦\pa] \\
\AppIT & (\pe~\px,ρ,μ,κ) & \smallstep & (\pe,ρ,μ,\ApplyF(\pa) \pushF κ) & \pa = ρ(\px) \\
\CaseIT & (\Case{\pe}{\Sel},ρ,μ,κ) & \smallstep & (\pe,ρ,μ,\SelF(ρ,\Sel) \pushF κ) & \\
\LookupT & (\px, ρ, μ, κ) & \smallstep & (\pe, ρ', μ, \UpdateF(\pa) \pushF κ) & \pa = ρ(\px),\ (ρ',\pe) = μ(\pa) \\
\AppET & (\Lam{\px}{\pe},ρ,μ, \ApplyF(\pa) \pushF κ) & \smallstep & (\pe,ρ[\px ↦ \pa],μ,κ) &  \\
\CaseET & (K'~\many{y},ρ,μ, \SelF(ρ',\Sel) \pushF κ) & \smallstep & (\pe_i,ρ'[\many{\px_i ↦ \pa}],μ,κ) & K_i = K',\ \many{\pa = ρ(\py)} \\
\UpdateT & (\pv, ρ, μ, \UpdateF(\pa) \pushF κ) & \smallstep & (\pv, ρ, μ[\pa ↦ (ρ,\pv)], κ) & \\
\bottomrule
\end{tabular}
} % resizebox
\caption{Lazy Krivine transition semantics $\smallstep$}
  \label{fig:lk-semantics}
\end{figure}

Let us try a different approach then and define a stronger notion of deadness
in terms of a small-step operational semantics such as the Mark II machine of
\citet{Sestoft:97} given in \Cref{fig:lk-semantics}, the semantic ground truth
for this work. (A close sibling for call-by-value would be a CESK machine
\citep{Felleisen:87} or a simpler derivative thereof.) It is a variant of
the Lazy Krivine (LK) machine implementing call-by-need, so for a meaningful
comparison to $\semscott{\wild}$, we ignore rules $\CaseIT, \CaseET, \UpdateT$
and the pushing of update frames in $\LookupT$ for now to recover a call-by-name
Krivine machine with explicit heap addresses.%
\footnote{Note that discarding update frames makes the heap entries immutable,
which makes the explicit heap unnecessary. Of course, for call-by-name we would
not need a heap to begin with, but the point is to get a glimpse at the effort
necessary for call-by-need.}

The configurations $σ$ in this transition system resemble abstract machine
states, consisting of control expression $\pe$, an environment $ρ$ mapping
lexically-scoped variables to their current heap address, a heap $μ$ listing a
closure for each address, and a stack of continuation frames $κ$.

The notation $f ∈ A \pfun B$ used in the definition of $ρ$ and $μ$ denotes a
finite map from $A$ to $B$, a partial function where the domain $\dom(f)$ is
finite and $\rng(f)$ denotes its range.
The literal notation $[a_1↦b_1,...,a_n↦b_n]$ denotes a finite map with domain
$\{a_1,...,a_n\}$ that maps $a_i$ to $b_i$. Function update $f[a ↦ b]$
maps $a$ to $b$ and is otherwise equal to $f$.

The initial machine state for a closed expression $\pe$ is given by the
injection function $inj(\pe) = (\pe,[],[],\StopF)$ and
the final machine states are of the form $(\pv,\wild,\wild,\StopF)$.
We bake into $σ$ the simplifying invariant of \emph{well-addressedness}: Any
address $\pa$ occuring in $ρ$, $κ$ or the range of $μ$ must be an element of
$\dom(μ)$. It is easy to see that the transition system maintains this invariant
and that it is still possible to observe scoping errors which are thus confined
to lookup in $ρ$.

Now we are able to define a notion of ``strong deadness'':

\begin{definition}[Deadness, Mark II]
  \label{defn:deadness2}
  Let $\pe$ be an expression and $\px$ a variable.
  $\px$ is \emph{dead} in $\pe$ if and only if
  for any evaluation context $(ρ,μ,κ)$ and expressions $\pe_1,\pe_2$
  (where $\px$ does not occur in the context)
  the sequences of transitions $(\Let{\px}{\pe_1}{\pe},ρ,μ,κ) \smallstep^*$
  and $(\Let{\px}{\pe_2}{\pe},ρ,μ,κ) \smallstep^*$ operate in lockstep.
  Otherwise, $\px$ is \emph{live}.
\end{definition}

This definition captures diverging behaviors correctly and straightforwardly
legitimises the transformation we want to perform, without any mention of
addresses. It is however unwieldy in a correctness proof due to its use of
bisimulation, so a bit of rejigging is in order:

\begin{lemma}[Without proof]
  $\px$ is dead in $\pe$ if and only if for any evaluation context $(ρ,μ,κ)$
  and $\pa \not∈ \dom(μ)$ there exists no sequence of transitions
  $(\pe,ρ[\px↦\pa],μ[\pa↦([],\Lam{z}{z})],κ) \smallstep^* (\py,ρ',μ',κ')$ such
  that $ρ'(\py) = \pa$.
\end{lemma}

This property is a bit easier to handle in a proof.
Unfortunately, it is still not compositional in $\pe$: Consider a variable
occurrence $y$; is $x$ dead in $y$? That depends on which expression $y$ is
bound to in the heap, but our deadness predicate has no notion of making
assumptions about free variables.
Consequently, it is impossible to prove that $\semusg{\pe}$ satisfies
\Cref{thm:semusg-correct-live} by direct structural induction on $\pe$ (in a way
that would be useful to the proof).

Instead, such proofs are often conducted by induction over the reflexive
transitive closure of the transition relation.
For that it is necessary to give an inductive hypothesis that considers
environments, stacks and heaps.
One way is to extend the analysis function $\semusg{\wild}$ to entire
configurations and then prove that if $σ_1 \smallstep σ_2$ we have $\semusg{σ_2}
⊑ F(\semusg{σ_1})$, where $F$ is the abstraction of the particular transition
rule taken and is often left implicit.
This is a daunting task, for multiple reasons:
First off, $\semusg{\wild}$ might be quite complicated in practice and extending
it to entire configurations multiplies this complexity.
Secondly, $\semusg{\wild}$ makes use of fixpoints in the let case and
undoubtedly needs some more fixpoints in its extension to the heap,
so $\semusg{σ_2} ⊑ F(\semusg{σ_1})$ relates fixpoints that are ``out of sync'',
implying a need for fixpoint induction for every transition that touches
the heap.

In call-by-need, there will be a fixpoint between the heap and stack due to
update frames acting like heap bindings whose right-hand side is under
evaluation (a point that is very apparent in the contextual semantics of
\citet{Ariola:95}), so fixpoint induction needs to be applied \emph{at every
case of the proof}, diminishing confidence in correctness unless the proof is
fully mechanised.

For an analogy with type systems: What we just tried is like attempting a proof
of preservation by referencing the result of an inference algorithm rather than
the declarative type system. So what is often done instead is to define a declarative and
more permissive \emph{correctness relation} $C(σ)$ to prove preservation $C(σ_1)
\Longrightarrow C(σ_2)$ (\eg, that $C$ is \emph{logical} \wrt $\smallstep$).
$C$ is chosen such that
  (1) it is strong enough to imply the property of interest (deadness)
  (2) it is weak enough that it is implied by the analysis result for an initial state ($\tr(\px) \not⊑ \semusg{\pe}_{\tr}$).

Examples of this approach are the
``well-annotatedness'' relation in \citep[Lemma 4.3]{cardinality-ext} or
$\sim_V$ in \citep[Theorem 2.21]{Nielson:99}).
We found it quite hard to come up with a suitable ad-hoc correctness relation
and postpone futher discussion to \Cref{sec:abstractions}, where the full
correctness relation in \Cref{thm:semusg-correct-2} and its proof is derived by
abstract interpretation~\citep{Cousot:21}.

Often, correctness proofs do not need to keep track about which address
is an activation of which let-bound program variable, in which case the
distinction between addresses and variables is unnecessary and the
environment component vanishes.
The stack can often be reflected back into the premises of the judgment
rules, so that the system distinguishes \emph{instruction transitions}
from \emph{search transitions}, a distinction which is made explicit in a
\emph{contextual semantics}.
Applying both these translations yields a CS machine~\citep{Felleisen:87}.
For call-by-need, the evaluation context corresponding to an update frame
is neither obvious nor simple~\citep{Ariola:95}.
For effect-free call-by-value and call-by-name calculi, the heap becomes
immutable and variables can be substituted immediately for their right-hand
sides/arguments rather than delaying lookup to the variable case, abolishing the
need for the heap altogether and yielding a contextual semantics where
states are a simple expression.

These refactorings are with the ultimate goal of simplifying proofs:
Instead of defining an coinductive well-formedness predicate on the heap, we
prove a substitution lemma.
Instead of a well-formedness predicate for the stack, we appeal to the
well-formedness of the search transition rule.

\subsection{Abstracting Abstract Machines}

Another way to work around the structure gap is to adopt the structure of the
semantics in the analysis; this is done in the Abstracting Abstract
Machines work \citep{aam}.
To our knowledge, its exclusive application seems to be control-flow analysis
\citep{Shivers:91}, so that the analyses and optimisations that follow do not
need to reason about arbitrary function call structure and can apply traditional
intraprocedural analysis techniques that are well-explored in the imperative
world.
Unfortunately, control-flow information is often invalidated during compiler
passes and we would expect re-running or interleaving the analysis after or
during each pass to be quite costly.

\section{Semantics}
\label{sec:semantics}

\subsection{Labelled Syntax}

Recall \Cref{fig:syntax}; it defines the syntax of $Λ$, an untyped lambda
calculus with recursive let bindings and data types in the style of
\citet{Launchbury:93} and \citet{Sestoft:97}.
Any (sub-)expression of $Λ$ has a unique \emph{label} (think of it as the AST node's
pointer identity) that we usually omit. For example, a correct labelling of
$\Let{x}{f~y}{f~x}$ would be
\[
  (\slbln{1} \Let{x}{(\slbln{2} (\slbln{3} f)~y)}{(\slbln{4} (\slbln{5} f)~x)}).
\]
Labels are there so that we do not conflate the (otherwise structurally equal)
sub-terms $(\slbln{3} f)$ and $(\slbln{5} f)$ as equivalent. This is an important
distinction for, \eg, control-flow analysis. Since labels introduce excessive
clutter, we will omit them unless they are distinctively important. If anything,
labels make it so that everything ``works as expected''.

\subsection{Transition System}

\Cref{fig:lk-semantics} gave an operational semantics for $Λ$ in terms of
a small-step transition system closest to the lazy Krivine machine
\citep{AgerDanvyMidtgaard:04} for Launchbury's language.
It is worth having a second look at the workings of our Gold Standard.

When the control expression $\mathit{ctrl}(σ)$ of a state $σ$ is a value $\pv$, we
call $σ$ a \emph{return} state and say that the continuation $\mathit{cont}(σ)$ drives
evaluation.
Otherwise, $σ$ is an \emph{evaluation} state and $\mathit{ctrl}(σ)$ drives evaluation.
The entries in the heap $μ$ are \emph{closures} of the form $(ρ,e)$, where the
environment $ρ$ closes over the expression $e$.
Finally, the $\mathit{cont}(σ)$ lists actions to be taken in a return state, such as
applying the result to an argument address or updating a heap entry with its
value.

Heap entries are introduced via $\BindT$ transitions under a \emph{fresh} address
$\pa \not∈ \dom(μ)$ that we call an \emph{activation} of the let-bound variable
$\px$. The lexical activation of every variable in scope is maintained
in $ρ$. The $\AppIT$ rule pushes an \emph{application frame} with the address of
the argument variable onto the stack, while the rule $\LookupT$ pushes an
\emph{update frame} with the address of the variable the heap entry of which is
accessed. When a return state is reached, the original heap entry is overwritten
with the value in the control.

Let us conclude with an example trace in this transition system, evaluating
$\pe \triangleq \Let{i}{\Lam{x}{x}}{i~i}$ to completion:
\[\begin{array}{c}
  \arraycolsep2pt
  \begin{array}{clclclcl}
             & (\pe, [], [], \StopF)         & \smallstep & (i~i, ρ_1, μ, \StopF)
             & \smallstep & (i, ρ_1, μ, κ_1) & \smallstep & (\Lam{x}{x}, ρ_1, μ, κ_2)
             \\
  \smallstep & (\Lam{x}{x}, ρ_1, μ, κ_1)     & \smallstep & (x, ρ_2, μ, \StopF) & \smallstep & (\Lam{x}{x}, ρ_1, μ, κ_3)
             & \smallstep & (\Lam{x}{x}, ρ_1, μ, \StopF) \\
  \end{array} \\
  \\[-0.5em]
  \quad \text{where} \quad \begin{array}{lll}
  ρ_1 = [i ↦ \pa_1] & ρ_2 = [i ↦ \pa_1, x ↦ \pa_1] & μ = [\pa_1 ↦ (ρ_1,\Lam{x}{x})] \\
  κ_1 = \ApplyF(\pa_1) \pushF \StopF & κ_2 = \UpdateF(\pa_1) \pushF κ_1 & κ_3 = \UpdateF(\pa_1) \pushF \StopF
  \end{array}
\end{array}\]

\subsection{Guarded Recursive Types}

The key to avoiding Domain Theory and $\bot$ as a semantic domain for our work
is use of a total type theory with \emph{guarded recursive types}, such as
Guarded Dependent Type Theory (GDTT) \citep{gdtt}.%
\footnote{Of course, in reality we are just using GDTT as a meta
language~\citep{Moggi:07} with a known domain theoretic interpretation in terms
of the topos of trees~\citep{gdtt}.
Thanks to GDTT this meta language is sufficiently expressive as a logic to
express proofs, though, justifying the view that we are extending ``math''
with the ability to conveniently reason about infinite data without needing to
think about topology and approximation directly.}
The fundamental construct of this theory is addition of the ``later''
modality $\later[κ]$ which allows to define coinductive data types with negative
occurrences, as first realised by \citet{Nakano:00}.
Whereas previous theories of coinduction require syntactic productivity
checks~\citep{Coquand:94}, requiring tiresome constraints on the form of guarded
recursive functions, the appeal of GDTT is that productivity is instead proven
semantically, in the type system.

The way that GDTT achieves this is roughly as follows: The type $\later T$
represents data of type $T$ that will become available after a finite amount
of computation, such as unrolling one layer of a fixpoint definition.
It comes with a general fixpoint combinator $\fix : \forall A.\ (\later A \to
A) \to A$ that can be used to define both coinductive \emph{types} (via guarded
recursive functions on the universe of types~\citep{BirkedalMogelbergEjlers:13})
as well as guarded recursive \emph{terms} inhabiting said types.
The classic example is that of coinductive streams:
\[
  Str = ℕ \times \later Str \qquad ones = \fix (r : \later Str).\ (1,r),
\]
where $ones : Str$ is the constant stream of $1$.
In particular, $Str$ is the fixpoint of a locally contractive functor $F(X) =
ℕ \times \later X$.
According to \citet{BirkedalMogelbergEjlers:13}, any type expression in simply
typed lambda calculus defines a locally contractive functor as long as any
occurrence of $X$ is under a $\later$, so we take that as the well-formedness
criterion of coinductive types in this work.

As a type constructor, $\later$ is an applicative
functor~\citep{McBridePaterson:08} via functions
\[
  \purelater : \forall A.\ A \to \later A \qquad \wild \aplater \wild : \forall A,B.\ \later (A \to B) \to \later A \to \later B,
\]
allowing us to apply a familiar framework of reasoning around $\later$.
In order not to obscure our work with pointless symbol pushing
in, \eg, \Cref{fig:semst}, we will often omit the idiom
brackets~\citep{McBridePaterson:08} $\idiom{}$ that would be necessary in a
mechanised form to indicate where the $\later$ ``effects'' happen; they should
be easy to infer anyway.
\sg{Perhaps we should add the elaborated version in the Appendix.}

\begin{figure}
\[\begin{array}{c}
 \begin{array}{rrclclrrclcl}
  \text{States}        & σ   & ∈ & \States        & =      & \Controls \times \Heaps
  &
  \text{Environments}  & ρ   & ∈ & \Environments  & =      & \Var \pfun \later\StateD
  \\
  \text{Controls}      & \pc & ∈ & \Controls      & ::=    & \pe \mid (\lbl, v)
  &
  \text{Heaps}         & μ   & ∈ & \Heaps         & =      & \Addresses \pfun \later\StateD
  \\
  \\[-0.5em]
 \end{array} \\
 \begin{array}{rrclclrrclcl}
  \text{Stateful traces} & τ      & ∈          & \STraces & ::= & \goodend{σ} \mid \stuckend{σ} \mid σ \cons τ^{\later}
  &
  \text{Delayed trace} & τ^{\later} & ∈ & \later\STraces &   &
  \\
  \text{Stateful domain} & d & ∈ & \StateD & = & \Heaps \to \STraces
  &
  \text{Delayed element} & d^{\later} & ∈ & \later\StateD &   &
  \\
  \\[-0.5em]
 \end{array} \\
 \begin{array}{rrclcl}
  \text{Semantic values} & v & ∈ & \StateV & ::= & \FunV(f ∈ (\later\StateD \to \later\StateD)) \mid \ConV(K,\many{d^{\later}}^{α_K}) \\
 \end{array} \\
  \\[-0.5em]
\end{array}\]
\[\begin{array}{c}
 \begin{array}{rcl}
  \multicolumn{3}{c}{ \ruleform{
    \begin{array}{c}
      (\betastep) : \later\STraces \to (\later\States \pfun \later\STraces) \to \later\STraces \quad  \mathit{ret} : (\Val \times \StateV) \to \StateD \\
      \mathit{deref} : \Addresses \to \later\StateD \quad \mathit{apply} : \later\STraces \to \later\StateD \to \later\STraces \\
      \mathit{select} : \later\STraces \to ((\later\STraces)^{α_K} \pfun \later\STraces)^* \to \later\STraces \\
    \end{array}
  }} \\
  \\[-0.5em]
  τ \betastep f & = & \begin{cases}
      τ :: f(σ) & \text{$τ = ... \cons \goodend{σ}$ and $σ ∈ \dom(f)$} \\
      τ' :: \stuckend{σ} & \text{$τ = τ' \cons \goodend{σ}$ and $σ \not∈ \dom(f)$} \\
      τ & \text{otherwise} \\
    \end{cases} \\
  \\[-0.5em]
  \mathit{ret}(\lbl,v)(μ) & = & \goodend{((\lbl,v),μ)} \\
  \\[-0.5em]
  \mathit{deref}(\pa)(μ) & = & μ(\pa)(μ) \betastep \fn{((\lbl,v),μ)}{\goodend{((\lbl,v),μ[\pa ↦ \mathit{ret}(\lbl,v)])}} \\
  \\[-0.5em]
  \mathit{apply}(τ,d) & = & τ \betastep \fn{((\wild,\FunV(f)),μ)}{f(d)(μ)} \\
  \\[-0.5em]
  \mathit{select}(τ,\many{f}) & = & τ \betastep \fn{((\wild,\ConV(K_s,\many{d_s})),μ)}{f_s(\many{d_s})} \\
  \\[-0.5em]
  \multicolumn{3}{c}{ \ruleform{ \semst{\wild} \colon \Exp → (\Var \pfun \later\StateD) → \StateD } } \\
  \\[-0.5em]
  \semst{\px}_ρ(μ) & = & \begin{cases}
    (\px,μ) \cons ρ(\px)(μ) & \px ∈ \dom(ρ) \\
    \stuckend{(\px,μ)} & \text{otherwise}
    \end{cases} \\
  \\[-0.5em]
  \semst{\slbl \Lam{\px}{\pe}}_ρ & = & \mathit{ret}(\lbl,\FunV(\fn{d}{\semst{\pe}_{ρ[\px↦d]}})) \\
  \\[-0.5em]
  \semst{\pe~\px}_ρ(μ) & = & \begin{cases}
      (\pe~\px,μ) \cons \mathit{apply}(\semst{\pe}_ρ(μ),ρ(\px)) & \px ∈ \dom(ρ) \\
      \stuckend{(\pe~\px,μ)} & \text{otherwise} \\
    \end{cases} \\
  \\[-0.5em]
  \semst{\Let{\px}{\pe_1}{\pe_2}}_ρ(μ) & = & \begin{letarray}
    \text{let} & ρ' = ρ[\px ↦ \mathit{deref}(\pa)] \quad \text{where $\pa \not∈ \dom(μ)$} \\
               & μ' = μ[\pa ↦ \semst{\pe_1}_{ρ'}] \\
    \text{in}  & (\Let{\px}{\pe_1}{\pe_2},μ) \cons \semst{\pe_2}_{ρ'}(μ') \\
  \end{letarray} \\
  \\[-0.5em]
  \semst{\slbl K~\many{\px}}_ρ & = & \mathit{ret}(\lbl,\ConV(K,\many{\semst{\px}_ρ})) \\
  \\[-0.5em]
  \semst{\pe@(\Case{\pe_s}{\Sel[r]})}_ρ(μ) & = & (\pe,μ) \cons \mathit{select}(\semst{\pe_s}_ρ(μ),\many{\fn{\many{d}^{α_K}}{\semst{\pe_r}_{ρ[\many{\px↦d}]}}})  \\
 \end{array}
  \\[-0.5em]
\end{array}\]
\caption{Structural call-by-need stateful trace semantics $\semst{-}$}
  \label{fig:semst}
\end{figure}

\subsection{Definition}

\Cref{fig:semst} finally gives the definition for $\semst{\wild}$, a function
defined by structural recursion on an input expression $\pe$. Given a denotation
for free variables $ρ$, $\semst{\pe}_ρ$ assigns $\pe$ a denotation $d$ in terms of
the semantic domain $\StateD$ of stateful call-by-need trace functions.
If such a trace function is supplied the heap $μ$ just before $\pe$ takes
control, then $d(μ)$ is a trace $τ$ starting at $\pe$ in that heap $μ$.
If evaluation of $\pe$ terminates, then $τ$ will be a finite list of states
ending with $\goodend{σ}$ for some return state $σ$. Otherwise, it might be
finite but stuck ($\stuckend{σ}$), or diverge without ever leaving $\pe$, in
which case $τ$ will be infinite.

Compared to the LK transition semantics, the most striking difference in state
structure is the lack of environment and stack components. The former is
maintained as a parameter to $\semst{\wild}$, while the latter is implicit in
the recursive call structure.
As we have seen in \Cref{sec:problem} at the example of $\semscott{\wild}$,
this reflection of machine state into ``math'' bears great potential for
program analysis, one we will exploit in \Cref{sec:abstractions}.
The second difference is that the environment $ρ$ and the heap $μ$ do not map to
addresses or syntactic closures but to delayed semantic values $\later \StateD$,
offering further abstraction possibilities compared to the rigid and indirect
syntactic domain.
The third difference is in control structure which explicitly signals the
distinction between evaluation states and return states.
In a return state $((\lbl,v),μ)$, the syntactic value's label $\lbl$ travels
together with a \emph{semantic} value $v ∈ \StateV$.
Crucially, this allows the embedding of function values $\FunV(f)$ in the
lambda case $\semst{\Lam{\px}{\pe}}$, enabling a compositional definition of the
application case $\semst{\pe~\px}$, just as in $\semscott{\wild}$.

It is worth noting that without guarded recursive types, the definition of
$\StateV$ would not be well-founded; a nuisance usually solved via restriction
to continuous functions on Scott domains and solving the corresponding domain
equation.

Note that we will continue to use the cons notation $τ \cons τ'$ quite liberally
to denote concatenation of traces. Doing so is unambiguous (because states are
distinct from traces) and well-defined via the following guarded fixpoint:
\[
  \fix (f : \later (\STraces \to \STraces \to \STraces)).\ λ(τ : \STraces)~(τ' : \STraces).\ \begin{cases}
    \stuckend{σ} & τ = \stuckend{σ} \\
    σ \cons τ'   & τ = \goodend{σ} \\
    σ \cons f \aplater τ^{\later} \aplater \purelater τ' & τ = σ \cons τ^{\later} \\
  \end{cases}
\]
A similar productive definition can be given for $\betastep$, which is to be
understood as yielding from the input trace $τ$ until it hits either end case of
the trace.

Intuitively, the heap lists the denotation of the expression bound at a
particular address, while the environment passed to $\semst{\wild}$ assigns each
free variable $\px$ a $\mathit{deref}(\pa)$ action for the particular address that $\px$
should be bound to.

Let us now understand $\semst{\wild}$ by way of evaluating the example program
from earlier, $\pe \triangleq \Let{i}{\Lam{x}{x}}{i~i}$:
\[\begin{array}{ll}
  \newcommand{\myleftbrace}[4]{\draw[mymatrixbrace] (m-#2-#1.west) -- node[right=2pt] {#4} (m-#3-#1.west);}
  \vcenter{\hbox{$
    \begin{tikzpicture}[mymatrixenv,anchor=center]
      \matrix [mymatrix] (m)
      {
        1 & (\pe, []) \cons {} & \hspace{3.7em} & \hspace{4.2em} & \hspace{3.9em} & \hspace{2.5em} \\
        2 & (i~i, μ) \cons {} & & & & \\
        3 & (i, μ) \cons {} & & & & \\
        4 & ((\Lam{x}{x}, \FunV(f)), μ) \cons {} & & & & \\
        5 & ((\Lam{x}{x}, \FunV(f)), μ) \cons {} & & & & \\
        6 & (x, μ) \cons {} & & & & \\
        7 & ((\Lam{x}{x}, \FunV(f)), μ) \cons {} & & & & \\
        8 & \goodend{((\Lam{x}{x}, \FunV(f)), μ)} & & & & \\
      };
      % Braces, using the node name prev as the state for the previous east
      % anchor. Only the east anchor is relevant
      \foreach \i in {1,...,\the\pgfmatrixcurrentrow}
        \draw[dotted] (m.east|-m-\i-\the\pgfmatrixcurrentcolumn.east) -- (m-\i-2);
      \myleftbrace{3}{1}{8}{$\semst{\pe}_{[]}$}
      \myleftbrace{4}{1}{2}{$\BindT$}
      \myleftbrace{4}{2}{8}{$\semst{i~i}_{ρ_1}$}
      \myleftbrace{5}{2}{3}{$\AppIT$}
      \myleftbrace{5}{3}{5}{$\semst{i}_{ρ_1}$}
      \myleftbrace{5}{5}{6}{$\AppET$}
      \myleftbrace{5}{6}{8}{$\semst{x}_{ρ_2}$}
      \myleftbrace{6}{3}{4}{$\LookupT$}
      \myleftbrace{6}{4}{5}{$\UpdateT$}
      \myleftbrace{6}{6}{7}{$\LookupT$}
      \myleftbrace{6}{7}{8}{$\UpdateT$}
  \end{tikzpicture}
  $}} &
  \!\!\!\!\text{where}  \begin{array}{ll}
  ρ_1 = [i ↦ \mathit{deref}(\pa_1)] & \\
  ρ_2 = ρ_1[x ↦ \mathit{deref}(\pa_1)] &  \\
  μ = [\pa_1 ↦ \semst{\Lam{x}{x}}_{ρ_1}] & \\
  f = d \mapsto \semst{\px}_{ρ_1[\px↦d]}
  \end{array}
\end{array}\]
The annotations to the right of the trace can be understood as denoting the
``call graph'' of $\semst{\pe}_{[]}$, with the corresponding LK transitions
as leaves.
Evaluation begins with a $\BindT$ transition from state 1 to state 2.
A fresh address $\pa_1$ is allocated for variable $i$ and the heap is extended
with $\semst{\Lam{x}{x}}_{ρ_1}$.
It is interesting to realise that this process does not involve a fixpoint
despite the recursive semantics of let.
Of course, the self-application in $\mathit{deref}$ does the job just as well, as we will
see in due course.

Evaluation recurses into the body $\semst{i~i}_{ρ_1}$ in the extended
environment $ρ_1$ to produce state 2, also yielding another $\AppIT$ transition
into $\semst{i}_{ρ_1}$.
Note that the final state 5 of $\semst{i}_{ρ_1}$ will later be fed (via
$\betastep$) into anonymous function in $\mathit{apply}$.
This scheme is quite common: Continuation items of the transition semantics
(``data'') are reflected into the call stack of the trace semantics (``code'').

$\semst{i}_{ρ_1}$ guides the trace through a heap lookup:
A $\LookupT$ goes straight into the $\mathit{deref}(\pa_1)$ action stored in the
environment entry for $i$, which performs the aforementioned self-application
to run the heap action $μ(\pa_1) = \semst{\Lam{x}{x}}_{ρ_1}$ starting in the
current heap $μ$.
Evaluation of $\semst{\Lam{x}{x}}_{ρ_1}$ terminates immediately in return state
4.
Returning to $\mathit{deref}(\pa_1)$, we witness for the first time a reduction
operation via $\betastep$:
Since the trace terminates, the anonymous function in $\mathit{deref}(\pa_1)$ is called
(in $\betastep$) with state 4, making an $\UpdateT$ transition to state 5.
Note that there is no observable change to the heap $μ$ because
$\mathit{ret}(\Lam{x}{x}, \FunV(f))$ is precisely the same as $\semst{\Lam{x}{x}}_{ρ_1}$.

After the heap update we leave $\semst{i}$ in return state 5, where $\betastep$
yields to the anonymous function in $\mathit{apply}$ (from the call to
$\semst{i~i}_{ρ_1}$), yielding another $\AppET$ reduction and giving control
to $f(ρ_1(i)) = \semst{x}_{ρ_1[x ↦ ρ_1(i)]}$.
Since $x$ is an alias for $i$, steps 6 to 8 just repeat the same same heap
update sequence we observed in steps 3 to 5, concluding the example.

It is useful to review another example involving an observable heap
update. The following trace begins right before the heap update occurs in
$\Let{i}{(\Lam{y}{\Lam{x}{x}})~i}{i~i}$, that is, after reaching the value
in $\semst{(\Lam{y}{\Lam{x}{x}})~i}_{ρ_1}$:
\[\begin{array}{ll}
  \newcommand{\myleftbrace}[4]{\draw[mymatrixbrace] (m-#2-#1.west) -- node[right=2pt] {#4} (m-#3-#1.west);}
  \vcenter{\hbox{$
    \begin{tikzpicture}[baseline={-0.5ex},mymatrixenv]
      \matrix [mymatrix] (m)
      {
        1 & ((\Lam{x}{x},\FunV(f)), μ_1) \cons {} & \hspace{4em} & \hspace{4em} & \hspace{2.5em} \\
        2 & ((\Lam{x}{x},\FunV(f)), μ_2) \cons {} & & & \\
        3 & (x, μ_2) \cons {} & & & \\
        4 & ((\Lam{x}{x}, \FunV(f)), μ_2) \cons {} & & & \\
        5 & \goodend{((\Lam{x}{x}, \FunV(f)), μ_2)} & & & \\
      };
      % Braces, using the node name prev as the state for the previous east
      % anchor. Only the east anchor is relevant
      \foreach \i in {1,...,\the\pgfmatrixcurrentrow}
        \draw[dotted] (m.east|-m-\i-\the\pgfmatrixcurrentcolumn.east) -- (m-\i-2);
      \myleftbrace{3}{1}{5}{$\semst{i~i}_{ρ_1}$}
      \myleftbrace{4}{1}{2}{$\semst{i}_{ρ_1}$}
      \myleftbrace{4}{2}{3}{$\AppET$}
      \myleftbrace{4}{3}{5}{$\semst{x}_{ρ_3}$}
      \myleftbrace{5}{1}{2}{$\UpdateT$}
      \myleftbrace{5}{3}{4}{$\LookupT$}
      \myleftbrace{5}{4}{5}{$\UpdateT$}
    \end{tikzpicture}
  $}} &
  \!\!\!\text{where} \begin{array}{l}
  ρ_1 = [i ↦ \pa_1] \\
  ρ_2 = [i ↦ \pa_1, y ↦ \pa_1] \\
  ρ_3 = [i ↦ \pa_1, y ↦ \pa_1, x ↦ \pa_1] \\
  μ_1 = [\pa_1 ↦ \semst{(\Lam{y}{\Lam{x}{x}})~i}_{ρ_1}] \\
  μ_2 = [\pa_1 ↦ \semst{\Lam{x}{x}}_{ρ_2}] \\
  f = \fn{d}{\semst{\px}_{ρ_2[\px↦d}} \\
  \end{array} \\
\end{array}\]
Note that both the denotation in the heap \emph{and} its environment are updated
in state 2, and that the new denotation is immediately visible on the next heap
lookup in state 3, so that $\semst{\Lam{x}{x}}_{ρ_2}$ takes control rather than
$\semst{(\Lam{y}{\Lam{x}{x}})~i}_{ρ_1}$, just as the transition system requires.

The handling of data types and case expressions is routine (if a bit
syntactically heavy) and not different to denotational semantics in call-by-name
or call-by-value, but it allows us to observe type errors other than scoping
errors.
Let us consider evaluation of the closed expression
$\pe \triangleq \Let{x}{\ttrue}{\ttrue~x}$
(where $\ttrue$ is one of two unary data constructors of the data type $\bool
::= \ttrue \mid \ffalse$).
$\semst{\wild}$ makes it is easy to observe that the trace gets stuck:
\[\begin{array}{ll}
  \newcommand{\myleftbrace}[4]{\draw[mymatrixbrace] (m-#2-#1.west) -- node[right=2pt] {#4} (m-#3-#1.west);}
  \vcenter{\hbox{$
    \begin{tikzpicture}[baseline={-0.5ex},mymatrixenv]
      \matrix [mymatrix] (m)
      {
        1 & (\pe, []) \cons {} & \hspace{4em} & \hspace{5em} & \hspace{2.5em} \\
        2 & (\ttrue~x, μ) \cons {} & & & \\
        3 & \stuckend{((\ttrue,\ConV(\ttrue)), μ)} & & & \\
      };
      % Braces, using the node name prev as the state for the previous east
      % anchor. Only the east anchor is relevant
      \foreach \i in {1,...,\the\pgfmatrixcurrentrow}
        \draw[dotted] (m.east|-m-\i-\the\pgfmatrixcurrentcolumn.east) -- (m-\i-2);
      \myleftbrace{3}{1}{3}{$\semst{\pe}_{[]}$}
      \myleftbrace{4}{1}{2}{$\BindT$}
      \myleftbrace{4}{2}{3}{$\semst{\ttrue~x}_{ρ}$}
      \myleftbrace{5}{2}{3}{$\AppIT$}
    \end{tikzpicture}
  $}} &
  \!\!\!\text{where} \begin{array}{l}
  ρ = [x ↦ \pa_1] \\
  μ = [\pa_1 ↦ \semst{\ttrue}_ρ] \\
  \end{array} \\
\end{array}\]
Crucially, $\betastep$ is equipped to propagate $\stuckend{\wild}$ up the call
stack (through potential $\UpdateT$ transitions, in particular), similar to
\citeauthor{Milner:78}'s $\mathbf{wrong}$.

Diverging traces hold no new surprises, other than they are observably different
to stuck traces.

\subsection{Maximal LK Traces}
\label{sec:maximal-traces}

It turns out that the traces $\semst{\pe}$ generates correspond to
\emph{maximal} traces in the LK transition system.
Let us make precise what that means.

A transition system is characterised by the set of \emph{traces} it generates.
An \emph{LK trace} is a trace in $(\smallstep)$, \eg, a non-empty and
potentially infinite sequence of LK states $(σ_i)_{i∈\overline{n}}$
(where $\overline{n} = \{ m ∈ ℕ_+ \mid m ≤ n \}$ when $n∈ℕ_+$, $\overline{ω} = ℕ_+$),
such that $σ_i \smallstep σ_{i+1}$ for $i,(i+1)∈\overline{n}$.

The \emph{source} state $σ_1$ exists for finite and infinite traces, while the
\emph{target} state $σ_n$ is only defined when $n \not= ω$ is finite.

For proofs, we will often regard $(σ_i)_{i∈\overline{n}}$ as an object of type
$∃n∈ℕ_ω.\ \overline{n} \to \States$, where $ℕ_ω$ is defined by guarded recursion
as $ℕ_ω = \{1\} + \later ℕ_ω$.
The constructor for the right sum alternative is written $1 + $ (a different
kind of $+$ than in the recursive equation for $ℕ_ω$).
Now $ℕ_ω$ contains all natural numbers (where $n$ is encoded as $(1+)^{n-1}(1)$) and
the transfinite limit ordinal $ω = 1 + (1 + ...)$.
Hence, when $(σ_i)_{i∈\overline{n}} ∈ \STraces$ is an LK trace and $n > 1$, then
$(σ_{i+1})_{i∈\overline{n-1}} ∈ \later \STraces$ is the guarded tail of the
trace with an associated induction principle.

An important kind of trace is one that never leaves the evaluation context of
its source state:

\begin{definition}[Deep, interior and balanced traces]
  An LK trace $(σ_i)_{i∈\overline{n}}$ is
  \emph{$κ$-deep} if every intermediate continuation
  $κ_i = \mathit{cont}(σ_i)$ extends $κ$ (so $κ_i = κ$ or $κ_i = ... \pushF κ$,
  abbreviated $κ_i = ...κ$).

  A trace $(σ_i)_{i∈\overline{n}}$ is called \emph{interior} if it is
  $\mathit{cont}(σ_1)$-deep.
  Furthermore, an interior trace $(σ_i)_{i∈\overline{n}}$ is
  \emph{balanced}~\citep{Sestoft:97} if the target state exists and is a return
  state with continuation $κ_1$.

  We notate $κ$-deep, interior and balanced traces as
  $\deep{κ}{(σ_i)_{i∈\overline{n}}}$, $\interior{(σ_i)_{i∈\overline{n}}}$ and
  $\balanced{(σ_i)_{i∈\overline{n}}}$, respectively.
\end{definition}

\begin{example}
  Let $ρ=[x↦\pa_1],μ=[\pa_1↦([], \Lam{y}{y})]$ and $κ$ an arbitrary
  continuation. The trace
  \[
     (x, ρ, μ, κ) \smallstep (\Lam{y}{y}, ρ, μ, \UpdateF(\pa_1) \pushF κ) \smallstep (\Lam{y}{y}, ρ, μ, \UpdateF(\pa_1) \pushF κ) \smallstep (\Lam{y}{y}, ρ, μ, κ)
  \]
  is interior and balanced. Its prefixes are interior but not balanced.
  The trace suffix
  \[
     (\Lam{y}{y}, ρ, μ, \UpdateF(\pa_1) \pushF κ) \smallstep (\Lam{y}{y}, ρ, μ, κ)
  \]
  is neither interior nor balanced.
\end{example}

We will say that the transition rules $\LookupT$, $\AppIT$, $\CaseIT$ and $\BindT$
are interior, because the lifting into a trace is, whereas the returning
transitions $\UpdateT$, $\AppET$ and $\CaseET$ are not.

A balanced trace starting at a focus expression $\pe$ and ending with $\pv$
loosely corresponds to a derivation of $\pe \Downarrow \pv$ in a natural
big-step semantics~\citep{Sestoft:97} or a non-$⊥$ result in a denotational
semantics.

It is when a derivation in a natural semantics does not exist that a small-step
semantics shows finesse, in that it differentiates two different kinds of
\emph{maximally interior} (or, just \emph{maximal}) traces:

\begin{definition}[Maximal trace]
  An LK trace $(σ_i)_{i∈\overline{n}}$ is \emph{maximal} if and only if it is
  interior and there is no $σ_{n+1}$ such that $(σ_i)_{i∈\overline{n+1}}$ is
  interior.
  More formally (and without a negative occurrence of ``interior''),
  \[
    \maxtrace{(σ_i)_{i∈\overline{n}}} \triangleq \interior{(σ_i)_{i∈\overline{n}}} \wedge (\not\exists σ_{n+1}.\ σ_n \smallstep σ_{n+1} \wedge \mathit{cont}(σ_{n+1}) = ...\mathit{cont}(σ_1))
  \]
  We notate maximal traces as $\maxtrace{(σ_i)_{i∈\overline{n}}}$.
\end{definition}

We call infinite and interior traces \emph{diverging}.
A maximally finite, but unbalanced trace is called \emph{stuck}.
Note that usually stuckness is associated with a state of a transition
system rather than a trace.
That is not possible in our framework; the following example clarifies.

\begin{example}[Stuck and diverging traces]
Consider the interior trace
\[
             (\ttrue~x, [x↦\pa_1], [\pa_1↦...], κ)
  \smallstep (\ttrue, [x↦\pa_1], [\pa_1↦...], \ApplyF(\pa_1) \pushF κ)
\]
It is stuck, but its singleton suffix is balanced.
An example for a diverging trace where $ρ=[x↦\pa_1]$ and $μ=[\pa_1↦(x,ρ,())]$ is
\[
  (\Let{x}{x}{x}, [], [], κ) \smallstep (x, ρ, μ, κ) \smallstep (x, ρ, μ, \UpdateF(\pa_1) \pushF κ) \smallstep ...
\]
\end{example}

A maximal trace that is not balanced either diverges or is stuck:

\begin{lemma}[Characterisation of maximal traces]
  An LK trace $(σ_i)_{i∈\overline{n}}$ is maximal if and only if it is balanced,
  diverging or stuck.
\end{lemma}
\begin{proof}
  $\Rightarrow$: Let $(σ_i)_{i∈\overline{n}}$ be maximal.
  If $n=ω$ is infinite, then it is diverging due to interiority, and if
  $(σ_i)_{i∈\overline{n}}$ is stuck, the goal follows immediately.
  So we assume that $(σ_i)_{i∈\overline{n}}$ is maximal, finite and not stuck,
  so it must be balanced by the definition of stuckness.

  $\Leftarrow$: Both balanced and stuck traces are maximal.
  A diverging trace $(σ_i)_{i∈\overline{n}}$ is interior and infintie,
  hence $n=ω$.
  Indeed $(σ_i)_{i∈\overline{ω}}$ is maximal, because the expression $σ_{ω+1}$
  is undefined and hence does not exist.
\end{proof}

Interiority guarantees that the particular initial stack $κ$ of a maximal trace
is irrelevant to execution, so maximal traces that differ only in the initial
stack are bisimilar.

One class of maximal traces is of particular interest:
The maximal trace starting in $inj(\pe)$ (or, rather whether it is infinite,
stuck or balanced) is the defining operational characteristic of $\pe$.

If we can show that $\semst{\pe}$ associates similar meaning to $\pe$, we have
proven it an adequate replacement for the LK transition system.

\subsection{Adequacy}

\Cref{fig:semst-correctness} shows the correctness predicate $\mathcal{C}$ in
our endeavour to prove $\semst{\wild}$ adequate.
It builds on the framework of maximal LK traces established in the last section;
specifically it encodes that an abstraction of every maximal LK trace can be
recovered by running $\semst{\wild}$ starting from the abstraction of an initial
state.

\begin{figure}
\[\begin{array}{rcl}
  α_\Environments(ρ) & = & \mathit{deref} \circ ρ \\
  α_\Heaps([\many{\pa ↦ (ρ,\pe)}]) & = & [\many{\pa ↦ \idiom{\semst{\pe}_{α_\Environments(ρ)}}}] \\
  α_\States(\slbl \Lam{\px}{\pe},ρ,μ,κ) & = & ((\lbl,\FunV(\fn{d}{\idiom{\semst{\pe}_{α_\Environments(ρ)[\px↦d]}}})),α_\Heaps(μ)) \\
  α_\States(\slbl K~\overline{\px},ρ,μ,κ) & = & ((\lbl,\ConV(K,\overline{\idiom{\semst{\px}_{α_\Environments(ρ)}}})),α_\Heaps(μ)) \\
  α_\States(\pe,ρ,μ,κ) & = & (\pe,α_\Heaps(μ)) \\
  α_{\STraces}((σ_i)_{i∈\overline{n}},κ) & = & \begin{cases}
    α_\States(σ_1) \cons \idiom{α_{\STraces}((σ_{i+1})_{i∈\overline{n-1}},κ)} & n > 1 \\
    \goodend{α_\States(σ_1)} & \mathit{ctrl}(σ_1) \text{ value } \wedge \mathit{cont}(σ_1) = κ \\
    \stuckend{α_\States(σ_1)} & \text{otherwise} \\
  \end{cases} \\
  \mathcal{C}((σ_i)_{i∈\overline{n}}) & = & \maxtrace{(σ_i)_{i∈\overline{n}}} \Longrightarrow ∀((\pe,ρ,μ,κ) = σ_1).\ α_{\STraces}((σ_i)_{i∈\overline{n}},κ) = \semst{\pe}_{α_\Environments(ρ)}(α_\Heaps(μ)) \\
\end{array}\]
\caption{Correctness predicate for $\semst{\wild}$, defining the family of functions $α_\LKStates$}
  \label{fig:semst-correctness}
\end{figure}

The family of abstraction functions (which we henceworth refer to as
$α_\LKStates$ and treat as overloaded for $\Environments,\Heaps,\States$ and
$\STraces$) makes precise the intuitive connection between the semantic objects
in $\semst{\wild}$ and the syntactic objects in the transition system.

We will sometimes need to disambiguate the clashing definitions from
\Cref{sec:semantics} and \Cref{sec:problem}.
We do so by adorning semantic objects with a tilde, so $\tr \triangleq
α_\Environments(ρ)$ denotes a semantic environment which in this instance is
defined to be the abstraction of a syntactic environment $ρ$.

%We will often talk about states that are well-elaborated and have a certain
%expression in control, hence we abbreviate this set as
%\[
%  \States_\pe \triangleq \{σ ∈ \States \mid \elabstate{σ} \wedge \mathit{ctrl}(σ)=\pe\},
%\]

Note first that $α_\STraces$ is effectively defined by guarded recursion over
the LK trace, in the sense defined in \Cref{sec:maximal-traces}.
As such, the expression $\idiom{α_{\STraces}((σ_{i+1})_{i∈\overline{n-1}},κ)}$ has type
$\later \tSTraces$ (the $\later$ in the type of $(σ_{i+1})_{i∈\overline{n-1}}$
maps through $α_\STraces$ via the idiom brackets).
Likewise, $=$ on $\tSTraces$ is defined in the obvious structural way by guarded
recursion (as it would be if it was a finite, inductive type).

Our first goal is to establish a few auxiliary lemmas showing what kind of
properties of LK traces are preserved by $α_\States$ and in which way.

\begin{lemma}[Preservation of length]
  \label{thm:abs-length}
  Let us define the length $\mathit{len} : \tSTraces \to ℕ_ω$ of a trace by
  guarded recursion
  \[
    \mathit{len}(τ) = \begin{cases}
      1 + \idiom{\mathit{len}(τ^{\later})} & τ = σ \cons τ^{\later} \\
      1 & \text{otherwise} \\
    \end{cases}
  \]
  Now let $(σ_i)_{i∈\overline{n}}$ be an arbitrary trace.
  Then $\mathit{len}(α_\STraces((σ_i)_{i∈\overline{n}},\mathit{cont}(σ_1))) = n$.
\end{lemma}
\begin{proof}
  This is quite simple to see and hence a good opportunity to familiarise
  ourselves with the concept of \emph{Löb induction}, the induction principle of
  guarded recursion.
  Löb induction arises simply from applying the guarded recursive fixpoint
  combinator to a proposition:
  \[
    \textsf{löb} = \fix : \forall P.\ (\later P \Longrightarrow P) \Longrightarrow P
  \]
  That is, we assume that our proposition holds \emph{later}, \eg
  \[
    IH ∈ (\later P \triangleq \later (
        \forall n ∈ ℕ_ω.\
        \forall (σ_i)_{i∈\overline{n}}.\
        \mathit{len}(α_\STraces((σ_i)_{i∈\overline{n}},\mathit{cont}(σ_1))) = n
      ))
  \]
  and use $IH$ to prove $P$.
  Let us assume $n$ and $(σ_i)_{i∈\overline{n}}$ are given, define
  $τ \triangleq α_\STraces((σ_i)_{i∈\overline{n}},\mathit{cont}(σ_1))$ and proceed by case analysis
  over $n$:
  \begin{itemize}
    \item \textbf{Case $n=1$}: Then either we have $τ = \goodend{α_\States(σ_1)}$
      or $τ = \stuckend{α_\States(σ_1)}$, $j = 1$, both of which map to $1$ under
      $\mathit{len}$.
    \item \textbf{Case $n>1$}: Then $n = 1+m$ and $m ∈ \later ℕ_ω$ and the
      first case of $α_\States$ applies, hence $τ = σ \cons τ^{\later}$ for some
      $σ∈\States, τ^{\later}∈\later \tSTraces$.
      Now we apply the inductive hypothesis, as follows:
      Let $(σ_{i+1})_{i∈\overline{m}} ∈ \later \STraces$ be the guarded
      tail of the LK trace $(σ_i)_{i∈\overline{n}}$.
      Then we can apply $IH \aplater m \aplater (σ_{i+1})_{i∈\overline{m}}$ and
      get a proof for $\later (\mathit{len}(τ^{\later}) = m)$.
      Now we can prove
      \[
        n = 1 + m = 1 + \mathit{len}(τ^{\later}) = \mathit{len}(τ).
      \]
  \end{itemize}
\end{proof}

\begin{lemma}[Preservation of components]
  \label{thm:abs-states}
  Let $(σ_i)_{i∈\overline{n}}$ be a trace and let $τ = α_\STraces((σ_i)_{i∈\overline{n}},\mathit{cont}(σ_1))$.
  Then for all $j∈\overline{n}$, $τ_j = α_\States(σ_j)$
  (where $τ_j$ denotes the $j$th state in $τ$).
\end{lemma}
\begin{proof}
  With \Cref{thm:abs-length} it is enough to regard the finite prefix of the
  trace $(σ_i)_{i∈\overline{j}}$, for which the proposition is easily shown by
  induction on $j$.
\end{proof}

\begin{lemma}[Preservation of characteristic]
  \label{thm:abs-max-trace}
  Let $(σ_i)_{i∈\overline{n}}$ be a maximal trace.
  Then $α_\STraces((σ_i)_{i∈\overline{n}}, cont(σ_1))$ is ...
  \begin{itemize}
    \item ... infinite if and only if $(σ_i)_{i∈\overline{n}})$ is diverging
    \item ... ending with $\goodend{\wild}$ if and only if $(σ_i)_{i∈\overline{n}}$ is balanced
    \item ... ending with $\stuckend{\wild}$ if and only if $(σ_i)_{i∈\overline{n}}$ is stuck
  \end{itemize}
\end{lemma}
\begin{proof}
  The first point follows by a similar inductive argument as in \Cref{thm:abs-length}.

  In the other cases, we may assume that $n$ is finite.
  If $(σ_i)_{i∈\overline{n}}$ is balanced, then $σ_n$ is a return state with
  continuation $\mathit{cont}(σ_1)$, so its control expression is a value.
  Then $α_\STraces$ will conclude with $\goodend{\wild}$.
  Conversely, if the trace ended with $\goodend{\wild}$, then $\mathit{cont}(σ_n) = \mathit{cont}(σ_1)$
  and $\mathit{ctrl}(σ_n)$ is a value, so $(σ_i)_{i∈\overline{n}}$ forms a
  balanced trace.
  The stuck case is similar.
\end{proof}

%\begin{lemma}[S3]
%  \label{thm:s3}
%  If $σ ∈ \States_\pe$, then $\elabtrace{\semst{\pe}(σ)}$.
%\end{lemma}
%\begin{proof}
%By coinduction, following~\citet{Czajka:2019}.
%The coinduction hypothesis is:
%\[
%  P(α) = ∀σ,\pe.\ σ ∈ \States_\pe \Longrightarrow \elabtracen{α}{\semst{\pe}(σ)}
%\]
%Where $α$ is a limit ordinal smaller than or equal to the closure ordinal $ζ$,
%$\mathit{ctrl}(σ)$ selects the control of $σ$ and $\elabtracen{α}{τ}$ is the
%$α$-approximant of the coinductive predicate $\elabtrace{τ}$.
%
%Let us assume that $\elabstate{σ}$ for an arbitrary $σ$ with control expression
%$\pe$. Whenever $f(σ)$ is undefined for some $f$, we can see that
%$step(f)(σ) = \goodend{σ}$ is well-elaborated. Furthermore, if both $g$ and $h$
%yield well-elaborated traces given a well-elaborated input state, then
%$g \sfcomp h$ does so, too.
%
%We abbreviate $τ \triangleq \semst{\pe}(σ)$ and proceed by case analysis over
%$\pe$. We have $src_\States(τ) = σ$ by (S1) which is thus well-elaborated.
%\begin{itemize}
%  \item \textbf{Case $\px$}:
%    Let us assume first that $look(\px)(σ)$ is undefined.
%    Then $step(look(\px))(σ) = \goodend{σ}$.
%    On the other hand, $step(upd)(σ) = \goodend{σ}$ because $upd$ is only defined on
%    return states.
%    Thus, $τ = \goodend{σ}$ and $\elabtracen{α+1}{τ}$.
%
%    When $look(\px)(σ)$ is defined, $σ$ must be of the form $(\px,ρ,μ,κ)$, and
%    $\pa,\pe,ρ',d$ must exist as in the definition of $look$ (shadowing $\pe =
%    \px$ from the assumption, for simplicity).
%    Clearly, $σ' \triangleq (\pe, ρ', μ, \UpdateF(\pa) \pushF κ)$ is
%    well-elaborated, because $σ$ is and the heap did not change.
%    Similarly, by $\elabstate{σ}$ we have $d = \semst{\pe}$.
%    Now we apply the coinduction hypothesis to $τ' = \semst{\pe}(σ')$
%    (noting that $\semst{\pe}$ is well-defined on $σ'$) and see that
%    $\elabtracen{α}{τ'}$, hence $\elabtracen{α+1}{(σ \cons τ')}$.
%
%    If $τ'$ is infinite, we are done.
%    Otherwise, $σ_u \triangleq tgt_\States(τ')$ is well-elaborated and if
%    $upd(σ_u)$ is undefined we are done, too, by our preceding considerations.
%    If $σ_v \triangleq upd(σ_u)$, the $σ_u$ must have the form
%    $((\pv,v),ρ,μ,\UpdateF(\pa) \pushF κ)$. We must show that
%    $step(val(\pv,v)) = \semst{\pv}$ in order to show that $σ_v$ is
%    well-elaborated.
%    That is the case: well-elaboratedness of $σ_u$ implies that $v = \FunV(f)$
%    for an $f$ just like that in the definition of $\semst{\pv}$.
%    Since $τ'$ is finite, we have $\elabtracen{α+1}{(σ \cons τ' \cons \goodend{σ_v})}$ by
%    reassociating the applications of the $\elabtrace{\wild}$ functional.
%  \item \textbf{Case $\Lam{\px}{\pe}$}:
%    The $\ValueT$ transition does not change the heap, but it transitions into a
%    return state. We have established in the previous point that the semantic
%    value $\FunV(f)$ fits our requirements for well-elaboratedness exactly,
%    hence well-elaboratedness for the trace follows.
%  \item \textbf{Case $\pe~\px$}:
%    The $\AppIT$ transition preserves well-elaboratedness to its target state, so
%    the input $σ'$ to $\semst{\pe}$ is well-elaborated. We apply the coinduction
%    hypothesis and and have $\elabtracen{α+1}{(σ \cons \semst{\pe}(σ'))}$.
%    (As before, if the resulting trace is infinite, we are done.)
%    Forward composition preserves well-elaboratedness, hence it suffices to
%    prove that given a well-elaborated input state $σ''$, $\mathit{apply}(σ'')$ is
%    well-elaborated.
%    That is the case if the subterm $f(\pa)(σ'')$ produces a well-elaborated
%    trace.
%    By $\elabstate{σ''}$, we know that $f$ must come from the
%    $\semst{\Lam{\px}{\pe'}}$ case.
%    $\AppET$ preserves well-elaboratedness to its target state and for
%    $\semst{\pe'}$ we can apply the coinductive hypothesis.
%    Now we have $\elabtracen{α+1}{(σ \cons \semst{\pe}(σ'))}$ and
%    $\elabtracen{α+1}{\mathit{apply}(σ'')}$, so the same must hold for its concatenation.
%  \item \textbf{Case $\Let{\px}{\pe_1}{\pe_2}$}:
%    The $\BindT$ transition updates the heap, but it is clear that
%    $bind(\semst{\pe_1})$ elaborates the correct $d_1$ into $σ'$.
%    For the recursive call, we apply the coinduction hypothesis
%    and conclude $\elabtracen{α+1}{(σ \cons \semst{\pe_2}(σ'))}$.
%\end{itemize}
%\end{proof}
%
%The preceding lemma implies a very desirable property:
%Whenever $σ$ is well-elaborated (and thus a total element of the approximation
%order), $\semst{\pe}(σ)$ is total, too.
%The approximation order restricted to total elements is discrete; hence we may
%now put domain theoretic considerations behind us in favor of a coinductive
%understanding.
%We can capture the termination properties of $\semst{\wild}$ as follows:
%
%\begin{corollary}
%  The restriction of $\semst{\pe}$ to $\States_\pe$ is a total function, defined
%  by \emph{guarded recursion}.
%\end{corollary}
%
%\begin{lemma}
%  \label{thm:step-interior}
%  If $f(σ) = τ$ implies $σ \smallstep src_\States(τ)$ interior and
%  $τ$ interior, then $step(f)(σ)$ interior.
%\end{lemma}
%\begin{proof}
%  Immediate. The case where $f$ is undefined is trivial.
%\end{proof}

With increased clarity, we go on to prove the correctness predicate:

\begin{theorem}[Correctness of $\semst{\wild}$]
  \label{thm:semst-correct}
  $\mathcal{C}$ from \Cref{fig:semst-correctness} holds everywhere.
  That is, whenever $(σ_i)_{i∈\overline{n}}$ is a maximal LK trace with source
  state $(\pe,ρ,μ,κ)$, we have
  $α_{\STraces}((σ_i)_{i∈\overline{n}},κ) = \semst{\pe}_{α_\Environments(ρ)}(α_\Heaps(μ))$.
\end{theorem}
\begin{proof}
By Löb induction, with the following hypothesis:
\[
  IH ∈ \later (\forall (σ_i)_{i∈\overline{n}}.\ C((σ_i)_{i∈\overline{n}}))
\]
Furthermore, we tacitly assume by (S3) that all occuring states are
well-elaborated. We will say that a state $σ$ is stuck if there is no applicable
rule in the transition system (\ie, $\goodend{σ}$ is a stuck maximal trace).

We abbreviate $τ \triangleq \semst{\pe}(σ)$ and proceed by case analysis over
$\pe$.
\begin{itemize}
  \item \textbf{Case $\px$}:
    Let us assume first that $look(\px)(σ)$ is undefined. Then the lookup
    $ρ(\px)$ must have failed and $\goodend{σ}$ is the result.
    On the other hand, $step(upd)(σ) = \goodend{σ}$ because $upd$ is only defined on
    return states, so $\goodend{σ}$ is the result of the composition, which is stuck
    and thus maximal.

    When $look(\px)(σ)$ is defined, $σ$ must be of the form $(\px,ρ,μ,κ)$, and
    $\pa,\pe,ρ',d$ must exist as in the definition of $look$.
    Clearly, $σ \smallstep (σ' \triangleq (\pe, ρ', μ, \UpdateF(\pa) \pushF κ))$
    by interior rule $\LookupT$.
    By $\elabstate{σ'}$ we have $d = \semst{\pe}$.
    Now we apply the coinduction hypothesis to $τ' \triangleq \semst{\pe}(σ')$
    and see that $\maxtracen{α}{τ'}$, hence with \Cref{thm:step-interior}
    $\interiorn{α+1}{σ \cons τ'}$.

    If $τ'$ is infinite, it is diverging and $\maxtracen{α+1}{σ \consτ'}$ follows.
    Otherwise, $σ_u \triangleq tgt_\States(τ')$ and either there is no
    transition $σ_u \smallstep σ_v$ or it leaves $\mathit{cont}(σ')$.
    The only two returning (\eg, non-interior) transitions are $\UpdateT$ and
    $\AppET$. Both pop a single continuation frame, thus $\mathit{cont}(σ_u) = \mathit{cont}(σ')$,
    because otherwise $\mathit{cont}(σ_u) = ... \pushF \mathit{cont}(σ')$ and popping one frame
    yields an interior transition.

    If $upd(σ_u)$ is defined, then the $\UpdateT$ transition exists as can
    easily be checked.
    Furthermore, since exactly one frame is popped, we must have
    $\mathit{cont}(σ) = \mathit{cont}(σ_v)$ and thus $\deepn{α+1}{\mathit{cont}(σ)}{(σ \cons τ \cons \goodend{σ_v})}$ by slight
    (corecursive) rearrangement of the proof for $\interiorn{α}{τ}$.
    $σ_v$ is a return state and any further transition must pop a continuation
    frame; hence $\maxtracen{α+1}{(σ \cons τ \cons \goodend{σ_v})}$.

    If $upd(σ_u)$ is undefined, then the $\UpdateT$ transition could not have
    fired. But the $\AppET$ transition can't have fired either, because if it
    could, we'd have $\mathit{cont}(σ_u) = \mathit{cont}(σ')$ by $\maxtracen{α}{τ'}$, but the top
    of $σ'$ is an update frame. Thus, again by maximality, there is no
    transition $σ_u \smallstep σ_v$ whatsoever and $\maxtracen{α+1}{σ \cons τ'}$.

  \item \textbf{Case $\Lam{\px}{\pe}$}:
    If $val(v, \FunV(f))$ is defined, then $σ \smallstep{\ValueT} σ'$ must
    exist. Furthermore, $σ \cons \goodend{σ'}$ is a maximal trace, as $σ'$ is a return
    state and any applicable transition leaves $\mathit{cont}(σ) = \mathit{cont}(σ')$.

    If $val(v, \FunV(f))$ is undefined, then $\mathit{ctrl}(σ) \not= \Lam{\px}{\pe}$, a
    contradiction.
  \item \textbf{Case $\pe~\px$}:
    If $app_1(\pe~\px)$ is undefined, then either $\mathit{ctrl}(σ) \not=\pe~\px$
    (contradiction), or $ρ(\px)$ was undefined, in which case $\goodend{σ}$ is stuck
    and thus maximal.

    If $app_1(\pe~\px)$ is defined, then
    $σ \smallstep{\AppIT} (σ' \triangleq (\pe,ρ,μ,\ApplyF(ρ(\px)) \pushF κ))$
    and $τ_1 \triangleq \semst{\pe}(σ')$ with $\maxtracen{α}{τ_1}$.

    Similar to the variable case, if $τ_\pe$ is infinite, $σ \cons τ_\pe$ is
    diverging and we are done. Otherwise, we have $σ_a \triangleq
    tgt_\States(τ_\pe)$ and either there is no transition $σ_a \smallstep σ_e$
    or it leaves $\mathit{cont}(σ')$, in which case we know $\mathit{cont}(σ') = \mathit{cont}(σ_a)$ and
    that the transition must have been $\AppET$.

    When the $\AppET$ transition exists, the first case of $\mathit{apply}(σ_a)$ matches.
    With $\elabstate{σ}$, we know that $f$ is defined just like in
    $\semst{\Lam{\px'}{\pe'}}$, where $\mathit{ctrl}(σ_a) = (\Lam{\px'}{\pe'}, \FunV(f))$.
    Similar to the variable case, we can see that
    $app_2$ matches and that $\maxtracen{α+1}{σ \cons τ_1 \cons \goodend{σ_e}}$, because
    $\mathit{cont}(σ_e) = \mathit{cont}(σ)$. By the coinductive hypothesis,
    $\maxtracen{α}{(τ_2 \triangleq \semst{\pe'}(σ_e))}$. By $\mathit{cont}(σ_e) = \mathit{cont}(σ)$
    and rearrangement we can see that $\interiorn{α+1}{σ \cons τ_1 \cons σ_e \cons τ_2}$ and
    maximality follows directly from maximality of $τ_2$.

    When the $\AppET$ transition does not exist, no other transition from $σ_a$
    does. Then the first case of $\mathit{apply}$ could not match, because the only
    syntactic values $\pv$ are lambdas and $\elabstate{σ_a}$ requires that
    $\FunV(f)$ matches accordingly.
    Thus, $σ \cons τ_1$ is the final, maximal trace.

  \item \textbf{Case $\Let{\px}{\pe_1}{\pe_2}$}:
    The $σ \smallstep{\BindT} σ'$ transition (which always exists for our choice
    of $σ$) does not push a new stack frame, hence $\mathit{cont}(σ) = \mathit{cont}(σ')$ and
    from the coinductive hypothesis $\maxtracen{α}{\semst{\pe_2}(σ')}$ we can
    immediately see $\maxtracen{α+1}{σ \cons \semst{\pe_2}(σ')}$.
\end{itemize}
\end{proof}

\Cref{thm:s2} is the key to proving a strong version of adequacy for
$\semst{\wild}$:

\begin{lemma}[Adequacy of $\semst{\wild}$]
  Let $τ = \semst{\pe}(inj(\pe))$.
  \begin{itemize}
    \item
      $τ$ is balanced iff there exists a final state $σ$ such that
      $inj(\pe) \smallstep^* σ$.
    \item
      $τ$ is stuck iff there exists a non-final state $σ$ such that
      $inj(\pe) \smallstep^* σ$ and there exists no $σ'$ such that $σ \smallstep
      σ'$.
    \item
      $τ$ is diverging iff for all $σ$ with $inj(\pe) \smallstep^* σ$ there
      exists $σ'$ with $σ \smallstep σ'$.
  \end{itemize}
\end{lemma}
\begin{proof}
  Note that $\maxtrace{τ}$ since $inj(\pe) ∈ \States_\pe$
  and the initial state $inj(\pe) = src_\States(τ)$ has an empty continuation.

  Clearly, the trace $τ'$ determined by a $σ$ such that $inj(\pe)
  \smallstep^* σ$ is an interior trace. Since $(\smallstep)$ is
  deterministic, it must be a prefix of the maximally interior trace $τ$.

  \begin{itemize}
    \item[$\Rightarrow$]
      \begin{itemize}
        \item
          If $τ$ is balanced, its target state $σ \triangleq tgt_\States(τ)$
          is a return state that must also have the empty continuation. Hence
          it is a final state and there exists $inj(\pe) \smallstep^* σ$ by
          $\validtrace{τ}$.
        \item
          If $τ$ is stuck, it is finite and maximal, but not balanced, so its
          target state $σ \triangleq tgt_\States(τ)$ cannot be a return state;
          otherwise maximality implies $σ$ has an (initial) empty continuation
          and the trace would be balanced. on the other hand, the only two
          returning transitions apply to return states, so maximality implies
          there is no $σ'$ such that $σ \smallstep σ'$ whatsoever.
        \item
          If $τ$ is diverging it is infinite and for every $σ$ with $inj(\pe)
          \smallstep^* σ$ determinism allows us to trace a finite prefix of
          $τ$, and there always exists $σ'$ such that $σ \smallstep σ'$ since
          $\validtrace{τ}$.
      \end{itemize}

    \item[$\Leftarrow$]
      \begin{itemize}
        \item
          If $σ$ is a final state, it has $\mathit{cont}(σ) = \mathit{cont}(inj(\pe)) = []$, so it
          is balanced.
        \item
          If $σ$ is not a final state, $τ'$ is not balanced. Since there is no
          $σ'$ such that $σ \smallstep^* σ'$, it is still maximal; hence it must
          be stuck.
        \item
          If for every choice of $σ$ (and thus $τ'$) there exists $σ'$ such that
          $σ \smallstep σ'$, there must be an infinite number of such $τ'$, all
          of which are prefixes of $τ$. Hence $τ$ must be infinite and interior,
          hence diverging.
      \end{itemize}
  \end{itemize}
\end{proof}

But there's more. Let us define the following binary relation $\equiv_\States$
by coinduction:
\[
\]

% \begin{theorem}[Full abstraction]
%   \label{thm:full-abstraction}
%  The restriction $S = \fn{\pe}{\semst{\pe}\restrict{\States_\pe}}$ is
%  \emph{fully abstract}~\citep{Plotkin:77}, meaning that
%
%  $\forall \pe_1,\pe_2.\ S(\pe_1) = S(\pe_2)
%
% \end{theorem}

\subsection{Discussion}

\begin{itemize}
  \item Generated traces are not enough to recover transition system. Unnecessary!
  \item Correctness predicate simpler to come up than for liveness analysis directly
  \item Compare to coinductive big-step \citep{LeroyGrall:09}? Perhaps in related work
  \item Full Abstraction. Do we want to prove it? Seems boring and obvious
\end{itemize}

\section{Stateless Semantics}
\label{sec:stateless}

In the previous section, we gave a compositional trace-based semantics and
proved it adequate \wrt the LK transition semantics.
With compositionality and structural induction we recover strong advantages of
denotational semantics.

It is striking that although $\semst{\wild}$ is as expressive as the LK
transition system, the information encoded in the states of the generated traces
is not enough to recover the LK transition system in the sense of
\citet[Chapter 43]{Cousot:21}.
For example, every return state $((\lbl,v),\tm)$ admits at least two possible
transitions:
Either a heap update or a $β$-reduction.
In the LK transition system, the particular transition is governed by the
continuation stack, of which $\semst{\wild}$ maintains no reification in its states.
It is however trivial to ``elaborate'' $\semst{\wild}$ to include the proper
stack manipulations and emit the current lexical environment as part of every
state.
We think it is rather uninteresting to give the closed, elaborated definition
of $\semst{\wild}$ which necessarily gives up simplicity.
It is more interesting to see what other parts of the traces we can \emph{omit}
before compromising on expressivity.

For example, in Control-Flow Analysis~\citep{Shivers:91}, we are interested in
observing in each state of a trace the label of the expression in control.
That makes $\semst{\wild}$ a suitable semantics for abstraction.
Yet for our usage analysis in \Cref{sec:problem}, we do not care so much
about the \emph{state} in which variable lookup happens, but rather about at
which \emph{address} the $\LookupT$ transition happened, as well as the $\BindT$
transition that tells us what $\mathbf{let}$ binding that particular address
refers to.

In this section, we will omit states --- including the explicitly maintained
heap --- altogether in favor of tracing the instantiations of transition rules,
called \emph{actions}.
The resulting \emph{stateless trace semantics} $\semless{\wild}$ is the
preferred framework of \citet{Cousot:21}, in which the heap will be
rematerialised as needed from the history of previous actions.

\begin{figure}
\[\begin{array}{c}
 \arraycolsep=3pt
 \begin{array}{rrclcl}
  \text{Actions} & a & ∈ & \Actions & ::=  & \AppIA(d) \mid \AppEA(\px↦d) \mid \BindA(\px,\pa↦d^{\later}) \mid \LookupA(\pa) \mid \UpdateA(\pa) \\
                 &   &   &          & \mid & \CaseIA(d) \mid \CaseEA(K,\many{\px↦d}) \mid \ValueA(v) \\
 \end{array} \\
 \arraycolsep=3pt
 \begin{array}{rrclcl@{\quad}rrclcl}
  \text{Initialisation Traces} & τ^i        & ∈ & \Traces^*               & ::= & [a_1,...,a_n], n ∈ ℕ_0
  &
  \text{Stateless domain}      & d          & ∈ & \MaxD                   & =   & \Traces^* \to \Traces^{\infty}
  \\
  \text{Program Traces}        & τ          & ∈ & \Traces^{\infty}        & ::= & \goodend{a} \mid \stuckend{} \mid a \cons τ^{\later}
  &
  \text{Delayed Trace}         & τ^{\later} & ∈ & \later \Traces^{\infty} &     &   \\
 \end{array} \\
 \arraycolsep=3pt
 \begin{array}{rrclcl}
  \text{Stateless values} & v & ∈ & \Values{} & ::= & \FunV(f ∈ \MaxD \to \later \MaxD) \mid \ConV(K,\many{d^{\later}}^{α_K}) \\
 \end{array} \\
 \\[-0.5em]
 \ruleform{ (\concat) : \Traces^{\infty} \to \later\Traces^{\infty} \to \Traces^{\infty} \quad (\fcomp) : \MaxD \to \later\MaxD \to \MaxD \quad \tell{\wild} : \Actions \to \MaxD} \\
 \\[-0.5em]
 \begin{array}{cc}
  \begin{array}{rcl}
   τ_1 \concat τ_2^{\later} & = & \begin{cases}
     a \cons \idiom{τ_1^{\later} \concat τ_2^{\later}} & τ_1 = a \cons τ_1^{\later} \\
     a \cons τ_2^{\later} & τ_1 = \goodend{a} \\
     \stuckend{} & τ_1 = \stuckend{} \\
   \end{cases} \\
  \end{array} &
  \begin{array}{rcl}
   \\[-0.5em]
   (d_1 \fcomp d_2^{\later})(τ^i)   & = & d_1(τ^i) \concat \idiom{d_2^{\later}(τ^i \concat d_1(τ^i))} \\
   \\[-0.5em]
   \tell{a}(τ^i) & = & \goodend{a} \\
  \end{array} \\
 \end{array} \\
 \\[-0.5em]
 \begin{array}{rcl}
  \multicolumn{3}{c}{ \ruleform{
    \begin{array}{c}
      β : (\Values{} \pfun \later\MaxD) \to \later\MaxD \quad μ : \Traces^{+} \to \Addresses \pfun \later\MaxD \\
      \deref : \Addresses \to \MaxD \quad \apply : \MaxD \to \later\MaxD \\
      \select : ((\MaxD)^{α_K} \pfun \MaxD)^* \to \later\MaxD \\
    \end{array}
  }} \\
  \\[-0.5em]
  β(f)(τ^i) & = & \begin{cases}
      \idiom{f(v)(τ^i)} & \text{$τ^i = [..., \ValueA(v), \many{\UpdateA(\wild)}]$ and $v ∈ \dom(f)$} \\
      \idiom{\stuckend{}} & \text{otherwise} \\
    \end{cases} \\
  \\[-0.5em]
  μ(τ^i)(\pa) & = & \begin{cases}
    \tell{\ValueA(v)} & \text{if $τ^i = [..., \ValueA(v), \many{\UpdateA(\wild),}\, \UpdateA(\pa),...]$} \\
    d & \text{if $τ^i = [..., \BindA(\px,\pa↦d), ...]$} \\
  \end{cases}  \\
  \\[-0.5em]
  \deref(\pa)(τ^i)   & = & (\tell{\LookupA(\pa)} \fcomp μ(τ^i)(\pa) \fcomp \tell{\UpdateA(\pa)})(τ^i) \\
  \apply(d_\px) & = & β(\fn{(\FunV(f))}{\idiom{f(d_\px)}}) \\
  \select(\many{f}) & = & β(\fn{(\ConV(K_s,\many{d_s}))}{\idiom{f_s(\many{d_s})}}) \\
 \end{array} \\
 \\[-0.5em]
 \begin{array}{rcl}
  \multicolumn{3}{c}{ \ruleform{ \semless{\wild} \colon \Exp → (\Var \pfun \MaxD) → \MaxD } } \\
  \\[-0.5em]
  \semless{\px}_ρ       & = & \begin{cases}
    ρ(\px) & \px ∈ \dom(ρ) \\
    \fn{\wild}{\stuckend{}}  & \text{otherwise} \\
  \end{cases} \\
  \\[-0.5em]
  \semless{\Lam{\px}{\pe}}_ρ & = & \tell{\ValueA(\FunV(\fn{d}{\tell{\AppEA(\px↦d)} \fcomp \semless{\pe}_{ρ[\px↦d]}}))} \\
  \\[-0.5em]
  \semless{\pe~\px}_ρ   & = & \begin{cases}
    \tell{\AppIA(ρ(x))} \fcomp \semless{\pe}_ρ \fcomp \apply(ρ(\px)) & \px ∈ \dom(ρ) \\
    \fn{\wild}{\stuckend{}}  & \text{otherwise} \\
  \end{cases} \\
  \\[-0.5em]
  \semless{\Let{\px}{\pe_1}{\pe_2}}_ρ(τ^i) & = &
    \begin{letarray}
      \text{let} & ρ' = ρ[\px ↦ \deref(\pa)] \quad \text{where $\pa \not∈ \dom(μ(τ^i))$} \\
      \text{in}  & (\tell{\BindA(\px,\pa↦\semless{\pe_1}_{ρ'})} \fcomp \semless{\pe_2}_{ρ'})(τ^i)
    \end{letarray} \\
  \\[-0.5em]
  \semless{K~\many{\px}}_ρ & = & \tell{\ValueA(\ConV(K,\many{\semless{\px}_ρ}))} \\
  \\[-0.5em]
  \semless{\Case{\pe_s}{\Sel[r]}}_ρ & = &
    \begin{letarray}
      \text{let} & \many{f_K = \fn{\many{d}^{α_K}}{\tell{\CaseEA(K,\many{\px↦d})} \fcomp \semless{\pe_r}_{ρ[\many{\px↦d}]}}} \\
      \text{in} & \tell{\CaseIA(\semless{\pe_s}_ρ)} \fcomp \semless{\pe_s}_ρ \fcomp \select(\many{f})  \\
    \end{letarray}
 \end{array} \\
\end{array}\]
\caption{Stateless Trace Semantics}
  \label{fig:semless}
\end{figure}

\subsection{Going stateless}

\Cref{fig:stateless-syntax} provides the new syntactic building blocks of the
semantics. A \emph{program point} $\pc$ is either a return point $\return$ or a
labelled expression; each program point defines one of finitely many
control-flow nodes of the program. Actions $a$ correspond one-to-one to the
transition rules of the LK machine, only they are now retained in an AST-like
fashion in the trace instead of eliminated (via functions $app_1$, $app_2$,
$bind$, \etc) to produce the successor state. Stateless traces $τ$ carry the
the transition taken as an action $a$. We re-use the same greek letter as in
\Cref{sec:stateful} and disambiguate as needed. We consider inductively-defined
finite traces $τ^+$ a subtype of the potentially infinite and thus
coinductively-defined traces $τ$.

Comparing the domain of maximal traces with the one for LK traces from
\Cref{fig:lk-domain}, we can see that a finite \emph{initialisation trace} takes
the place of the initial state as a parameter to a domain element $d$.
Similarly, the use of $\Addresses$ in the definition of semantic values $v$ is
replaced in favor of a seemingly more complex argument of type $\MaxD$; in
practice the $d$ passed will be one-to-one to the address passed in the stateful
semantics.

% Note [Design of Maximal Trace Semantics]
% ~~~~~~~~~~~~~~~~~~~~~~~~~~~~~~~~~~~~~~~~
% A stateless (prefix) trace semantics' defining feature is the absence of
% full-blown configurations between actions. However, states can be
% reconstructed from looking at the history of actions in the initialisation
% trace. So it is quite vital that the actions carry enough information.
%
% The need for a value action
% ---------------------------
% Values have after labels, so that when we reach at(v) we make one value
% transition to announce the value of at(v) in the trace. Without this
% action, it would be hard to differentiate between stuck and successful
% maximal traces. Furthermore, it plays a crucial role in the memoisation
% mechanism, where we reconstruct the memoised value from the trace.
%
% A different design would be to have controls `expr | (lbl, sem value)`,
% but then we still need the value transition: Otherwise, callers of
% S[λx.e] must supply the semantic value of the lambda at the end of the
% initialisation trace which defeats the point.
%
% The need for an update action
% -----------------------------
% Memoisation could well work by looking for a balanced sub-trace in the
% initialisation trace that started with look(τ_k). In fact, earlier
% versions did exactly that, and it worked great!
% Unfortunately, it has the following drawbacks:
%   * CESK semantics does have update frames and we want to match those
%     rather simply. It is a matter of producing "complete" states, see
%     "Which info do we need to attach to an action?"
%   * We need to define relatively early what a balanced trace is.
%     The semantics itself should not depend on that...
%   * It is simpler to define the abstraction to stateful prefix trace semantics
%     when there are corresponding update actions
%   * It is better to be explicit to announce exactly when "the heap binding
%     changes" for the perspective of weak memory models and interleavings of
%     parallel traces (TODO: Read up on that!), which is one of the prime
%     reasons to consider stateless trace semantics in the first place.
%
% Why is the stateless semantics not simply (V, T+ -> T+∞)?
% ---------------------------------------------------------
% ... and in the process could abandon the value action (if we were willing
% to detect stuckness by looking at V)?
%
% Because then we can't see the prefix T+ when we have to extend ρ at let
% bindings. But this is our primary mechanism for sharing! Similarly for
% call-by-value.
%
% So we are stuck with value actions. Still, we could decide to return the
% value "out of band", in a pair, T+ -> (V, T+∞). That yields the worse of
% both worlds; the definition is similar to T+ -> T+∞ but we often need to
% adjust the second component of the pair; plus, in `memo`, we still have
% to "execute" the semantics S for its value, because we can't recover it
% from the trace.
%
% Which info do we need to attach to an action?
% ---------------------------------------------
% TLDR; that is determined by transition semantics that we want to
% be to abstract a trace into. The reasoning is as follows:
%
%   * "The transition semantics" is really the semantics we get by
%     applying the transition abstraction α_τ to a *stateful* trace
%     semantics, where the states capture enough information for
%     the resulting transition system to become deterministic.
%     (The transition system we get by abstracting the *stateless* trace
%     semantics isn't very useful precisely for that reason; taking
%     labels as state yields too many spurious transitions.)
%   * So determinism of the abstracted transition system is a quality
%     of the semantic richness of states (given that the sequence of states
%     is fixed); let's call state structure that allows for deterministic
%     abstraction "complete"
%   * (Are all complete state structures isomorph? Probably not)
%   * The stateful trace semantics is an abstraction of the stateless trace
%     semantics by way of α_S. We want to produce (at least one) stateful
%     semantics where the state space is complete. To produce such states, the
%     necessary information must be part of the stateless trace, otherwise we
%     can't write the abstraction function from stateless to stateful.
%
% So given the completeness of the states produced by α_S as a goal, we can
% make the following claims for action kinds in a trace:
%
%   * AppI, AppE, Lookup, Bind are all necessary actions because they make
%     a step from one label to a label of a subexpression.
%   * Val actions are the trace semantics' means of communicating a successful
%     (e.g., not stuck) execution as well as playing the role of `Just value`.
%     They correspond to Val transtitions in CESK or STG's ReturnCon
%   * We do *not* strictly need Update actions -- we just need update frames
%     in the stateful trace, but those could be reconstructed from when a
%     Lookup's balanced trace ended. Update actions make our life simpler in
%     other ways, though: See "The need for an update action".
%
% In fact, given that each action corresponds to a CESK transition, α_S can be
% defined inductively (by prefixes) on actions and states:
%
%   * The data on actions is simply erased and the corresopnding CESK transition
%     is taken. (Vital to realise that a well-formed stateless trace results in
%     an ok stateful trace.)
%   * For the state after prefix τ, we simply call \varrho(τ). It is a function
%     that Cousot uses throughout his book, and so should we.
%
% Now as to what information we need on the actions:
%
%   * AppI: We need the Var so that α_S can produce an Apply frame
%   * AppE: We need both the Var *and* the D so that varrho can produce the
%           proper environment.
%   * Lookup: We need the address so that we can push an update frame in α_S.
%             Also we need it to find the corresponding Bind.
%   * Bind: We need the address, so α_S can find it when encountering a Lookup
%           at that address. Then we also need the Var and the D for varrho.
%   * Update: The address is convenient (as are update actions to begin with),
%             otherwise we'd have to fiddle with balanced traces in memo to find
%             the corresponding Lookup.
%   * Values: It is convenient to attach values to Val actions; this is not strictly
%     necessary, just convenient. (See page 4 of 61e6f8a, quite ugly.)

\subsection{Abstraction}

% 1. Move Lookup into Env; have Env = Name :-> D and let its action look into
%    the Heap. Requires a bit of tricky setup in let_
% 2. Move the Env to meta level
% 3. Realise that we can now get rid of the stack, since everything happens
%    on the meta call stack
% 4. Materialise the state as needed during memo

\begin{figure}
\[\begin{array}{rcl}
  μ_ρ(τ)(\pa) & = & \begin{cases}
    \varrho(τ_1) & \text{if $τ_1 \act{\UpdateA(\pa)} \return \concat τ_2 = τ$} \\
    \varrho(τ_1 \act{\BindA(\px,\pa↦\pe,d)} \pc) & \text{if $τ_1 \act{\BindA(\px,\pa↦\pe,d)} \pc \concat τ_2 = τ$} \\
  \end{cases}  \\
  \varrho(τ \act{\BindA(\px,\pa↦\pe,d)} \pc\trend)(\px) & = & deref(\pa) \\
  \varrho(τ \act{\AppEA(\px↦d)} \pc\trend)(\px) & = & d \\
  \varrho(τ \act{\LookupA(\pa)} \pc\trend)(\px) & = & μ_ρ(τ)(\pa)(\px) \\
  \varrho(τ \act{a} \pc\trend)(\px) & = & \varrho(τ)(\px) \\
\end{array}\]
\caption{Materialising heap and environment from an initialisation trace}
  \label{fig:materialisation}
\end{figure}

\begin{figure}
\[\begin{array}{c}
 \begin{array}{rcl}
  \multicolumn{3}{c}{ \ruleform{ α^{\States} : ((\Var \pfun \MaxD) \to \MaxD) \to \StateD } } \\
  \\[-0.5em]
  deref^{-1}(d) & = & \pa \text{ such that $d = deref(\pa)$} \\
  tgt_\States^{-1}(σ) & = & τ \text{ such that $\validtrace{τ}$ and $tgt_\States(τ) = σ$} \\
  \\[-0.5em]
  α_{\STraces}(τ^i, \pc\trend) & = & α_{\States}(τ^i \concat \pc\trend)\trend \\
  α_{\STraces}(τ^i, τ \act{a} \pc\trend) & = & α_{\STraces}(τ^i,τ); α_{\States}(τ^i \concat τ \act{a} \pc\trend)\trend \\
  α_{\States}(τ) & = & (α_{\Controls}(τ,tgt(τ)), α_{\Environments}(\varrho(τ)), α_{\Heaps}(τ) \circ μ(τ), α_{\Continuations}(τ)) \\
  α_{\Controls}(τ, \return) & = & (\pv,α_{\StateV}(v)) \text{ where $\getval{τ}{(\pv,v)}$} \\
  α_{\Controls}(τ, \pe) & = & \pe \\
  α_{\Environments}(ρ) & = & deref^{-1} \circ ρ \\
  α_{\Heaps}(τ)([\many{\pa ↦ (\pe,d)}]) & = & [\many{\pa ↦ (\pe, μ_ρ(τ)(\pa), α_{\StateD}(d))}] \\
  α_{\Continuations}(\pc\trend) & = & \StopF \\
  α_{\Continuations}(τ \act{\LookupA(\pa)} \pc\trend) & = & \UpdateF(\pa) \pushF α_{\Continuations}(τ) \\
  α_{\Continuations}(τ \act{\UpdateA(\pa)} \pc\trend) & = & κ \text{ where $α_{\Continuations}(τ) = \UpdateF(\pa) \pushF κ$} \\
  α_{\Continuations}(τ \act{\AppIA(d)} \pc\trend) & = & \ApplyF(deref^{-1}(d)) \pushF α_{\Continuations}(τ) \\
  α_{\Continuations}(τ \act{\AppEA(\px↦d)}   \pc\trend) & = & κ \text{ where $α_{\Continuations}(τ) = \ApplyF(deref^{-1}(d)) \pushF κ$} \\
  α_{\Continuations}(τ \act{} \pc\trend) & = & α_{\Continuations}(τ) \\
  α_{\StateD}(d) & = & (\fn{τ}{α_{\STraces}(τ,d(τ))) \circ γ_{\STraces} \circ tgt_\States^{-1}} \\
  α_{\StateV}(\FunV(f)) & = & \FunV(α_{\StateD} \circ f \circ deref) \\
  γ_{\StateD}(d) & = & γ_{\STraces} \circ d \circ tgt_\States \circ α_{\STraces} \\
  γ_{\StateV}(\FunV(f)) & = & \FunV(γ_{\StateD} \circ f \circ deref^{-1}) \\
  γ_{\STraces}(σ\trend) & = & γ_{\States}(σ)\trend \\
  γ_{\STraces}(τ; σ\trend) & = & γ_{\STraces}(τ) \act{γ_{\Actions}(tgt_\States(τ) \smallstep σ)} γ_{\States}(τ; σ\trend)\trend \\
  γ_{\States}(γ,\wild,\wild,\wild) & = & α_{\Controls}(γ) \\
  γ_{\Controls}((\pv,v)) & = & \return \\
  γ_{\Controls}(\pe) & = & \pe \\
  γ_{\Actions}(\BindT(\px,\pa,\pe_1,d)) & = & \BindA(\px,\pa↦\pe_1,γ_{\StateD}(d)) \\
  γ_{\Actions}(\UpdateT(\pa)) & = & \UpdateA(\pa) \\
  γ_{\Actions}(\LookupT(\pa)) & = & \LookupA(\pa) \\
  γ_{\Actions}(\AppIT(\pa)) & = & \AppIA(deref(\pa)) \\
  γ_{\Actions}(\AppET(\px,\pa)) & = & \AppEA(\px,deref(\pa)) \\
  γ_{\Actions}(\ValueT(\pv,v)) & = & \ValueA(\pv,γ_{\StateV}(v)) \\
  \\
 \end{array}
\end{array}\]
\caption{Prefix Trace Abstraction}
  \label{fig:semantics}
\end{figure}

%\section{Related Work}
\label{sec:related-work}

%\sg{Move to related work.}
%There have been attempts to discern crashes from other kinds of loops, such as
%\cite{imprecise-exceptions}. Unfortunately, in Section 5.3 they find it
%impossible give non-terminating programs a denotation other than $\bot$, which
%still encompasses all possible exception behaviors.
%
% eval/apply or push/enter?
% Given an expr like $f x$, we first push $ρ(x)$ onto the stack and then
% evaluate $f$, which will look it up (pushing an udpate frame) and evaluate its
% RHS. Since we will never return to the "eval site" of $f$, IMO this qualifies
% as push/enter rather than eval/apply. Which is in contrast to what the Krivine
% paper says, which dubs return states as "apply" transitions


\nocite{*}

\clearpage
\bibliography{references}

\fi % \ifmain

% Appendix
\ifappendix
\appendix
\section{Appendix}\label{sec:appendix}
%include custom.fmt

\renewcommand\thefigure{\thesection.\arabic{figure}}

\subsection{Proofs}

\providecommand{\tr}{\ensuremath{\tilde{ρ}}}

\begin{proof}[Of \Cref{thm:semusg-correct-live}]
  \label{prf:semusg-correct-live}
  Let us fix $\pe$ and $\px$ and let us assume that there exists $\tr$ such that
  $\tr(\px) \not⊑ \semusg{\pe}$. The goal is to show that $\px$ is dead in $\pe$,
  so we are given an arbitrary $ρ$, $d_1$ and $d_2$ and have to show that
  $\semscott{\pe}_{ρ[\px↦d_1]} = \semscott{\pe}_{ρ[\px↦d_2]}$.
  We proceed by induction on $\pe$:
  \begin{itemize}
    \item \textbf{Case $\pe\triangleq\py$}: If $\px=\py$, then
      $\tr(\px) \not⊑ \semusg{\py}_{\tr} = \tr(\py) = \tr(\px)$, a contradiction.
      If $\px \not= \py$, then varying the entry for $\px$ won't matter; hence
      $\px$ is dead in $\py$.
    \item \textbf{Case $\pe\triangleq\Lam{\py}{\pe'}$}: The equality follows from
      pointwise equality on functions, so we pick an arbitrary $d$ to show
      $\semscott{\pe'}_{ρ[\px↦d_1][\py↦d]} = \semscott{\pe'}_{ρ[\px↦d_2][\py↦d]}$.

      This is simple to see if $\px=\py$. Otherwise, $\tr[\py↦\bot]$ (recall
      that $\bot \colon \UsgD$ is defined pointwise, $\bot(\pz) \triangleq
      \bot_u = 0$) witnesses the fact that
      \[
        \tr[\py↦\bot](\px) = \tr(\px) \not⊑
        \semusg{\Lam{\px}{\pe'}}_{\tr} = \semusg{\pe'}_{\tr[\py↦\bot]}
      \]
      so we can apply the induction hypothesis to see that $\px$ must be dead in
      $\pe'$, hence the equality on $\semscott{\pe'}$ holds.
    \item \textbf{Case $\pe\triangleq\pe'~\py$}:
      From $\tr(\px) \not⊑ \semusg{\pe'}_{\tr} + ω*\tr(\py)$ we can see that
      $\tr(\px) \not⊑ \semusg{\pe'}_{\tr}$ and $\tr(\px) \not⊑ \tr(\py)$ by
      monotonicity of $+$ and $*$.
      If $\px=\py$ then the latter inequality leads to a contradiction.
      Otherwise, $\px$ must be dead in $\pe'$, hence both cases of
      $\semscott{\pe'~\py}$ evaluate equally, differing only in
      the environment. It remains to be shown that
      $ρ[\px↦d_1](\py) = ρ[\px↦d_2](\py)$, and that is easy to see since
      $\px \not= \py$.
    \item \textbf{Case $\pe\triangleq\Let{\py}{\pe_1}{\pe_2}$}:
      We have to show that
      \[
        \semscott{\pe_2}_{ρ[\px↦d_1][\py↦d'_1]} = \semscott{\pe_2}_{ρ[\px↦d_2][\py↦d'_2]}
      \]
      where $d'_i$ satisfy $d'_i = \semscott{\pe_1}_{ρ[\px↦d_i][\py↦d'_i]}$.
      The case $\px = \py$ is simple to see, because $ρ[\px↦d_i](\px)$ is never
      looked at.
      So we assume $\px \not= \py$ and see that $\tr(\px) = \tr'(\px)$, where
      $\tr' = \operatorname{fix}(\fn{\tr'}{\tr ⊔ [\py ↦ \semusg{\pe_1}_{\tr'}]})$.

      We know that
      \[
        \tr'(\px) = \tr(\px) \not⊑ \semusg{\pe}_{\tr} = \semusg{\pe_2}_{\tr'}
      \]
      So by the induction hypothesis, $\px$ is dead in $\pe_2$.

%      Now consider the predicate $P(\tr) = \tr(\px) ⊑ \semusg{\pe_1}_{\tr}$.
%      We must prove it admissable to see that it holds (by fixpoint induction)
%      for $\tr'$. That is clearly the case because it is a composition of
%      continuous functions ($\tr, \semusg{\pe_1}$) and admissable predicates
%      ($⊑$).
%
%      SG: I think we don't need to prove that P above is admissable because
%      we never try to prove it through fixpoint induction; we simply apply LEM.

      We proceed by cases over $\tr(\px) = \tr'(\px) ⊑ \semusg{\pe_1}_{\tr'}$.
      \begin{itemize}
        \item \textbf{Case $\tr'(\px) ⊑ \semusg{\pe_1}_{\tr'}$}: Then
          $\tr'(\px) ⊑ \tr'(\py)$ and $\py$ is also dead in $\pe_2$ by the above
          inequality.
          Both deadness facts together allow us to rewrite
          \[
            \semscott{\pe_2}_{ρ[\px↦d_1][\py↦d'_1]} = \semscott{\pe_2}_{ρ[\px↦d_1][\py↦d'_2]} = \semscott{\pe_2}_{ρ[\px↦d_2][\py↦d'_2]}
          \]
          as requested.
        \item \textbf{Case $\tr'(\px) \not⊑ \semusg{\pe_1}_{\tr'}$}:
          Then $\px$ is dead in $\pe_1$ and $d'_1 = d'_2$. The goal follows
          from the fact that $\px$ is dead in $\pe_2$.
      \end{itemize}
  \end{itemize}
\end{proof}

\subsection{Detailed Semantics Specification in ML}

\lstinputlisting[style=hsyl-style,language=ML,firstline=3,lastline=90]{semantics.sml}

\fi % \ifappendix

\end{document}

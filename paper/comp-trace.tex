% -*- mode: LaTeX -*-
%% For double-blind review submission, w/o CCS and ACM Reference (max submission space)
%\documentclass[acmsmall,review,anonymous,natbib=false]{acmart}\settopmatter{printfolios=true,printccs=false,printacmref=false}
%% For double-blind review submission, w/ CCS and ACM Reference
%\documentclass[acmsmall,review,anonymous]{acmart}\settopmatter{printfolios=true}
%% For single-blind review submission, w/o CCS and ACM Reference (max submission space)
\documentclass[draft,acmsmall,review]{acmart}\settopmatter{printfolios=true,printccs=false,printacmref=false}
%% For single-blind review submission, w/ CCS and ACM Reference
%\documentclass[acmsmall,review]{acmart}\settopmatter{printfolios=true}
%% For final camera-ready submission, w/ required CCS and ACM Reference
%\documentclass[acmsmall,screen]{acmart}\settopmatter{}

%\documentclass[acmsmall,review,anonymous]{acmart}\settopmatter{printfolios=true,printccs=false,printacmref=false}

%% Journal information
%%
\setcopyright{rightsretained}
\acmPrice{}
\acmDOI{10.1145/1111111}
\acmYear{2024}
\copyrightyear{2024}
\acmSubmissionID{popl24main-p11-p}
\acmJournal{PACMPL}
\acmVolume{1}
\acmNumber{POPL}
\acmArticle{1}
\acmMonth{1}

%% Bibliography style
\bibliographystyle{ACM-Reference-Format}
%% Citation style
%% Note: author/year citations are required for papers published as an
%% issue of PACMPL.
\citestyle{acmauthoryear}   %% For author/year citations

%%%%%%%
\usepackage{apxproof}
\usepackage{array} % \newcolumntype
\usepackage{color}
\usepackage{ifdraft}
%\usepackage[svgnames]{xcolor}
\usepackage{cleveref}
\usepackage{xspace}
\usepackage{url}
\usepackage{varwidth}
\usepackage{galois}
\usepackage{hsyl-listing} % SML listing style, hsyl-style
%\usepackage{scalerel}
%\usepackage[all]{xy}
\usepackage{relsize} % relscale
\usepackage{xfp} % fpeval
%\usepackage{stackengine}
\usepackage{mathtools} % xhookrightarrow
\usepackage{trimclip} % clipbox
\usepackage{mathpartir} % inference rules
\usepackage{subcaption}
\usepackage{mathbbol} % \bbcolon and \bbquestionmark
\usepackage{stmaryrd} % \lightning
\usepackage{tikz}
\usetikzlibrary{cd} % commutative diagrams
\usetikzlibrary{calc}
\usetikzlibrary{fit}
\usetikzlibrary{patterns}
\usetikzlibrary{matrix}
\usetikzlibrary{decorations.pathreplacing}
\usetikzlibrary{decorations.pathmorphing}
\usepackage{placeins} % flush floats with \FloatBarrier
\usepackage[T1]{fontenc} % https://tex.stackexchange.com/a/181119
\usepackage[mark]{gitinfo2}

\usepackage{utf8-symbols}
% Theorems
\theoremstyle{plain} % default
\newtheorem{theorem}{Theorem}
\newtheorem{lemma}[theorem]{Lemma}
\newtheorem{proposition}[theorem]{Proposition}
\crefname{proof}{Proof}{Proofs}

\theoremstyle{definition}
\newtheorem{example}[theorem]{Example}
\newtheorem{definition}[theorem]{Definition}

% Abbrev
\newcommand\eg{\emph{e.g.}\ }
\newcommand\etc{\emph{etc.}}

% Comments and notes
\newcommand{\slpj}[1]{\emph{SLPJ: #1}}
\newenvironment{slpjenv}{\em SLPJ:}{}
\newcommand{\sg}[1]{\emph{SG: #1}}
\newcommand{\todo}[1]{\textcolor{red}{TODO: #1}}

% Auxiliary
\newcommand{\many}[1]{\overline{#1}}
\newcommand{\wild}{\ensuremath{\mathunderscore}}
\newcommand{\pfun}{\rightharpoonup}
\newcommand{\ruleform}[1]{\fbox{$#1$}}
\newcommand{\highlight}[1]{\setlength{\fboxsep}{2pt}\colorbox[gray]{0.8}{\ensuremath{#1}}}
\newcommand{\nelems}[1]{\lvert#1\rvert}
\newcommand{\fn}[2]{\ensuremath{λ#1.#2}}
\newcommand{\constfn}[1]{\fn{\mathunderscore}{#1}}
%\newcommand{\repeat}[2]{\foreach \n in {1,...,#1}{#2}} % https://tex.stackexchange.com/a/16190/52414
% https://latex.org/forum/viewtopic.php?p=62177&sid=26890ac77d076f3338384a47a2ffd4bc#p62177
\makeatletter
\newcommand{\concat}{%
  \mathbin{\mathop{\cdot}\limits^{\vbox to -1.7\ex@{\kern-\tw@\ex@
   \hbox{\scriptsize$\smallfrown$}\vss}}}}
\makeatother

% Syntax
%% LC
\newcommand{\Con}{\mathsf{Con}}
\newcommand{\Var}{\mathsf{Var}}
\newcommand{\Lab}{\mathsf{Lab}}
\newcommand{\Exp}{\mathsf{Exp}}
\newcommand{\Let}[3]{\mathbf{let}~{#1}={#2}~\mathbf{in}~{#3}}
\newcommand{\LLet}[4]{\mathbf{let}_{#1}~{#2}={#3}~\mathbf{in}~{#4}}
\newcommand{\Case}[2]{\mathbf{case}~{#1}~\mathbf{of}~{#2}}
\newcommand{\Lam}[2]{\lambda #1. #2}
\newcommand{\LLam}[3]{\lambda_{#1} #2. #3}
\newcommand{\dom}{\mathop\mathsf{dom}}
\newcommand{\fv}{\mathop\mathsf{fv}}
\newcommand{\bv}{\mathop\mathsf{bv}}
\newcommand{\fresh}[2]{#1\mathbin{\#}#2}
\newcommand{\delvar}[1]{{\setminus_{#1}}}
%% Actions
\newcommand{\Actions}{\mathbb{A}}
\newcommand{\ValA}{\mathsf{val}}
\newcommand{\BindA}{\mathsf{bind}}
\newcommand{\LookupA}{\mathsf{look}}
\newcommand{\AppA}{\mathsf{app}}
\newcommand{\BetaA}{\mathsf{beta}}
%% Labels
\newcommand{\lbl}{\ensuremath{\ell}}
\newcommand{\lbln}[1]{\ensuremath{\ell_{#1}}}
\newcommand{\slbl}{\raisebox{0.08em}{$\;\scriptscriptstyle{\ell}\;$}}
\newcommand{\slbln}[1]{\raisebox{0.1em}{$\;\scriptscriptstyle{\ell_{#1}}\;$}}
%% Traces
\newcommand{\Traces}{\mathbb{T}}
\newcommand{\act}[1]{\xrightarrow{#1}}
\newcommand{\balanced}[1]{#1\;\mathsf{bal}}
%% Values
\newcommand{\Values}{\mathbb{V}}
\newcommand{\FunV}{\mathsf{fun}}
%% Domains
\newcommand{\Domain}[1]{\mathbb{D}^{#1}}
\newcommand{\PrefD}{\Domain{*}}
\newcommand{\MaxD}{\Domain{+\infty}}
\newcommand{\AbsD}{\Domain{\raisebox{0.1em}{\scalebox{0.4}{$\square$}}}}

% Math
\newcommand{\poset}[1]{\wp(#1)}
\newcommand{\Nat}{\mathbb{N}}

% Order theory
\DeclareMathOperator*{\lub}{\sqcup}
\DeclareMathOperator*{\Lub}{\bigsqcup}
\DeclareMathOperator*{\glb}{\sqcap}
\DeclareMathOperator*{\Glb}{\bigsqcap}
\newcommand{\lfp}{\mathsf{lfp}}

% Semantics
\newcommand{\denot}[1]{\llbracket {#1} \rrbracket}
\newcommand{\sempref}[1]{\mathcal{S}^*\denot{#1}}
\newcommand{\seminf}[1]{\mathcal{S}^{+\infty}\denot{#1}}



% Tables should have the caption above
\floatstyle{plaintop}
\restylefloat{table}

\begin{document}

%\special{papersize=8.5in,11in}
\setlength{\pdfpageheight}{\paperheight}
\setlength{\pdfpagewidth}{\paperwidth}

\title{A Compositional Trace Semantics for Lambda Calculus}
\subtitle{Or: Denotational semantics for Call-by-need Lambda Calculus}

\author{Sebastian Graf}
\affiliation{%
  \institution{Karlsruhe Institute of Technology}
  \city{Karlsruhe}
  \country{Germany}
}
\email{sgraf1337@gmail.com}

\author{Simon Peyton Jones}
\affiliation{%
  \institution{Epic Games}
  \city{Cambridge}
  \country{UK}
}
\email{simon.peytonjones@gmail.com}

% Some conditional build stuff for handling the Appendix

\newif\ifmain
\newif\ifappendix

% Builds only the main paper by default.
\maintrue
\appendixfalse
% But we provide a switch to build the Appendix only.
\def\appendixonly{\mainfalse{}\appendixtrue{}}

% .. so that you can comment out the following line to build the Appendix only
% This is done by the `make appendix` target.
%\appendixonly

% Same thing for an extended version that includes the Appendix
\def\extended{\maintrue{}\appendixtrue{}}
%\extended

\ifmain

\begin{abstract}
  Motivated by observing
\end{abstract}

%% 2012 ACM Computing Classification System (CSS) concepts
%% Generate at 'http://dl.acm.org/ccs/ccs.cfm'.
\begin{CCSXML}
<ccs2012>
   <concept>
       <concept_id>10011007.10011006.10011041</concept_id>
       <concept_desc>Software and its engineering~Compilers</concept_desc>
       <concept_significance>500</concept_significance>
       </concept>
   <concept>
       <concept_id>10011007.10011006.10011073</concept_id>
       <concept_desc>Software and its engineering~Software maintenance tools</concept_desc>
       <concept_significance>300</concept_significance>
       </concept>
   <concept>
       <concept_id>10011007.10011006.10011008.10011024.10011035</concept_id>
       <concept_desc>Software and its engineering~Procedures, functions and subroutines</concept_desc>
       <concept_significance>100</concept_significance>
       </concept>
   <concept>
       <concept_id>10011007.10011006.10011008.10011024.10011032</concept_id>
       <concept_desc>Software and its engineering~Constraints</concept_desc>
       <concept_significance>300</concept_significance>
       </concept>
   <concept>
       <concept_id>10011007.10011006.10011008.10011009.10011012</concept_id>
       <concept_desc>Software and its engineering~Functional languages</concept_desc>
       <concept_significance>300</concept_significance>
       </concept>
   <concept>
       <concept_id>10011007.10011006.10011008.10011009.10011021</concept_id>
       <concept_desc>Software and its engineering~Multiparadigm languages</concept_desc>
       <concept_significance>300</concept_significance>
       </concept>
 </ccs2012>
\end{CCSXML}

\ccsdesc[500]{Software and its engineering~Compilers}
\ccsdesc[300]{Software and its engineering~Software maintenance tools}
\ccsdesc[100]{Software and its engineering~Procedures, functions and subroutines}
\ccsdesc[300]{Software and its engineering~Constraints}
\ccsdesc[300]{Software and its engineering~Functional languages}
\ccsdesc[300]{Software and its engineering~Multiparadigm languages}
%% End of generated code

%% Keywords
%% comma separated list
\keywords{Programming language semantics}  %% \keywords are mandatory in final camera-ready submission

\maketitle

\section{Introduction}
\label{sec:introduction}

As an implementor of a programming language, it is often useful to automatically
glean facts about a program such as ``this program is well-typed'', ``this
higher-order function is always called with argument $\Lam{x}{x+1}$'' or ``this
program never evaluates $x$'' by way of \emph{static (program) analysis}.

If the implementation language is a functional one, then usually such static
analyses are formulated as a function defined by \emph{structural recursion} on
the input term.
For example, given an application expression $(\pe_1~\pe_2)$,
the property ``$(\pe_1~\pe_2)$ never evaluates its free variable $x$'' can be
\emph{conservatively approximated} (here: It's OK to say ``No'' more often) by
the property ``$\pe_1$ and $\pe_2$ never evaluate $x$''.

It is quite convenient to formulate \emph{potential liveness analysis}
structurally:
(1) Structural recursion gives an immediate proof of termination and can
    further be exploited in other inductive proofs, all within decidable
    territory.
(2) A structurally-defined function $f$ is often \emph{compositional}, meaning that
    the result of $f\denot{\pe_1~\pe_2}$ is a function of $f\denot{\pe_1}$ and
    $f\denot{\pe_2}$, but not of the structure of either $\pe_1$ or $\pe_2$.
    Compositionality makes it easy for humans to understand and reason about the
    function, as well as allowing to abstract away program syntax in the first
    place.

For static analyses, especially more complicated ones, it is good practice to
provide a proof of correctness of some sort. If the correctness statement can
be expressed in terms of a \emph{denotational
semantics}~\cite{ScottStrachey:71}, then the recursion structure of analysis
function and semantics function line up nicely. As a result, the proof of
correctness can be conducted by simple induction over the expression and
abstract interpretation~\cite{Cousot:21} provides a neat framework to derive
static analyses that are proven correct by simple calculation.

\subsection{Domain Theory is leaky and too abstract}
Alas, there are several shortcomings to traditional denotational semantics:
\begin{itemize}
  \item Infinite program behaviors are denoted by $\bot$, the least element of
        the approximation order on elements of an algebraic domain.
        This encoding necessitates semantic properties to be proven
        \emph{admissable} before they can be used. This means that although
        the denotational semantics might be ``standard'', the semantic domain
        can't be treated like a black box --- the abstraction is leaky and
        $\bot$ is lurking everywhere to make life complicated.
  \item Traditional domain theory proved non-compositional~\cite{WrightFelleisen:94},
        in that language features such as exceptions, state and nondeterminism
        require individual, complicated adjustments where it is far from clear
        how to combine these adjustments.
  \item There have been denotational semantics for call-by-name and
        call-by-value, but none for call-by-need. That is a symptom of the fact
        that traditional domain theory eschews any account of operational
        detail, such as evaluation cardinality.
\end{itemize}

The third point is important in the context of quantitative type
systems~\cite{Atkey:18} and optimising compilers for call-by-need languages such
as the Glasgow Haskell Compiler.
The first two points have been noted by \citep{WrightFelleisen:94}, establishing
the now standard approach of operational semantics as the Gold Standard.
The past twenty years have proven that the methodology scales to
big and complicate type systems, building on techniques such as
\emph{step-indexing}~\cite{AppelMcAllester:01,DreyerAhmedBirkedal:11}.

\subsection{Rewrite systems are syntactic and not structural}
However, transition systems don't easily admit a compositional definition as a
function of the input expression \sg{Cousot has derived trace-based fixpoint
semantics from a transition system~\cite{Cousot:02}, but the resulting function
is not structurally-recursive in the expression}. In giving in to operational
semantics, semanticists have chosen predictable step-wise reasoning over elegant
equational reasoning backed by compositionality.

In preservation proofs of type systems, step-wise reasoning necessitates
substitution lemmas to cope with the non-structural recursion during variable
lookup and beta reduction, an inconvenience that is readily accepted when the
alternative would be to tame domain theory.

It is not so simple in lattice-based compiler analyses, because recursive let
bindings invite fixpoint iteration. If you think about ``well-annotatedness''
of a machine configuration according to an analysis as a type system, the
corresponding ``preservation lemma'' needs to relate fixpoints before and after
\emph{every} kind of machine transition! This necessitates a dreadful application of fixpoint
induction, unfolding the whole analysis function body before and after and thus
culminating in beaurocratic nightmare.
Of course, the definition of ``well-annotatedness'' needs to encompass whole
machine configurations, including the mutually recursive stack and heap, see the
proofs in~\cite{cardinality-ext} for a taste. It is hard to raise confidence in
such a proof without full mechanisation.
A denotational semantics would arguably simplify the proof, unfortunately one
allowing to observe evaluation cardinality of call-by-need is lacking to this
day.

When there is no semantics that structurally matches our analysis, we could
try to rearrange our analysis. That is the idea of \emph{Abstracting Abstract
Machines}~\cite{aam} which abandons compositional analysis in favour of a
convenient proof by abstract interpretation of the transition semantics.
Not only is it deeply unsatisfying to be pushed into such rearrangements, it
is also impossible to perform in a production-grade compiler because of the
inevitable regressions it introduces.

\subsection{Summary of Contributions}

Our ambitious goal is to resolve the aforementioned tensions via the following
contributions:
\begin{itemize}
  \item In \Cref{sec:problem}, we highlight the main points raised in the
    Introduction by way of example analyses.
  \item In \Cref{sec:semantics}, we give a compositional semantics
    for call-by-need lambda calculus that generates potentially infinite
    small-step traces. Unlike traditional formulations of denotational
    semantics, our semantics is defined by \emph{guarded recursion} and thus
    total as a mathematical function. Our semantics is also distinct from
    similar ones for call-by-name or call-by-value and allows to observe
    evaluation cardinality as needed.
    We believe that our semantics is the first with the aforementioned two
    qualities and prove it correct \wrt a standard operational semantics. The
    idea borrows heavily from the idea of a maximal prefix trace semantics
    advocated by \citep{Cousot:21}.
  \item The semantics in \Cref{sec:semantics} is one generating \emph{stateful}
    traces in a standard operational semantics, and serves mostly as a
    convenient bridge for proving bisimulation \sg{No longer}. In
    \Cref{sec:stateless} we will define an equivalent, but more convenient
    \emph{stateless} semantics and we will see how to recover necessary state
    from program history as needed.
  \item We employ the stateless semantics as a collecting semantics and derive
    $\semlive{\wild}$ by calculational design \cite{Cousot:21}.
    Similar derivations will be made for a simple type system as well as for
    control flow analysis. \sg{Hopefully :)}
  \item Talk about prototype in Haskell?
  \item Related Work \Cref{sec:related-work}
\end{itemize}

% TO RECYCLE:
%
%\paragraph{Analysis follows structure}
%Alas, even when the denotational semantics is ``standard'', the hard part is in
%coming up with a suitable correctness predicate!
%
%Traditionally, denotational semantics denotes diverging and stuck programs with
%$\bot$, corresponding to the function that is undefined everywhere.
%There is rich and complicated literature on defining an algebraic
%domain~\cite{Scott:71} that is suitable to denote untyped lambda calculus.
%The key is embedding the subclass of \emph{continuous} functions between domain
%elements into the domain itself. All computable functions are continuous,
%so an algebraic domain is a sufficient substrate for giving meaning to all
%computable functions.
%
%Now, to prove a predicate $P(d)$ of some denotation $d$, such as ``$d$ has type
%$τ$'' by structural induction, one has to show first that $P$ is compatible
%with continuity; perhaps by proving that $P$ characterises a sub-domain or an
%ideal in the domain~\cite{Milner:78}.
%
%Note that this is all \emph{before} even attempting the proof! Often, the
%denotational semantics (for example to distinguish \textbf{wrong} behaviors from
%diverging ones~\cite{Milner:78}) or even the domain itself needs adjustments.
%The former case needs a proof of continuity, while the latter can be quite
%involved and non-compositional for effects such as exceptions, concurrency and
%state, as \citep{WrightFelleisen:94} noted.
%
%We think that most of the troubles in the application of domain theory are
%ultimately caused by the non-commital nature of the approximation order, in that
%any predicate on total elements also needs to accept its partial approximations;
%hence it is desirable to strive for \emph{total} descriptions of the potentially
%program infinite behaviors.
%
%\citep{WrightFelleisen:94} give a detailed account of the shortcomings of
%partial denotational semantics and propose to use small-step operational
%semantics paired with the now standard syntactic approach to type soundness.
%They demonstrate that their approach scales well to composition of different
%language features such as state, exceptions and nondeterminism. The past
%twenty years have scaled the approach to big and complicated type systems,
%
%What's more, small-step operational semantics distinguish infinite
%and stuck behaviors quite naturally; a distinction that is crucial
%in (a correctness property of) control-flow and liveness analysis.
%
%the crucial property that any
%, and it was
%a great achievement of \citep{Scott:71} turns out that Domain Theory is the result of equipping Doing so invites
%and stuck programs. Consider the following program using recursive let that is
%infinitely-looping
%\begin{equation}
%  \label{eqn:loop}
%  \pe_{loop} \triangleq \Let{id}{\Lam{x}{x}}{\Let{loop}{id~loop}{loop}}
%\end{equation}
%The traditional denotational semantics after Scott and Strachey would equate all
%of the following programs:
%$\semscott{\pe_{loop}}_ρ = \semscott{\Let{loop}{id~loop}{loop}}_ρ =
%\semscott{\mathsf{segfault}}_ρ = \bot$.
%Note that the first program has an infinite loop, the second one is not
%well-scoped (thus stuck at some point) and the last one is a straight out crash
%in the style of an imprecise exception \cite{imprecise-exceptions}.
%To a compiler developer, this conflation is both a reason for joy (more
%optimisation opportunities) and a reason for ... reflection (users didn't expect
%their infinite loop to be optimised into a crash). Such issues come up in
%practice \sg{cite GHC issues}.
%
%More seriously, it is impossible to prove by way of a denotational semantics
%that a static analysis does not misoptimise infinite behaviors.
%(a) The \emph{potential liveness} analysis that says ``$\pe_{loop}$ never
%evaluates $id$'' could be proven ``correct''.
%(b) A type analysis that says ``$\Let{loop}{id~loop}{loop}$ is closed and
%well-typed'' can still be proven progressing as long as $\pe_{loop}$ can
%be well-typed (which would not be too surprising), because the latter is
%denotationally equivalent to the former.
%(c) Imagine that $id$ was supplied as a parameter to $loop$ instead, \eg
%$...\ \Let{loop}{\Lam{f}{f~(loop~f)}}{loop~id}$. Then a control-flow analysis
%\cite{Shivers:91} that says ``$f$ is never bound to $id$'' can be proven correct
%in terms of the denotational semantics.
%
%Furthermore, although it is sensible (in the terminating case) to ask whether or
%not $x$ is \emph{never} evaluated in terms of the denotational semantics, asking
%whether $x$ is evaluated \emph{at most once} is not, for the same reason that
%traditional denotational semantics is not able to discern call-by-name from
%call-by-need. Yet, to the Glasgow Haskell Compiler, this distinction is very
%much of concern!
%
%When denotational semantics fails the compiler developer, they turn to a
%correctness criterion in terms of a \emph{structural operational semantics}.
%This was the approach taken by \cite{cardinality} to prove evaluation
%cardinality properties such as potential liveness. The drawback of this proof
%framework is the immense complexity arising from the disconnect between a
%structural definition and a transition system; matters such as substitution,
%multiple heap activations of the let binding, non-determinism and fixpoint
%induction abound.
%One could adopt the approach of \emph{Abstracting Abstract Machines} \cite{aam}
%and let the structure of the semantics dictate the structure of the
%analysis for a re-usable proof of correctness via abstract interpretation
%\cite{Cousot:21}.
%However, that is not how the static analyses work that the authors are familiar
%with.
%For example, it would be quite an effort to rewrite the neat,
%structurally-defined analyses of the Glasgow Haskell Compiler into a fixpoint
%iteration on the approximated states of an abstract transition system.

\section{Problem Statement}
\label{sec:problem}

By way of the poster child example of a compositional definition of \emph{usage
analysis}, we showcase how the operational detail available in traditional
denotational semantics is too coarse to substantiate a correctness criterion,
although the \emph{proof} of (a weaker notion of) correctness is simple and direct.
While operational semantics observe sufficient detail to formulate a correctness
criterion, it is quite complicated to come up with a suitable inductive
hypothesis for the correctness proof.

\subsection{Usage Analysis and Deadness, Intuitively}

\begin{figure}
\begin{minipage}{\textwidth}
\[\begin{array}{c}
 \arraycolsep=3pt
 \begin{array}{rrclcrrclcl}
  \text{Variables}    & \px, \py & ∈ & \Var        &     & \quad \text{Constructors} &        K & ∈ & \Con        &     & \text{with arity $α_K ∈ ℕ$} \\
  \text{Labels}       &     \lbl & ∈ & \Labels     &     & \quad \text{Values}       &      \pv & ∈ & \Val        & ::= & \highlight{\Lam{\px}{\pe}} \mid K~\many{\px}^{α_K} \\
  \text{Expressions}  &      \pe & ∈ & \Exp        & ::= & \multicolumn{6}{l}{\highlight{\slbl \px \mid \slbl \pv \mid \slbl \pe~\px \mid \slbl \Let{\px}{\pe_1}{\pe_2}} \mid \slbl \Case{\pe}{\SelArity}} \\
  \\[-0.5em]
 \end{array} \\
 \\[-0.5em]
 \begin{array}{rrclcl}
  \text{Scott Domain}      &  d & ∈ & \ScottD & =   & [\ScottD \to_c \ScottD]_\bot \\
  \text{Usage cardinality} &  u & ∈ & \Card & =   & \{ 0 ⊏ 1 ⊏ ω \} ⊂ ℕ_ω \\
  \text{Usage Domain}      &  d & ∈ & \UsgD & =   & \Var \to \Card \\
 \end{array} \quad
 \begin{array}{rcl}
   (ρ_1 ⊔ ρ_2)(\px) & = & ρ_1(\px) ⊔ ρ_2(\px) \\
   (ρ_1 + ρ_2)(\px) & = & ρ_1(\px) + ρ_2(\px) \\
   (u * ρ_1)(\px)   & = & u * ρ_1(\px) \\
 \end{array}
 \\[-0.5em]
\end{array}\]
\subcaption{Syntax of expressions and semantic domains}
  \label{fig:syntax}
\newcommand{\scalefactordenot}{0.92}
\scalebox{\scalefactordenot}{%
\begin{minipage}{0.49\textwidth}
\arraycolsep=0pt
\[\begin{array}{rcl}
  \multicolumn{3}{c}{ \ruleform{ \semscott{\wild} \colon \Exp → (\Var \to \ScottD) → \ScottD } } \\
  \\[-0.5em]
  \semscott{\px}_ρ & {}={} & ρ(\px) \\
  \semscott{\Lam{\px}{\pe}}_ρ & {}={} & d ↦ \semscott{\pe}_{ρ[\px ↦ d]} \\
  \semscott{\pe~\px}_ρ & {}={} & \begin{cases}
     f(ρ(x)) & \text{if $\semscott{\pe} = f$}  \\
     \bot   & \text{otherwise}  \\
   \end{cases} \\
  \semscott{\Letsmall{\px}{\pe_1}{\pe_2}}_ρ & {}={} &
    \begin{letarray}
      \text{letrec}~ρ'. & ρ' = ρ \mathord{⊔} [\px \mathord{↦} d_1] \\
                        & d_1 = \semscott{\pe_1}_{ρ'} \\
      \text{in}         & \semscott{\pe_2}_{ρ'}
    \end{letarray} \\
\end{array}\]
\subcaption{\relscale{\fpeval{1/\scalefactordenot}} Denotational semantics after Scott}
  \label{fig:denotational}
\end{minipage}%
\quad
\begin{minipage}{0.56\textwidth}
\arraycolsep=0pt
\[\begin{array}{rcl}
  \multicolumn{3}{c}{ \ruleform{ \semusg{\wild} \colon \Exp → (\Var → \UsgD) → \UsgD } } \\
  \\[-0.5em]
  \semusg{\px}_ρ & {}={} & ρ(\px) \\
  \semusg{\Lam{\px}{\pe}}_ρ & {}={} & ω*\semusg{\pe}_{ρ[\px ↦ \bot]} \\
  \semusg{\pe~\px}_ρ & {}={} & \semusg{\pe} + ω*ρ(\px)
    \phantom{\begin{cases}
       f(ρ(x)) & \text{if $\semscott{\pe} = f$}  \\
       \bot   & \text{otherwise}  \\
     \end{cases}} \\
  \semusg{\Letsmall{\px}{\pe_1}{\pe_2}}_ρ& {}={} & \begin{letarray}
      \text{letrec}~ρ'. & ρ' = ρ \mathord{⊔} [\px \mathord{↦} d_1] \\
                        & d_1 = [\px\mathord{↦}1] \mathord{+} \semscott{\pe_1}_{ρ'} \\
      \text{in}         & \semusg{\pe_2}_{ρ'}
    \end{letarray}
\end{array}\]
\subcaption{\relscale{\fpeval{1/\scalefactordenot}} Naïve usage analysis}
  \label{fig:usage}
\end{minipage}
}
\end{minipage}
  \label{fig:intro}
\caption{Connecting usage analysis to denotational semantics}
\end{figure}

\Cref{fig:syntax} defines the labelled syntax of a lambda calculus with
recursive let bindings and algebraic data types, reminiscent of
\citet{Sestoft:97}. The calculus is factored into \emph{administrative normal
form}, that is, the arguments of applications are restricted to be variables, so
the difference between call-by-name and call-by-value manifests purely in the
semantics of $\mathbf{let}$.
In this section, only the highlighted parts are relevant; we will ignore labels
and data types for brevity.

We give a standard call-by-name denotational semantics $\semscott{\wild}$ in
\Cref{fig:denotational} \citep{ScottStrachey:71}, assigning meaning to our
syntax by means of the infamous Scott domain $\ScottD$.
Squinting a bit, we find that it looks quite similar to the function
to its right in \Cref{fig:usage}, depicting a \emph{usage analysis}
$\semusg{\wild}$, a static analysis for estimating an upper bound on how often
a variable is evaluated. The given analysis is naïve in that its treatment of
function application assumes that every function deeply evaluates its argument.

Assuming that all program variables are distinct (a silent assumption from
here on throughout), the result of $\semusg{\wild}$ is an element $d ∈ \UsgD$,
an environment that maps to each variable an upper bound on its \emph{evaluation
cardinality}, that is, how often the variable is evaluated over the cause of any
of its activations.
Whenever $0 ⊏ d(x)$, we say that it is \emph{potentially live} in $d$ and
extend this meaning to a program $\pe$ whenever $\semusg{\pe} = d$.
Likewise, when $d(x) = 0$ we say that $x$ is \emph{dead} in $d$/$\pe$, \eg,
\emph{never evaluated}.
In this way, $\semusg{\wild}$ can be used to infer facts of the form ``$\pe$
never evaluates $x$'' from the introduction.

\subsection{Denotational Deadness, Continuity and Divergence}

The requirement (in the sense of informal specification) on an assertion
such as ``$x$ is dead'' in a program like $\Let{x}{\pe_1}{\pe_2}$ is that we
may rewrite to $\Let{x}{panic}{\pe_2}$ and perhaps even to $\pe_2$ without
observing any change in semantics. Doing so reduces code size and heap
allocation.

This can be made formal in the following definition of deadness in terms of
$\semscott{\wild}$:

\begin{definition}[Deadness]
  \label{defn:deadness}
  A variable $\px$ is \emph{dead} in an expression $\pe$ if and only
  if, for all $ρ ∈ \Var \to \ScottD$ and $d_1, d_2 ∈ \ScottD$, we have
  $\semscott{\pe}_{ρ[\px↦d_1]} = \semscott{\pe}_{ρ[\px↦d_2]}$.
  Otherwise, $\px$ is \emph{live}.
\end{definition}

Indeed, if we know that $x$ is dead, then the following equation justifies our
rewrite above: $\semscott{\Let{x}{\pe_1}{\pe_2}}_ρ = \semscott{\pe_2}_{ρ[x↦d]} =
\semscott{\pe_2}_ρ$ (for all $ρ$ and the suitable $d$).
So our definition of deadness is not only simple to grasp, but also simple to
exploit.

We can now try to prove our usage analysis correct as a liveness analysis in
terms of this notion of deadness. After a bit of trial and error, we could
arrive at the following theorem:

\begin{theorem}[$\semusg{\wild}$ is a correct potential liveness analysis]
  \label{thm:semusg-correct-live}
  Let $\pe$ be an expression and $\px$ a variable.
  Then $\px$ is dead in $\pe$ whenever
  there exists $\tr ∈ \Var \to \UsgD$ such that
  $\tr(\px) \not⊑ \semusg{\pe}_{\tr}$.
\end{theorem}
\begin{proof}
  By induction over $\pe$. The full proof can be found in
  \Cref{prf:semusg-correct-live}.
\end{proof}

Let us stop and reflect about this theorem for a bit.
Deadness is witnessed by a particular $\tr$ and it helps to think of this
witness as the ``diagonal'' $\tr(\px) \triangleq \fn{\py}{\ternary{\px =
\py}{1}{0}}$, because then the intuitive notion of deadness applies.%
\footnote{In fact, it can be proven that if \emph{any} $\tr$ exists, then the
diagonal is also a witness.}

It is surprising that the theorem does not relate $\tr$ with $ρ$; after all,
$ρ(\py)$ (for $\py \not= \px$) might be bound to the \emph{meaning} of an
expression that is potentially live in $\px$, such as $\semscott{\px}_{ρ'}$, and
we have no way to observe the dependency on $\px$ just through $ρ(\py)$.
The key is to realise that our notion of deadness varies $ρ(\px)$ (the meaning
of $\px$), but that does not vary $ρ(\py)$, because that only sees $ρ'(\px)$,
so for all intents and purposes, the proof may assume that $ρ(\py)$ is dead in
$\px$.
The analysis, on the other hand, encodes transitive deadness relationships via
$\tr(\px) ⊑ \tr(\py)$ in case $\px$ occurs in the RHS of a $\mathsf{let}$-bound
$\py$ to encode that deadness of $\py$ is a necessary condition for deadness of
$\px$.

The proof capitalises on the similarities in structure by using induction on the
program expression, hence it is simple and direct, at just under a page of
accessible prose. Often, such a proof needs to strengthen the induction
hypothesis for the application case, or prove admissability of a predicate to
apply fixpoint induction in the let case, but for deadness and our very simple
analysis we do not need to be so crafty.

Nevermind our confidence in the ultimate correctness of $\semusg{\wild}$,
note that our notion of deadness has a blind spot \wrt diverging computations:
A looping program is automatically dead in all its free variables, even though
any of them might influence which particular endless loop is taken.

This is not a curiosity of $\semusg{\wild}$; it also applies to the original
control-flow analysis work~\citep[p. 23]{Shivers:91} where it is remedied
by the introduction of a \emph{non-standard semantic interpretation} that
assigns meaning to diverging programs where the denotational semantics only
says $\bot$. Credibility of this approach solely rests on the structural
similarity to the standard denotational semantics.

So the issue is not with $\semusg{\wild}$ but with traditional denotational
semantics because it (necessarily) assigns $\bot$ to any diverging computation.
Furthermore, as is often done, $\semscott{\wild}$ abuses $\bot$ as a collecting
pool for error cases.
This shows in the following example:
$x$ is dead in $(\Lam{y}{\Lam{z}{z}})~x$, but dropping the $\mathbf{let}$
binding in $\Let{x}{\pe_1}{(\Lam{y}{\Lam{z}{z}})~x}$
introduces a scoping error which is not observable under $\semscott{\wild}$.

We could take inspiration in the work of \citet{Milner:78}
and navigate around the issue by introducing a $\mathbf{wrong}$ denotation for
errors which is propagated strictly; then we would notice when we optimise a
looping program into one that has a scoping error (the only kind of stuckness
that our calculus admits without data types).

However, $\bot$ is still there as the denotation of diverging computations;
hence a predicate such as ``Denotation $d$ will get stuck and not diverge'' is
not an admissable one, because an admissable predicate would be true for $\bot$.

\subsection{Evaluation Cardinality and Call-by-need}
Blind spots notwithstanding, the notion of deadness above is quite reasonable.
But our usage analysis infers more detailed cardinality information; for
example, it can infer whether a binding is evaluated at most once.
This information can be useful under call-by-need to omit pushing of update
frames~\citep{cardinality-ext}.%
\footnote{A more useful application of the ``at most once'' cardinality is the
notion of a \emph{one-shot} lambda~\citep{cardinality-ext}, a function which is
called at most once for every activation, because it allows floating of heap
allocations from a hot code path into cold function bodies.
Simplicity prohibits $\semusg{\wild}$ from inferring such properties.}
Thus, our usage analysis should satisfy the following generalisation of
\Cref{thm:semusg-correct-live}:

\begin{theorem}[Correctness of $\semusg{\wild}$]
  \label{thm:semusg-correct-2}
  Let $\pe$ be an expression and $\px$ a variable.
  Then $\pe$ evaluates $\px$ at most $u$ times whenever
  there exists $\tr ∈ \Var \to \UsgD$ such that
  $(u+1)*\tr(\px) \not⊑ \semusg{\pe}_{\tr}$.
\end{theorem}

Unfortunately, our denotational semantics does not allow us to express the
operational property ``$\pe$ evaluates $\px$ at most $u$ times'', so
this theorem cannot be proven correct.

% We should probably mvoe this RElated Work? Don't want to discuss it here
%The problem of observable cardinality also comes up in Quantitative Type
%Theory~\citep{Atkey:18}, where the solution is to give a categorical
%semantics that postulates observability of cardinality in a suitable
%\emph{$R$-Quantitative Category with Families} without giving a concrete
%model.

\begin{figure}
\[\begin{array}{c}
 \begin{array}{rrclcl}
  \text{LK States}     & σ   & ∈ & \States        & =      & \Exp \times \Environments \times \Heaps \times \Continuations \\
  \text{Environments}  & ρ   & ∈ & \Environments  & =      & \Var \pfun \Addresses \\
  \text{Addresses}     & \pa & ∈ & \Addresses     & \simeq & ℕ \\
  \text{Heaps}         & μ   & ∈ & \Heaps         & =      & \Addresses \pfun \Environments \times \Exp \\
  \text{Continuations} & κ   & ∈ & \Continuations & ::=    & \StopF \mid \ApplyF(\pa) \pushF κ \mid \SelF(ρ,\SelArity) \pushF κ \mid \UpdateF(\pa) \pushF κ \\
 \end{array} \\
  \\[-0.5em]
\end{array}\]

\newcolumntype{L}{>{$}l<{$}} % math-mode version of "l" column type
\newcolumntype{R}{>{$}r<{$}} % math-mode version of "r" column type
\newcolumntype{C}{>{$}c<{$}} % math-mode version of "c" column type
\resizebox{\textwidth}{!}{%
\begin{tabular}{LR@{\hspace{0.4em}}C@{\hspace{0.4em}}LL}
\toprule
\text{Rule} & σ_1 & \smallstep & σ_2 & \text{where} \\
\midrule
\BindT & (\Let{\px}{\pe_1}{\pe_2},ρ,μ,κ) & \smallstep & (\pe_2,ρ',μ[\pa↦(ρ',\pe_1)], κ) & \pa \not∈ \dom(μ),\ ρ'\! = ρ[\px↦\pa] \\
\AppIT & (\pe~\px,ρ,μ,κ) & \smallstep & (\pe,ρ,μ,\ApplyF(\pa) \pushF κ) & \pa = ρ(\px) \\
\CaseIT & (\Case{\pe}{\Sel},ρ,μ,κ) & \smallstep & (\pe,ρ,μ,\SelF(ρ,\Sel) \pushF κ) & \\
\LookupT & (\px, ρ, μ, κ) & \smallstep & (\pe, ρ', μ, \UpdateF(\pa) \pushF κ) & \pa = ρ(\px),\ (ρ',\pe) = μ(\pa) \\
\AppET & (\Lam{\px}{\pe},ρ,μ, \ApplyF(\pa) \pushF κ) & \smallstep & (\pe,ρ[\px ↦ \pa],μ,κ) &  \\
\CaseET & (K'~\many{y},ρ,μ, \SelF(ρ',\Sel) \pushF κ) & \smallstep & (\pe_i,ρ'[\many{\px_i ↦ \pa}],μ,κ) & K_i = K',\ \many{\pa = ρ(\py)} \\
\UpdateT & (\pv, ρ, μ, \UpdateF(\pa) \pushF κ) & \smallstep & (\pv, ρ, μ[\pa ↦ (ρ,\pv)], κ) & \\
\bottomrule
\end{tabular}
} % resizebox
\caption{Lazy Krivine transition semantics $\smallstep$}
  \label{fig:lk-semantics}
\end{figure}

Let us try a different approach then and define a stronger notion of deadness
in terms of a small-step operational semantics such as the Mark II machine of
\citet{Sestoft:97} given in \Cref{fig:lk-semantics}, the semantic ground truth
for this work. (A close sibling for call-by-value would be a CESK machine
\citep{Felleisen:87} or a simpler derivative thereof.) It is a variant of
the Lazy Krivine (LK) machine implementing call-by-need, so for a meaningful
comparison to $\semscott{\wild}$, we ignore rules $\CaseIT, \CaseET, \UpdateT$
and the pushing of update frames in $\LookupT$ for now to recover a call-by-name
Krivine machine with explicit heap addresses.%
\footnote{Note that discarding update frames makes the heap entries immutable,
which makes the explicit heap unnecessary. Of course, for call-by-name we would
not need a heap to begin with, but the point is to get a glimpse at the effort
necessary for call-by-need.}

The configurations $σ$ in this transition system resemble abstract machine
states, consisting of control expression $\pe$, an environment $ρ$ mapping
lexically-scoped variables to their current heap address, a heap $μ$ listing a
closure for each address, and a stack of continuation frames $κ$.

The notation $f ∈ A \pfun B$ used in the definition of $ρ$ and $μ$ denotes a
finite map from $A$ to $B$, a partial function where the domain $\dom(f)$ is
finite and $\rng(f)$ denotes its range.
The literal notation $[a_1↦b_1,...,a_n↦b_n]$ denotes a finite map with domain
$\{a_1,...,a_n\}$ that maps $a_i$ to $b_i$. Function update $f[a ↦ b]$
maps $a$ to $b$ and is otherwise equal to $f$.

The initial machine state for a closed expression $\pe$ is given by the
injection function $\inj(\pe) = (\pe,[],[],\StopF)$ and
the final machine states are of the form $(\pv,\wild,\wild,\StopF)$.
We bake into $σ∈\States$ the simplifying invariant of \emph{well-addressedness}:
Any address $\pa$ occuring in $ρ$, $κ$ or the range of $μ$ must be an element of
$\dom(μ)$.
It is easy to see that the transition system maintains this invariant and that
it is still possible to observe scoping errors which are thus confined to lookup
in $ρ$.

Now we are able to define a notion of ``strong deadness'':

\begin{definition}[Deadness, Mark II]
  \label{defn:deadness2}
  Let $\pe$ be an expression and $\px$ a variable.
  $\px$ is \emph{dead} in $\pe$ if and only if
  for any evaluation context $(ρ,μ,κ)$ and expressions $\pe_1,\pe_2$
  (where $\px$ does not occur in the context)
  the sequences of transitions $(\Let{\px}{\pe_1}{\pe},ρ,μ,κ) \smallstep^*$
  and $(\Let{\px}{\pe_2}{\pe},ρ,μ,κ) \smallstep^*$ operate in lockstep.
  Otherwise, $\px$ is \emph{live}.
\end{definition}

This definition captures diverging behaviors correctly and straightforwardly
legitimises the transformation we want to perform, without any mention of
addresses. It is however unwieldy in a correctness proof due to its use of
bisimulation, so a bit of rejigging is in order:

\begin{lemma}[Without proof]
  $\px$ is dead in $\pe$ if and only if for any evaluation context $(ρ,μ,κ)$
  and $\pa \not∈ \dom(μ)$ there exists no sequence of transitions
  $(\pe,ρ[\px↦\pa],μ[\pa↦([],\Lam{z}{z})],κ) \smallstep^* (\py,ρ',μ',κ')$ such
  that $ρ'(\py) = \pa$.
\end{lemma}

This property is a bit easier to handle in a proof.
Unfortunately, it is still not compositional in $\pe$: Consider a variable
occurrence $y$; is $x$ dead in $y$? That depends on which expression $y$ is
bound to in the heap, but our deadness predicate has no notion of making
assumptions about free variables.
Consequently, it is impossible to prove that $\semusg{\pe}$ satisfies
\Cref{thm:semusg-correct-live} by direct structural induction on $\pe$ (in a way
that would be useful to the proof).

Instead, such proofs are often conducted by induction over the reflexive
transitive closure of the transition relation.
For that it is necessary to give an inductive hypothesis that considers
environments, stacks and heaps.
One way is to extend the analysis function $\semusg{\wild}$ to entire
configurations and then prove that if $σ_1 \smallstep σ_2$ we have $\semusg{σ_2}
⊑ F(\semusg{σ_1})$, where $F$ is the abstraction of the particular transition
rule taken and is often left implicit.
This is a daunting task, for multiple reasons:
First off, $\semusg{\wild}$ might be quite complicated in practice and extending
it to entire configurations multiplies this complexity.
Secondly, $\semusg{\wild}$ makes use of fixpoints in the let case and
undoubtedly needs some more fixpoints in its extension to the heap,
so $\semusg{σ_2} ⊑ F(\semusg{σ_1})$ relates fixpoints that are ``out of sync'',
implying a need for fixpoint induction for every transition that touches
the heap.

In call-by-need, there will be a fixpoint between the heap and stack due to
update frames acting like heap bindings whose right-hand side is under
evaluation (a point that is very apparent in the semantics of
\citet{Ariola:95}), so fixpoint induction needs to be applied \emph{at every
case of the proof}, diminishing confidence in correctness unless the proof is
fully mechanised.

For an analogy with type systems: What we just tried is like attempting a proof
of preservation by referencing the result of an inference algorithm rather than
the declarative type system. So what is often done instead is to define a declarative and
more permissive \emph{correctness relation} $C(σ)$ to prove preservation $C(σ_1)
\Longrightarrow C(σ_2)$ (\eg, that $C$ is \emph{logical} \wrt $\smallstep$).
$C$ is chosen such that
  (1) it is strong enough to imply the property of interest (deadness)
  (2) it is weak enough that it is implied by the analysis result for an initial state ($\tr(\px) \not⊑ \semusg{\pe}_{\tr}$).
Examples of this approach are the
``well-annotatedness'' relation in \citep[Lemma 4.3]{cardinality-ext} or
$\sim_V$ in \citep[Theorem 2.21]{Nielson:99}).
We found it quite hard to come up with a suitable ad-hoc correctness relation
and postpone futher discussion to \Cref{sec:abstractions}, where the full
correctness relation in \Cref{thm:semusg-correct-2} and its proof is derived by
abstract interpretation~\citep{Cousot:21}.

Often, correctness proofs do not need the full operational detail of a
CESK-style machine, in which case a simpler machine definition leads to simpler
proof.
If the correctness statement does not need to keep track about which address
is an activation of which let-bound program variable, the distinction between
addresses and variables is unnecessary and the environment component vanishes.
The stack can often be reflected back into the premises of the judgment
rules, distinguishing \emph{instruction transitions} from \emph{search
transitions}, a distinction which is made explicit in a \emph{contextual
semantics}.
Applying both these transformations yields a CS machine~\citep{Felleisen:87}.
However for call-by-need, the evaluation context corresponding to an update
frame is neither obvious nor simple~\citep{Ariola:95}.
For effect-free call-by-value and call-by-name calculi, the heap becomes
immutable and variables can be substituted immediately for their right-hand
sides/arguments rather than delaying lookup to the variable case, abolishing the
need for the heap altogether and yielding a contextual semantics where
states are a simple expression.

These refactorings help in simplifying proofs:
Instead of defining an coinductive well-formedness predicate on the heap, we
prove a substitution lemma.
Instead of a well-formedness predicate for the stack, we appeal to the
well-formedness of the search transition rule.

\subsection{Abstracting Abstract Machines}

Another way to work around the structure gap is to adopt the structure of the
semantics in the analysis; this is done in the Abstracting Abstract
Machines work \citep{aam}.
To our knowledge, its exclusive application is control-flow analysis
\citep{Shivers:91}, so that the analyses and optimisations that follow can apply
traditional intraprocedural analysis techniques that are well-explored in the
imperative world.
Unfortunately, control-flow information is often invalidated during compiler
passes and we expect re-running after or interleaving the analysis during each
pass to be quite costly.

\section{A Denotational Semantics for Call-by-need}
\label{sec:vanilla}

%\subsection{Labelled Syntax}
%
%Recall the syntax definition of our object language in
%\Cref{sec:usage-intuition} in the style of \citet{Launchbury:93} and
%\citet{Sestoft:97}.
%Any (sub-)expression has a unique \emph{label} (think of it as the AST node's
%pointer identity) that we usually omit. For example, a correct labelling of
%$\Let{x}{f~y}{f~x}$ would be
%\[
%  (\slbln{1} \Let{x}{(\slbln{2} (\slbln{3} f)~y)}{(\slbln{4} (\slbln{5} f)~x)}).
%\]
%Labels are there so that we do not conflate the (otherwise structurally equal)
%sub-terms $(\slbln{3} f)$ and $(\slbln{5} f)$ as equivalent. This is an important
%distinction for, \eg, control-flow analysis. Since labels introduce excessive
%clutter, we will omit them unless they are distinctively important. If anything,
%labels make it so that everything ``works as expected''.

\subsection{Transition System}

\Cref{fig:lk-semantics} gave an operational semantics in terms of
a small-step transition system closest to the lazy Krivine machine
\citep{AgerDanvyMidtgaard:04} for Launchbury's language as presented
in \citet{Sestoft:97}.
It is worth having a second look at the workings of our gold standard.

When the control expression of a state $σ$ (selected via $\ctrl(σ)$) is a value
$\pv$, we call $σ$ a \emph{return} state and say that the continuation (selected
via $\cont(σ)$) drives evaluation.
Otherwise, $σ$ is an \emph{evaluation} state and $\ctrl(σ)$ drives evaluation.
The entries in the heap $μ$ are \emph{closures} of the form $(ρ,e)$, where the
environment $ρ$ closes over the expression $e$.
Finally, $\cont(σ)$ lists actions to be taken in a return state, such as
applying the result to an argument address or updating a heap entry with its
value.

Heap entries are introduced via $\BindT$ transitions under a \emph{fresh} address
$\pa \not∈ \dom(μ)$ that we call an \emph{activation} of the let-bound variable
$\px$. The lexical activation of every variable in scope is maintained
in $ρ$. The $\AppIT$ rule pushes an \emph{application frame} with the address of
the argument variable onto the stack, while the rule $\LookupT$ pushes an
\emph{update frame} with the address of the variable the heap entry of which is
accessed. When a return state is reached, the original heap entry is overwritten
with the value in the control.
Similarly to application, evaluating a $\mathbf{case}$ expression pushes a
\emph{selector frame} onto the stack via $\CaseIT$.
When subsequent evaluation reaches a data constructor value, a $\CaseET$
transition looks up the corresponding case alternative in the selector frame.
For our examples, we will assume that we have defined the data type
$\bool ::= \ttrue \mid \ffalse$, where $\ttrue$ and $\ffalse$ are two nullary
data constructors.

Let us conclude with an example trace in this transition system, evaluating
$\pe \triangleq \Let{i}{\Lam{x}{x}}{i~i}$:
\[\begin{array}{c}
  \arraycolsep2pt
  \begin{array}{clclclcl}
             & (\pe, [], [], \StopF)         & \smallstep & (i~i, ρ_1, μ, \StopF)
             & \smallstep & (i, ρ_1, μ, κ_1) & \smallstep & (\Lam{x}{x}, ρ_1, μ, κ_2)
             \\
  \smallstep & (\Lam{x}{x}, ρ_1, μ, κ_1)     & \smallstep & (x, ρ_2, μ, \StopF) & \smallstep & (\Lam{x}{x}, ρ_1, μ, κ_3)
             & \smallstep & (\Lam{x}{x}, ρ_1, μ, \StopF) \\
  \end{array} \\
  \\[-0.5em]
  \quad \text{where} \quad \begin{array}{lll}
  ρ_1 = [i ↦ \pa_1] & ρ_2 = [i ↦ \pa_1, x ↦ \pa_1] & μ = [\pa_1 ↦ (ρ_1,\Lam{x}{x})] \\
  κ_1 = \ApplyF(\pa_1) \pushF \StopF & κ_2 = \UpdateF(\pa_1) \pushF κ_1 & κ_3 = \UpdateF(\pa_1) \pushF \StopF
  \end{array}
\end{array}\]

\subsection{(Algebraic) Domain Theory}
\label{sec:domain-theory}

The challenge that domain theory sets out to solve is that the ``inductive
datatype''
\[
  D ::= \FunV(f ∈ D \to D) \mid \bot
\]
is ill-defined:
The usual interpretation of such a declaration as the least fixed-point of
the implied set-valued functional
$F(X) \triangleq \{ f \mid ∀a∈X.\ ∃b∈X.\ f(a) = b \} ∪ \{ \bot \}$
does not exist.

To see that, suppose $μF$ was that set.
Then there exists an injection (``data constructor'') $\FunV$ from $μF \to μF =
μF^{μF}$ into $μF$.
We can see that $\{\bot, (\fn{\wild}{\bot}) \} \subseteq μF$, so there are at
least two elements in $μF$.
Then to accomodate $μF^{μF}$, $μF$ must be at least as large as $2^{μF}$, the
set of two-valued functions on $μF$.
But this latter set is one-to-one with $\poset{μF}$ and it is a known result by
Cantor that $\poset{μF}$ has greater cardinality than $μF$, in contradiction to
the existence of the injection $\FunV$.

Domain theory, on the other hand, interprets the implied recursion equation
in terms of topology and continuous functions, where the fixed-point exists
when restricted to \emph{algebraic domains}~\citep{Scott:70}.
At the same time, algebraic domains are expressive enough to encode any
computable function as a continuous function.

\subsection{(Synthetic) Guarded Domain Theory}

As we have discussed in \Cref{sec:continuity}, there are a few strings attached
to working with continuity and partiality in the context of denotational
semantics.

The key to getting rid of partiality and thus denoting infinite computations
with total elements is to avoid working with algebraic domains altogether and
instead work in a total type theory with \emph{guarded recursive types}, such
as Guarded Dependent Type Theory (GDTT)~\citep{gdtt} or Ticked Cubical Type
Theory~\citep{tctt}.%
\footnote{Of course, in reality we are just using GDTT as a meta
language~\citep{Moggi:07} with a known domain-theoretic model in terms
of the topos of trees~\citep{gdtt}.
This meta language is sufficiently expressive as a logic to
express proofs, though, justifying the view that we are extending ``math''
with the ability to conveniently reason about computable functions on infinite
data without needing to think about topology and approximation directly.}
The fundamental innovation of these theories is the integration of the
``later'' modality $\later$ which allows to define coinductive data types
with negative recursive occurrences such in our ``data type'' $D$ from
\Cref{sec:domain-theory}, as first realised by \citet{Nakano:00}.

%GDTT walks a fine line:
%The theory guarantees that all well-typed functions (naturally) correspond
%to continuous functions in the underlying model, while it also allows for
%embedding of a sufficiently expressive restriction to give meaning to $D$.

Whereas previous theories of coinduction require syntactic productivity
checks~\citep{Coquand:94}, requiring tiresome constraints on the form of guarded
recursive functions, the appeal of GDTT is that productivity is instead proven
semantically, in the type system.

The way that GDTT achieves this is roughly as follows: The type $\later T$
represents data of type $T$ that will become available after a finite amount
of computation, such as unrolling one layer of a fixpoint definition.
It comes with a general fixpoint combinator $\fix : \forall A.\ (\later A \to
A) \to A$ that can be used to define both coinductive \emph{types} (via guarded
recursive functions on the universe of types~\citep{BirkedalMogelbergEjlers:13})
as well as guarded recursive \emph{terms} inhabiting said types.
The classic example is that of coinductive streams:
\[
  Str = ℕ \times \later Str \qquad ones = \fix (r : \later Str).\ (1,r),
\]
where $ones : Str$ is the constant stream of $1$.
In particular, $Str$ is the fixpoint of a locally contractive functor $F(X) =
ℕ \times \later X$.
According to \citet{BirkedalMogelbergEjlers:13}, any type expression in simply
typed lambda calculus defines a locally contractive functor as long as any
occurrence of $X$ is under a $\later$, so we take that as the well-formedness
criterion of coinductive types in this work.
The most exciting consequence is that
$D ::= \FunV(f ∈ \later D \to \later D) \mid \bot$ (where $\bot$ is interpreted
as a plain nullary data constructor rather than as the least element of
some partial order) is a sound coinductive encoding of the data type in
\Cref{sec:domain-theory}.
Even unguarded positive occurrences as in
$D ::= \FunV(f ∈ \later D \to D) \mid \bot$ are permissible as

As a type constructor, $\later$ is an applicative
functor~\citep{McBridePaterson:08} via functions
\[
  \purelater : \forall A.\ A \to \later A \qquad \wild \aplater \wild : \forall A,B.\ \later (A \to B) \to \later A \to \later B,
\]
allowing us to apply a familiar framework of reasoning around $\later$.
In order not to obscure our work with pointless symbol pushing
in, \eg, \Cref{fig:semvan}, we will often omit the idiom
brackets~\citep{McBridePaterson:08} $\idiom{\wild}$
to indicate where the $\later$ ``effects'' happen.
Rest assured, all $\later$ are present in the Guarded Cubical Agda
development for \Cref{fig:semvan,fig:semevt} in the Supplement.

\begin{figure}
\[\begin{array}{c}
 \begin{array}{rrclclrrcrclcl}
  \text{Environment}  & ρ   & ∈ & \Environments  & =      & \Var \pfun \Addresses
  &
  \text{Heap}         & μ   & ∈ & \Heaps         & =      & \Addresses \pfun \later\VanD
  \\
  \\[-1em]
  \text{Trace} & τ      & ∈          & \VTraces & ::= & \goodend{v,μ} \mid \stuckend \mid \laterC~τ^{\later}
  &
  \multicolumn{3}{r}{\text{Delayed trace}} & τ^{\later} & ∈ & \later\VTraces &   &
  \\
  \text{Domain} & d & ∈ & \VanD & = & \Heaps \to \VTraces
  &
  \multicolumn{3}{r}{\text{Delayed element}} & d^{\later} & ∈ & \later\VanD &   &
  \\
  \\[-1em]
 \end{array} \\
 \begin{array}{rrclcl}
  \text{Value} & v & ∈ & \VanV & ::= & \FunV(f ∈ (\Addresses \to \later\VanD)) \mid \ConV(K,\many{\pa}^{α_K}) \\
 \end{array} \\
\end{array}\]
\[\begin{array}{c}
 \begin{array}{rcl}
  \multicolumn{3}{c}{ \ruleform{
    \begin{array}{c}
      (\betastep) : \VanD \to (\VanV \pfun \VanD) \to \VanD \quad \ret : \VanV \to \VanD \quad \apply : \VanD \times \Addresses \to \VanD \\
      \memo : \Addresses \times \VanD \to \VanD \quad \select : \VanD \times ((K:\Con) \times \pa^{α_K} \pfun \VanD) \to \VanD \\
    \end{array}
  }} \\
  \\[-0.5em]
  (d \betastep f)(μ) & = & \begin{cases}
      \laterC^n~f(v,μ')  & \text{$d(μ) = \laterC^n~\goodend{v,μ'}$ and $(v,μ') ∈ \dom(f)$} \\
      \laterC^n~\stuckend  & \text{$d(μ) = \laterC^n~\goodend{v,μ'}$ and $(v,μ') \not∈ \dom(f)$} \\
      d(μ) & \text{otherwise} \\
    \end{cases} \\
  \\[-0.5em]
  \ret(v)(μ) & = & \goodend{v,μ} \\
  \apply(d,\pa) & = & d \betastep \fn{(\FunV(f))}{\laterC~f(\pa)} \\
  \select(d,\alts) & = & d \betastep \fn{(\ConV(K_s,\many{\pa}))}{\alts(K_s, \many{\pa})} \quad \text{where } (K_s, \many{\pa}) ∈ \dom(\alts) \\
  \memo(\pa,d) & = & d \betastep \fn{v}{(\fn{μ}{\laterC~\goodend{v,μ[\pa ↦ \memo(\pa,\ret(v))]}})} \\
 \end{array} \\
  \\[-0.5em]
 \begin{array}{rcl}
  \multicolumn{3}{c}{ \ruleform{ \semvan{\wild} \colon \Exp → (\Var \pfun \Addresses) → \VanD } } \\
  \\[-0.5em]
  \semvan{\px}_ρ & = & \begin{cases}
    \laterC~μ(ρ(\px))(μ) & \px ∈ \dom(ρ) \\
    \stuckend & \text{otherwise}
    \end{cases} \\
  \\[-0.5em]
  \semvan{\Lam{\px}{\pe}}_ρ & = & \ret(\FunV(\fn{\pa}{\semvan{\pe}_{ρ[\px↦\pa]}})) \\
  \\[-0.5em]
  \semvan{\pe~\px}_ρ(μ) & = & \begin{cases}
      \laterC~\apply(\semvan{\pe}_ρ(μ),\pa) & \px ∈ \dom(ρ) \\
      \stuckend & \text{otherwise} \\
    \end{cases} \\
  \\[-0.5em]
  \semvan{\Let{\px}{\pe_1}{\pe_2}}_ρ(μ) & = & \begin{letarray}
    \text{let} & ρ' = ρ[\px↦\pa] \quad \text{where $\pa \not∈ \dom(μ)$} \\
    \text{in}  & \laterC~\semvan{\pe_2}_{ρ'}(μ[\pa ↦ \memo(\pa,\semvan{\pe_1}_{ρ'})]) \\
  \end{letarray} \\
  \\[-0.5em]
  \semvan{K~\many{\px}}_ρ & = & \ret(\ConV(K,\many{ρ(\px)})) \\
  \\[-0.5em]
  \semvan{\Case{\pe_s}{\Sel[r]}}_ρ(μ) & = &
    \begin{letarray}
      \text{let} & \alts = \fn{(K_i,\many{\pa})}{\laterC~\semevt{\pe_{r_i}}_{ρ[\many{\px_i↦\pa}]}} \\
      \text{in} & \laterC~\select(\semevt{\pe_s}_ρ, \alts)  \\
    \end{letarray}
 \end{array}
  \\[-0.5em]
\end{array}\]
\caption{Call-by-need Denotational Semantics $\semvan{-}$}
  \label{fig:semvan}
\end{figure}

\subsection{Definition}

\Cref{fig:semvan} defines $\semvan{\wild}$ by structural recursion on an input
expression $\pe$. Given a free variable environment $ρ$, $\semvan{\pe}_ρ$
assigns $\pe$ a denotation $d$ in terms of the semantic domain $\VanD$ of
stateful call-by-need trace functions. If such a trace function is supplied the
heap $μ$ just before $\pe$ takes control, then $d(μ)$ is a trace $τ$ starting
at $\pe$ in that heap $μ$. As in the LK transition system $(\smallstep)$, the
job of the environment $ρ$ is to assign meaning to free variables of $\pe$ via
address tokens $\pa$ bound in the heap.
If evaluation of $\pe$ terminates, then $τ$ will be a finite list of $\laterC$s
ending with $\goodend{v,μ'}$ for some return semantic value $v$ and heap $μ'$.
Otherwise, it might be finite but stuck ($\stuckend$), or diverge without
ever leaving $\pe$, in which case $τ$ will be an infinite layering of $\laterC$s.

One can think of $\laterC$ (read as ``later'' or ``delayed'') as introducing
a finite portion of latency into the trace -- an atomic computation step.
It makes crucial use of the later modality $\later$ to naturally encode
divergence as an infinite sequence of productive steps.

Compared to the LK transition semantics, the most striking difference is that
the returned trace does not contain any intermediate information such as an
environment or control stack component.
The entire state is internal to the definition of $\semvan{\wild}$:
The stack in particular is implicitly encoded in call structure while the
environment follows lexical structure.
%As we have seen in \Cref{sec:problem} at the example of $\semscott{\wild}$,
%this reflection of machine state into ``math'' bears great potential for program
%analysis, one we will exploit in \Cref{sec:abstractions}.
We will see that for every LK transition, the trace will have one $\laterC$
step.

The second difference is that the heap $μ$ does not map to syntactic closures
but to delayed semantic denotations $\later \VanD$.
This is so that the transitive negative occurrence arising through
$\VanD \to \Heaps \to \VanD$ is broken and the definition ``type-checks''.
With algebraic domain theory, we would instead have to justify that this
domain equation has a solution and that $\semvan{\wild}$ is in fact continuous.

The choice to have environments map to addresses leads to function values
$\FunV(f)$ that take said addresses as parameter, rather than a (necessarily
guarded) $\VanD$ as in $\semscott{\wild}$.%
%\footnote{An earlier version of this paper had
%$\Environments = \Var \pfun \VanD$ instead, where every entry would simply look
%up a particular address in the heap.
%Though it felt more ``semantic'', it was ultimately more taxing than useful,
%because many later proofs will have to pose additional preconditions on $ρ$.}
This allows the embedding of function values $\FunV(f)$ in the lambda case
$\semvan{\Lam{\px}{\pe}}$, enabling a compositional definition of the
application case $\semvan{\pe~\px}$, just as in $\semscott{\wild}$.

Let us now understand $\semvan{\wild}$ by way of evaluating the example program
from earlier, $\pe \triangleq \Let{i}{\Lam{x}{x}}{i~i}$:
\[\begin{array}{ll}
  \newcommand{\myleftbrace}[4]{\draw[mymatrixbrace] (m-#2-#1.west) -- node[right=2pt] {#4} (m-#3-#1.west);}
  \vcenter{\hbox{$
    \begin{tikzpicture}[mymatrixenv,anchor=center]
      \matrix [mymatrix] (m)
      {
        1 & \laterC & \hspace{3.7em} & \hspace{4.2em} & \hspace{3.9em} & \hspace{2.5em} \\
        2 & \laterC & & & & \\
        3 & \laterC & & & & \\
        4 & \laterC & & & & \\
        5 & \laterC & & & & \\
        6 & \laterC & & & & \\
        7 & \laterC & & & & \\
        8 & \goodend{\FunV(f), μ} & & & & \\
      };
      % Braces, using the node name prev as the state for the previous east
      % anchor. Only the east anchor is relevant
      \foreach \i in {1,...,\the\pgfmatrixcurrentrow}
        \draw[dotted] (m.east|-m-\i-\the\pgfmatrixcurrentcolumn.east) -- (m-\i-2);
      \myleftbrace{3}{1}{8}{$\semvan{\pe}_{[]}$}
      \myleftbrace{4}{1}{2}{$\BindT$}
      \myleftbrace{4}{2}{8}{$\semvan{i~i}_{ρ_1}$}
      \myleftbrace{5}{2}{3}{$\AppIT$}
      \myleftbrace{5}{3}{5}{$\semvan{i}_{ρ_1}$}
      \myleftbrace{5}{5}{6}{$\AppET$}
      \myleftbrace{5}{6}{8}{$\semvan{x}_{ρ_2}$}
      \myleftbrace{6}{3}{4}{$\LookupT$}
      \myleftbrace{6}{4}{5}{$\UpdateT$}
      \myleftbrace{6}{6}{7}{$\LookupT$}
      \myleftbrace{6}{7}{8}{$\UpdateT$}
  \end{tikzpicture}
  $}} &
  \!\!\!\!\text{where}  \begin{array}{ll}
  ρ_1 = [i ↦ \pa_1] & \\
  ρ_2 = ρ_1[x ↦ \pa_1] &  \\
  μ = [\pa_1 ↦ \memo(\pa_1,\semvan{\Lam{x}{x}}_{ρ_1})] & \\
  f = \pa \mapsto \semvan{\px}_{ρ_1[\px↦\pa]}
  \end{array}
\end{array}\]
The annotations to the right of the trace can be understood as denoting the
``call graph'' of $\semvan{\pe}_{[]}$, with the corresponding LK transitions
as leaves.
Evaluation begins at timestamp 1 with a $\BindT$ transition to timestamp 2,
which $\semvan{\pe}_{[]}$ acknowledges by emitting an $\laterC$.
A fresh address $\pa_1$ is allocated for variable $i$ and the heap is extended
with $\memo(\pa,\semvan{\Lam{x}{x}}_{ρ_1})$.
It is interesting to realise that this process does not involve a fixpoint
despite the recursive semantics of $\mathbf{let}$.
Of course, the self-application in $\semvan{\px}$ does the job just as well, as
we will see in a moment.

Evaluation recurses into the body $\semvan{i~i}_{ρ_1}$, yielding another
$\AppIT$ transition (corresponding to a bland $\laterC$) into $\semvan{i}_{ρ_1}$.
Note that the target state $\semvan{i}_{ρ_1}$ will later be fed (via
$\betastep$) into the anonymous function in $\apply$.
This scheme is quite common: Continuation items of the transition semantics
(``data'') are reflected into the call stack of the trace semantics (``code'').
The reverse process can be recognised as defunctionalisation~\citep{Reynolds:72}.

$\semvan{i}_{ρ_1}$ guides the trace through a heap lookup:
After yielding an $\laterC$ corresponding to the $\LookupT$ transition,
$μ$ is dereferenced at $ρ(i)$ where we find $\memo(\pa,\semvan{\Lam{x}{x}}_{ρ_1})$.
This heap entry is ``self-applied'' to $μ$, effectively ``tying the knot''.
The memoisation action will run $\semvan{\Lam{x}{x}}_{ρ_1}$ to completeness
through $(\betastep)$ on heap $μ$ at timestamp 4.
Since that is already a value, $(\betastep)$ calls its second argument with
$\goodend{\FunV(f),μ}$ and we witness for the first time a reduction operation,
making an $\UpdateT$ transition to update $μ(\pa)$.
Note that this update has no effect on the heap $μ$ because
$\ret(\FunV(f))$ is precisely the same as $\semvan{\Lam{x}{x}}_{ρ_1}$.

After the heap update, we leave $\semvan{i}$ in timestamp 5,
where $\betastep$ yields to the anonymous function in $\apply$ (from the earlier
call to $\semvan{i~i}_{ρ_1}$), yielding another $\AppET$ reduction and giving
control to $f(ρ_1(i)) = \semvan{x}_{ρ_1[x ↦ ρ_1(i)]}$.
Since $x$ is an alias for $i$, steps 6 to 8 just repeat the same same heap
update sequence we observed in steps 3 to 5, concluding the example.

It is useful to review another example involving an observable heap
update. The following trace begins right before the heap update occurs in
$\Let{i}{(\Lam{y}{\Lam{x}{x}})~i}{i~i}$, that is, after reaching the value
in $\semvan{(\Lam{y}{\Lam{x}{x}})~i}_{ρ_1}$.
We will indicate the heap before a transition takes place by
writing it before $\laterC$:
\[\begin{array}{ll}
  \newcommand{\myleftbrace}[4]{\draw[mymatrixbrace] (m-#2-#1.west) -- node[right=2pt] {#4} (m-#3-#1.west);}
  \vcenter{\hbox{$
    \begin{tikzpicture}[baseline={-0.5ex},mymatrixenv]
      \matrix [mymatrix] (m)
      {
        1 & (μ_1)~\laterC & \hspace{4em} & \hspace{4em} & \hspace{2.5em} \\
        2 & (μ_2)~\laterC & & & \\
        3 & (μ_2)~\laterC & & & \\
        4 & (μ_2)~\laterC & & & \\
        5 & \goodend{\FunV(f), μ_2} & & & \\
      };
      % Braces, using the node name prev as the state for the previous east
      % anchor. Only the east anchor is relevant
      \foreach \i in {1,...,\the\pgfmatrixcurrentrow}
        \draw[dotted] (m.east|-m-\i-\the\pgfmatrixcurrentcolumn.east) -- (m-\i-2);
      \myleftbrace{3}{1}{5}{$\semvan{i~i}_{ρ_1}$}
      \myleftbrace{4}{1}{2}{$\semvan{i}_{ρ_1}$}
      \myleftbrace{4}{2}{3}{$\AppET$}
      \myleftbrace{4}{3}{5}{$\semvan{x}_{ρ_3}$}
      \myleftbrace{5}{1}{2}{$\UpdateT$}
      \myleftbrace{5}{3}{4}{$\LookupT$}
      \myleftbrace{5}{4}{5}{$\UpdateT$}
    \end{tikzpicture}
  $}} &
  \!\!\!\text{where} \begin{array}{l}
  ρ_1 = [i ↦ \pa_1] \\
  ρ_2 = [i ↦ \pa_1, y ↦ \pa_1] \\
  ρ_3 = [i ↦ \pa_1, y ↦ \pa_1, x ↦ \pa_1] \\
  μ_1 = [\pa_1 ↦ \memo(\pa_1,\semvan{(\Lam{y}{\Lam{x}{x}})~i}_{ρ_1})] \\
  μ_2 = [\pa_1 ↦ \memo(\pa_1,\semvan{\Lam{x}{x}}_{ρ_2})] \\
  f = \fn{d}{\semvan{\px}_{ρ_2[\px↦d]}} \\
  \end{array} \\
\end{array}\]
Note that both the denotation in the heap $μ_1$ \emph{and} its environment are updated
in timestamp 2, and that the new denotation is immediately visible on the next heap
lookup in timestamp 3, so that $\semvan{\Lam{x}{x}}_{ρ_2}$ takes control rather than
$\semvan{(\Lam{y}{\Lam{x}{x}})~i}_{ρ_1}$, just as the transition system requires.

The handling of data types and case expressions is routine (if a bit
syntactically heavy) and not so different to denotational semantics in
call-by-name or call-by-value.
It is a useful inclusion because it allows us to observe type errors other than
scoping errors, as well as enable more interesting examples.
Let us consider evaluation of the closed expression
$\pe \triangleq \Let{x}{\ttrue}{\ttrue~x}$.
$\semvan{\wild}$ makes it is easy to observe that the trace gets stuck
(indicating the control expression in front of every $\laterC$):
\[\begin{array}{ll}
  \newcommand{\myleftbrace}[4]{\draw[mymatrixbrace] (m-#2-#1.west) -- node[right=2pt] {#4} (m-#3-#1.west);}
  \vcenter{\hbox{$
    \begin{tikzpicture}[baseline={-0.5ex},mymatrixenv]
      \matrix [mymatrix] (m)
      {
        1 & (\pe)      \laterC & \hspace{4em} & \hspace{5em} & \hspace{2.5em} \\
        2 & (\ttrue~x) \laterC & & & \\
        3 & \stuckend & & & \\
      };
      % Braces, using the node name prev as the state for the previous east
      % anchor. Only the east anchor is relevant
      \foreach \i in {1,...,\the\pgfmatrixcurrentrow}
        \draw[dotted] (m.east|-m-\i-\the\pgfmatrixcurrentcolumn.east) -- (m-\i-2);
      \myleftbrace{3}{1}{3}{$\semvan{\pe}_{[]}$}
      \myleftbrace{4}{1}{2}{$\BindT$}
      \myleftbrace{4}{2}{3}{$\semvan{\ttrue~x}_{ρ}$}
      \myleftbrace{5}{2}{3}{$\AppIT$}
    \end{tikzpicture}
  $}} &
  \!\!\!\text{where} \begin{array}{l}
  ρ = [x ↦ \pa_1] \\
  μ = [\pa_1 ↦ \semvan{\ttrue}_ρ] \\
  \end{array} \\
\end{array}\]
Crucially, $\betastep$ is equipped to propagate $\stuckend$ up the call
stack (through potential $\UpdateT$ transitions, in particular), similar to
\citeauthor{Milner:78}'s $\mathbf{wrong}$.

Diverging traces hold no new surprises, other than they are observably different
to stuck traces.

\subsection{Maximal LK Traces}
\label{sec:maximal-traces}

It turns out that the traces $\semvan{\pe}$ generates correspond to
\emph{maximal} traces in the LK transition system.
Let us make precise what that means.

A transition system is characterised by the set of \emph{traces} it generates.
An \emph{LK trace} is a trace in $(\smallstep)$, \ie, a non-empty and
potentially infinite sequence of LK states $(σ_i)_{i∈\overline{n}}$
(where $\overline{n} = \{ m ∈ ℕ_+ \mid m ≤ n \}$ when $n∈ℕ$, $\overline{ω} = ℕ$),
such that $σ_i \smallstep σ_{i+1}$ for $i,(i+1)∈\overline{n}$.

The \emph{source} state $σ_0$ exists for finite and infinite traces, while the
\emph{target} state $σ_n$ is only defined when $n \not= ω$ is finite.

For proofs, we will often regard $(σ_i)_{i∈\overline{n}}$ as an object of type
$\STraces \triangleq ∃n∈ℕ_ω.\ \overline{n} \to \States$, where $ℕ_ω$ is defined by guarded recursion
as $ℕ_ω = \{\mathit{Z}\} + \later ℕ_ω$.
The constructor for the right sum alternative is written $\mathit{succ}$.
Now $ℕ_ω$ contains all natural numbers (where $n$ is encoded as
$(\mathit{succ})^{n}(\mathit{Z})$) and the transfinite limit ordinal
$ω = \mathit{succ}(\mathit{succ}(...))$.
We will write $1+m$ to denote $\mathit{succ}(m)$ (a different kind of $+$ than
in the recursive equation for $ℕ_ω$), just as we are used to from $ℕ$.
Hence, when $(σ_i)_{i∈\overline{n}} ∈ \STraces$ is an LK trace and $n > 0$, then
$(σ_{i+1})_{i∈\overline{n-1}} ∈ \later \STraces$ is the guarded tail of the
trace with an associated induction principle.

An important kind of trace is one that never leaves the evaluation context of
its source state:

\begin{definition}[Deep, interior and balanced traces]
  An LK trace $(σ_i)_{i∈\overline{n}}$ is
  \emph{$κ$-deep} if every intermediate continuation
  $κ_i = \cont(σ_i)$ extends $κ$ (so $κ_i = κ$ or $κ_i = ... \pushF κ$,
  abbreviated $κ_i = ...κ$).

  A trace $(σ_i)_{i∈\overline{n}}$ is called \emph{interior} if it is
  $\cont(σ_0)$-deep.
  Furthermore, an interior trace $(σ_i)_{i∈\overline{n}}$ is
  \emph{balanced}~\citep{Sestoft:97} if the target state exists and is a return
  state with continuation $\cont(σ_0)$.

  We notate $κ$-deep, interior and balanced traces as
  $\deep{κ}{(σ_i)_{i∈\overline{n}}}$, $\interior{(σ_i)_{i∈\overline{n}}}$ and
  $\balanced{(σ_i)_{i∈\overline{n}}}$, respectively.
\end{definition}

\begin{example}
  Let $ρ=[x↦\pa_1],μ=[\pa_1↦([], \Lam{y}{y})]$ and $κ$ an arbitrary
  continuation. The trace
  \[
     (x, ρ, μ, κ) \smallstep (\Lam{y}{y}, ρ, μ, \UpdateF(\pa_1) \pushF κ) \smallstep (\Lam{y}{y}, ρ, μ, \UpdateF(\pa_1) \pushF κ) \smallstep (\Lam{y}{y}, ρ, μ, κ)
  \]
  is interior and balanced. Its prefixes are interior but not balanced.
  The trace suffix
  \[
     (\Lam{y}{y}, ρ, μ, \UpdateF(\pa_1) \pushF κ) \smallstep (\Lam{y}{y}, ρ, μ, κ)
  \]
  is neither interior nor balanced.
\end{example}

We will say that the transition rules $\LookupT$, $\AppIT$, $\CaseIT$ and $\BindT$
are interior, because the lifting into a trace is, whereas the returning
transitions $\UpdateT$, $\AppET$ and $\CaseET$ are not.

A balanced trace starting at a focus expression $\pe$ and ending with $\pv$
loosely corresponds to a derivation of $\pe \Downarrow \pv$ in a natural
big-step semantics~\citep{Sestoft:97} or a non-$⊥$ result in a denotational
semantics.

It is when a derivation in a natural semantics does not exist that a small-step
semantics shows finesse, in that it differentiates two different kinds of
\emph{maximally interior} (or, just \emph{maximal}) traces:

\begin{definition}[Maximal trace]
  An LK trace $(σ_i)_{i∈\overline{n}}$ is \emph{maximal} if and only if it is
  interior and there is no $σ_{n+1}$ such that $(σ_i)_{i∈\overline{n+1}}$ is
  interior.
  More formally (and without a negative occurrence of ``interior''),
  \[
    \maxtrace{(σ_i)_{i∈\overline{n}}} \triangleq \interior{(σ_i)_{i∈\overline{n}}} \wedge (\not\exists σ_{n+1}.\ σ_n \smallstep σ_{n+1} \wedge \cont(σ_{n+1}) = ...\cont(σ_0))
  \]
  We notate maximal traces as $\maxtrace{(σ_i)_{i∈\overline{n}}}$.
\end{definition}

We call infinite and interior traces \emph{diverging}.
A maximally finite, but unbalanced trace is called \emph{stuck}.
Note that usually stuckness is associated with a state of a transition
system rather than a trace.
That is not possible in our framework; the following example clarifies.

\begin{example}[Stuck and diverging traces]
Consider the interior trace
\[
             (\ttrue~x, [x↦\pa_1], [\pa_1↦...], κ)
  \smallstep (\ttrue, [x↦\pa_1], [\pa_1↦...], \ApplyF(\pa_1) \pushF κ)
\]
It is stuck, but its singleton suffix is balanced.
An example for a diverging trace where $ρ=[x↦\pa_1]$ and $μ=[\pa_1↦(ρ,x)]$ is
\[
  (\Let{x}{x}{x}, [], [], κ) \smallstep (x, ρ, μ, κ) \smallstep (x, ρ, μ, \UpdateF(\pa_1) \pushF κ) \smallstep ...
\]
\end{example}

A maximal trace that is not balanced either diverges or is stuck:

\begin{lemmarep}[Characterisation of maximal traces]
  An LK trace $(σ_i)_{i∈\overline{n}}$ is maximal if and only if it is balanced,
  diverging or stuck.
\end{lemmarep}
\begin{proof}
  $\Rightarrow$: Let $(σ_i)_{i∈\overline{n}}$ be maximal.
  If $n=ω$ is infinite, then it is diverging due to interiority, and if
  $(σ_i)_{i∈\overline{n}}$ is stuck, the goal follows immediately.
  So we assume that $(σ_i)_{i∈\overline{n}}$ is maximal, finite and not stuck,
  so it must be balanced by the definition of stuckness.

  $\Leftarrow$: Both balanced and stuck traces are maximal.
  A diverging trace $(σ_i)_{i∈\overline{n}}$ is interior and infinite,
  hence $n=ω$.
  Indeed $(σ_i)_{i∈\overline{ω}}$ is maximal, because the expression $σ_ω$
  is undefined and hence does not exist.
\end{proof}

Interiority guarantees that the particular initial stack $κ$ of a maximal trace
is irrelevant to execution, so maximal traces that differ only in the initial
stack are bisimilar.

One class of maximal traces is of particular interest:
The maximal trace starting in $\inj(\pe)$!
Whether it is infinite, stuck or balanced is the defining operational
characteristic of $\pe$.
If we can show that $\semvan{\pe}$ distinguishes these behaviors of $\pe$, we
have proven it an adequate replacement for the LK transition system.

\subsection{Adequacy}

\Cref{fig:semvan-correctness} shows the correctness predicate $\correct$ in
our endeavour to prove $\semvan{\wild}$ adequate.
It builds on the framework of maximal LK traces established in the last section;
specifically it encodes that an \emph{abstraction} of every maximal LK trace can be
recovered by running $\semvan{\wild}$ starting from the abstraction of an initial
state.

\begin{figure}
\[\begin{array}{rcl}
  α_\Heaps([\many{\pa ↦ (ρ,\pe)}]) & = & [\many{\pa ↦ \memo(\pa,\semvan{\pe}_{ρ})}] \\
  α_\VanV(\Lam{\px}{\pe},ρ,μ,κ) & = & \goodend{\FunV(\fn{\pa}{\semvan{\pe}_{ρ[\px↦\pa]}}),α_\Heaps(μ)} \\
  α_\VanV(K~\overline{\px},ρ,μ,κ) & = & \goodend{\ConV(K,\overline{ρ(\pa)}),α_\Heaps(μ)} \\
  α_{\STraces}((σ_i)_{i∈\overline{n}},κ) & = & \begin{cases}
    \laterC~\idiom{α_{\STraces}((σ_{i+1})_{i∈\overline{n-1}},κ)} & n > 0 \\
    α_\VanV(σ_0) & \ctrl(σ_0) \text{ value } \wedge \cont(σ_0) = κ \\
    \stuckend & \text{otherwise} \\
  \end{cases} \\
  \correct((σ_i)_{i∈\overline{n}}) & = & \maxtrace{(σ_i)_{i∈\overline{n}}} \Longrightarrow ∀((\pe,ρ,μ,κ) = σ_0).\ α_{\STraces}((σ_i)_{i∈\overline{n}},κ) = \semvan{\pe}_{ρ}(α_\Heaps(μ)) \\
\end{array}\]
\caption{Correctness predicate for $\semvan{\wild}$}
  \label{fig:semvan-correctness}
\end{figure}

The family of abstraction functions makes precise the intuitive connection
between the semantic objects in $\semvan{\wild}$ and the syntactic objects in
the transition system.

We will sometimes need to disambiguate the clashing definitions from
\Cref{sec:vanilla} and \Cref{sec:problem}.
We do so by adorning semantic objects with a tilde, so $\tm \triangleq
α_\Heaps(μ)$ denotes a semantic heap which in this instance is
defined to be the abstraction of a syntactic heap $μ$.

Note first that $α_\STraces$ is defined by guarded recursion over
the LK trace, in the sense defined in \Cref{sec:maximal-traces}.
As such, the expression $\idiom{α_{\STraces}((σ_{i+1})_{i∈\overline{n-1}},κ)}$ has type
$\later \VTraces$ (the $\later$ in the type of $(σ_{i+1})_{i∈\overline{n-1}}$
maps through $α_\STraces$ via the idiom brackets).
Likewise, $=$ on $\VTraces$ is defined in the obvious structural way by guarded
recursion (as it would be if it was a finite, inductive type).

Our first goal is to establish a few auxiliary lemmas showing what kind of
properties of LK traces are preserved by $α_\STraces$ and in which way.

Let us warm up by defining a length function on traces:
\begin{definition}[Length of a trace]
  \label{defn:length}
  The \emph{length} $\len : \VTraces \to ℕ_ω$ of a trace is defined by
  guarded recursion
  \[
    \len(τ) = \begin{cases}
      1 + \purelater \len \aplater τ^{\later} & τ = \laterC~τ^{\later} \\
      0 & \text{otherwise} \\
    \end{cases}
  \]
\end{definition}

\begin{lemma}[Preservation of length]
  \label{thm:abs-length}
  Let $(σ_i)_{i∈\overline{n}}$ be an arbitrary trace.
  Then $\len(α_\STraces((σ_i)_{i∈\overline{n}},\cont(σ_0))) = n$.
\end{lemma}
\begin{proof}
  This is quite simple to see and hence a good opportunity to familiarise
  ourselves with the concept of \emph{Löb induction}, the induction principle of
  guarded recursion.
  Löb induction arises simply from applying the guarded recursive fixpoint
  combinator to a proposition:
  \[
    \textsf{löb} = \fix : \forall P.\ (\later P \Longrightarrow P) \Longrightarrow P
  \]
  That is, we assume that our proposition holds \emph{later}, \eg
  \[
    IH ∈ (\later P \triangleq \later (
        \forall n ∈ ℕ_ω.\
        \forall (σ_i)_{i∈\overline{n}}.\
        \len(α_\STraces((σ_i)_{i∈\overline{n}},\cont(σ_0))) = n
      ))
  \]
  and use $IH$ to prove $P$.
  Let us assume $n$ and $(σ_i)_{i∈\overline{n}}$ are given, define
  $τ \triangleq α_\STraces((σ_i)_{i∈\overline{n}},\cont(σ_0))$ and proceed by case analysis
  over $n$:
  \begin{itemize}
    \item \textbf{Case $n=0$}: Then we have $j = 0$ and either $τ = \goodend{α_\States(σ_0)}$
      or $τ = \stuckend$, both of which map to $0$ under
      $\len$.
    \item \textbf{Case $n>0$}: Then $n = 1+m$ and $m ∈ \later ℕ_ω$ and the
      first case of $α_\States$ applies, hence $τ = σ \cons τ^{\later}$ for some
      $σ∈\States, τ^{\later}∈\later \VTraces$.
      Now we apply the inductive hypothesis, as follows:
      Let $(σ_{i+1})_{i∈\overline{m}} ∈ \later \STraces$ be the guarded
      tail of the LK trace $(σ_i)_{i∈\overline{n}}$.
      Then we can apply $IH \aplater m \aplater (σ_{i+1})_{i∈\overline{m}}$ and
      get a proof for $\later (\len(τ^{\later}) = m)$.
      Now we can prove
      \[
        n = 1 + m = 1 + \len(τ^{\later}) = \len(τ).
      \]
  \end{itemize}
\end{proof}

\begin{lemmarep}[Preservation of characteristic]
  \label{thm:abs-max-trace}
  Let $(σ_i)_{i∈\overline{n}}$ be a maximal trace.
  Then $α_\STraces((σ_i)_{i∈\overline{n}}, cont(σ_0))$ is ...
  \begin{itemize}
    \item ... infinite if and only if $(σ_i)_{i∈\overline{n}}$ is diverging
    \item ... ending with $\goodend{\wild,\wild}$ if and only if $(σ_i)_{i∈\overline{n}}$ is balanced
    \item ... ending with $\stuckend$ if and only if $(σ_i)_{i∈\overline{n}}$ is stuck
  \end{itemize}
\end{lemmarep}
\begin{proof}
  The first point follows by a similar inductive argument as in \Cref{thm:abs-length}.

  In the other cases, we may assume that $n$ is finite.
  If $(σ_i)_{i∈\overline{n}}$ is balanced, then $σ_n$ is a return state with
  continuation $\cont(σ_0)$, so its control expression is a value.
  Then $α_\STraces$ will conclude with $\goodend{\wild,\wild}$.
  Conversely, if the trace ended with $\goodend{\wild,\wild}$, then $\cont(σ_n) = \cont(σ_0)$
  and $\ctrl(σ_n)$ is a value, so $(σ_i)_{i∈\overline{n}}$ forms a
  balanced trace.
  The stuck case is similar.
\end{proof}

The previous lemma is interesting as it allows us to apply the classifying
terminology of interior traces to a $τ$ that is an abstraction of a
\emph{maximal} LK trace.
For such a maximal $τ$ we will say that it is balanced when it ends with
$\goodend{\wild,\wild}$, stuck if ending in $\stuckend$ and diverging if
infinite.

With increased clarity, we go on to prove the correctness predicate:

\begin{theoremrep}[Correctness of $\semvan{\wild}$]
  \label{thm:semvan-correct}
  $\correct$ from \Cref{fig:semvan-correctness} holds.
  That is, whenever $(σ_i)_{i∈\overline{n}}$ is a maximal LK trace with source
  state $(\pe,ρ,μ,κ)$, we have
  $α_{\STraces}((σ_i)_{i∈\overline{n}},κ) = \semvan{\pe}_{ρ}(α_\Heaps(μ))$.
\end{theoremrep}
\begin{proof}
By Löb induction, with $IH ∈ \later C$ as the hypothesis.

We will say that an LK state $σ$ is stuck if there is no applicable rule in the
transition system (\ie, the singleton LK trace $σ$ is maximal and stuck).

Now let $(σ_i)_{i∈\overline{n}}$ be a maximal LK trace with source state
$σ_0=(\pe,ρ,μ,κ)$ and let $τ = \semvan{\pe}_{α_\Environments(ρ)}(α_\Heaps(μ))$.
Then the goal is to show $α_{\STraces}((σ_i)_{i∈\overline{n}},κ) = τ$.
We do so by cases over $\pe$, abbreviating $\tm \triangleq α_\Heaps(μ)$:
\begin{itemize}
  \item \textbf{Case $\px$}:
    Let us assume first that $σ_0$ is stuck. Then $\px \not∈ \dom(ρ)$ (because
    $\LookupT$ is the only transition that could apply), so
    $τ = \stuckend$ and the goal follows from
    \Cref{thm:abs-max-trace}.

    Otherwise, $σ_1 \triangleq (\pe', ρ', μ, \UpdateF(\pa) \pushF κ), σ_0 \smallstep σ_1$
    via $\LookupT$, and $ρ(\px) = \pa, μ(\pa) = (ρ', \pe')$.
    To show that the tails equate, it suffices to show that they equate \emph{later}.

    We can infer that $\tm(\pa) = \memo(\pa,\semvan{\pe'}_{ρ'})$ from the
    definition of $α_\Heaps$, so
    \[
      \tm(ρ(\px))(\tm) = (\semvan{\pe'}_{ρ'} \betastep \fn{v}{(\fn{μ}{\laterC~\goodend{v,μ[\pa ↦ \memo(\pa,\ret(v))]}})})(\tm)
    \]

    Let us define $τ^\later \triangleq \idiom{\semvan{\pe'}_{ρ'}(\tm)}$ and
    apply the induction hypothesis $IH$ to the maximal trace starting at $σ_1$.
    This yields an equality
    \[
      IH \aplater (σ_{i+1})_{i∈\overline{m}} ∈ \idiom{α_\STraces((σ_{i+1})_{i∈\overline{m}},\UpdateF(\pa) \pushF κ) = τ^{\later}}
    \]
    When $τ^{\later}$ is infinite, we are done. Similarly, if $τ^{\later}$ ends
    in $\stuckend$ then $(\betastep)$ will return $τ^{\later}$, indicating
    by \Cref{thm:abs-length} and \Cref{thm:abs-max-trace} that
    $(σ_{i+1})_{i∈\overline{n-1}}$ is stuck and hence $(σ_i)_{i∈\overline{n}}$
    is, too.

    Otherwise $τ^{\later}$ ends after $m-1$ $\laterC$s with $\goodend{v,\tm_m}$ and
    by \Cref{thm:abs-max-trace} $(σ_{i+1})_{i∈\overline{m}}$ is balanced; hence
    $\cont(σ_m) = \UpdateF(\pa) \pushF κ$ and $\ctrl(σ_m)$ is a value.
    So $σ_m = (\pv,ρ_m,μ_m,\UpdateF(\pa) \pushF κ)$ and the
    $\UpdateT$ transition fires, reaching $(\pv,ρ_m,μ_m[\pa ↦ (ρ_m, \pv)],κ)$
    and this must be the target state $σ_n$ (so $m = n-2$), because it remains
    a return state and has continuation $κ$, so $(σ_i)_{i∈\overline{n}}$ is
    balanced.
    Likewise, by the second argument of $(\betastep)$, we call do another
    memoisation step on $\goodend{v,\tr_m}$, updating the heap to
    \[
      \goodend{v,\tm_m[\pa↦\memo(\pa,\ret(v))]} = \goodend{v,\tm_m[\pa↦\memo(\pa,\semvan{\pv}_{ρ_m})]} = α_\VanV(σ_n),
    \]
    and this equality concludes the proof.

  \item \textbf{Case $\pe~\px$}:
    The cases where $τ$ gets stuck or diverges before finishing evaluation of
    $\pe$ are similar to the variable case.

    So let us focus on the situation when $τ^{\later} \triangleq
    \idiom{\semvan{\pe}_{ρ}(\tm)}$ returns and let $σ_m$ be LK state at the
    end of the finite maximal trace $(σ_{i+1})_{i∈\overline{m-1}}$ through $\pe$
    starting in stack $\ApplyF(\pa) \pushF κ$.
    Since it is maximal, any transition $σ_m \smallstep σ_{m+1}$ must leave
    the stack $\ApplyF(\pa) \pushF κ$, necessarily by an $\AppET$ transition.
    That in turn means that the value in $\ctrl(σ_m)$ must be a lambda
    $\Lam{\py}{\pe'}$, hence $τ^{\later}$ ends in $\goodend{\FunV(\fn{\pa}{\idiom{\semvan{\pe'}_{ρ_m[\py ↦ \pa]}}}), \tm_m} = σ_\VanV(σ_m)$
    (where $\tm_m$ corresponds to the heap in $σ_m$ in the usual way).

    Now let $(σ_{i+m+1})_{i∈\overline{k}}$ be the maximal trace starting at
    $σ_{m+1}=(\pe',ρ_m[\py↦\pa], μ_m,κ)$.
    We can apply the induction hypothesis to this LK trace and
    $τ_2^{\later} \triangleq \idiom{\semvan{\pe'}_{ρ_m[\py↦\pa]}(\tm_m)}$
    to get the equality
    $\idiom{α_\STraces((σ_{i+m+1})_{i∈\overline{k}},κ) = τ_2^{\later}}$.
    From this and our earlier equalities, we get
    $α_\STraces((σ_i)_{i∈\overline{n}},κ) = τ$, concluding the proof.

  \item \textbf{Case $\Case{\pe_s}{\Sel[r]}$}:
    Similar to the application and lookup case.

  \item \textbf{Cases $\Lam{\px}{\pe}$, $K~\many{\px}$}:
    The length of both traces is $n = 0$ and the goal follows by simple calculation.

  \item \textbf{Case $\Let{\px}{\pe_1}{\pe_2}$}:
    Let $σ_0 = (\Let{\px}{\pe_1}{\pe_2},ρ,μ,κ)$.
    Then $σ_1 = (\pe_2, ρ', μ',κ)$ by $\BindT$, where $ρ' = ρ[\px↦\pa], μ'
    = μ[\pa↦(ρ',\pe_1)]$.
    Since the stack does not grow, maximality from the tail $(σ_{i+1})_{i∈\overline{n-1}}$
    transfers to $(σ_{i})_{i∈\overline{n}}$.
    Straightforward application of the induction hypothesis to
    $(σ_{i+1})_{i∈\overline{n-1}}$ yields the equality for the tail (after a bit
    of calculation for the updated environment and heap), which concludes the
    proof.
\end{itemize}
\end{proof}

\Cref{thm:semvan-correct} and \Cref{thm:abs-max-trace} are the key to proving a
strong version of adequacy for $\semvan{\wild}$, where $σ$ is defined to be a
\emph{final} state if $\ctrl(σ)$ is a value and $\cont(σ) = \StopF$.

\begin{theoremrep}[Adequacy of $\semvan{\wild}$]
  \label{thm:semvan-adequate}
  Let $τ = \semvan{\pe}_{[]}([])$.
  \begin{itemize}
    \item
      $τ$ ends with $\goodend{\wild,\wild}$ (is balanced) iff there exists a final
      state $σ$ such that $\inj(\pe) \smallstep^* σ$.
    \item
      $τ$ ends with $\stuckend$ (is stuck) iff there exists a non-final
      state $σ$ such that $\inj(\pe) \smallstep^* σ$ and there exists no $σ'$
      such that $σ \smallstep σ'$.
    \item
      $τ$ is infinite (is diverging) iff for all $σ$ with $\inj(\pe)
      \smallstep^* σ$ there exists $σ'$ with $σ \smallstep σ'$.
  \end{itemize}
\end{theoremrep}
\begin{proof}
  There exists a maximal trace $(σ_i)_{i∈\overline{n}}$ starting
  from $σ_0 = \inj(\pe)$, and by \Cref{thm:semvan-correct} we have
  $α_\STraces((σ_i)_{i∈\overline{n}},\StopF) = τ$.
  \begin{itemize}
    \item[$\Rightarrow$]
      \begin{itemize}
        \item
          If $(σ_i)_{i∈\overline{n}}$ is balanced, its target state $σ_n$
          is a return state that must also have the empty continuation, hence it
          is a final state.
        \item
          If $(σ_i)_{i∈\overline{n}}$ is stuck, it is finite and maximal, but not balanced, so its
          target state $σ_n$ cannot be a return state;
          otherwise maximality implies $σ_n$ has an (initial) empty continuation
          and the trace would be balanced. On the other hand, the only returning
          transitions apply to return states, so maximality implies there is no
          $σ'$ such that $σ \smallstep σ'$ whatsoever.
        \item
          If $(σ_i)_{i∈\overline{n}}$ is diverging, $n=ω$ and for every $σ$ with
          $\inj(\pe) \smallstep^* σ$ there exists an $i$ such that $σ = σ_i$ by
          determinism.
      \end{itemize}

    \item[$\Leftarrow$]
      \begin{itemize}
        \item
          If $σ_n$ is a final state, it has $\cont(σ) = \cont(\inj(\pe)) = []$,
          so the trace is balanced.
        \item
          If $σ$ is not a final state, $τ'$ is not balanced. Since there is no
          $σ'$ such that $σ \smallstep^* σ'$, it is still maximal; hence it must
          be stuck.
        \item
          Suppose that $n∈ℕ_ω$ was finite.
          Then, if for every choice of $σ$ there exists $σ'$ such that $σ
          \smallstep σ'$, then there must be $σ_{n+1}$ with $σ_n \smallstep
          σ_{n+1}$, violating maximality of the trace.
          Hence it must be infinite.
          It is also interior, because every stack extends the empty stack,
          hence it is diverging.
      \end{itemize}
  \end{itemize}
\end{proof}

\subsection{Discussion}

We can already give perspective on two of the goals we set ourselves in
\Cref{sec:problem}:

As the domain $\VanD$ of our semantics $\semvan{\wild}$ is one defined by
guarded recursion, the approximation order between elements of the domain is
discrete and all elements are total.
$\semvan{\wild}$ is formulated entirely within this internal language and as
such, looping programs are denoted by total elements as well,
fullfilling Goal 2.

\section{Eventful Semantics}
\label{sec:eventful}

In the previous section, we gave a trace semantics and proved it adequate \wrt
the LK transition semantics.
With compositionality and structural induction we recover strong advantages of
denotational semantics.

It is striking that although $\semst{\wild}$ is as expressive as the LK
transition system, the information encoded in the states of the generated traces
is not enough to recover the LK transition system in the sense of
\citet[Chapter 43]{Cousot:21}.
For example, every return state $((\lbl,v),\tm)$ admits at least two possible
transitions:
Either a heap update or a $β$-reduction.
In the LK transition system, the particular transition is governed by the
continuation stack, of which $\semst{\wild}$ maintains no reification in its states.
It is however easily possible to ``elaborate'' $\semst{\wild}$ to include the
proper stack manipulations and emit the current lexical environment as part of
every state.
We think it is rather uninteresting to give the closed, elaborated definition
of $\semst{\wild}$ which necessarily gives up simplicity \sg{Give it in the Appendix}.
It is more interesting to see what other parts of the traces we can \emph{omit}
before compromising on expressivity and retain just as much information as
needed by the particular analysis we want to prove correct.

For example, in Control-Flow Analysis~\citep{Shivers:91}, we are ultimately
interested in observing the different lambda labels a variable occurrence might
evaluate to and thus construct a conservative but useful approximation of
interprocedural control-flow.
We do not need the lexical environment or the continuation stack, but the
call-by-need nature of our semantics needs the state in $\betastep$.
That makes $\semst{\wild}$ a suitable concrete semantics for abstraction.

Yet for our usage analysis in \Cref{sec:problem}, we do not care so much
about the \emph{state} in which variable lookup happens, but rather about at
which \emph{address} the $\LookupT$ transition happened, as well as the $\BindT$
transition that tells us what $\mathbf{let}$ binding that particular address
refers to.

In this section, we will omit states in favor of tracing the instantiations of
transition rules, calling those instantiations \emph{events}.
The resulting \emph{eventful trace semantics} $\semevt{\wild}$ is (closer to)
the preferred framework of \citet{Cousot:21}.%
\footnote{Cousot's \emph{stateless} semantics even goes one step further
and stops passing around a materialised mapping for the heap, instead
rematerialising the heap as needed from the history of previous events in
an effort to model weak memory models.}
In essence, we shift attention from the \emph{nodes} of the control-flow graph
to the \emph{edges} modelling the transitions.

\begin{figure}
\[\begin{array}{c}
 \arraycolsep=3pt
 \begin{array}{rrclcl}
  \text{Event} & ε & ∈ & \Events & ::=  & \BindA(\px,\pa↦d) \mid \LookupA(\pa) \mid \UpdateA(\pa↦d) \\
               &   &   &          & \mid & \AppIA(d) \mid \AppEA(\px↦d) \mid \CaseIA(d) \mid \CaseEA(K,\many{\px↦d}) \\
 \end{array} \\
 \\[-0.5em]
 \arraycolsep=3pt
 \begin{array}{rrclcl@{\quad}rrclcl}
  \text{Heap}         & μ   & ∈ & \Heaps         & =      & \Addresses \pfun \later\EventD
  &
  \text{Delayed event}        & ε^{\later} & ∈ & \later \Events         &     &
  \\
  \text{Program trace}        & τ          & ∈ & \Traces        & ::= & \goodend{v,μ} \mid \stuckend{} \mid ε^{\later} \cons τ^{\later}
  &
  \text{Delayed trace}         & τ^{\later} & ∈ & \later \Traces &     &
  \\
  \text{Eventful domain}      & d          & ∈ & \EventD                   & =   & \Heaps \to \Traces
  &
  \text{Delayed element}       & d^{\later} & ∈ & \later \EventD            &     &
 \end{array} \\
 \\[-0.5em]
 \arraycolsep=3pt
 \begin{array}{rrclcl}
  \text{Eventful value} & v & ∈ & \EventV & ::= & \FunV(f ∈ \later\EventD \to \EventD) \mid \ConV(K,\many{d^\later}^{α_K}) \\
 \end{array} \\
 \\[-0.5em]
 \begin{array}{rcl}
  \multicolumn{3}{c}{ \ruleform{
    \begin{array}{c}
      (\betastep) : \EventD \to (\EventV \pfun \EventD) \to \EventD \quad \ret : \EventV \to \EventD \quad \apply : \EventD \to \EventD \to \EventD \\
      \deref : \Addresses \to \EventD \quad \select : \EventD \to ((K:\Con) \to (\later\EventD)^{α_K} \pfun \EventD) \to \EventD \\
    \end{array}
  }} \\
  \\[-0.5em]
  (d \betastep f)(μ) & = & \begin{cases}
    \many{ε \cons {}} f(v)(μ') & \text{$d(μ) = \many{ε \cons {}} \goodend{v,μ'}$ and $v ∈ \dom(f)$} \\
    \many{ε \cons {}} \stuckend{} & \text{$d(μ) = \many{ε \cons {}} \goodend{v,μ'}$ and $v \not∈ \dom(f)$} \\
    d(μ) & \text{otherwise} \\
  \end{cases} \\
  \\[-0.5em]
  \ret(v)(μ) & = & \goodend{v,μ} \\
  \deref(\pa)(μ)   & = & \LookupA(\pa) \cons (μ(\pa) \betastep \fn{v\,μ'}{\UpdateA(\pa↦\ret(v)) \cons \goodend{v,μ'[\pa↦\ret(v)]}})(μ) \\
  \apply(d_\pe,d_\px) & = & d_\pe \betastep \fn{(\FunV(f))}{f(d_\px)} \\
  \select(d,\alts) & = & d \betastep \fn{(\ConV(K_s,\many{d_s^\later}))}{\alts(K_s, \many{d_s^\later})} \quad \text{where } (K_s, \many{d_s^\later}) ∈ \dom(\alts) \\
 \end{array} \\
 \\[-0.5em]
 \begin{array}{rcl}
  \multicolumn{3}{c}{ \ruleform{ \semevt{\wild} \colon \Exp → (\Var \pfun \EventD) → \EventD } } \\
  \\[-0.5em]
  \semevt{\px}_ρ       & = & \begin{cases}
    ρ(\px) & \px ∈ \dom(ρ) \\
    \fn{\wild}{\stuckend{}}  & \text{otherwise} \\
  \end{cases} \\
  \\[-0.5em]
  \semevt{\Lam{\px}{\pe}}_ρ & = & \ret(\FunV(\fn{d^\later}{\AppEA(\px↦d^\later) \cons \semevt{\pe}_{ρ[\px↦d^\later]}})) \\
  \\[-0.5em]
  \semevt{\pe~\px}_ρ   & = & \begin{cases}
    \AppIA(ρ(x)) \cons \apply(\semevt{\pe}_ρ, ρ(\px)) & \px ∈ \dom(ρ) \\
    \fn{\wild}{\stuckend{}}  & \text{otherwise} \\
  \end{cases} \\
  \\[-0.5em]
  \semevt{\Let{\px}{\pe_1}{\pe_2}}_ρ(μ) & = &
    \begin{letarray}
      \text{let} & ρ' = ρ[\px ↦ \deref(\pa)] \quad \text{where $\pa \not∈ \dom(μ)$} \\
                 & d_1^\later = \semevt{\pe_1}_{ρ'} \\
      \text{in}  & \BindA(\px,\pa↦d_1^\later) \cons \semevt{\pe_2}_{ρ'}(μ[\pa ↦ d_1^\later])
    \end{letarray} \\
  \\[-0.5em]
  \semevt{K~\many{\px}}_ρ & = & \ret(\ConV(K,\many{\semevt{\px}_ρ})) \\
  \\[-0.5em]
  \semevt{\Case{\pe_s}{\Sel[r]}}_ρ & = &
    \begin{letarray}
      \text{let} & \alts = \fn{K}{\fn{\many{d^\later}}{\CaseEA(K,\many{\px↦d^\later}) \cons \semevt{\pe_r}_{ρ[\many{\px↦d^\later}]}}} \\
      \text{in} & \CaseIA(\semevt{\pe_s}_ρ) \cons \select(\semevt{\pe_s}_ρ, \alts)  \\
    \end{letarray}
 \end{array} \\
\end{array}\]
\caption{Eventful Trace Semantics}
  \label{fig:semevt}
\end{figure}

\subsection{Reading Between the Nodes}

\Cref{fig:semevt} gives the definition for the eventful trace semantics
$\semevt{\wild}$.
The domain of eventful maximal traces $\EventD$ maps a heap
to a possibly infinite or stuck program trace $τ$, just as the stateful
semantics $\semst{\wild}$.

The fundamental difference to the stateful semantics manifests in the
information encoded in a trace $τ$:
Instead of a sequence of states $σ$ there is a (possibly empty, if stuck)
sequence of \emph{events} $ε$.
With the exception of $\ValueA$ events which we will explain in due course,
each kind of event corresponds to a transition rule in the LK semantics.
In fact, each event carries with it part of the ``closure'' of the
particular transition rule which is readily available for analysis.
For example, the $\LookupA$ event carries the heap address which is accessed,
$\BindA$ events carry the $\mathsf{let}$-bound variable, its fresh address and
the denotation of the right-hand side.

We will occassionally write $ι$ to denote a finite list of events contained in
a finite program trace such as in $\betastep$.
We re-use the same greek letter as in \Cref{sec:stateful} and disambiguate as
needed, as before using tilde $\tilde{τ}$.

\begin{figure}
\[\begin{array}{rcl}
  α_\Environments(ρ) & = & \deref \circ ρ \\
  α_\Heaps([\many{\pa ↦ (ρ,\pe)}]) & = & [\many{\UpdateA(\pa ↦ \semst{\pe}_{α_\Environments(ρ)}}] \\
  α_\EventV(ρ,\Lam{\px}{\pe}) & = & \FunV(\fn{d}{\idiom{\AppEA(\px↦d)} \cons \idiom{\semevt{\pe}_{α_\Environments(ρ)[\px↦d]}}}) \\
  α_\EventV(ρ,K~\many{\px}) & = & \ConV(K,\idiom{\many{\semevt{\px}_{α_\Environments(ρ)}}}) \\
  α_\Events(σ) & = & \begin{cases}
    \BindA(\px,\pa↦\semevt{\pe_1}_{α_\Environments(ρ[\px↦\pa])}) & σ = (\Let{\px}{\pe_1}{\wild},ρ,μ,\wild) \wedge \pa \not∈ \dom(μ) \\
    \AppIA(\deref(\pa)) & σ = (\wild~\px,ρ,\wild,\ApplyF(\pa) \pushF \wild) \\
    \CaseIA(\semevt{\pe_s}_{α_\Environments(ρ)}) & σ = (\Case{\pe_s}{\wild},ρ,\wild, \wild)\\
    \LookupA(ρ(\px)) & σ = (\px,ρ,\wild,\UpdateF(\pa) \pushF \wild) \\
    \AppEA(\px↦\deref(\pa)) & σ = (\Lam{\px}{\wild},\wild,\wild,\ApplyF(\pa) \pushF \wild) \\
    \CaseEA(K',\many{\px_i ↦ \deref(ρ(\py))}) & σ = (K'~\many{\py}, ρ, \wild, \SelF(\wild,\Sel) \pushF \wild) \wedge K' = K_i \\
    \UpdateA(\pa↦\ret(α_\EventV(ρ,\pv))) & σ = (\pv,ρ,\wild,\UpdateF(\pa) \pushF \wild) \\
  \end{cases} \\
  α_{\Traces}((σ_i)_{i∈\overline{n}},κ) & = & \begin{cases}
    \idiom{α_\Events(σ_1)} \cons \idiom{α_{\Traces}((σ_{i+1})_{i∈\overline{n-1}},κ)} & n > 1 \\
    \goodend{α_\EventV(ρ,\pv)} & n = 1 \wedge σ_1 = (\pv, ρ, \wild, κ) \\
    \stuckend{} & \text{otherwise} \\
  \end{cases} \\
  \mathcal{C} & = & ∀(σ_i)_{i∈\overline{n}}.\ \maxtrace{(σ_i)_{i∈\overline{n}}} \Longrightarrow ∀((\pe,ρ,μ,κ) = σ_1).\ α_{\Traces}((σ_i)_{i∈\overline{n}},κ) = \semst{\pe}_{α_\Environments(ρ)}(α_\Heaps(μ)) \\
\end{array}\]
\caption{Correctness predicate for $\semevt{\wild}$}
  \label{fig:semevt-correctness}
\end{figure}

\sg{Justify why we can't abstract $\semst{\wild}$ (where to take the address in the lookup case from?).
Although, perhaps we could; by zipping up the original heaps and stacks. But a direct proof is simpler.}

Similarly to before, we can define preservation lemmas:

\begin{lemma}[Preservation of length]
  \label{thm:abs-length-less}
  Let us define the length $\mathit{len} : \Traces \to ℕ_ω$ of a trace by
  guarded recursion
  \[
    \mathit{len}(τ) = \begin{cases}
      1 + \idiom{\mathit{len}(τ^{\later})} & τ = a^\later \cons τ^{\later} \\
      1 & \text{otherwise} \\
    \end{cases}
  \]
  Now let $(σ_i)_{i∈\overline{n}}$ be an arbitrary trace.
  Then $\mathit{len}(α_\STraces((σ_i)_{i∈\overline{n}},\cont(σ_1))) = n$.
\end{lemma}
\begin{proof}
  Similar to \Cref{thm:abs-length}.
\end{proof}

\begin{lemma}[Preservation of components]
  \label{thm:abs-components}
  Let $(σ_i)_{i∈\overline{n}}$ be a trace and let $τ = α_\Traces((σ_i)_{i∈\overline{n}},\cont(σ_1))$.
  Then for all $j∈\overline{n-1}$, $τ_j = α_\Events(σ_j)$
  (where $τ_j$ denotes the $j$th event in $τ$)
  and if $τ$ ends in $\goodend{v}$, then $v = α_\EventV(ρ,\pv)$, where $σ_n = (\pv, ρ, \wild, \wild)$.
\end{lemma}
\begin{proof}
  Similar to \Cref{thm:abs-states}.
\end{proof}

\begin{lemma}[Preservation of characteristic]
  \label{thm:abs-max-trace-less}
  Let $(σ_i)_{i∈\overline{n}}$ be a maximal trace.
  Then $α_\Traces((σ_i)_{i∈\overline{n}}, cont(σ_1))$ is ...
  \begin{itemize}
    \item ... infinite if and only if $(σ_i)_{i∈\overline{n}})$ is diverging
    \item ... ending with $\goodend{\wild}$ if and only if $(σ_i)_{i∈\overline{n}}$ is balanced
    \item ... ending with $\stuckend{\wild}$ if and only if $(σ_i)_{i∈\overline{n}}$ is stuck
  \end{itemize}
\end{lemma}
\begin{proof}
  Similar to \Cref{thm:abs-max-trace}.
\end{proof}

\begin{theorem}[Correctness of $\semevt{\wild}$]
  \label{thm:semevt-correct}
  $\mathcal{C}$ from \Cref{fig:semevt-correctness} holds.
  That is, whenever $(σ_i)_{i∈\overline{n}}$ is a maximal LK trace with source
  state $(\pe,ρ,μ,κ)$, we have
  $α_{\Traces}((σ_i)_{i∈\overline{n}},κ) = \semst{\pe}_{α_\Environments(ρ)}(α_\Heaps(μ))$.
\end{theorem}
\begin{proof}
  Similar to \Cref{thm:semst-correct}. \sg{We probably want to carry out the proof explicitly.}
\end{proof}

Again, this allows us to prove a strong version of adequacy:

\begin{lemma}[Adequacy of $\semevt{\wild}$]
  \label{thm:semevt-adequate}
  Let $τ = \semevt{\pe}_{[]}([])$.
  \begin{itemize}
    \item
      $τ$ ends with $\goodend{\wild}$ (is balanced) iff there exists a final
      state $σ$ such that $\inj(\pe) \smallstep^* σ$.
    \item
      $τ$ ends with $\stuckend{\wild}$ (is stuck) iff there exists a non-final
      state $σ$ such that $\inj(\pe) \smallstep^* σ$ and there exists no $σ'$
      such that $σ \smallstep σ'$.
    \item
      $τ$ is infinite (is diverging) iff for all $σ$ with $\inj(\pe)
      \smallstep^* σ$ there exists $σ'$ with $σ \smallstep σ'$.
  \end{itemize}
\end{lemma}
\begin{proof}
  Same as for \Cref{thm:semst-adequate}.
\end{proof}

It should be clear that $α_\Events$ could elaborate the returned events with
more information from the machine states without a noticable effect on the
correctness proof.
For example it could interleave the list of events with program labels.
The only drawback of doing so would be that $\semevt{\wild}$ becomes more
complicated.
The same applies to $α_\States$ from \Cref{sec:stateful} and $\semst{\wild}$, of
course, where this increase in complexity can be witnessed by having a look at
the generalisation in the Appendix \sg{Write that}.

\section{The Imperative Essence of Lazy Evaluation}
\label{sec:essence}

Before we can talk about abstraction, we have to record a few meta-theoretic
properties.
We will prove these properties in terms of the eventful semantics but they apply
just the same for the stateful one, or even the ``bare-bones'' version from
the Appendix.

As we have observed in \Cref{sec:problem}, the semantics of call-by-need is more
complicated than the semantics for call-by-name and call-by-value because it
relies on a heap to model memoisation.
This leads to a more complicated domain model that essentially passes around
heaps as state.

On the other hand, memoisation is a very ``benign'' kind of state and often we
can reason rather naturally about call-by-need programs without thinking too
much about heaps, in contrast to a calculus with arbitrary assignments to ref
cells.

For example, we can observe that heaps in our semantics evolve in a very
specific way:
When $\semevt{\pe}_ρ(μ) = ... \cons \goodend{v,μ'}$, we intuit that $μ'$ must be
``at least as evaluated'' as $μ$, \eg, for all $\pa ∈ \dom(μ)$ either $μ(\pa) =
μ'(\pa)$ or $μ'(\pa)$ is ``the value of'' $μ(\pa)$.
We make this observation precise via the following definitions:

\begin{definition}[Big-step]
  \label{defn:eval-d}
  For all $d ∈ \EventD$, $v ∈ \EventV$, $μ, μ' ∈ \Heaps$ we say that
  $d$ \emph{evaluates} in $μ$ to $(v,μ')$ (written $\bigstep{d}{μ}{v}{μ'}$) if
  and only if $d(μ) = ... \cons \goodend{v,μ'}$.
\end{definition}

Big-step notation deliberately throws away much information about the trace,
retaining only whether it was balanced and what the final value and heap is,
hence the name ``big-step'' is fitting.

\begin{definition}[Heap forcing]
  \label{defn:force-heap}
  For all $μ_1, μ_2 ∈ \Heaps$ we say that $μ_1$ \emph{forces} to $μ_2$
  (written $μ_1 \forcesto μ_2$) if and only if
  \begin{enumerate}
    \item $\dom(μ_1) ⊆ \dom(μ_2)$
    \item For all $\pa$, either
      \begin{itemize}
        \item $μ_1(\pa) = μ_2(\pa)$, or
        \item there exists $v,μ'_1$ such
          that $\bigstep{μ_1(\pa)}{μ_1}{v}{μ'_1}$ and $μ_2(\pa) = \ret(v)$ as well as
          $μ'_1 \forcesto μ_2$ if $μ_1 \not= μ'_1$.
      \end{itemize}
  \end{enumerate}
\end{definition}

Our earlier intuition can now be formulated as follows:
If $\bigstep{\semevt{\pe}_ρ}{μ}{v}{μ'}$, then $μ \forcesto μ'$.

For the remainder of this work, we will identify heaps modulo consistent
readdressing.
Of course, a rigorous treatment would have to carry around readdressing
substitutions and apply them to adjust the domain of a heap, its entries,
as well as the heaps and values in the returned traces.
\sg{Perhaps I should give a figure with the definitions I have in mind, maybe
in the Appendix.}

%We write $μ_1 \forcessym μ_2$ if either $μ_1 \forcesto μ_2$ or
%$μ_2 \forcesto μ_1$ holds.
When a heap entry is evaluated, it stays that way during forcing:

\begin{lemma}
  \label{thm:force-heap-val}
  For all $μ_1,μ_2,\pa,v$ such that $μ_1 \forcesto μ_2$ and $μ_1(\pa) = \ret(v)$,
  then $μ_2(\pa) = \ret(v)$.
\end{lemma}
\begin{proof}
  Immediate proof by contraposition and unfolding of $\ret$.
\end{proof}

So once an entry becomes a value during forcing, it stays that way.

The heap forcing relation is reflexive (it is always $μ(\pa) = μ(\pa)$) and
also antisymmetric:

\begin{lemma}[$(\forcesto)$ is antisymmetric]
  \label{thm:force-heap-trans}
  Let $μ_1,μ_2$ be lazy heaps and $μ_1 \forcesto μ_2, μ_2 \forcesto μ_1$.
  Then $μ_1 = μ_2$.
\end{lemma}
\begin{proof}
  It is easy to see that both $μ_1$ and $μ_2$ must have the same domain.

  Now suppose that there exists $\pa$ such that $μ_1(\pa) \not= μ_2(\pa)$.
  Then there exists $v_1$ such that $\bigstep{μ_1(\pa)}{μ_1}{v_1}{\wild}$ and
  $μ_2(\pa) = \ret(v_1)$.
  By \Cref{thm:force-heap-val} applied to $μ_2 \forcesto μ_1$, we get $μ_1(\pa)
  = v_1 = μ_2(\pa)$, a contradiction.
\end{proof}

(It is worth noting that antisymmetry only holds modulo readdressing.)
However, $(\forcesto)$ is not easily proven transitive without further
characterisation of the domain elements returned by $\semevt{\wild}$.

\begin{definition}[Lazy elements]
  \label{defn:lazy-d}
  A heap $μ$ is \emph{lazy} if and only if its entries are lazy elements.
  An element $d ∈ \EventD$ is \emph{lazy} if and only if
  \begin{enumerate}
    \item \textup{(Forces)} For all lazy $μ,μ' ∈ \Heaps, v ∈ \EventV$ such that $d(μ) = ... \cons \goodend{v,μ'}$, it is the case that $μ \forcesto μ'$.
     Furthermore,
     \begin{itemize}
       \item When $v = \FunV(f)$ (for some $f$), then $f(d')$ is lazy whenever $d'$ is lazy.
       \item When $v = \ConV(K,\many{d})$ (for some $K,\many{d}$), then all $\many{d}$ are lazy.
     \end{itemize}
    \item \textup{(Postpone)} For all lazy $μ_1,μ_2 ∈ \Heaps$ such that $μ_1 \forcesto μ_2$ and
     $\bigstep{d}{μ_1}{v_1}{μ'_1}$ (for some $v_1,μ'_1$), we also have
     $\bigstep{d}{μ_2}{v_2}{μ'_2}$ (for some $v_2,μ'_2$) and
     $v_1 = v_2$ as well as $μ'_1 \forcesto μ'_2$.
    \item \textup{(Speculate)} For all lazy $μ_1,μ_2 ∈ \Heaps$ such that $μ_1 \forcesto μ_2$ and
     $\bigstep{d}{μ_2}{v_2}{μ'_2}$ (for some $v_2,μ'_2$), we also have
     $\bigstep{d}{μ_1}{v_1}{μ'_1}$ (for some $v_1,μ'_1$) and
     $v_1 = v_2$ as well as $μ'_1 \forcesto μ'_2$.
  \end{enumerate}
\end{definition}

So when an element $d$ is \emph{lazy}, every input heap forces to the output
heap, it evaluates to the same value whether or not the input heap is more or
less forced, and it homomorphically transfers the forcedness relation from input
heaps to output heaps.

When we restrict ourselves to lazy heaps (\eg, heaps where every entry is a lazy
element), we can prove transitivity.

\begin{lemma}[$(\forcesto)$ is transitive]
  \label{thm:force-heap-trans}
  Let $μ_1,μ_2,μ_3$ be lazy heaps and $μ_1 \forcesto μ_2, μ_2 \forcesto μ_3$.
  Then $μ_1 \forcesto μ_3$.
\end{lemma}
\begin{proof}
  We need a termination measure on $(\forcesto)$ before we can proceed.
  Note that in successive steps $μ_1 \forcesto ... \forcesto μ_n$ where all
  $μ_i$ are lazy heaps, heap entries progress from being undefined,
  then to a non-value denotation, and finally to a value element $\ret(v)$.

  More formally, we can define a measure $m : \Heaps \to \dom(μ_n) \pfun \{0,1,2\}$
  \[
    m(μ)(\pa) = \begin{cases}
        0 & μ(\pa) = \ret(\wild) \\
        1 & \pa ∈ \dom(μ) \\
        2 & \text{otherwise} \\
      \end{cases}
  \]
  with the pointwise partial order on $\dom(μ_n) \pfun \{0,1,2\}$.
  When $\dom(μ_n)$ is finite, this partial order has finite height, which can
  thus serve as a termination measure.

  We proceed by well-founded induction with the above measure $m$ defined on
  $\dom(μ_3)$.
  More precisely, show that the proposition is satisfied for $μ_1,μ_2,μ_3$
  assuming that the proposition holds for any $μ_1',μ_2',μ_3'$
  such that $m(μ_1') - m(μ_3') < m(μ_1) - m(μ_3)$.

  We can see that $\dom(μ_1) ⊆ \dom(μ_2) ⊆ \dom(μ_3)$, proving the first property
  of \Cref{defn-force-heap}.
  For the second property, fix an arbitrary $\pa$.
  When $μ_1(\pa) = μ_2(\pa)$ and $μ_2(\pa) = μ_3(\pa)$, we have $μ_1(\pa) = μ_3(\pa)$.
  Otherwise, either $μ_1(\pa) \not= μ_2(\pa)$ or $μ_2(\pa) \not= μ_3(\pa)$.

  Suppose that $μ_1(\pa) \not= μ_2(\pa)$.
  Then there exists $v,μ'_1$ such that $\bigstep{μ_1(\pa)}{μ_1}{v}{μ'_1}$ and
  $μ_2(\pa) = \ret(v)$ as well as $μ'_1 \forcesto μ_2$.
  By \Cref{thm:force-heap-val} applied to $μ_2 \forcesto μ_3$, we have
  $μ_2(\pa) = μ_3(\pa) = \ret(v)$.
  If $μ_1' \not= μ_1$, then $m(μ_1') < m(μ_1)$ and by well-founded induction
  we can combine $μ'_1 \forcesto μ_2$ and $μ_2 \forcesto μ_3$ to
  $μ'_1 \forcesto μ_3$.
  Hence we have shown that $μ_1 \forcesto μ_3$.

  Suppose that $μ_1(\pa) = μ_2(\pa)$, but $μ_2(\pa) \not= μ_3(\pa)$.
  Then there exists $v,μ'_2$ such that $\bigstep{μ_2(\pa)}{μ_2}{v}{μ'_2}$ and
  $μ_3(\pa) = \ret(v)$ as well as $μ'_2 \forcesto μ_3$.
  We know that $μ_1(\pa) = μ_2(\pa)$, so $\bigstep{μ_1(\pa)}{μ_2}{v}{μ'_2}$.
  Since $μ_1(\pa)$ is lazy and $μ_1 \forcesto μ_2$, we get
  $\bigstep{μ_1(\pa)}{μ_1}{v}{μ'_1}$ such that $μ_1' \forcesto μ_2'$ by
  definition of laziness.
  By well-founded induction, we can combine $μ'_1 \forcesto μ_2'$ and
  $μ_2' \forcesto μ_3$ to $μ'_1 \forcesto μ_3$.
  If $μ_1' \not= μ_1$, then $m(μ_1') < m(μ_1)$ and by well-founded induction
  we can combine $μ'_1 \forcesto μ_2'$ and $μ_2' \forcesto μ_3$ to
  $μ'_1 \forcesto μ_3$.
  Hence have shown that $μ_1 \forcesto μ_3$.
\end{proof}

\begin{corollary}
  $(\forcesto)$ is a partial order on lazy heaps.
\end{corollary}

Another neat consequence of lazy elements is the following confluence property:

\begin{corollary}[Evaluating lazy elements is confluent]
  \label{thm:force-heap-confluent}
  Let $μ_1,μ_1',μ_2$ be lazy heaps, $d$ a lazy element and $v$ a value.
  If $\bigstep{d}{μ_1}{v}{μ_1'}$ and $μ_1 \forcesto μ_2$,
  then there exists a lazy heap $μ_2'$ such that $μ_1' \forcesto μ_2'$
  and $\bigstep{d}{μ_2}{v}{μ_2'}$. As a commutative diagram:
  \[
  \begin{tikzcd}
    μ_1 \arrow[Rightarrow]{d}[swap]{\langle d, \wild\rangle}{\langle v, \wild\rangle} \arrow[rightsquigarrow]{r}{} & μ_2 \arrow[Rightarrow]{d}{\langle d, \wild\rangle}[swap]{\langle v, \wild\rangle} \\
    μ_1' \arrow[rightsquigarrow]{r}{} & μ_2' \\
  \end{tikzcd}
  \]
\end{corollary}
\begin{proof}
  This is just the Postpone property in \Cref{defn:lazy-d}.
\end{proof}

And now we finally prove that $\semevt{\wild}$ indeed produces a lazy element:

\begin{theorem}
  \label{thm:semevt-lazy}
  For all lazy environments $ρ$ and expressions $\pe$, $\semevt{\pe}_ρ$ is a lazy element.
\end{theorem}
\begin{proof}
  By induction on $\pe$.
  \begin{itemize}
    \item \textbf{Case $\px$}: $ρ(\px)$ is lazy, and
      $\fn{\wild}{\stuckend{}}$ is trivially so as it never produces a balanced
      trace.
    \item \textbf{Case $\Lam{px}{\pe'}, K~\many{\px}$}:
      For any lazy $μ$, we have $\bigstep{\semevt{\pe}_ρ}{μ}{v}{μ}$, where
      $v$ is a $\FunV$ or $\ConV$ value which is independent of $μ$.
      Clearly, $μ \forcesto μ$ by reflexivity.
      The Postpone and Speculate properties follow by assumption.
    \item \textbf{Case $\pe'~\px$}:
      First note that $d \betastep f$ is lazy when $d$ is lazy and $f(v)$ is
      lazy when defined.
      This result makes use of transitivity in the first case of $(\betastep)$.

      For any lazy $μ$, we have $\bigstep{\semevt{\pe}_ρ}{μ}{v}{μ}$, where
      $v$ is a $\FunV$ or $\ConV$ value which is independent of $μ$.
      Clearly, $μ \forcesto μ$ by reflexivity.
      The Postpone and Speculate properties follow by assumption.
  \end{itemize}
\end{proof}

\section{Abstract Interpretation}
\label{sec:abstractions}

\subsection{Lazy Denotational Deadness}

Let us now finally try to reformulate semantic deadness in terms of
$\semevt{\wild}$ and $\equiv$:

\begin{definition}[Denotational deadness, lazily]
  \label{defn:deadness3}
  An address $\pa$ is \emph{dead} in a denotation $d$ if and only if,
  for all $μ_1 \approx μ_2$ such that $\pa$ is dead in $\rng(μ_i)$,
  $\dom(μ_1) ∪ \{\pa\} ⊦ d$ and $d_1,d_2$, we have
  \[
    \dom(μ_1) ⊦ d(μ_1[\pa↦d_1]) \sim d(μ_2[\pa↦d_2]).
  \]
  A variable $\px$ is \emph{dead} in an expression $\pe$ if and only if
  any $\pa$ is dead in $\semevt{\pe}_{ρ[\px↦\pa]}$ for any $ρ$ such
  that $\pa \not∈ \rng(ρ)$.
  Otherwise, $\px$ is \emph{live}.
\end{definition}

\begin{lemmarep}[Deadness implies irrelevance]
  If $\px$ is dead in $\pe$,
  then for all $ρ, \pe_1,\pe_2$,
  \[\semevt{\Let{\px}{\pe_1}{\pe}}_{ρ} \equiv \semevt{\Let{\px}{\pe_2}{\pe}}_{ρ}.\]
\end{lemmarep}
\begin{proof}
  Assume that $\pa$ is dead in $\semevt{\pe}_{ρ[\px↦\pa]}$ for any
  $ρ$ such that $\pa \not∈ \rng(ρ)$.
  To show that
  \[
    \semevt{\Let{\px}{\pe_1}{\pe}}_ρ \equiv \semevt{\Let{\px}{\pe_1}{\pe}}_ρ,
  \]
  we assume that $μ_1 \approx μ_2$ for arbitrary $μ_1,μ_2$ (satisfying
  well-addressedness constraints) and show that
  \[
    \dom(μ_1) ⊦ \semevt{\Let{\px}{\pe_1}{\pe}}_{ρ}(μ_1) \sim \semevt{\Let{\px}{\pe_2}{\pe}}_{ρ}(μ_2)
  \]
  We unfold the $\mathbf{let}$ case of $\semevt{\wild}$ and show
  \[
    \dom(μ_1) ⊦ \semevt{\pe}_{ρ[\px↦\pa]}(μ_1[\pa↦d_1]) \sim \semevt{\pe}_{ρ[\px↦\pa]}(μ_2[\pa↦d_2])
  \]
  For arbitrary $d_1,d_2$ (which we may set to
  $d_i \triangleq \memo(\pa,\semevt{\pe_i}_{ρ[\px↦\pa]})$).
  But $\pa \not∈ \rng(ρ)$ due to \Cref{thm:well-addressedness},
  so $\pa$ is dead in $\semevt{\pe}_{ρ[\px↦\pa]}$ per assumption.
  Hence we can show the goal, applying to $μ_1 \approx μ_2$.
\end{proof}

So if $x$ is dead in $\pe_2$, we can justify the following rewrite by
irrelevance:
\[
  \Let{x}{\pe_1}{\pe_2} \equiv \Let{x}{\mathit{panic}}{\pe_2}
\]
A syntactic \emph{occurrence analysis} could subsequently figure out whether the
binding for $x$ can be dropped without introducing scoping errors, a feat
that becomes far simpler once huge expressions $\pe_1$ are turned into small
$\mathit{panic}$s.

We can now prove \Cref{thm:semusg-correct-live} in terms of this new
characterisation of deadness by induction:

\begin{toappendix}
\begin{lemma}[Heap partitioning]
  \label{thm:heap-partitioning}
  Let $μ$ be a heap and $\pa ∈ \dom(μ)$.
  Then there exists a heap $μ_h$ with domain $\dom(μ_h) \setminus \{ \pa \}$
  and $μ \approx μ_h[\pa↦μ(\pa)]$.
\end{lemma}
\begin{proof}
  Define $μ_h \triangleq \mathit{hide}_\pa(μ)$, where
  \[\begin{array}{rcl}
    \mathit{hide}_\pa(μ)(\pa')(μ') & = & \begin{cases}
      μ(\pa')(μ'[\pa↦μ(\pa)]) & \pa' ∈ \dom(μ) \setminus \{ \pa \} \\
      \mathit{undefined} & \text{otherwise} \\
    \end{cases}
  \end{array}\]
  We abbreviate $μ_h \triangleq \mathit{hide}_\pa(μ)$.
  Clearly, $\dom(μ_h) = \dom(μ) \setminus \{ \pa \}$.

  By Löb induction, we prove
  \[
    μ_h \text{ is lazy and extensional} \wedge μ \approx μ_h[\pa↦μ(\pa)]
  \]
  So we assume that the property holds later.

  Consider $d \triangleq μ_h(\pa') = \fn{μ'}{μ(\pa')(μ'[\pa↦μ(\pa)])}$ for some $\pa' ∈ \dom(μ_h)$.
  Then $d(μ_h) = μ(\pa')(μ_h[\pa↦μ(\pa)])$ which is defined, because
  $μ(\pa')(μ)$ is defined and $μ(\pa')$ is extensional, which we can apply to
  $μ \approx μ_h[\pa↦μ(\pa)]$ later.

  Furthermore, $d$ is lazy for a similar argument that the $\mathbf{let}$ case
  of \Cref{thm:semevt-lazy} is.
  Extensionality follows because the heap extension $[\pa↦μ(\pa)]$ in $d$ is
  always with the same (extensional) element.

  For $μ \approx μ_h[\pa↦μ(\pa)]$, we unfold $\mathit{hide}_\pa$, fix
  $\pa' ∈ \dom(μ)$ and show
  \[
    \dom(μ) ⊦ μ(\pa')(μ) \sim μ(\pa')(μ_h[\pa↦μ(\pa)])
  \]
  Which can be shown by extensionality of $μ(\pa')$ applied to the induction
  hypothesis.
\end{proof}

\begin{lemma}
  \label{thm:adom-dead}
  Let $A ⊦ d$. Then any $\pa \not∈ A$ is dead in $d$.
\end{lemma}
\begin{proof}
  Assume $μ_1 \approx μ_2$ such that $\pa$ is dead in $\rng(μ_i)$ and $\dom(μ_1) ∪ \{\pa\} ⊦ d$
  and fix arbitrary $d_1,d_2$.
  We have to show that
  \[
    \dom(μ_1) ⊦ d(μ_1[\pa↦d_1]) \sim d(μ_2[\pa↦d_2])
  \]

  We assume that $A ⊆ \dom(μ_i)$.
  Otherwise, we can extend $μ_i$ to heaps $μ_i'$ (mapping to equal entries, for
  example $d$) such that $A ⊆ \dom(μ_i')$.
  Then $μ_i \forcesto μ_i'$ and we can apply \Cref{thm:lazy-force-bisimilar}.

  W.l.o.g., $A ⊆ \dom(μ_i)$.
  By \Cref{thm:heap-partitioning}, there exist $μ_i'$ such that $μ_i[\pa↦d_i] \approx μ_i'[\pa↦d_i]$
  and $\pa \not∈ \dom(μ_i')$, so $μ_i' \forcesto μ_i'[\pa↦d_i]$.
  By extensionality of $d$, we know that
  \[
    \dom(μ_1') ⊦ d(μ_1') \sim d(μ_2')
  \]
  Since $A ⊆ \dom(μ_i')$ and $A ⊦ d$, we can apply \Cref{thm:lazy-force-bisimilar}, getting
  \[
    \dom(μ_1') ⊦ d(μ_1'[\pa↦d_i]) \sim d(μ_2'[\pa↦d_i])
  \]
  And again by extensionality applied to $μ_i[\pa↦d_i] \approx μ_i'[\pa↦d_i]$
  and transitivity, we show the goal.
\end{proof}
\end{toappendix}

\begin{theoremrep}[$\semusg{\wild}$ is a correct deadness analysis]
  \label{thm:semusg-correct-live-3}
  Let $\pe$ be an expression, $\px$ a variable and $\tr$ a usage environment.
  If $\tr(\px) \not⊑ \semusg{\pe}_{\tr}$
  then $\px$ is dead in $\pe$.
\end{theoremrep}
\begin{proof}
  We fix $\pe$, $\px$ and $\tr$ such that $\tr(\px) \not⊑ \semusg{\pe}_{\tr}$.
  The goal is to show that $\px$ is dead in $\pe$.
  So we assume that an address $\pa$ is dead in the range of an
  environment $ρ$ and show that $\pa$ is dead in
  $\semevt{\pe}_{ρ[\px↦\pa]}$.

  We assume that $μ_1 \approx μ_2$ such that $\pa$ is dead in $\rng(μ_i)$, $\dom(μ_1) ∪ \{\pa\} ⊦ \semevt{\pe}_{ρ[\px↦\pa]}$ and
  abbreviate $A \triangleq \dom(μ_1)$. Then the goal is to show
  \[
    A ⊦ \semevt{\pe}_{ρ[\px↦\pa]}(μ_1[\pa↦d_1]) \sim \semevt{\pe}_{ρ[\px↦\pa]}(μ_2[\pa↦d_2])
  \]
  for any $d_1,d_2$ and $ρ$ such that $\pa$ is dead in $\rng(ρ)$.
  We proceed by induction on $\pe$.
  \begin{itemize}
    \item \textbf{Case $\pe = \py$}: If $\px=\py$, then
      $\semusg{\py}_{\tr} = \semusg{\px}_{\tr}$, contradicting the inequality.
      If $\px \not= \py$, then $\pa$ is dead in $\semevt{\py}_{ρ[\px↦\pa]}=ρ(\py)$.

    \item \textbf{Case $\pe = \Lam{\py}{\pe'}$}:
      We abbreviate
      $μ_i' \triangleq μ_i[\pa↦d_i]$, $ρ' \triangleq ρ[\px↦\pa]$
      and have to show
      \[
        A ⊦ \goodend{\FunV(\fn{d}{\semevt{\pe'}_{ρ'[\py↦d]}}), μ_1'} \sim \goodend{\FunV(\fn{d}{\semevt{\pe'}_{ρ'[\py↦d]}}), μ_2'},
      \]
      which requires us to show
      \[
        \forall d.\ \later (A ⊦ d) ⟹  A ⊦ \semevt{\pe'}_{ρ'[\py↦d]}(μ_1') \sim \semevt{\pe'}_{ρ'[\py↦d]}(μ_2')
      \]
      Crucially, the address restriction on $d$ means that $\pa$ is dead in $d$
      per \Cref{thm:adom-dead}, so the proof is a simple matter of applying the
      inductive hypothesis.

    \item \textbf{Case $\pe = \pe'~\py$}:
      $\pa$ is trivially dead in the stuck case, so we assume that there exists
      $d_\py \triangleq ρ(\py)$ and $\pa$ is dead in $d_\py$ per assumption.
      We unfold $\apply$ and abbreviate
      $d_{\pe'} \triangleq \semevt{\pe'}_{ρ[\px↦\pa]}$,
      $k \triangleq \fn{(\FunV(f))}{f(d_\py)}$, so the goal is to prove
      \[
        A ⊦ (d_{\pe'} \betastep k)(μ_1[\pa↦d_1]) \sim (d_{\pe'} \betastep k)(μ_2[\pa↦d_2])
      \]
      Since $d_{\pe'} \equiv d_{\pe'}$, the only interesting case is when
      $\bigstep{d_{\pe'}}{μ_i[\pa↦d_i]}{\FunV(f_i)}{μ_i'}$, in which case we
      have to show that
      \[
        A ⊦ f_1(d_\py)(μ_1') \sim f_2(d_\py)(μ_2').
      \]
      But $A ⊦ d_\py$ by \Cref{thm:well-addressedness}, so we can apply the
      premise of the $\eqfun$ judgment at the end of the $d_{\pe'}$ trace.

    \item \textbf{Case $\pe = \Let{\py}{\pe_1}{\pe_2}$}:
      We abbreviate
      ${ρ' \triangleq ρ[\px↦\pa]}$,
      $d_{\pe_i} \triangleq \semevt{\pe_i}_{ρ'[\py↦\pa']}$,
      ${μ_i' \triangleq μ_i[\pa↦d_i]}$,
      so the goal is to prove
      \[
        A ⊦ d_{\pe_2}(μ_1[\pa↦d_1][\pa'↦d_{\pe_1}]) \sim d_{\pe_2}(μ_2[\pa↦d_2][\pa'↦d_{\pe_1}])
      \]
      where $\pa' \not∈ \dom(μ_i)$ and $\pa' \not= \pa$.

      Since there can be no shadowing, it is $\px \not= \py$ and we see that
      $\tr(\px) = \tr'(\px)$, where
      $\tr' \triangleq {\operatorname{fix}(\fn{\tr'}{\tr ⊔ [\py ↦
      [\py↦1]+\semusg{\pe_1}_{\tr'}]})}$.
      We know that
      \[
        \tr'(\px) = \tr(\px) \not⊑ \semusg{\pe}_{\tr} = \semusg{\pe_2}_{\tr'}
      \]
      but we cannot apply the induction hypothesis yet, because
      $\pa$ might not be dead in $d_{\pe_1}$ and thus not in
      $μ_i[\pa'↦d_{\pe_1}]$.

      We proceed by cases over $\tr(\px) = \tr'(\px) ⊑ \semusg{\pe_1}_{\tr'}$.
      \begin{itemize}
        \item \textbf{Case $\tr'(\px) ⊑ \semusg{\pe_1}_{\tr'}$}: Then
          $\tr'(\px) ⊑ \tr'(\py)$ and $\py$ is also dead in $\pe_2$ by the above
          inequality.
          Both deadness facts together allow us to rewrite $d_{\pe_1}$ to
          $d_{\pe_1}' \triangleq \semevt{\Lam{x}{x}}_{[]}$, and $\pa$ is dead in that.
          Then we apply the induction hypothesis to the heaps
          $μ_i[\pa'↦d_{\pe_1}']$ (noting that $μ_1[\pa'↦d_{\pe_1}'] \approx
          μ_2[\pa'↦d_{\pe_1}']$ and that $\pa$ is dead in the range of both) and
          then rewrite back to $d_{\pe_1}$:
          \[\begin{array}{rcl}
            A & ⊦    & \semevt{\pe_2}_{ρ'[\py↦\pa']}(μ_1[\pa'↦d_{\pe_1}]) \\
              & \sim & \semevt{\pe_2}_{ρ'[\py↦\pa']}(μ_1[\pa'↦d_{\pe_1}']) \\
              & \sim & \semevt{\pe_2}_{ρ'[\py↦\pa']}(μ_2[\pa'↦d_{\pe_1}']) \\
              & \sim & \semevt{\pe_2}_{ρ'[\py↦\pa']}(μ_2[\pa'↦d_{\pe_1}]) \\
          \end{array}\]
          as requested.
        \item \textbf{Case $\tr'(\px) \not⊑ \semusg{\pe_1}_{\tr'}$}:
          Then $\pa$ is dead in $d_{\pe_1}$ and thus in the range of
          $μ_i[\pa'↦d_{\pe_i}']$ and we may invoke the induction hypothesis
          directly to show the goal.
      \end{itemize}
  \end{itemize}
\end{proof}

\subsection{Discussion}

The main proof of \Cref{thm:semusg-correct-live-3} is hardly longer than
\Cref{thm:semusg-correct-live}, but we have to admit that we needed to prove
quite a few metatheoretic properties to get there, so we declare only partial
victory on Goal 1 from \Cref{sec:problem}.
For obvious reasons, it seems preferable to stick to a simpler call-by-name
semantics without a heap if the property in question (\eg, deadness) can be
understood there as well.

Still, with \Cref{defn:deadness3} we were at least able to break
down the proof into manageable intermediate steps.
That is a huge step forward compared to the operational deadness definition of
\Cref{defn:deadness2} where we weren't even able to come up with a suitable
correctness relation.

In hindsight, we could trace back the intermediate steps to come up with at
least one proof strategy with \Cref{defn:deadness2}:
The structural induction principle is an important enabling factor and
the focus on maximal traces in \Cref{fig:semvan-correctness} suggests that
we should likely strive for a correctness relation characterising a property of
maximal traces.
This was not obvious to us when we first set out to do the proof; the gap was
too large to see how to get to the other side due to the structural mismatch.
A nice consequence of successfully avoiding structural mismatch, which we set
out to in Goal 4.

\subsection{Improvement}

Proving that dead bindings can be soundly rewritten is a nice litmus test
for the semantics.
But does it really make the program faster, or at least not slower?

We can affirm that indirectly:
A diverging program $\mathit{loop}$ takes more steps than a stuck program
$\mathit{panic}$, hence the former runs ``slower'' than the latter.
The stuck and the diverging program are not semantically equivalent,
but $\Let{x}{\mathit{panic}}{\pe}$ is semantically equivalent to
$\Let{x}{\mathit{loop}}{\pe}$ when $x$ is dead in $\pe$.
Since $(\betastep)$ runs sub-programs to completion, we would observe
execution of ${\mathit{panic}}$ or ${\mathit{loop}}$, which is the only
way in which we could have made the program slower or faster.
Since there is no semantic difference, the performance of the program must be
unaffected.

This argument is quite vague.
In order to put it on firm ground, we define an \emph{improvement relation}
$(\faster)$ in the style of \citet{MoranSands:99} in \Cref{fig:improv}.

\begin{figure}
\[\begin{array}{c}
 \ruleform{ A ⊦_n τ_1 \lesssim τ_2 \qquad μ_1 \lessapprox μ_2 \qquad d_1 \faster d_2 }
 \\
 \\[-0.75em]
 \inferrule*[right=\implrcons]
    {\later (A ⊦_n τ_1 \lesssim τ_2)}
    {A ⊦_{n} \wild \cons τ_1 \lesssim \wild \cons τ_2}
 \quad
 \inferrule*[right=\imprcons]
    {A ⊦_{n+1} τ_1 \lesssim τ_2}
    {A ⊦_{n} τ_1 \lesssim \wild \cons τ_2}
 \quad
 \inferrule*[right=\implcons]
    {A ⊦_{n} τ_1 \lesssim τ_2}
    {A ⊦_{n+1} \wild \cons τ_1 \lesssim τ_2}
 \\
 \\[-0.75em]
 \inferrule*[right=\impstuck]
    {\quad}
    {A ⊦_{0} \stuckend{} \lesssim \stuckend{}}
 \quad
 \inferrule*[right=\impcon]
    {\many{A ⊦_{n} μ_1(\pa_1) \lesssim μ_2(\pa_2)})}
    {A ⊦_{\Sigma \{\many{n}\}} \goodend{\ConV(K, \many{\pa_1}), μ_1} \lesssim \goodend{\ConV(K, \many{\pa_2}), μ_2}}
 \\
 \\[-0.75em]
 \inferrule*[right=\impfun]
    {\forall \pa.\ \pa ∈ A ⟹  A ⊦_{n} f_1(\pa)(μ_1) \lesssim f_2(\pa)(μ_2)}
    {A ⊦_{n} \goodend{\FunV(f_1), μ_1} \lesssim \goodend{\FunV(f_2), μ_2}}
 \\
 \\[-0.75em]
 \inferrule*[right=\impheap]
    {\dom(μ_1) = \dom(μ_2) \quad \forall \pa.\ \later(\dom(μ_1) ⊦_{0} μ_1(\pa)(μ_1) \lesssim μ_2(\pa)(μ_2))}
    {μ_1 \lessapprox μ_2}
 \\
 \\[-0.75em]
 \inferrule*[right=\impdenot]
    {\forall μ_1,μ_2.\ μ_1 \lessapprox μ_2 \wedge  \dom(μ_1) ⊦ d_1,d_2 ⟹  \dom(μ_1) ⊦_{0} d_1(μ_1) \lesssim d_2(μ_2)}
    {d_1 \faster d_2}
\end{array}\]
\vspace{-1em}
\caption{Improvement relation}
  \label{fig:improv}
\end{figure}

Structurally, $(\faster)$ is very similar to $(\equiv)$; the main differences are in
the rules for $(\cons)$ and the resulting tracking of \emph{skew credits} $n∈ℕ$,
\eg, we count the applications of $\imprcons$ to spend them on $\implcons$ or
when a value is further scrutinised.
Naturally, it is easier to prove $A ⊦_n τ_1 \lesssim τ_2$ the more credits we
have at our expense and thus the larger $n$ is.

We write $d_1 \lockstep d_2$ when $d_1 \faster d_2$ and $d_2 \faster d_1$, in
which case both denotations operate in lockstep.
%Already it is the case that $d_1 \faster d_2$ implies $d_1 \equiv d_2$,
%so $(\faster)$ corresponds to the strong notion of improvement in
%\citet{MoranSands:99}.
%The weaker notion can be recovered by defining $\imprcons$ and $\implcons$ by
%coinduction, thus accepting skew credits in $ℕ_ω$ instead of $ℕ$.
%
We conjecture that $(\faster)$ is a sub-relation of the strong contextual
improvement relation of \citet{MoranSands:99}, just as $(\equiv)$ is finer than
contextual equivalence.

We could now once again refine our notion of deadness, using lockstep
simulation.
It is then easy to see that $\semusg{\wild}$ is correct \wrt to this stronger
notion of deadness, because we can in large parts reuse the proof for
\Cref{thm:semusg-correct-live-3}.

%\begin{definition}[Denotational deadness, improving]
%  \label{defn:deadness4}
%  An address $\pa$ is \emph{dead} in a denotation $d$ if and only if,
%  for all $μ_1 \lessapprox\!\gtrapprox μ_2$ such that $\pa$ is dead in $\rng(μ_i)$,
%  $\dom(μ_1) ∪ \{\pa\} ⊦ d$ and $d_1,d_2$, we have
%  \[
%    \dom(μ_1) ⊦_0 d(μ_1[\pa↦d_1]) \lesssim\!\gtrsim d(μ_2[\pa↦d_2]).
%  \]
%  A variable $\px$ is \emph{dead} in an expression $\pe$ if and only if
%  any $\pa$ is dead in $\semevt{\pe}_{ρ[\px↦\pa]}$ for any $ρ$ such
%  that $\pa \not∈ \rng(ρ)$.
%  Otherwise, $\px$ is \emph{live}.
%\end{definition}
%
%And we will now assume that we have proven $\semusg{\wild}$ correct \wrt to this
%new notion of deadness:
%\begin{theorem}[$\semusg{\wild}$ is an improving deadness analysis]
%  \label{thm:semusg-correct-live-4}
%  Let $\pe$ be an expression, $\px$ a variable and $\tr$ a usage environment.
%  If $\tr(\px) \not⊑ \semusg{\pe}_{\tr}$
%  then $\px$ is dead in $\pe$.
%\end{theorem}
%
%The proof is much the same as \Cref{thm:semusg-correct-live-3}, because
%intuitively, any derivation of $A ⊦ τ_1 \sim τ_2$ can be rewritten to never use
%$\eqlcons$ and $\eqrcons$.
%Such a derivation can be directly transformed into a proof for
%$A ⊦_0 τ_1 \lesssim τ_2$, as we will never need $\implcons$ and $\imprcons$
%and skew credits remain at $0$.

\subsection{Evaluation Cardinality}

Since our new semantics is able to express evaluation cardinality and thunk
update, we may add a new $\mathbf{let1}$ construct to our language that opts out
of memoisation:
\[
 \begin{array}{rcl}
  ε ∈ \Events   & ::= & ... \mid \BindOE(\px,\pa↦d) \\
  \\[-0.5em]
  \semevt{\Letn{\px}{\pe_1}{\pe_2}}_ρ(μ) & = &
    \begin{letarray}
      \text{let} & ρ' = ρ[\px ↦ \pa] \quad \text{where $\pa \not∈ \dom(μ)$} \\
      \text{in}  & \BindOE(\px,\pa↦\semevt{\pe_1}_{ρ'}) \cons \semevt{\pe_2}_{ρ'}(μ[\pa ↦ \highlight{\semevt{\pe_1}_{ρ'}}])
    \end{letarray} \\
 \end{array}
\]
Any program in which we switch from memoised $\mathbf{let}$ to $\mathbf{let1}$
is semantically equivalent after we adjust the definition of lazy heaps
accordingly.
This is a simple consequence of the fact that $\memo(\pa,d) \equiv d$
and compositionality.

However, omitting thunk memoisation has measurable effect on performance
if the same variable is evaluated repeatedly!
We should rather show that whenever $x$ is \emph{evaluated at most
once}, it is an improvement to rewrite $\Let{x}{\pe_1}{\pe_2}$ to
$\Letn{x}{\pe_1}{\pe_2}$.

We can sharpen this statement by making use of \emph{tick algebra}
\citep{MoranSands:99}.
For that, we need to add a notion of ticks to
our language (postfix, to mirror $\memo$), a routine extension:
\[
 \begin{array}{rcl}
  ε ∈ \Events   & ::= & ... \mid \TickE \\
  \\[-0.5em]
  \semevt{\pe \tick}_ρ(μ) & = & \semevt{\pe}_{ρ}(μ) \betastep \fn{v}{\fn{μ}{\TickE \cons \goodend{v,μ}}} \\
 \end{array}
\]
Whenever $x$ is ``evaluated at most once'', we
have $\semevt{\Let{x}{\pe_1}{\pe_2}}_ρ \lockstep
      \semevt{\Letn{x}{\pe_1\tick}{\pe_2}}_ρ$.
In \Cref{sec:problem}, specifically \Cref{thm:semusg-correct-2}, we have
proclaimed that $x$ is evaluated at most once whenever
$\semusg{\pe_2}_{\tr_Δ}(x) ⊑ 1$.
We can now make good on that claim:

\begin{toappendix}
\begin{figure}
\[\begin{array}{c}
 \ruleform{ \correct_L(τ_1,τ_2) }
 \\
 \\[-0.5em]
 \inferrule*[right=\corstuck]
    {\quad}
    {\correct_L(\stuckend{},\stuckend{})}
 \quad
 \inferrule*[right=\corcons]
    {\later (\correct_L(τ_1,τ_2))}
    {\correct_L(\wild :: τ_1,\wild :: τ_2)}
 \\
 \\[-0.5em]
 \inferrule*[right=\corfun]
    {\dom(μ_1) = \dom(μ_2) \quad \forall \pa ∈ (\dom(μ_i) \setminus L).\ \correct_L(f_1(\pa)(μ_1), f_2(\pa)(μ_2))}
    {\correct_L(\goodend{\FunV(f_1),μ_1},\goodend{\FunV(f_2),μ_2})}
 \\
 \\[-0.5em]
\end{array}\]
\caption{Correctness relation for \Cref{thm:semusg-by-name}}
  \label{fig:semusg-correct}
\end{figure}

\begin{lemma}
  \label{thm:pe1-dead}
  If $ω*\tr(\px) \not⊑ \semusg{\Let{\px}{\pe_1}{\pe_2}}_{\tr}$, then $\px$ is
  dead in either $\pe_1$ or $\pe_2$.
\end{lemma}
\begin{proof}
  Unfold $\semusg{\wild}$ once and consider the case that $\px$ is not dead in $\pe_1$,
  so we have $\tr(\px) ⊑ \semusg{\pe_1}_{\tr}$.

  We have
  $ω*\tr_1(\px) \not⊑ \semusg{\pe_2}_{\tr_1}$ where
  $\tr_1 \triangleq \fix(\fn{\tr_1}{\tr ⊔ [\px ↦ [\px↦1]+\semusg{\pe_1}_{\tr_1}]})$.

  Clearly, $\tr_1(\px) = ω$, so
  $\tr_1(\px) \not⊑ \semusg{\pe_2}_{\tr_1}$, so $\px$ must be dead in $\pe_2$.
\end{proof}

\begin{definition}
  $d$ is \emph{$L$-once} if and only if, for all $μ$ such that $\rng(μ)$ is
  $L$-once, $L$ is dead in $μ(\dom(μ) \setminus L)$ and for all $\pa ∈ L$ such
  that $\pa$ is dead in $d'$,
  \[
    \correct_{L}(d(μ[\pa↦ d'\tick]), d(μ[\pa↦\memo(\pa,d')])).
  \]
  (Where $(d'\tick )(μ) \triangleq d'(μ) \betastep \fn{v}{\fn{μ}{\TickE \cons \goodend{v,μ}}}$.)
\end{definition}
\end{toappendix}

\begin{theoremrep}[Update avoidance for $\semusg{\wild}$]
  \label{thm:semusg-by-name}
  Let $\Let{\px}{\pe_1}{\pe_2}$ be an expression
  such that \\
  ${\semusg{\pe_1}_{\tr_Δ}(\px) + \semusg{\pe_2}_{\tr_Δ}(\px) ⊑ 1}$.
  Then
    $\semevt{\Let{\px}{\pe_1}{\pe_2}}_ρ \lockstep
     \semevt{\Letn{\px}{\pe_1\tick}{\pe_2}}_ρ$.
\end{theoremrep}
\begin{proof}
  Unless explicitly stated otherwise, we will always apply the improvement
  judgments in a symmetrical manner.
  That is, whenever we have to prove $a ⊑ b$, the proof should also hold for
  $b ⊑ a$ (where $(⊑)$ is either of $(\faster),(\lessapprox),(\lesssim)$).

  Since $u_1 + u_2 ⊑ 1$ implies that either $u_i$ must $0$ and the other $1$,
  either $\px$ must be dead in $\pe_1$ or $\pe_2$.
  If $\px$ is dead in $\pe_2$, the rewrite follows by deadness.
  So we assume that $\px$ is dead in $\pe_1$.

  We prove $(\lockstep)$ by rule $\impdenot$.
  After unfolding the definitions of $\semevt{\wild}$ for $\mathbf{let}$ and
  $\mathbf{let1}$, the goal is to show
  \[
    \dom(μ_i) ⊦_0 \semevt{\pe_2}_{ρ[\px↦\pa]}(μ_1[\pa↦d_1]) \lesssim\!\gtrsim \semevt{\pe_2}_{ρ[\px↦\pa]}(μ_2[\pa↦d_2])
  \]
  where $\pa \not∈\dom(μ_i)$,
  $\dom(μ_i) ∪ \{\pa\} ⊦ \semevt{\pe_2}_{ρ[\px↦\pa]}$,
  $μ_1 \lessapprox\!\gtrapprox μ_2$.
  Clearly,
  \[
    \dom(μ_i) ⊦_0 \semevt{\pe_2}_{ρ[\px↦\pa]}(μ_1[\pa↦\highlight{d_2}]) \lesssim\!\gtrsim \semevt{\pe_2}_{ρ[\px↦\pa]}(μ_2[\pa↦d_2])
  \]
  holds by reflexivity because $μ_1[\pa↦d_2] \lessapprox\!\gtrapprox μ_2[\pa↦d_2]$, so the goal becomes
  \[
    \dom(μ_i) ⊦_0 τ_1 \lesssim\!\gtrsim τ_2,
  \]
  abbreviating $τ_i \triangleq \semevt{\pe_2}_{ρ[\px↦\pa]}(μ_{\highlight{\scriptstyle 1}}[\pa↦d_i])$.

  \noindent
  We will prove the correctness predicate $\correct_{\{\pa\}}(τ_1,τ_2)$ instead, where
  $\correct$ is defined in \Cref{fig:semusg-correct}.

  Since $\pa \not ∈ \dom(μ_1)$,
  $\corfun$ implies $\impfun$ on $\dom(μ_i)$ and the goal follows.

  We generalise our assumption about $\semusg{\wild}$ as follows:
  There exists a $\tr$ (\ie, $\tr_Δ$) such that
  $ω*\tr(\px) \not⊑ \semusg{\Let{\px}{\pe_1}{\pe_2}}_{\tr}$.

  Then
  $ω*\tr_1(\px) \not⊑ \semusg{\pe_2}_{\tr_1}$ where
  $\tr_1 \triangleq \fix(\fn{\tr_1}{\tr ⊔ [\px ↦ [\px↦1]+\semusg{\pe_1}_{\tr_1}]})$.
  Let us also abbreviate $ρ_1 \triangleq ρ[\px↦\pa]$.

  We show that $\semevt{\pe_2}_{ρ[\px↦\pa]}$ is $\{\pa\}$-once
  assuming that $ω*\tr_1(\px) \not⊑ \semusg{\pe_2}_{\tr_1}$.
  By induction on $\pe_2$, generalising to $L$-onceness and
  maintaining that $\tr_1(\px) \not⊑ \tr_1(\py) \Longleftrightarrow ρ(\py) \not∈ L$
  for all $\py∈ \dom(ρ)$.
  \begin{itemize}
    \item \textbf{Case $\pe_2 = \py$}:
      Fix $μ$ such that $\rng(μ)$ is $L$-once and for all $\pa' ∈ L$ such that $\pa'$ is dead in $d'$,
      we need to show that
      \[
        \correct_{L}(μ_1(ρ(\py))(μ_1), μ_2(ρ(\py))(μ_2)).
      \]
      where $μ_1 \triangleq μ[\pa'↦d'\tick ], μ_2 = μ[\pa'↦\memo(\pa',d')]$.
      This is easy to see when $ρ(\py) \not= \pa'$, so assume $ρ(\py) = \pa'$:
      \[
        \correct_{L}(d'(μ_1) \cons \TickE, d'(μ_2) \betastep \fn{v}{(\fn{μ'}{\UpdateE(...) \cons \goodend{v,μ'[\pa'↦\memo(\pa',\ret(v))]}})}).
      \]
      We can strip away the trailing transition, leaving behind just the heap update.

      Now, without loss of generality, we will assume that $L$ is dead in
      $\rng(μ)$.
      If that was not the case, we could apply \Cref{heap-partitioning}
      to hide all entries in which $L$ is live ``inside $d'$'' and allocate them
      on first execution.
      % Turn into lemma. Sounds useful

      When $L$ is dead in $\rng(μ)$, we can rewrite the entry for $\pa'$ freely
      without affecting $\correct$.
      In the interesting case we have $\bigstep{d'}{μ_i}{\FunV(f_i)}{μ_i'}$.
      Because of deadness, we must have
      \[
        \correct_{L}(d'(μ_1), d'(μ_2)).
      \]
      That implies
      \[
        \correct_{L}(f_1(\pa_a)(μ_1'), f_2(\pa_a)(μ_2')).
      \]
      for all $\pa_a ∈ \dom(μ_i') \setminus L$.
      Now it suffices to show the situation after heap update
      \begin{equation}
        \label{eqn:usg-var-goal}
        \correct_{L}(f_2(\pa_a)(μ_2'), f_2(\pa_a)(μ_2'[\pa'↦\memo(\pa',\ret(\FunV(f_2)))])).
      \end{equation}
      In particular, note that
      $μ_2 \forcesto μ_3 \triangleq μ[\pa'↦\memo(\pa',\ret(\FunV(f_2)))]$
      (again, we can properly hide any free address that $f_2$ needs in
      $μ_3(\pa')$), so $\bigstep{d'}{μ_3}{\FunV(f_2)}{μ_3'}$ with
      $μ_2' \forcesto μ_3'$ by the Postpone property.
      By deadness we can follow
      \[
        \correct_{L}(d'(μ_2), d'(μ_3)).
      \]
      and since $\bigstep{d'}{μ_3}{\FunV(f_2)}{μ_3'}$ (NB: $f_2$ due to the
      forcing relationship), that implies
      \[
        \correct_{L}(f_2(\pa_a)(μ_2'), f_2(\pa_a)(μ_3')).
      \]
      for all $\pa_a ∈ \dom(μ_i') \setminus L$.

      Due to forcing, it must be that $μ_3'(\pa) = μ_3(\pa)$.
      Again by forcing, $μ_2'(\pa)$ must be either
      $μ_2(\pa)$ or $μ_3'(\pa)$.

      Since the number of steps taken for $f_2(\pa_a)(μ_2'[\pa↦μ_3'(\pa)])$ must be between $f_2(\pa_a)(μ_2')$
      and $f_2(\pa_a)(μ_3')$ (due to forcing), this shows the goal \Cref{eqn:usg-var-goal}.

    \item \textbf{Case $\pe_2 = \pe~\py$}:
      Fix $μ$ such that $\rng(μ)$ is $L$-once and for all $\pa' ∈ L$ such that $\pa'$ is dead in $d'$,
      we need to show that
      \[
        \correct_{L}(\semevt{\pe~\py}_{ρ_1}(μ[\pa'↦ d'\tick]), \semevt{\pe~\py}_{ρ_1}(μ[\pa'↦\memo(\pa',d')])).
      \]

      From $ω*\tr_1(\px) \not⊑ \semusg{\pe}_{\tr_1} + ω*\tr_1(\py)$ we can see that
      $ω*\tr_1(\px) \not⊑ \semusg{\pe}_{\tr_1}$ and $\tr_1(\px) \not⊑ \tr_1(\py)$ by
      monotonicity of $+$ and $*$.
      By induction, we know that the correctness statement holds for
      $\semevt{\pe}_{ρ_1}$.
      We also know that $ρ(\py) \not∈ L$ because $\tr_1(\px) \not⊑ \tr_1(\py)$.

      In the interesting case that $\bigstep{\semevt{\pe}_{ρ_1}}{μ[\pa'↦...]}{\FunV(f_i)}{μ_i'}$,
      we have
      \[
        \correct_{L}(f_1(ρ(\py))(μ_1), f_2(ρ(\py))(μ_2))
      \]
      for all $\pa' ∈ L$, because $ρ(\py) \not∈ L$.
      This shows the goal.

    \item \textbf{Case $\pe_2 = \Lam{\py}{\pe}$}:
      Fix $μ$ such that $\rng(μ)$ is $L$-once and for all $\pa' ∈ L$ such that $\pa'$ is dead in $d'$.
      To apply $\corfun$, we need to show that
      \[
        \correct_{L}(\semevt{\pe}_{ρ_1[\py↦\pa_a]}(μ[\pa'↦ d'\tick]), \semevt{\pe}_{ρ_1[\py↦\pa_a]}(μ[\pa'↦\memo(\pa,d')]))
      \]
      for all $\pa_a ∈ (\dom(μ) \setminus L)$.

      Since $\pa_a \not∈ L$ we maintain the invariant on $ρ_1[\py↦\pa_a]$ and
      $\tr_1(\px) \not⊑ \bot = \tr_1[\py↦\bot](\py)$ and may apply the induction hypothesis
      to the above situation since $ω*\tr_1[\py↦\bot](\px) \not⊑ \semusg{\pe}_{\tr_1[\py↦⊥]}$.
      This shows the goal.

    \item \textbf{Case $\pe_2 = \Let{\py}{\pe_1'}{\pe_2'}$}:
      Fix $μ$ such that $\rng(μ)$ is $L$-once and for all $\pa' ∈ L$ such that $\pa'$ is dead in $d'$.
      We need to show that
      \[
        \correct_{L}(\semevt{\Let{\py}{\pe_1'}{\pe_2'}}_{ρ_1}(μ[\pa'↦ d'\tick]), \semevt{\Let{\py}{\pe_1'}{\pe_2'}}_{ρ_1}(μ[\pa'↦\memo(\pa,d')])).
      \]
      We unfold the definition of $\semevt{\wild}$ and see that we need to prove
      \[
        \correct_{L}(\semevt{\pe_2'}_{ρ_2}(μ[\pa'↦ d'\tick][\pa_l↦d_1]), \semevt{\pe_2'}_{ρ_2}(μ[\pa'↦\memo(\pa,d')][\pa_l↦d_1])),
      \]
      where $ρ_2 \triangleq ρ_1[\py↦\pa_l]$,
      $d_1 \triangleq \memo(\pa_l,\semevt{\pe_1'}_{ρ_1[\py↦\pa_l]})$.

      From $\px \not= \py$ (otherwise shadowing) we see that
      $\tr_1(\px) = \tr_2(\px)$, \\
      where
      ${\tr_2 \triangleq \fix(\fn{\tr_2}{\tr_1 ⊔ [\py ↦
      [\py↦1]+\semusg{\pe_1'}_{\tr_2}]})}$.
      It is clear that $ω*\tr_2(\px) = ω*\tr_1(\px) \not⊑ \semusg{\pe_2}_{\tr_1} = \semusg{\pe_2'}_{\tr_2}$,
      so we want try to apply the induction hypothesis to this situation,
      but that requires us to show that the invariant on $L$ is satisfied
      as well as that $d_1$ is $L$-once.

      We proceed by cases over $\tr_1(\px) = \tr_2(\px) ⊑ \semusg{\pe_1'}_{\tr_2}$.
      \begin{itemize}
        \item \textbf{Case $ω*\tr_2(\px) ⊑ \semusg{\pe_1'}_{\tr_2}$}:
          Then $ω*\tr_2(\px) ⊑ \tr_2(\py) \not⊑ \semusg{\pe_2'}_{\tr_2}$ and
          hence $\py$ is dead in $\pe_2'$, which allows us to rewrite its bindings
          in $μ$ with something that is dead in $L$.
          (It is easy to see that deadness is compatible with $\correct$ in this
          way.)
          That again makes it possible to apply the induction hypothesis
          (noting that now $\tr(\px) \not⊑ \tr_2(\py)$ and hence $\pa_l \not∈ L$),
          showing the goal.
        \item \textbf{Case $\tr_2(\px) \not⊑ \semusg{\pe_1'}_{\tr_2}$}:
          Then $\px$ is dead in $\pe_1$, $\tr_2(\px) \not⊑  \tr_2(\py)$.
          This implies that $L$ is dead in $\semevt{\pe_1'}_{ρ_2}$:
          For any $\pa'$, either there exists $\py$ such that $\pa' = ρ(\py)$
          in which case $\tr_2(\px) ⊑  \tr_2(\py)$ leads to a contradiction.
          Otherwise, there is no variable in scope that refers to $\pa'$ and hence
          $\pa'$ is dead in $\semevt{\pe_1'}_{ρ_2}$ by \Cref{thm:addr-dom-sem}.

          From $\tr_2(\px) \not⊑ \tr_2(\py)$ we conclude $ρ_2(\py) = \pa_l
          \not∈L$ and $L$ is compatible with the latter inequality.
          $L$-deadness implies that $d_1$ is $L$-once as well.
          Hence we may apply the induction hypothesis to show the goal.
        \item \textbf{Case $\tr_2(\px) ⊑ \semusg{\pe_1'}_{\tr_2}$}:
          Then $\tr_2(\px) ⊑ \tr_2(\py)$ and hence
          $ω*\tr_2(\py) \not⊑ \semusg{\pe_2}_{\tr_2}$.

          To apply the induction hypothesis, we need to show that $μ$ is also
          $L'$-once, where $L' \triangleq L ∪ \{\pa_l\}$.
          Since $\pa_l$ is dead in $\rng(\pa_l)$, that is the case.
          Furthermore, we need to show that $\semevt{\pe_1'}_{ρ_2}$ is
          $L'$-once, but that follows by the induction hypothesis applied
          to $L'$ and $ω*\tr_2(\px) ⊑ \semusg{\pe_1}_{\tr_2}$.

          So it is the case that $\rng(μ[\pa_l↦\semevt{\pe_1'}_{ρ_2}])$ is
          $L'$-once.
          (Concerns about definedness can be overcome with
          \Cref{thm:heap-partitioning}.)
          Hence apply the induction hypothesis.

          We still need to show that we can weaken $L'$ back to $L$.
          Since $L'$ was extended for $\pa_l$ which is free in $\dom(μ_i)$,
          all rule applications can be rewritten without conflict.
      \end{itemize}
  \end{itemize}
\end{proof}

But $\semusg{\Let{x}{\pe_1}{\pe_2}}_{\tr_Δ}(x) ⊑ 1$ is just a sufficient condition for
the semantic property of ``evaluates $x$ at most once''; it is not a suitable
\emph{definition} of that property by far.
For example, $((\Lam{y}{x})~x)$ evaluates $x$ at most once, but is not recognised
as such by $\semusg{\wild}$.

Intuitively, we need a function $\mathit{count}_\pa : \Traces \to ℕ_ω$ that counts
$\LookupE(\pa)$ actions at a particular address $\pa$ over the course of a
head-normal form reduction and take the upper bound of this function in
arbitrary evaluation contexts.

We distribute these reponsibilities between the two functions $\usg$ and $\ctx$
in \Cref{fig:usg-abs}:

%Then, ``$\pa$ is evaluated at most once'' roughly corresponds to
%$\mathit{count}_\pa(\semevt{\pE[\pe_2]}_{[\px↦\pa]}([\pa↦\semevt{\pe_1}_{[\px↦\pa]}])) \leq 1$
%in arbitrary contexts such that $\pE$ does not capture $\px$ nor preoccupies $\pa$.

%This sketch has two flaws:
%The first is that the proposition and thus $\mathit{count}$ needs to know the
%address it should count, so we were forced to unfold the definition of
%$\semevt{\Let{\px}{\pe_1}{\pe_2}}_ρ(μ)$ without knowing $μ$, leading to new side
%conditions begging for an intricate definition, such as ``$\pE$ preoccupies
%$\pa$''.
%The second flaw is that $\pe_1$ can't have any free variables besides $\px$ this
%way, so it is not a faithful model of $\mathbf{let}$-binding and perhaps too
%weak to prove lockstep bisimilarity.

%Both flaws are a result of the need to escape and re-enter the syntactic (or,
%perhaps \emph{static}) realm of $\pE$ and $\pe_2$ to name the semantic (or
%\emph{dynamic}) address which the proposition needs to refer to.

%One solution to this problem is to extend the semantics function to evaluation
%contexts, with functionality
%$\semevt{\pE} : ((\Var \to \Addresses) \to \EventD) \to ((\Var \to \Addresses) \to \EventD)$
%such that $\semevt{\pE[\pe]}_ρ = \semevt{\pE}_ρ(\semevt{\pe})$.
%This equation has a solution because of compositionality
%and would allow for a faithful model of $\mathbf{let}$-binding in the hole of
%the context.
%A closed definition can be found in \Cref{fig:semevt-context}.
%
%\begin{toappendix}
%\begin{figure}
%\[
% \begin{array}{rcl}
%  \multicolumn{3}{c}{ \ruleform{ \semevt{\wild} \colon \Ctx → ((\Var \pfun \Addresses) → \EventD) → (\Var \pfun \Addresses) → \EventD } } \\
%  \\[-0.5em]
%  \semevt{\hole}_ρ(S) & = & S_ρ \\
%  \\[-0.5em]
%  \semevt{\pE~\px}_ρ(S)   & = & \begin{cases}
%    \AppIE(ρ(x)) \cons \apply(\semevt{\pE}_ρ(S), ρ(\px)) & \px ∈ \dom(ρ) \\
%    \fn{\wild}{\stuckend{}}  & \text{otherwise} \\
%  \end{cases} \\
%  \\[-0.5em]
%  \semevt{\Let{\px}{\pe}{\pE}}_ρ(S)(μ) & = &
%    \begin{letarray}
%      \text{let} & ρ' = ρ[\px ↦ \pa] \quad \text{where $\pa \not∈ \dom(μ)$} \\
%                 & d_1^\later = \semevt{\pe}_{ρ'} \\
%      \text{in}  & \BindE(\px,\pa↦d_1^\later) \cons \semevt{\pE}_{ρ'}(S)(μ[\pa ↦ \memo(\pa,d_1^\later)])
%    \end{letarray} \\
%  \\[-0.5em]
%  \semevt{\Let{\px}{\pE_1}{\pE_2[\px]}}_ρ(S)(μ) & = &
%    \begin{letarray}
%      \text{let} & ρ' = ρ[\px ↦ \pa] \quad \text{where $\pa \not∈ \dom(μ)$} \\
%                 & d_1^\later = \semevt{\pE_1}_{ρ'}(S) \\
%      \text{in}  & \BindE(\px,\pa↦d_1^\later) \cons \semevt{\pE_2[\px]}_{ρ'}(μ[\pa ↦ \memo(\pa,d_1^\later)])
%    \end{letarray} \\
%  \\[-0.5em]
%  \semevt{\Case{\pE}{\Sel[r]}}_ρ(S) & = &
%    \begin{letarray}
%      \text{let} & \alts = \fn{(K_i, \many{\pa})}{\CaseEE(K_i,\many{\px_i↦\pa}) \cons \semevt{\pe_{r_i}}_{ρ[\many{\px_i↦\pa}]}} \\
%      \text{in} & \CaseIE(\semevt{\pE}_ρ(S)) \cons \select(\semevt{\pe_s}_ρ, \alts)  \\
%    \end{letarray}
% \end{array}
%\]
%\caption{Eventful Semantics of Evaluation Contexts}
%\label{fig:semevt-context}
%\end{figure}
%\end{toappendix}

\begin{figure}
\[\begin{array}{c}
 \arraycolsep=3pt
 \begin{array}{rclcl@{\quad}rclcl}
  μ & ∈ & \Heaps^{\Look} & =   & \Addresses \pfun \later\Domain{\Look}
  &
  d & ∈ & \Domain{\Look} & =   & \Heaps^{\Look} \to \LookTraces \\
  u & ∈ & \Usg & =   & \{ 0 ⊏ 1 ⊏ ω \} ⊂ ℕ_ω
  &
  τ & ∈ & \LookTraces & ::=  & (l ∈ \Addresses \to \Usg) \lcons \someend{v,μ} \\
  \\[-0.75em]
 \end{array} \\
 \begin{array}{rclcl}
  v & ∈ & \Values{\Look} & ::=   & \FunV(f ∈ \Addresses \to \Domain{\Look}) \\
 \end{array} \\
 \\[-0.5em]
 \begin{array}{lcl}
  \multicolumn{3}{c}{ \ruleform{ \usg_{\Events} : \Events \to (\Addresses \to \Usg) \quad \usg_{\Traces} : \Traces \to \LookTraces } } \\
  \\[-0.5em]
  \usg_{\Events}(ε) & = & \begin{cases}
      [\pa↦1] & ε = \LookupT(\pa) \\
      \constfn{0} & \text{otherwise}
    \end{cases} \\
  l_1 +_1 (l_2 \lcons \someend{\tilde{v},\tm}) & = & (l_1+l_2) \lcons \someend{\tilde{v},\tm} \\
  \usg_{\Traces}(ε \cons τ) & = & \usg_{\Events}(ε) +_1 \usg_{\Traces}(τ) \\
  \usg_{\Traces}(\goodend{\FunV(f),μ}) & = & \constfn{0} \lcons \someend{\FunV(\usg_{\EventD} \circ f), \usg_\Heaps(μ)} \\
  \usg_{\Traces}(\stuckend{}) & = & \constfn{0} \lcons \someend{\bot_{\Values{\UsgD}}, (\constfn{\bot_\EventD}) \circ μ} \\
  \usg_{\Heaps}(μ) & = & \usg_\EventD \circ μ \\
  \usg^{⊣}_{\Heaps}(\tm) & = & \bigcup \{ μ \mid \usg_{\Heaps}(μ) ⊑ \tm \} \\
  \usg_{\EventD}(d) & = & \usg_\Traces \circ d \circ \usg^{⊣}_{\Heaps} \\
  \\[-0.5em]
 \end{array} \\
 \begin{array}{lcl}
  \multicolumn{3}{c}{ \ruleform{ \ctx_{\wild} : \poset{\Addresses} \to \LookTraces \to \Addresses \to \Usg } } \\
  \\[-0.5em]
  \ctx_A(l \lcons \someend{\FunV(\tilde{f}),\tm}) & = & l +
    \Lub_{\pa∈A} \{ ω*\ctx_A(\tilde{f}(\pa)(\tm)) \}  \\
%  \multicolumn{3}{c}{ \ruleform{ α : \Traces \to \UsgD } } \\
%  \\[-0.5em]
%  α(τ) & = & fst(intra(usg_\Traces(blub(τ)(\fn{\px}{[\px↦1]}))(τ))) \\
 \end{array}
\end{array}\]
\vspace{-1em}
\caption{Usage abstraction}
\label{fig:usg-abs}
\end{figure}

\begin{definition}[Usage cardinality]
  \label{defn:usg-card}
  A denotation $d$ evaluates an address $\pa$ \emph{at most $u$ times} (where
  $u∈\Usg$) if and only if, for all $μ$ such that $\pa$ is dead in $μ$,
  \[
    \ctx_{\dom(μ)}(\usg_\Traces(d(μ)))(\pa) ⊑ u.
  \]
  An expression $\pe$ evaluates $\px$ \emph{at most $u$ times} if and only if,
  for $\pa \not∈ \rng(ρ)$, $\semevt{\pe}_{ρ[\px↦\pa]}$ evaluates
  $\pa$ at most $u$ times.
\end{definition}

%Happily and unlike \Cref{defn:deadness3}, this definition does
%not need to relate executions of different heaps, thus it is
%a simple \emph{trace property} rather than a \emph{program
%property}~\citep{Cousot:21} (also called
%\emph{hyperproperty} by~\citet{ClarksonSchneider:10}).

The \emph{semantic usage abstraction} $\usg_\Traces : \Traces \to \LookTraces$
is the most precise usage analysis and is induced componentwise on
$\Traces, \Heaps,\EventD$ and $\EventV$ by the event abstraction $\usg_\Events$
in \Cref{fig:usg-abs}.

A $\LookTraces$ trace can then be folded by $\ctx_\LookTraces$ into
an $\Addresses \to \Usg$ mapping.%
\footnote{It is worth noting that for brevity we play fast and loose with guardedness
conditions for $\LookTraces$.
I.e., $\Let{loop}{\Let{\px}{\ttrue}{loop}}{loop}$ generates a diverging trace
for which $\usg_\Traces$ would only produce an $\Addresses \to \Usg$ mapping
after an infinite amount of time. A rigorous treatment would have each event
emit its own $l$ in a guarded fashion.}
Thus, $\ctx_\LookTraces$ continues a $\LookTraces$ in all possible evaluation
contexts, returning the least upper bound of all continued traces.

\begin{toappendix}
%\begin{figure}
%\[\begin{array}{c}
% \ruleform{ \correct_{\pa,u}(τ_1,τ_2) }
% \\
% \\[-0.5em]
% \inferrule*[right=\corstuck]
%    {\quad}
%    {\correct_{\pa,u}(\stuckend{},\stuckend{})}
% \quad
% \inferrule*[right=\corcons]
%    {\later (\correct_{\pa,u}(τ_1,τ_2))}
%    {\correct_{\pa,u}(\wild :: τ_1,\wild :: τ_2)}
% \\
% \\[-0.5em]
% \inferrule*[right=\corfun]
%    {\dom(μ_1) = \dom(μ_2) \quad \forall \pa' ∈ \dom(μ_i).\ \usg_\Traces(\correct_{\pa,u}(f_1(\pa)(μ_1), f_2(\pa)(μ_2))}
%    {\correct_{\pa,u}(\goodend{\FunV(f_1),μ_1},\goodend{\FunV(f_2),μ_2})}
% \\
% \\[-0.5em]
%\end{array}\]
%\caption{Correctness relation for \Cref{thm:usg-by-name}}
%  \label{fig:semusg-correct2}
%\end{figure}
\end{toappendix}

%Rather than to let syntactic evaluation contexts dictate how $\pe_2$ might be
%evaluated to produce (essentially first-order) traces that we count
%$\LookupE(\pa)$ actions in, $\usg$ lifts the counting of addresses through the
%whole domain, including function values.

%\begin{lemma}[Heap extension preserves improvement]
%  If $A ⊦_0 d_1(μ_1) \lesssim d_2(μ_2)$, then $A ⊦_0 d_1(μ_1[\pa↦d_1]) \lesssim d_2(μ_2[\pa↦d_2])$.
%\end{lemma}

\begin{theoremrep}[Update avoidance]
  \label{thm:usg-by-name}
  Let $\Let{\px}{\pe_1}{\pe_2}$ be an expression such that $\pe_2$ evaluates $\px$
  at most once and $\px$ is dead in $\pe_1$.
  Then
    $\semevt{\Let{\px}{\pe_1}{\pe_2}}_ρ \lockstep
     \semevt{\Letn{\px}{\pe_1\tick}{\pe_2}}_ρ$.
\end{theoremrep}
\begin{proof}
  Not done yet. It should be similar to \Cref{thm:semusg-by-name},
  with a suitably adjusted correctness predicate.
  We hope to provide a complete the proof by October.
%  The setup is quite similar to \Cref{thm:semusg-by-name}:
%
%  We prove $(\lockstep)$ by rule $\impdenot$.
%  After unfolding the definitions of $\semevt{\wild}$ for $\mathbf{let}$ and
%  $\mathbf{let1}$, the goal is to show
%  \[
%    \dom(μ_i) ⊦_0 \semevt{\pe_2}_{ρ[\px↦\pa]}(μ_1[\pa↦d_1]) \lesssim\!\gtrsim \semevt{\pe_2}_{ρ[\px↦\pa]}(μ_2[\pa↦d_2])
%  \]
%  where $\pa \not∈\dom(μ_i)$,
%  $\dom(μ_i) ∪ \{\pa\} ⊦ \semevt{\pe_2}_{ρ[\px↦\pa]}$,
%  $μ_1 \lessapprox\!\gtrapprox μ_2$.
%  Clearly,
%  \[
%    \dom(μ_i) ⊦_0 \semevt{\pe_2}_{ρ[\px↦\pa]}(μ_1[\pa↦\highlight{d_2}]) \lesssim\!\gtrsim \semevt{\pe_2}_{ρ[\px↦\pa]}(μ_2[\pa↦d_2])
%  \]
%  holds by reflexivity because $μ_1[\pa↦d_2] \lessapprox\!\gtrapprox μ_2[\pa↦d_2]$, so the goal becomes
%  \[
%    \dom(μ_i) ⊦_0 τ_1 \lesssim\!\gtrsim τ_2,
%  \]
%  abbreviating $τ_i \triangleq \semevt{\pe_2}_{ρ[\px↦\pa]}(μ_{\highlight{\scriptstyle 1}}[\pa↦d_i])$.
%
%  \noindent
%  We will prove the correctness predicate $\correct_{\{\pa\}}(τ_1,τ_2)$ instead, where
%  $\correct$ is defined in \Cref{fig:semusg-correct2}.
%
%  Since $\pa \not ∈ \dom(μ_1)$,
%  $\corfun$ implies $\impfun$ on $\dom(μ_i)$ and the goal follows.
%
%  (Same setup until here.)
%
%  Since $\pe_2$ evaluates $\px$ at most once and $\pa$ is dead in
%  $μ_1[\pa↦d_i]$ by assumption that $x$ is dead in $\pe_1$, we have
%  $\ctx_\LookTraces(\usg_\Traces(τ_i))(\pa) ⊑ 1$.
%  Let us also abbreviate $ρ_1 \triangleq ρ[\px↦\pa]$.

%  We show that $\semevt{\pe_2}_{ρ[\px↦\pa]}$ is $\{\pa\}$-once
%  assuming that $ω*\tr_1(\px) \not⊑ \semusg{\pe_2}_{\tr_1}$.
%  By induction on $\pe_2$, generalising to $L$-onceness and
%  maintaining that $\tr_1(\px) \not⊑ \tr_1(\py) \Longleftrightarrow ρ(\py) \not∈ L$
%  for all $\py∈ \dom(ρ)$.
%  \begin{itemize}
%    \item \textbf{Case $\pe_2 = \py$}:
%      Fix $μ$ such that $\rng(μ)$ is $L$-once and for all $\pa' ∈ L$ such that $\pa'$ is dead in $d'$,
%      we need to show that
%      \[
%        \correct_{L}(μ_1(ρ(\py))(μ_1), μ_2(ρ(\py))(μ_2)).
%      \]
%      where $μ_1 \triangleq μ[\pa'↦d'\tick ], μ_2 = μ[\pa'↦\memo(\pa',d')]$.
%      This is easy to see when $ρ(\py) \not= \pa'$, so assume $ρ(\py) = \pa'$:
%      \[
%        \correct_{L}(d'(μ_1) \cons \TickE, d'(μ_2) \betastep \fn{v}{(\fn{μ'}{\UpdateE(...) \cons \goodend{v,μ'[\pa'↦\memo(\pa',\ret(v))]}})}).
%      \]
%      We can strip away the trailing transition, leaving behind just the heap update.
%
%      Now, without loss of generality, we will assume that $L$ is dead in
%      $\rng(μ)$.
%      If that was not the case, we could apply \Cref{heap-partitioning}
%      to hide all entries in which $L$ is live ``inside $d'$'' and allocate them
%      on first execution.
%      % Turn into lemma. Sounds useful
%
%      When $L$ is dead in $\rng(μ)$, we can rewrite the entry for $\pa'$ freely
%      without affecting $\correct$.
%      In the interesting case we have $\bigstep{d'}{μ_i}{\FunV(f_i)}{μ_i'}$.
%      Because of deadness, we must have
%      \[
%        \correct_{L}(d'(μ_1), d'(μ_2)).
%      \]
%      That implies
%      \[
%        \correct_{L}(f_1(\pa_a)(μ_1'), f_2(\pa_a)(μ_2')).
%      \]
%      for all $\pa_a ∈ \dom(μ_i') \setminus L$.
%      Now it suffices to show the situation after heap update
%      \begin{equation}
%        \label{eqn:usg-var-goal}
%        \correct_{L}(f_2(\pa_a)(μ_2'), f_2(\pa_a)(μ_2'[\pa'↦\memo(\pa',\ret(\FunV(f_2)))])).
%      \end{equation}
%      In particular, note that
%      $μ_2 \forcesto μ_3 \triangleq μ[\pa'↦\memo(\pa',\ret(\FunV(f_2)))]$
%      (again, we can properly hide any free address that $f_2$ needs in
%      $μ_3(\pa')$), so $\bigstep{d'}{μ_3}{\FunV(f_2)}{μ_3'}$ with
%      $μ_2' \forcesto μ_3'$ by the Postpone property.
%      By deadness we can follow
%      \[
%        \correct_{L}(d'(μ_2), d'(μ_3)).
%      \]
%      and since $\bigstep{d'}{μ_3}{\FunV(f_2)}{μ_3'}$ (NB: $f_2$ due to the
%      forcing relationship), that implies
%      \[
%        \correct_{L}(f_2(\pa_a)(μ_2'), f_2(\pa_a)(μ_3')).
%      \]
%      for all $\pa_a ∈ \dom(μ_i') \setminus L$.
%
%      Due to forcing, it must be that $μ_3'(\pa) = μ_3(\pa)$.
%      Again by forcing, $μ_2'(\pa)$ must be either
%      $μ_2(\pa)$ or $μ_3'(\pa)$.
%
%      Since the number of steps taken for $f_2(\pa_a)(μ_2'[\pa↦μ_3'(\pa)])$ must be between $f_2(\pa_a)(μ_2')$
%      and $f_2(\pa_a)(μ_3')$ (due to forcing), this shows the goal \Cref{eqn:usg-var-goal}.
%
%    \item \textbf{Case $\pe_2 = \pe~\py$}:
%      Fix $μ$ such that $\rng(μ)$ is $L$-once and for all $\pa' ∈ L$ such that $\pa'$ is dead in $d'$,
%      we need to show that
%      \[
%        \correct_{L}(\semevt{\pe~\py}_{ρ_1}(μ[\pa'↦ d'\tick]), \semevt{\pe~\py}_{ρ_1}(μ[\pa'↦\memo(\pa',d')])).
%      \]
%
%      From $ω*\tr_1(\px) \not⊑ \semusg{\pe}_{\tr_1} + ω*\tr_1(\py)$ we can see that
%      $ω*\tr_1(\px) \not⊑ \semusg{\pe}_{\tr_1}$ and $\tr_1(\px) \not⊑ \tr_1(\py)$ by
%      monotonicity of $+$ and $*$.
%      By induction, we know that the correctness statement holds for
%      $\semevt{\pe}_{ρ_1}$.
%      We also know that $ρ(\py) \not∈ L$ because $\tr_1(\px) \not⊑ \tr_1(\py)$.
%
%      In the interesting case that $\bigstep{\semevt{\pe}_{ρ_1}}{μ[\pa'↦...]}{\FunV(f_i)}{μ_i'}$,
%      we have
%      \[
%        \correct_{L}(f_1(ρ(\py))(μ_1), f_2(ρ(\py))(μ_2))
%      \]
%      for all $\pa' ∈ L$, because $ρ(\py) \not∈ L$.
%      This shows the goal.
%
%    \item \textbf{Case $\pe_2 = \Lam{\py}{\pe}$}:
%      Fix $μ$ such that $\rng(μ)$ is $L$-once and for all $\pa' ∈ L$ such that $\pa'$ is dead in $d'$.
%      To apply $\corfun$, we need to show that
%      \[
%        \correct_{L}(\semevt{\pe}_{ρ_1[\py↦\pa_a]}(μ[\pa'↦ d'\tick]), \semevt{\pe}_{ρ_1[\py↦\pa_a]}(μ[\pa'↦\memo(\pa,d')]))
%      \]
%      for all $\pa_a ∈ (\dom(μ) \setminus L)$.
%
%      Since $\pa_a \not∈ L$ we maintain the invariant on $ρ_1[\py↦\pa_a]$ and
%      $\tr_1(\px) \not⊑ \bot = \tr_1[\py↦\bot](\py)$ and may apply the induction hypothesis
%      to the above situation since $ω*\tr_1[\py↦\bot](\px) \not⊑ \semusg{\pe}_{\tr_1[\py↦⊥]}$.
%      This shows the goal.
%
%    \item \textbf{Case $\pe_2 = \Let{\py}{\pe_1'}{\pe_2'}$}:
%      Fix $μ$ such that $\rng(μ)$ is $L$-once and for all $\pa' ∈ L$ such that $\pa'$ is dead in $d'$.
%      We need to show that
%      \[
%        \correct_{L}(\semevt{\Let{\py}{\pe_1'}{\pe_2'}}_{ρ_1}(μ[\pa'↦ d'\tick]), \semevt{\Let{\py}{\pe_1'}{\pe_2'}}_{ρ_1}(μ[\pa'↦\memo(\pa,d')])).
%      \]
%      We unfold the definition of $\semevt{\wild}$ and see that we need to prove
%      \[
%        \correct_{L}(\semevt{\pe_2'}_{ρ_2}(μ[\pa'↦ d'\tick][\pa_l↦d_1]), \semevt{\pe_2'}_{ρ_2}(μ[\pa'↦\memo(\pa,d')][\pa_l↦d_1])),
%      \]
%      where $ρ_2 \triangleq ρ_1[\py↦\pa_l]$,
%      $d_1 \triangleq \memo(\pa_l,\semevt{\pe_1'}_{ρ_1[\py↦\pa_l]})$.
%
%      From $\px \not= \py$ (otherwise shadowing) we see that
%      $\tr_1(\px) = \tr_2(\px)$, \\
%      where
%      ${\tr_2 \triangleq \fix(\fn{\tr_2}{\tr_1 ⊔ [\py ↦
%      [\py↦1]+\semusg{\pe_1'}_{\tr_2}]})}$.
%      It is clear that $ω*\tr_2(\px) = ω*\tr_1(\px) \not⊑ \semusg{\pe_2}_{\tr_1} = \semusg{\pe_2'}_{\tr_2}$,
%      so we want try to apply the induction hypothesis to this situation,
%      but that requires us to show that the invariant on $L$ is satisfied
%      as well as that $d_1$ is $L$-once.
%
%      We proceed by cases over $\tr_1(\px) = \tr_2(\px) ⊑ \semusg{\pe_1'}_{\tr_2}$.
%      \begin{itemize}
%        \item \textbf{Case $ω*\tr_2(\px) ⊑ \semusg{\pe_1'}_{\tr_2}$}:
%          Then $ω*\tr_2(\px) ⊑ \tr_2(\py) \not⊑ \semusg{\pe_2'}_{\tr_2}$ and
%          hence $\py$ is dead in $\pe_2'$, which allows us to rewrite its bindings
%          in $μ$ with something that is dead in $L$.
%          (It is easy to see that deadness is compatible with $\correct$ in this
%          way.)
%          That again makes it possible to apply the induction hypothesis
%          (noting that now $\tr(\px) \not⊑ \tr_2(\py)$ and hence $\pa_l \not∈ L$),
%          showing the goal.
%        \item \textbf{Case $\tr_2(\px) \not⊑ \semusg{\pe_1'}_{\tr_2}$}:
%          Then $\px$ is dead in $\pe_1$, $\tr_2(\px) \not⊑  \tr_2(\py)$.
%          This implies that $L$ is dead in $\semevt{\pe_1'}_{ρ_2}$:
%          For any $\pa'$, either there exists $\py$ such that $\pa' = ρ(\py)$
%          in which case $\tr_2(\px) ⊑  \tr_2(\py)$ leads to a contradiction.
%          Otherwise, there is no variable in scope that refers to $\pa'$ and hence
%          $\pa'$ is dead in $\semevt{\pe_1'}_{ρ_2}$ by \Cref{thm:addr-dom-sem}.
%
%          From $\tr_2(\px) \not⊑ \tr_2(\py)$ we conclude $ρ_2(\py) = \pa_l
%          \not∈L$ and $L$ is compatible with the latter inequality.
%          $L$-deadness implies that $d_1$ is $L$-once as well.
%          Hence we may apply the induction hypothesis to show the goal.
%        \item \textbf{Case $\tr_2(\px) ⊑ \semusg{\pe_1'}_{\tr_2}$}:
%          Then $\tr_2(\px) ⊑ \tr_2(\py)$ and hence
%          $ω*\tr_2(\py) \not⊑ \semusg{\pe_2}_{\tr_2}$.
%
%          To apply the induction hypothesis, we need to show that $μ$ is also
%          $L'$-once, where $L' \triangleq L ∪ \{\pa_l\}$.
%          Since $\pa_l$ is dead in $\rng(\pa_l)$, that is the case.
%          Furthermore, we need to show that $\semevt{\pe_1'}_{ρ_2}$ is
%          $L'$-once, but that follows by the induction hypothesis applied
%          to $L'$ and $ω*\tr_2(\px) ⊑ \semusg{\pe_1}_{\tr_2}$.
%
%          So it is the case that $\rng(μ[\pa_l↦\semevt{\pe_1'}_{ρ_2}])$ is
%          $L'$-once.
%          (Concerns about definedness can be overcome with
%          \Cref{thm:heap-partitioning}.)
%          Hence apply the induction hypothesis.
%
%          We still need to show that we can weaken $L'$ back to $L$.
%          Since $L'$ was extended for $\pa_l$ which is free in $\dom(μ_i)$,
%          all rule applications can be rewritten without conflict.
%      \end{itemize}
%  \end{itemize}
\end{proof}

Similarly, one could prove that a binding is dead when it is evaluated at most 0
times.
An alternative correctness proof for $\semusg{\wild}$ can then be formulated by
\emph{abstract interpretation}, proving it correct simultaneously as a deadness
and a sharing analysis:

\begin{theoremrep}[$\semusg{\wild}$ approximates semantic usage]
  \label{thm:semusg-correct-3}
  Let $\pe$ be an expression, $\px$ a variable, $\pa$ an address
  that is dead in a heap $μ$ and $ρ$ an environment such that $\pa \not∈
  \rng(ρ)$.
  Then
  \[
    \ctx_{\dom(μ)}(\usg_\Traces(\semevt{\pe}_{ρ[\px↦\pa]}(μ)))(\pa) ⊑ \semusg{\pe}_{\tr_Δ}(\px)
  \]
\end{theoremrep}
\begin{proof}
  Not done yet (deadline is in 30 minutes). Here's what we've got so far:

  We first get rid of the concrete heap by the following estimate:
  \[
    \ctx_{D}(\usg_\Traces(\semevt{\pe}_{ρ[\px↦\pa]}(μ)))(\pa) ⊑ \ctx_{D}(\usg_\EventD(\semevt{\pe}_{ρ[\px↦\pa]})(\usg_\Heaps(μ)))(\pa)
  \]
  for $D \triangleq \dom(μ)$.
  This holds because $μ ∈ \usg^⊣_\Heaps(\usg_\Heaps(μ))$, exploiting that
  $\usg_\Heaps$ is left adjoint to $\usg^⊣_\Heaps$.

  By induction on $\pe$, we show that
  \[
    \ctx_{D}(\usg_\EventD(\semevt{\pe}_{ρ[\px↦\pa]})(\tm))(\pa) ⊑ \semusg{\pe}_{\tr}(\px)
  \]
  for all $\px$, maintaining the invariant that $\pa \not∈ \rng(ρ)$ and $\pa$ is
  dead in $\tm(D)$
  (so for all $\pa' ∈ D.\ \ctx_{D}(\tm(\pa')(\tm))(\pa) = 0$),
  as well as
  \begin{equation}
    \label{eqn:semusg-rho}
    ∀\py.\ \tr_Δ(\py)(\px) + \ctx_D(\tm(ρ[\px↦\pa](\py))(\tm))(\pa) ⊑ \tr(\py)(\px)
  \end{equation}
  Initially, $D = \dom(μ), \tr = \tr_Δ$.
  We abbreviate
  $u_l \triangleq \ctx_{D}(\usg_\EventD(\semevt{\pe}_ρ)(\tm))(\pa)$,
  ${u_r \triangleq \semusg{\pe}_{\tr}(\px)}$ and
  ${ρ_1 \triangleq ρ[\px↦\pa]}$.
  \begin{itemize}
    \item \textbf{Case $\pe = \py$}:
      The stuck case is clear, because then $u_l = 0$.

      When $\px = \py$:
      \[\begin{WithArrows}
              & u_l \Arrow{definition of $\semevt{\px}$} \\
        {}={} & 1 + \ctx_{D}(\tm(ρ_1(\px))(\tm))(\pa) \Arrow{definition of $\tr_Δ(\px)$} \\
        {}={} & \tr_Δ(\px)(\px) + \ctx_{D}(\tm(ρ_1(\px))(\tm))(\pa) \Arrow{\Cref{eqn:semusg-rho}} \\
        {}⊑{} & \tr(\px)(\px) \Arrow{definition of $\semusg{\px}_{\tr}$} \\
        {}={} & u_r
      \end{WithArrows}\]

      When $\px \not= \py$:
      \[\begin{WithArrows}
              & u_l \Arrow{definition of $\semevt{\px}, ρ(\px) \not=ρ(\py)$} \\
        {}={} & 0 + \ctx_{D}(\tm(ρ_1(\px))(\tm))(\pa) \Arrow{definition of $\tr_Δ(\px)$} \\
        {}={} & \tr_Δ(\py)(\px) + \ctx_{D}(\tm(ρ_1(\py))(\tm))(\pa) \Arrow{\Cref{eqn:semusg-rho}} \\
        {}⊑{} & \tr(\px)(\px) \Arrow{definition of $\semusg{\py}_{\tr}$} \\
        {}={} & u_r
      \end{WithArrows}\]
      %$\begin{WithArrows}
      %        & u_l & \\
      %  {}={} & \ctx_{D}(\tm(ρ_1(\px))(\tm))(\pa) & \hspace{5em} \Lbag\text{definition of $\semevt{\py}, ρ(\px) \not=ρ(\py)$}\Rbag \\
      %  {}={} & \tr_Δ(\px)(\px) + \ctx_{D}(\tm(ρ_1(\px))(\tm))(\pa) & \hspace{5em} \Lbag\text{definition of $\tr_Δ(\px)$}\Rbag \\
      %  {}⊑{} & \tr(\px)(\px) & \hspace{5em} \Lbag\text{\Cref{eqn:semusg-rho}}\Rbag \\
      %  {}={} & u_r & \hspace{5em} \Lbag\text{definition of $\semusg{\px}_{\tr}$}\Rbag
      %\end{WithArrows}$

    \item \textbf{Case $\pe = \pe'~\py$}:
      If $\tr(\py)(\px) \not= 0$, then $u_r = ω$, the top element, and the goal follows.

      Otherwise, $\tr(\py)(\px) = 0$
      and hence
      $\ctx_D(\tm(ρ_1(\py))(\tm))(\pa) = 0$ by \Cref{eqn:semusg-rho}.

      Let us define $l \lcons \someend{\FunV(\tilde{f}),\tm'} \triangleq \usg_\EventD(\semevt{\pe'}_{ρ_1})(\tm)$
      and $D' \triangleq D ∪ \{ ρ_1(\py) \}$.
      We get
      \begin{DispWithArrows*}
            & u_l \Arrow{definition of $\semevt{\pe'~\py}$, drop $\AppIE$} \\
        ={} & \ctx_{D}(\usg_\EventD(\apply(\semevt{\pe'}_{ρ_1},ρ_1(\py)))(\tm)) \Arrow{definition of $\apply,\betastep,l,\tilde{f},\tm$} \\
        ={} & l + \ctx_{D}(\tilde{f}(ρ_1(\py))(\tm')) \Arrow{$D ⊆ D'$ and monotonicity of $\ctx$} \\
        ⊑{} & l + \ctx_{D'}(\tilde{f}(ρ_1(\py))(\tm')) \Arrow{$ω*$ is increasing} \\
        ⊑{} & l + ω*\ctx_{D'}(\tilde{f}(ρ_1(\py))(\tm')) \Arrow{$ρ_1(\py) ∈ D'$} \\
        ⊑{} & \ctx_{D'}(l \lcons \someend{\tilde{f},\tm'}) \Arrow{definition of $\usg_\EventD(\semevt{\pe'}_{ρ_1})$} \\
        ={} & \ctx_{D'}(\usg_\EventD(\semevt{\pe'}_{ρ_1})(\tm)) \Arrow{induction hypothesis for $D'$} \\
        ⊑{} & 0 + \semusg{\pe'}_{\tr}(\px) \Arrow{$\tr(\py)(\px) = 0$} \\
        ={} & u_r
      \end{DispWithArrows*}

    \item \textbf{Case $\pe = \Lam{\py}{\pe'}$}:
      Unfolding $\ctx_D$ and $\semusg{\Lam{\py}{\pe'}}_{\tr}$ once, we have
      \[\begin{array}{l}
        u_l = \Lub_{\pa'∈D} \{ ω*\ctx_{D}(\usg_\EventD(\semevt{\pe'}_{ρ_1[\py↦\pa']})(\tm))(\pa) \} \\
        u_r = ω*\semusg{\pe'}_{\tr[\py↦\bot]}(\px) \\
      \end{array}\]
      So we show
      \[
        \ctx_{D}(\usg_\EventD(\semevt{\pe'}_{ρ_1[\py↦\pa']})(\tm))(\pa) ⊑
        \semusg{\pe'}_{\tr[\py↦\bot]}(\px)
      \]
      for all $\pa' ∈ D$, the rest follows by definition of the upper bound and
      monotonicity of $ω*$.

      We want to invoke the induction hypothesis for $D$ in this situation.
      In order to do that, we have to show \Cref{eqn:semusg-rho} for $\py$:
      \begin{DispWithArrows*}
            & \tr_Δ(\py)(\px) + \ctx_D(\tm(ρ_1[\py↦\pa'](\py))(\tm))(\pa) \Arrow{definition of $\tr_Δ$, $\px\not=\py$} \\
        ={} & 0 + \ctx_D(\tm(\pa')(\tm))(\pa) \Arrow{$\pa'∈D$} \\
        ={} & 0 + 0 \Arrow{simplify} \\
        ={} & 0  \Arrow{simplify} \\
        ={} & \tr[\py↦\bot]
      \end{DispWithArrows*}
      The induction hypothesis shows the goal.

    \item \textbf{Case $\pe = \Let{\py}{\pe_1'}{\pe_2'}$}:
      Unfolding $\ctx_D$ and $\semusg{\Let{\py}{\pe_1'}{\pe_2'}}_{\tr}$ once, we have
      \[\begin{array}{l}
        u_l = \ctx_{D}(\usg_\EventD(\semevt{\pe_2'}_{ρ_1[\py↦\pa']})(\tm[\pa'↦\td_1)(\pa) \} \\
        u_r = \semusg{\pe_2'}_{\tr'}(\px) \\
      \end{array}\]
      where $\td_1 \triangleq \usg_\EventD(\semevt{\pe_1'}_{ρ_1[\py↦\pa']}),
             \tr' \triangleq \fix(\fn{\tr'}{\tr ⊔ [\py ↦ [\py↦1]+\semusg{\pe_1'}_{\tr'}]})$.

      When $\semusg{\pe_1'}_{\tr'}(\px) = 0$, we need to add $\pa'$ to $D$.
      \sg{Do we really? It's not necessary I think, in contrast to the lambda case.} \\
      Before we can apply the induction hypothesis, we must verify that
      \Cref{eqn:semusg-rho} holds for the new entry $\py$ (the others hold by an
      address domain argument):
      \begin{DispWithArrows*}
            & \tr_Δ(\py)(\px) + \ctx_D(\tm[\pa'↦d_1](ρ_1[\py↦\pa'](\py))(\tm[\pa'↦d_1]))(\pa) \Arrow{definition of $\tr_Δ$, $\px\not=\py$} \\
        ={} & 0 + \ctx_D(d_1(\tm[\pa'↦d_1]))(\pa) \Arrow{} \\
        % Want to apply induction hypothesis, but can't without this proof.
        % Perhaps we can by fixpoint induction!
        % E.g., for initial \tr'(\py) = \bot, it is clear.
        % Now imagine we have proven for \tr'_1 and want to show for iteration \tr'_2.
        % Case 0 is clear. Case 1: try with \tr' =
        ={} & \tr'(\py)
      \end{DispWithArrows*}

      We proceed by cases over $\tr(\px) ⊑ \semusg{\pe_1'}_{\tr'}$.
      \begin{itemize}
        \item \textbf{Case $ω*\tr_2(\px) ⊑ \semusg{\pe_1'}_{\tr_2}$}:
        \item \textbf{Case $\tr_2(\px) \not⊑ \semusg{\pe_1'}_{\tr_2}$}:
        \item \textbf{Case $\tr_2(\px) ⊑ \semusg{\pe_1'}_{\tr_2}$}:
      \end{itemize}
  \end{itemize}
\end{proof}

A more elegant proof would continually abstract $\ctx_A \circ \usg_\Traces$
without relating to the structure of $\semevt{\wild}$ or $\semusg{\wild}$ at
all, encoding the strengthened inductive hypothesis (\ie, the logical relation)
of the proof above in a chain of Galois connections. Then the above proof
is simply by direct, structural induction on $\pe$ and equational reasoning
-- \citet{Cousot:21} calls this process ``calculational design''.
We want to attempt such a proof in the future and are confident that it will
yield insightful abstraction functions, for example encoding the transition
from heap-based call-by-need to call-by-name or the transition from
interprocedural to intraprocedural analysis.

\section{Related Work}
\label{sec:related-work}

%\sg{Move to related work.}
%There have been attempts to discern crashes from other kinds of loops, such as
%\citep{imprecise-exceptions}. Unfortunately, in Section 5.3 they find it
%impossible give non-terminating programs a denotation other than $\bot$, which
%still encompasses all possible exception behaviors.
%
% eval/apply or push/enter?
% Given an expr like $f x$, we first push $ρ(x)$ onto the stack and then
% evaluate $f$, which will look it up (pushing an udpate frame) and evaluate its
% RHS. Since we will never return to the "eval site" of $f$, IMO this qualifies
% as push/enter rather than eval/apply. Which is in contrast to what the Krivine
% paper says, which dubs return states as "apply" transitions
%
% Mention how >>β= is similar to the bind operator on interaction trees and probably
% corresponds to Moggi's monadic semantics.i
% Traces are somewhat like the simpler monad `M e a = Good a | Bad e | Later |>(M e a)`
% which combines the delay monad with the error monad

% Compare to coinductive big-step \citep{LeroyGrall:09}
%
% Compare to Definitional Interpreter work; we are using GDTT as the defining language
%
% Compare to Clairvoyant blah, Garden of Forking Paths
%
% Compare to Pitts chapter in TAPL2, "largest congruence relation"


\nocite{*}

\begin{acks}
We would like to thank the anonymous POPL reviewers for their feedback.
%, as well as Bohdan Liesnikov and Sebastian Ullrich.
\end{acks}

\clearpage
\bibliography{references}

\fi % \ifmain

% Appendix
\ifappendix
\appendix
\section{Appendix}\label{sec:appendix}
%include custom.fmt

\renewcommand\thefigure{\thesection.\arabic{figure}}

\subsection{Proofs}

\begin{proof}[Proof of \Cref{thm:semusg-correct-live}]
  \label{prf:semusg-correct-live}
  By induction on $\pe$.
  We fix $\pe$, $\px$ and $\tr$ such that $\tr(\px) \not⊑ \semusg{\pe}_{\tr}$.
  The goal is to show that $\px$ is dead in $\pe$,
  so we are given an arbitrary $ρ$, $d_1$ and $d_2$ and have to show that
  $\semscott{\pe}_{ρ[\px↦d_1]} = \semscott{\pe}_{ρ[\px↦d_2]}$.
  By cases on $\pe$:
  \begin{itemize}
    \item \textbf{Case $\pe\triangleq\py$}: If $\px=\py$, then
      $\tr(\px) \not⊑ \semusg{\py}_{\tr} = \tr(\py) = \tr(\px)$, a contradiction.
      If $\px \not= \py$, then varying the entry for $\px$ won't matter; hence
      $\px$ is dead in $\py$.
    \item \textbf{Case $\pe\triangleq\Lam{\py}{\pe'}$}: The equality follows from
      pointwise equality on functions, so we pick an arbitrary $d$ to show
      $\semscott{\pe'}_{ρ[\px↦d_1][\py↦d]} = \semscott{\pe'}_{ρ[\px↦d_2][\py↦d]}$.

      This is simple to see if $\px=\py$. Otherwise, $\tr[\py↦\bot]$ witnesses the fact that
      \[
        \tr[\py↦\bot](\px) = \tr(\px) \not⊑
        \semusg{\Lam{\px}{\pe'}}_{\tr} = \semusg{\pe'}_{\tr[\py↦\bot]}
      \]
      so we can apply the induction hypothesis to see that $\px$ must be dead in
      $\pe'$, hence the equality on $\semscott{\pe'}$ holds.
    \item \textbf{Case $\pe\triangleq\pe'~\py$}:
      From $\tr(\px) \not⊑ \semusg{\pe'}_{\tr} + ω*\tr(\py)$ we can see that
      $\tr(\px) \not⊑ \semusg{\pe'}_{\tr}$ and $\tr(\px) \not⊑ \tr(\py)$ by
      monotonicity of $+$ and $*$.
      If $\px=\py$ then the latter inequality leads to a contradiction.
      Otherwise, $\px$ must be dead in $\pe'$, hence both cases of
      $\semscott{\pe'~\py}$ evaluate equally, differing only in
      the environment. It remains to be shown that
      $ρ[\px↦d_1](\py) = ρ[\px↦d_2](\py)$, and that is easy to see since
      $\px \not= \py$.
    \item \textbf{Case $\pe\triangleq\Let{\py}{\pe_1}{\pe_2}$}:
      We have to show that
      \[
        \semscott{\pe_2}_{ρ[\px↦d_1][\py↦d'_1]} = \semscott{\pe_2}_{ρ[\px↦d_2][\py↦d'_2]}
      \]
      where $d'_i$ satisfy $d'_i = \semscott{\pe_1}_{ρ[\px↦d_i][\py↦d'_i]}$.
      The case $\px = \py$ is simple to see, because $ρ[\px↦d_i](\px)$ is never
      looked at.
      So we assume $\px \not= \py$ and see that $\tr(\px) = \tr'(\px)$, where
      $\tr' = \operatorname{fix}(\fn{\tr'}{\tr ⊔ [\py ↦ \semusg{\pe_1}_{\tr'}]})$.

      We know that
      \[
        \tr'(\px) = \tr(\px) \not⊑ \semusg{\pe}_{\tr} = \semusg{\pe_2}_{\tr'}
      \]
      So by the induction hypothesis, $\px$ is dead in $\pe_2$.

%      Now consider the predicate $P(\tr) = \tr(\px) ⊑ \semusg{\pe_1}_{\tr}$.
%      We must prove it admissable to see that it holds (by fixpoint induction)
%      for $\tr'$. That is clearly the case because it is a composition of
%      continuous functions ($\tr, \semusg{\pe_1}$) and admissable predicates
%      ($⊑$).
%
%      SG: I think we don't need to prove that P above is admissable because
%      we never try to prove it through fixpoint induction; we simply apply LEM.

      We proceed by cases over $\tr(\px) = \tr'(\px) ⊑ \semusg{\pe_1}_{\tr'}$.
      \begin{itemize}
        \item \textbf{Case $\tr'(\px) ⊑ \semusg{\pe_1}_{\tr'}$}: Then
          $\tr'(\px) ⊑ \tr'(\py)$ and $\py$ is also dead in $\pe_2$ by the above
          inequality.
          Both deadness facts together allow us to rewrite
          \[
            \semscott{\pe_2}_{ρ[\px↦d_1][\py↦d'_1]} = \semscott{\pe_2}_{ρ[\px↦d_1][\py↦d'_2]} = \semscott{\pe_2}_{ρ[\px↦d_2][\py↦d'_2]}
          \]
          as requested.
        \item \textbf{Case $\tr'(\px) \not⊑ \semusg{\pe_1}_{\tr'}$}:
          Then $\px$ is dead in $\pe_1$ and $d'_1 = d'_2$. The goal follows
          from the fact that $\px$ is dead in $\pe_2$.
      \end{itemize}
  \end{itemize}
\end{proof}

\begin{proof}[Proof of \Cref{thm:semusg-correct-live-3}]
  \label{prf:semusg-correct-live-3}
  By induction on $\pe$.
  We fix $\pe$, $\px$ and $\tr$ such that $\tr(\px) \not⊑ \semusg{\pe}_{\tr}$.
  The goal is to show that $\px$ is dead in $\pe$,
  so we are given an arbitrary $ρ, d_1, d_2, μ, \pa$ and have to show that
  $\semevt{\pe}_{ρ[\px↦\deref(\pa)]}(μ[\pa↦d_1]) = \semevt{\pe}_{ρ[\px↦\deref(\pa)]}(μ[\pa↦d_2])$.
  By cases on $\pe$:
  \begin{itemize}
    \item \textbf{Case $\pe\triangleq\py$}: If $\px=\py$, then
      $\semusg{\py}_{\tr} = \semusg{\px}_{\tr}$, a contradiction.
      If $\px \not= \py$, then varying the heap entry for $\pa$ won't matter,
      because $ρ(\py) \not= \deref(\pa)$ due to the freshness constraint; hence
      $\px$ is dead in $\py$.
    \item \textbf{Case $\pe\triangleq\Lam{\py}{\pe'}$}: The equality follows from
      pointwise equality on functions, so we pick arbitrary $μ, d^\later$ to show
      % Urgh, the μ that we start with is not the same as that reaches the subexpression.
      % It might already have evaluated d_1, d_2 which might cause them to eqaute.
      $\semevt{\pe'}_{ρ[\px↦\deref(\pa)][\py↦d^\later](μ[\pa↦])} = \semevt{\pe'}_{ρ[\px↦\deref(\pa)][\py↦d^\later]}$.

      This is simple to see if $\px=\py$. Otherwise, $\tr[\py↦\bot]$ witnesses the fact that
      \[
        \tr[\py↦\bot](\px) = \tr(\px) \not⊑
        \semusg{\Lam{\px}{\pe'}}_{\tr} = \semusg{\pe'}_{\tr[\py↦\bot]}
      \]
      so we can apply the induction hypothesis to see that $\px$ must be dead in
      $\pe'$, hence the equality on $\semevt{\pe'}$ holds.
    \item \textbf{Case $\pe\triangleq\pe'~\py$}:
      From $\tr(\px) \not⊑ \semusg{\pe'}_{\tr} + ω*\tr(\py)$ we can see that
      $\tr(\px) \not⊑ \semusg{\pe'}_{\tr}$ and $\tr(\px) \not⊑ \tr(\py)$ by
      monotonicity of $+$ and $*$.
      If $\px=\py$ then the latter inequality leads to a contradiction.
      Otherwise, $\px$ must be dead in $\pe'$, hence both cases of
      $\semevt{\pe'~\py}$ evaluate equally, differing only in
      the environment. It remains to be shown that
      $ρ[\px↦d_1](\py) = ρ[\px↦d_2](\py)$, and that is easy to see since
      $\px \not= \py$.
    \item \textbf{Case $\pe\triangleq\Let{\py}{\pe_1}{\pe_2}$}:
      We have to show that
      \[
        \semevt{\pe_2}_{ρ[\px↦d_1][\py↦d'_1]} = \semevt{\pe_2}_{ρ[\px↦d_2][\py↦d'_2]}
      \]
      where $d'_i$ satisfy $d'_i = \semevt{\pe_1}_{ρ[\px↦d_i][\py↦d'_i]}$.
      The case $\px = \py$ is simple to see, because $ρ[\px↦d_i](\px)$ is never
      looked at.
      So we assume $\px \not= \py$ and see that $\tr(\px) = \tr'(\px)$, where
      $\tr' = \operatorname{fix}(\fn{\tr'}{\tr ⊔ [\py ↦ \semusg{\pe_1}_{\tr'}]})$.

      We know that
      \[
        \tr'(\px) = \tr(\px) \not⊑ \semusg{\pe}_{\tr} = \semusg{\pe_2}_{\tr'}
      \]
      So by the induction hypothesis, $\px$ is dead in $\pe_2$.

%      Now consider the predicate $P(\tr) = \tr(\px) ⊑ \semusg{\pe_1}_{\tr}$.
%      We must prove it admissable to see that it holds (by fixpoint induction)
%      for $\tr'$. That is clearly the case because it is a composition of
%      continuous functions ($\tr, \semusg{\pe_1}$) and admissable predicates
%      ($⊑$).
%
%      SG: I think we don't need to prove that P above is admissable because
%      we never try to prove it through fixpoint induction; we simply apply LEM.

      We proceed by cases over $\tr(\px) = \tr'(\px) ⊑ \semusg{\pe_1}_{\tr'}$.
      \begin{itemize}
        \item \textbf{Case $\tr'(\px) ⊑ \semusg{\pe_1}_{\tr'}$}: Then
          $\tr'(\px) ⊑ \tr'(\py)$ and $\py$ is also dead in $\pe_2$ by the above
          inequality.
          Both deadness facts together allow us to rewrite
          \[
            \semevt{\pe_2}_{ρ[\px↦d_1][\py↦d'_1]} = \semevt{\pe_2}_{ρ[\px↦d_1][\py↦d'_2]} = \semevt{\pe_2}_{ρ[\px↦d_2][\py↦d'_2]}
          \]
          as requested.
        \item \textbf{Case $\tr'(\px) \not⊑ \semusg{\pe_1}_{\tr'}$}:
          Then $\px$ is dead in $\pe_1$ and $d'_1 = d'_2$. The goal follows
          from the fact that $\px$ is dead in $\pe_2$.
      \end{itemize}
  \end{itemize}
\end{proof}

\subsection{Bare-bones Denotational Semantics for Call-by-need}

\begin{figure}
\[\begin{array}{c}
 \begin{array}{rrclclrrclcl}
 \end{array} \\
 \begin{array}{rrclclrrcrclcl}
  \text{Environment}  & ρ   & ∈ & \Environments  & =      & \Var \pfun \BareD
  &
  \text{Heap}         & μ   & ∈ & \Heaps         & =      & \Addresses \pfun \later\BareD
  \\
  \\[-0.5em]
  \text{Trace} & τ      & ∈          & \BTraces & ::= & \goodend{v,μ} \mid \stuckend{} \mid \laterC(τ^{\later})
  &
  \multicolumn{3}{r}{\text{Delayed trace}} & τ^{\later} & ∈ & \later\BTraces &   &
  \\
  \text{Domain} & d & ∈ & \BareD & = & \Heaps \to \BTraces
  &
  \multicolumn{3}{r}{\text{Delayed element}} & d^{\later} & ∈ & \later\BareD &   &
  \\
  \\[-0.5em]
 \end{array} \\
 \begin{array}{rrclcl}
  \text{Value} & v & ∈ & \BareV & ::= & \FunV(f ∈ (\later\BareD \to \BareD)) \mid \ConV(K,\many{d^{\later}}^{α_K}) \\
 \end{array} \\
  \\[-0.5em]
\end{array}\]
\[\begin{array}{c}
 \begin{array}{rcl}
  \multicolumn{3}{c}{ \ruleform{
    \begin{array}{c}
      (\betastep) : \BTraces \to (\BareV \times \Heaps \pfun \BTraces) \to \BTraces \quad  \ret : \BareV \to \BareD \\
      \deref : \Addresses \to \BareD \quad \apply : \BTraces \to \BareD \to \BTraces \\
    \end{array}
  }} \\
  \\[-0.5em]
  τ \betastep f & = & \begin{cases}
      \laterC^n(f(v,μ)) & \text{$τ = \laterC^n(\goodend{v,μ}$) and $(v,μ) ∈ \dom(f)$} \\
      \laterC^n(\stuckend{}) & \text{$τ = \laterC^n(\goodend{v,μ}$) and $(v,μ) \not∈ \dom(f)$} \\
      τ & \text{otherwise} \\
    \end{cases} \\
  \\[-0.5em]
  \ret(\lbl,v)(μ) & = & \goodend{v,μ} \\
  \deref(\pa)(μ) & = & μ(\pa)(μ) \betastep \fn{(v,μ)}{\goodend{(v,μ[\pa ↦ \ret(v)])}} \\
 \end{array} \\
 \\[-0.5em]
 \begin{array}{rcl}
  \multicolumn{3}{c}{ \ruleform{ \sembare{\wild} \colon \Exp → (\Var \pfun \BareD) → \BareD } } \\
  \\[-0.5em]
  \sembare{\px}_ρ & = & \begin{cases}
    ρ(\px) & \px ∈ \dom(ρ) \\
    \stuckend{} & \text{otherwise}
    \end{cases} \\
  \\[-0.5em]
  \sembare{\Lam{\px}{\pe}}_ρ & = & \ret(\FunV(\fn{d^\later}{\sembare{\pe}_{ρ[\px↦d^\later]}})) \\
  \\[-0.5em]
  \sembare{\pe~\px}_ρ(μ) & = & \begin{cases}
      \sembare{\pe}_ρ(μ) \betastep \fn{(\FunV(f),μ)}{f(ρ(\px))(μ)} & \px ∈ \dom(ρ) \\
      \stuckend{} & \text{otherwise} \\
    \end{cases} \\
  \\[-0.5em]
  \sembare{\Let{\px}{\pe_1}{\pe_2}}_ρ(μ) & = & \begin{letarray}
    \text{let} & ρ' = ρ[\px ↦ \deref(\pa)] \quad \text{where $\pa \not∈ \dom(μ)$} \\
    \text{in}  & \sembare{\pe_2}_{ρ'}(μ[\pa ↦ \sembare{\pe_1}_{ρ'}]) \\
  \end{letarray} \\
  \\[-0.5em]
  \sembare{K~\many{\px}}_ρ & = & \ret(\ConV(K,\many{\sembare{\px}_ρ})) \\
  \\[-0.5em]
  \sembare{\Case{\pe_s}{\Sel[r]}}_ρ(μ) & = & \sembare{\pe_s}_ρ(μ) \betastep \fn{(\ConV(K',\many{d^\later}),μ)}{\sembare{\pe_r}_{ρ[\many{\px↦d^\later}]}(μ)} \\
                                     &   & \hspace{10em} \text{where } K_i = K' \\
 \end{array}
  \\[-0.5em]
\end{array}\]
\caption{Bare-bones Denotational Semantics $\sembare{-}$}
  \label{fig:sembare}
\end{figure}


\fi % \ifappendix

\end{document}

\section{Related Work}
\label{sec:related-work}

% Compare to
% - Operational semantics: CESK Felleisen, Launchbury, Sestoft, Krivine
%        eval/apply or push/enter?
%        Given an expr like $f x$, we first push $ρ(x)$ onto the stack and then
%        evaluate $f$, which will look it up (pushing an udpate frame) and evaluate its
%        RHS. Since we will never return to the "eval site" of $f$, IMO this qualifies
%        as push/enter rather than eval/apply. Which is in contrast to what the Krivine
%        paper says, which dubs return states as "apply" transitions
% - Clairvoyant blah, Garden of Forking Paths
% - Imprecise exceptions
%        There have been attempts to discern panices from other kinds of loops, such as
%        \citep{imprecise-exceptions}. Unfortunately, in Section 5.3 they find it
%        impossible give non-terminating programs a denotation other than $\bot$, which
%        still encompasses all possible exception behaviors.
% - coinductive big-step \citep{LeroyGrall:09}
% - Interaction trees
%        Mention how >>β= is similar to the bind operator on interaction trees and probably
%        corresponds to Moggi's monadic semantics.
%        Traces are somewhat like the simpler monad `M e a = Good a | Bad e | L |>(M e a)`
%        which combines the delay monad with the error monad
% - Definitional Interpreter work; we are using TCTT as the defining language
% - Pitts chapter in TAPL2, "largest congruence relation"
% - Nielson: correctness predicate
% - SGDT:
%     * Nakano:00 introduced modality
%     * DreyerAhmedBirkedal:11 refined and applied to step-indexing
%     * Birkedal:12 discovered how to hide step indices. Applied to System F like language with store. Later also depedent type theory. Connection to Kripke worlds
%     * BirkedalMogelbergEjlers:13 first described how to encode guarded recusive types ``syntactically'', e.g., as we use them in meta ::=-notation
%     * gdtt was all about lifting to dependent product types containing the later modality. For example `f <*> t` has to substitute `t` in `f`'s type, solved via delayed substitutions.
%       Important for properties! Hence guaded DEPENDENT type theory
%     * Guarded Cubical Type Theory: Reasoning about equality in GDTT can be undecidable, fix that. But no clock qunatification!
%     * Clocks are ticking: Clock quantification via tick binders which act similar to intervals in Cubical. (Still uses those delayed substitutions!)
%     * TCTT not only introduces ticked cubical, it also covers bisimilarity of guarded labelled transition systems. Our traces are just that, plus a bit more.
%       Also shows how to define a mixed guarded/inductive data type.
%     * Later credits: May solve the unguarded positive occurrence
% - AAM
%     * Our approach at some point simply gets rid of the heap. AAM bounds it to finite size.
%       That is inelegant, as it would never allow us to treat variables as a non-aliased representative of their address
%     * AAM

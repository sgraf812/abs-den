\section{Abstract Interpretation}
\label{sec:abstractions}

\subsection{Lazy Denotational Deadness}

Let us now finally try to reformulate semantic deadness in terms of
$\semevt{\wild}$ and $\cong$:

\begin{definition}[Denotational deadness, lazily]
  \label{defn:deadness3}
  An address $\pa$ is \emph{dead} in a denotation $\semevt{\pe}_ρ$
  (written $\dead(\pa,\semevt{\pe}_ρ)$)
  if and only if, for all $d$,%
  \footnote{Minor detail: This $d$ is assumed to be of the form
  $\memo(\pa,\semevt{\pe'}_{ρ'})$, which is not much of a restriction in
  practice.}
  we have
  \[
    (ρ_\cong=ρ) \wedge \dead(\pa,μ_\cong) \wedge \pa \not∈ A_\cong ⊦ \semevt{\pe}_{ρ_\cong}(μ_\cong) \cong \semevt{\pe}_{ρ_\cong}(μ_\cong[\pa↦d])
  \]
  (See \Cref{defn:sem-equiv-abbrev} for notation.)
  An address $\pa$ is \emph{dead} in a heap $μ$ (written $\dead(\pa,μ)$) if and only if
  \[
    \later(\forall \pa' ∈ \dom(μ).\ ∃\pe',ρ'.\ μ(\pa')=\memo(\pa',\semevt{\pe'}_{ρ'}) \wedge (\pa' \not=\pa \Longrightarrow \dead(\pa,\semevt{\pe'}_{ρ'})))
  \]
  A variable $\px$ is \emph{dead} in an expression $\pe$ if and only if
  \[
    \forall ρ.\ \pa \not∈ \rng(ρ) \Longrightarrow \dead(\pa,\semevt{\pe}_{ρ[\px↦\pa]})
  \]
  Otherwise, $\pa/\px$ is \emph{live}.
\end{definition}

\begin{lemmarep}[Deadness implies irrelevance]
  \label{thm:dead-irrelevant}
  If $\px$ is dead in $\pe$,
  then for all $\pe_1,\pe_2$,
  \[\Let{\px}{\pe_1}{\pe} \equiv \Let{\px}{\pe_2}{\pe}.\]
\end{lemmarep}
\begin{proof}
  Assume that $\px$ is dead in $\pe$. The goal is to show
  \[
    \semevt{\Let{\px}{\pe_1}{\pe}} \cong \semevt{\Let{\px}{\pe_2}{\pe}},
  \]
  so we unfold the let case of $\semevt{\wild}$ and simplify to
  \[
    \semevt{\pe}_{ρ_\cong[\px↦\pa]}(μ_\cong[\pa↦d_1]) \cong \semevt{\pe}_{ρ_\cong[\px↦\pa]}(μ_\cong[\pa↦d_2]).
  \]
  where $d_i \triangleq \memo(\pa,\semevt{\pe_i}_{ρ_\cong[\px↦\pa]})$, $\pa \not∈ \rng(ρ_\cong)$,
  % TODO: Lemma: May drop arbitrary many steps
  $\pa \not∈ \dom(μ_\cong)$ and the additional $\BindE$ step has no effect on
  divergence and hence was dropped.
  % TODO: Lemma: No occ implies deadness
  Since $\pa$ cannot occur in $μ_\cong$, we have $\dead(\pa,μ_\cong[\pa↦d_1])$.
  Since $\px$ is assumed dead in $\pe$, $\pa$ is dead in
  $\semevt{\pe}_{ρ_\cong[\px↦\pa]}$ and we can apply to $d \triangleq d_2$ to
  show the goal.
\end{proof}

So if $x$ is dead in $\pe_2$, we can justify the following rewrite by
irrelevance:
\[
  \Let{x}{\pe_1}{\pe_2} \equiv \Let{x}{\mathit{panic}}{\pe_2},
\]
where $\mathit{panic}$ denotes a small term that is immediately stuck.
A syntactic \emph{occurrence analysis} could subsequently figure out whether the
binding for $x$ can be dropped without introducing scoping errors, a feat
that becomes far simpler once other huge, dead expressions in $\pe_2$ that close
over $x$ have been turned into small $\mathit{panic}$s.

\begin{toappendix}
It is convenient to abbreviate alias-freeness for $\px$, as follows:
\begin{abbreviation}
  We write $\noalias(\px,\pa,ρ)$ to mean that
  $\forall \py.\ ρ(\py) = \pa \Longleftrightarrow \py = \px$, \eg, that there
  is no other variable in $ρ$ with the same address $\pa$ as $\px$.
\end{abbreviation}

We can use $\noalias$ to simplify the statement $\dead(\px,\pe)$:

\begin{lemma}
  Let $\px$ be a variable and $\pe$ an expression.
  If for all $\pa,d$,
  \[
    \noalias(\px,\pa,ρ_\cong) \wedge \dead(\pa,μ_\cong) \wedge \pa \not∈ A_\cong ⊦ \semevt{\pe}_{ρ_\cong}(μ_\cong) \cong \semevt{\pe}_{ρ_\cong}(μ_\cong[\pa↦d]),
  \]
  then $\px$ is dead in $\pe$.
\end{lemma}

Let us furthermore abbreviate this antecedent into a predicate for ``good contexts'':
\begin{abbreviation}
  We abbreviate
  \[\begin{array}{lcl}
    \ok(\px,\pa,ρ,μ,A) & \triangleq & \noalias(\px,\pa,ρ) \wedge \dead(\pa,μ) \wedge \pa \not∈ A \\
    \ok(\px,\pE) & \triangleq & \ok(\px,ρ(\pE)(\px),ρ(\pE),μ(\pE),A(\pE)) \\
  \end{array}\]
  where $ρ(\pE),μ(\pE),A(\pE)$ are defined as in \Cref{defn:sem-equiv}.
\end{abbreviation}

\begin{definition}
  \label{defn:dead-ctx}
  A variable is dead in a context $\pE$ if and only if whenever $\px$ is dead in
  an expression $\pe$, then $\px$ is dead in $\pE[\pe]$.
\end{definition}

\begin{lemma}[OK-ness-preserving contexts are dead]
  \label{thm:ok-ctx-dead}
  Let $\pE$ be an evaluation context and $\px$ a variable.

  If for all contexts $\pE'$ such that $\ok(\px,\pE')$ it is $\ok(\px,\pE'[\pE])$
  then $\px$ is dead in $\pE$.
\end{lemma}
\begin{proof}
Note that whenever $\ok(\px,\pa,ρ,μ,A)$ then also
$\ok(\px,\pa,ρ,μ[\pa↦d],A)$ for any $d$.
We will use this fact below as $(\star)$.

We abbreviate $\diverges(τ) \triangleq \len(τ) = ω$.

We assume that $\px$ dead in some expression $\pe$ and show
\begin{DispWithArrows*}[fleqn,mathindent=1em]
                        & \ok(\px,\pa,ρ_\cong,μ_\cong,A_\cong) ⊦ \semevt{\pe}_{ρ_\cong}(μ_\cong) \cong \semevt{\pe}_{ρ_\cong}(μ_\cong[\pa↦d]) \Arrow{Unfold $(\cong)$} \\
  \Longleftrightarrow{} & \forall \pE_1.\ \ok(\px,\pE_1) \Longrightarrow \left(\begin{array}{ll}
                                                & \diverges(\semevt{\pE_1}(\fn{ρ\,μ}{\semevt{\pe}_ρ(μ)})_{[]}([])) \\
                            \Longleftrightarrow & \diverges(\semevt{\pE_1}(\fn{ρ\,μ}{\semevt{\pe}_ρ(μ[\pa↦d])})_{[]}([]))
                          \end{array}\right)
                          \Arrow{$\pE_1 \triangleq \pE_2[\pE]$} \\
  \Longrightarrow{}     & \forall \pE_2.\ \ok(\px,\pE_2) \Longrightarrow \left(\begin{array}{ll}
                                                & \diverges(\semevt{\pE_2}(\semevt{\pE}(\fn{ρ\,μ}{\semevt{\pe}_ρ(μ)}))_{[]}([])) \\
                            \Longleftrightarrow & \diverges(\semevt{\pE_2}(\semevt{\pE}(\fn{ρ\,μ}{\semevt{\pe}_ρ(μ[\pa↦d])}))_{[]}([]))
                          \end{array}\right)
                          \Arrow{$(\star)$} \\
  \Longleftrightarrow{} & \forall \pE_2.\ \ok(\px,\pE_2) \Longrightarrow \left(\begin{array}{ll}
                                                & \diverges(\semevt{\pE_2}(\fn{ρ\,μ}{\semevt{\pE}(\semevt{\pe})_ρ(μ)})_{[]}([])) \\
                            \Longleftrightarrow & \diverges(\semevt{\pE_2}(\fn{ρ\,μ}{\semevt{\pE}(\semevt{\pe})_ρ(μ[\pa↦d])})_{[]}([]))
                          \end{array}\right)
                          \Arrow{Fold $(\cong)$} \\
                        & \ok(\px,\pa,ρ_\cong,μ_\cong,A_\cong) ⊦ \semevt{\pE}(\semevt{\pe})_{ρ_\cong}(μ_\cong) \cong \semevt{\pE}(\semevt{\pe})_{ρ_\cong}(μ_\cong[\pa↦d]) \Arrow{\Cref{thm:semevt-comp}} \\
                        & \ok(\px,\pa,ρ_\cong,μ_\cong,A_\cong) ⊦ \semevt{\pE[\pe]}_{ρ_\cong}(μ_\cong) \cong \semevt{\pE[\pe]}_{ρ_\cong}(μ_\cong[\pa↦d])
\end{DispWithArrows*}
So $\px$ is dead in $\pE[\pe]$, too, hence $\px$ is dead in $\pE$.
\end{proof}

\begin{lemma}[Dead let 1]
  \label{thm:dead-let-1}
  If $\px$ is dead in $\pe_1$ and $\pe_2$ and $\px \not= \py$, then $\px$ is also
  dead in $\Let{\py}{\pe_1}{\pe_2}$.
\end{lemma}
\begin{proof}
We show that $\px$ is dead in the context $\Let{\py}{\pe_1}{\hole}$ by
\Cref{thm:ok-ctx-dead} to show the goal.

So fix $ρ,μ,\pa,A$ such that $\ok(\px,\pa,ρ,μ,A)$.
We will show that all calls to $S$ in $\semevt{\Let{\py}{\pe_1}{\hole}}(S)_ρ(μ)$
are with some $ρ',μ'$ such that $\ok(\px,\pa,ρ',μ',A)$.

Hence we unfold the definition of $\semevt{\Let{\py}{\pe_1}{\hole}}(S)_ρ(μ)$ and
see the call $S(ρ[\py↦\pa'])(μ[\pa'↦\memo(\pa',\semevt{\pe_1}_{ρ[\py↦\pa']})])$.
Since $\py \not= \px$ and $\pa' \not∈ \dom(μ)$, we have $\noalias(\px,\pa,ρ[\py↦\pa'])$.
Furthermore, since $\px$ is dead in $\pe_1$, we have that
$\pa$ is dead in $\semevt{\pe_1}_{ρ[\py↦\pa']}$; hence
$\dead(\pa,μ[\pa'↦\memo(\pa',\semevt{\pe_1}_{ρ[\py↦\pa']})])$.
$A$ is unchanged and hence $\pa \not∈ A$.
Thus
\[
  \ok(\px,\pa,ρ[\py↦\pa'],μ[\pa'↦\memo(\pa',\semevt{\pe_1}_{ρ[\py↦\pa']})],A),
\]
showing the goal.
\end{proof}

\begin{lemma}[Dead let 2]
  \label{thm:dead-let-2}
  If $\px$ and $\py$ are dead in $\pe_2$,
  then $\px$ is also dead in $\Let{\py}{\pe_1}{\pe_2}$.
\end{lemma}
\begin{proof}
  The following reasoning proves that $\px$ is dead in
  $\Let{\py}{\pe_1}{\pe_2}$:
  \[\begin{array}{l}
    \noalias(\px,\pa,ρ_\cong) \wedge \dead(\pa,μ_\cong) \wedge \pa \not∈ A_\cong \\
    \begin{WithArrows}[format=rl]
      \quad ⊦    {} & \semevt{\Let{\py}{\pe_1'}{\pe_2'}}_{ρ_\cong[\px↦\pa]}(μ_\cong) \Arrow{Unfold $\semevt{\wild}$} \\
      \quad \cong{} & ε \cons \semevt{\pe_2'}_{ρ_\cong[\px↦\pa,\py↦\pa']}(μ_\cong[\pa'↦d_1']) \Arrow{$\dead(\py,\pe_2')$} \\
      \quad \cong{} & ε \cons \semevt{\pe_2'}_{ρ_\cong[\px↦\pa,\py↦\pa']}(μ_\cong[\pa'↦d_{\bullet}]) \Arrow{$\dead(\pa,d_{\bullet})$} \\
      \quad \cong{} & ε \cons \semevt{\pe_2'}_{ρ_\cong[\px↦\pa,\py↦\pa']}(μ_\cong[\pa↦d,\pa'↦d_{\bullet}]) \Arrow{$\dead(\py,\pe_2')$} \\
      \quad \cong{} & ε \cons \semevt{\pe_2'}_{ρ_\cong[\px↦\pa,\py↦\pa']}(μ_\cong[\pa↦d,\pa'↦d_1']) \Arrow{Refold $\semevt{\wild}$} \\
      \quad \cong{} & \semevt{\Let{\py}{\pe_1'}{\pe_2'}}_{ρ_\cong[\px↦\pa]}(μ_\cong[\pa↦d]
    \end{WithArrows}
  \end{array}\]
  where $\pa' \not∈ \dom(μ_\cong)$
  (thus $\ok(\py,\pa',ρ_\cong[\py↦\pa'],μ_\cong[\pa'↦d_1'],A)$)
  and for some $ε$, $d_1'$ and a $d_{\bullet}$ that $\pa$ is dead in, for
  example $\memo(\pa',\semevt{\Lam{z}{z}}_{[]})$.
  Thus, $\px$ is dead in $\pe$.
\end{proof}

\begin{lemma}[Dead application]
  \label{thm:dead-app}
  If $\px$ is dead in $\pe$ and $\px \not= \py$, then $\px$ is also
  dead in $\pe~\py$.
\end{lemma}
\begin{proof}
We show that $\px$ is dead in the context $\hole~\py$ by
\Cref{thm:ok-ctx-dead} to show the goal.

So fix $ρ,μ,\pa,A$ such that $\ok(\px,\pa,ρ,μ,A)$
and unfold the definition of $\semevt{\hole~\py}(S)_ρ(μ)$ to see that $S$ is
called at the same $ρ,μ$.
Additionally, $\hole~\py$ adds an innermost frame that passes $ρ(\py)$.
But since $\px \not= \py$, we have $\pa = ρ(\px) \not= ρ(\py)$, so
$\pa \not∈ ρ(\py) ∪ A$.
It follows that $\ok(\px,\pa,ρ,μ,ρ(\py) ∪ A)$; the goal.
\end{proof}

\begin{lemma}
  If $\px$ is dead in $\pe_1$ and $\pe_1 \equiv \pe_2$, then $\px$ is dead in $\pe_2$.
\end{lemma}
\begin{proof}
  Note that when $P_1$ implies $P_2$,
  $P_2(ρ_\cong,μ_\cong,A_\cong) ⊦ S_1 \cong S_2$ implies $P_1(ρ_\cong,μ_\cong,A_\cong) ⊦ S_1 \cong S_2$.
  The proposition follows by the following chain of implications, where $P$ is chosen freely:
  \[
    \pe_1 \equiv \pe_2 \Longleftrightarrow \mathsf{true} ⊦ \semevt{\pe_1} \cong \semevt{\pe_1} \Longrightarrow P(ρ_\cong,μ_\cong,A_\cong) ⊦ \semevt{\pe_1} \cong \semevt{\pe_1}
  \]
  We can use this identity to rewrite $\pe_1$ in the definition of deadness to
  $\pe_2$ on both sides of $\wild ⊦ \wild \cong \wild$ by transitivity.

  We will make liberal use of this result to maintain a high level of
  abstraction in proofs.
\end{proof}

\begin{lemma}
  \label{thm:var-dead}
  $\px$ is dead in $\py$ if $\px \not= \py$.
\end{lemma}
\begin{proof}
We fix $\pa,d$ and show
\[
  \noalias(\px,\pa,ρ_\cong) \wedge \dead(\pa,μ_\cong) \wedge \pa \not∈ A_\cong ⊦ \semevt{\py}_{ρ_\cong[\px↦\pa]}(μ_\cong) \cong \semevt{\py}_{ρ_\cong[\px↦\pa]}(μ_\cong[\pa↦d])
\]
The stuck case is simple to see.
In the interesting case, note that $\px \not= \py$ implies
$\pa' \triangleq ρ_\cong(\py) \not= \pa$ because $\noalias(\px,\pa,ρ_\cong)$.
We can discard the lookup step and simplify the goal to
\[
  \noalias(\px,\pa,ρ_\cong) \wedge \dead(\pa,μ_\cong) \wedge \pa \not∈ A_\cong ⊦ \later(\memo(\pa',\semevt{\pe'}_{ρ'})(μ_\cong) \cong \memo(\pa',\semevt{\pe'}_{ρ'})(μ_\cong[\pa↦d]))
\]
for some $\semevt{\pe'}_{ρ'}$ that $\pa$ is known to be dead in.

Without loss of generality, we may assume that $\pe'$ is a value $\pv$, because
$\pe' \equiv \pv$ for whatever value $\pe'$ reduces to (NB: the equivalence
holds trivially when $\pe'$ diverges), and we can apply this rewrite to both
sides of $(\cong)$.
(Both sides need an equal number of transitions because $\pa$ is dead in
$\semevt{\pe'}_{ρ'}$.)

For a value $\pv$, $\memo$ has no observable effect:
It will overwrite the heap entry at $\pa'$ with $\memo(\pa',\semevt{\pv}_ρ')$,
definitionally equal to the previous entry at $\pa'$.
Thus we can simply discard the extra transition on both sides. The goal becomes
\[
  \noalias(\px,\pa,ρ_\cong) \wedge \dead(\pa,μ_\cong) \wedge \pa \not∈ A_\cong ⊦ \later(\semevt{\pv}_{ρ'})(μ_\cong) \cong \semevt{\pv}_{ρ'})(μ_\cong[\pa↦d]))
\]
and that follows simply from deadness of $\pa$ in $\semevt{\pv}_{ρ'}$.
\end{proof}
\end{toappendix}

\begin{theoremrep}[$\semusg{\wild}$ is a correct deadness analysis]
  \label{thm:semusg-correct-live-3}
  Let $\pe$ be an expression and $\px$ a variable. \\
  If $\semusg{\pe}_{\tr_Δ}(\px) = 0$
  then $\px$ is dead in $\pe$.
\end{theoremrep}
\begin{proof}
The setting is the same as in \Cref{thm:semusg-correct-live}:
We proceed by induction on $\pe$, generalising $\tr_Δ$ in the assumption
$\semusg{\pe}_{\tr_Δ}(\px) = 0$ to ``there exists a $\tr$ such that
$\tr(\px) \not⊑ \semusg{\pe}_{\tr}$'' with the goal to show that
$\px$ is dead in $\pe$.
Stuck cases are boring and hence omitted.
\begin{itemize}
  \item \textbf{Case $\pe = \py$}:
    If $\px=\py$, then $\semusg{\py}_{\tr} = \semusg{\px}_{\tr}$,
    contradicting the inequality.
    If $\px \not= \py$, then we can apply \Cref{thm:var-dead}.

  \item \textbf{Case $\pe = \Let{\py}{\pe_1'}{\pe_2'}$}:
    Since there is no shadowing, it is $\px \not= \py$ and we see that
    $\tr(\px) = \tr'(\px)$, where
    $\tr' \triangleq {\operatorname{fix}(\fn{\tr'}{\tr ⊔ [\py ↦
    [\py↦1]+\semusg{\pe_1'}_{\tr'}]})}$.
    We know that
    \[
      \tr'(\px) = \tr(\px) \not⊑ \semusg{\pe}_{\tr} = \semusg{\pe_2'}_{\tr'}
    \]
    but we cannot apply the induction hypothesis yet, because we cannot
    float out the binding for $\py$ because $\pe_1'$ might be free in $\px$.

    We proceed by cases over $\tr(\px) = \tr'(\px) ⊑ \semusg{\pe_1'}_{\tr'}$.
    \begin{itemize}
      \item \textbf{Case $\tr'(\px) \not⊑ \semusg{\pe_1}_{\tr'}$}:
        Then $\px$ is dead both in $\pe_1'$ and $\pe_2'$ by the induction
        hypothesis and the result follows by \Cref{thm:dead-let-1}.
      \item \textbf{Case $\tr'(\px) ⊑ \semusg{\pe_1'}_{\tr'}$}: Then
        $\tr'(\px) ⊑ \tr'(\py)$ and $\py$ is also dead in $\pe_2'$ by the above
        inequality and the result follows by \Cref{thm:dead-let-2}.
    \end{itemize}

  \item \textbf{Case $\pe = \pe'~\py$}:
    Since $\tr(\px) \not⊑ \tr(\py)$, we have $\px \not= \py$ and
    hence $ρ(\px) \not= ρ(\py)$.
    It is also $\tr(\px) \not⊑ \semusg{\pe'}_\tr$, so $\px$ is dead in $\pe'$.

    To show that $\px$ is dead in $\pe'~\py$ it suffices to show that $\px$
    is dead in the context $\hole~\py$, and that follows by \Cref{thm:dead-app}.

  \item \textbf{Case $\pe = \Lam{\py}{\pe'}$}:
    Note that $\px \not= \py$ because there is no shadowing.
    Since $\tr(\px) \not⊑ \semusg{\pe}_{\tr}$, we also have
    \[
      \tr(\px) = \tr[\py↦⊥](\px) \not⊑ \semusg{\pe'}_{\tr[\py↦⊥]} ⊑ \semusg{\pe}_{\tr}.
    \]
    By induction, $\px$ is dead in $\pe'$.

    We need to show that $\px$ is dead in $\Lam{\py}{\pe'}$, so we
    fix some $\pa,d$ and show
    \[
      \ok(\px,\pa,ρ_\cong,μ_\cong,A_\cong) ⊦ \semevt{\Lam{\py}{\pe'}}_{ρ_\cong}(μ_\cong) \cong \semevt{\Lam{\py}{\pe'}}_{ρ_\cong}(μ_\cong[\pa↦d])
    \]
    Unfolding $(\cong)$ yields
    \[
      \forall \pE.\ \ok(\px,\pE) \Longrightarrow \left(\begin{array}{ll}
                            & \diverges(\semevt{\pE}(\fn{ρ\,μ}{\semevt{\Lam{\py}{\pe'}}_ρ(μ)})_{[]}([])) \\
        \Longleftrightarrow & \diverges(\semevt{\pE}(\fn{ρ\,μ}{\semevt{\Lam{\py}{\pe'}}_ρ(μ[\pa↦d])})_{[]}([]))
      \end{array}\right)
    \]
    which we will show by induction on the size of $\pE$, doing a case
    distinction on the innermost frame.
    \begin{itemize}
      \item \textbf{Case $\pE = \hole$}: Then both sides do exactly 0 steps.
      \item \textbf{Case $\pE = \pE'[\hole~\pz]$}:
        Then we have
        \[
          \left(\begin{array}{ll}
                                      & \len(\semevt{\pE'}(\semevt{\hole~\pz}(\fn{ρ\,μ}{\semevt{\Lam{\py}{\pe'}}_ρ(μ)}))_{[]}([])) = ω \\
                  \Longleftrightarrow & \len(\semevt{\pE'}(\semevt{\hole~\pz}(\fn{ρ\,μ}{\semevt{\Lam{\py}{\pe'}}_ρ(μ[\pa↦d])}))_{[]}([])) = ω
                \end{array}\right)
        \]
        Now we unfold $\semevt{\hole~\pz}$ and $\semevt{\Lam{\py}{\pe'}}$ and simplify:
        \[
          \left(\begin{array}{ll}
                                      & \len(\semevt{\pE'}(\fn{ρ\,μ}{ε \cons \semevt{\pe'}_{ρ[\py↦ρ(\pz)]}(μ)}))_{[]}([])) = ω \\
                  \Longleftrightarrow & \len(\semevt{\pE'}(\fn{ρ\,μ}{ε \cons \semevt{\pe'}_{ρ[\py↦ρ(\pz)]}(μ[\pa↦d])}))_{[]}([])) = ω
                \end{array}\right)
        \]
        The additional transition can be discarded on both sides.
        By $\ok(\px,\pE)$ we have that $ρ(\px) \not= ρ(\pz)$, implying that
        $\noalias(\px,\pa,ρ[\py↦ρ(\pz)])$.
        Hence the above equivalence follows inductively by deadness of $\pe'$.
      \item \textbf{Case $\pE = \pE_1[\Let{\pz}{\pe''}{\hole}]$}:
        This frame just extends the heap with another dead binding ($\ok(\px,\pE)$,
        so follows by induction on $\pE_1$.
      \item \textbf{Case $\pE = \pE_1[\Let{\pz}{\hole}{\pE_2[\pz]}]$}:
        By Löb induction, we assume that $\later(\dead(\px,\Lam{\py}{\pe'}))$
        to show that $\dead(\px,\Lam{\py}{\pe'})$ (which is our goal anyway).
        (Note that we only need the deadness condition on $\Lam{\py}{\pe'}$
        \emph{later}, for $\dead(\pa,μ)$.)
        Then $\Let{\pz}{\Lam{\py}{\pe'}}{\hole}$ preserves OK-ness of $\pE_1$
        and we have $\ok(\px,\pE_1[\Let{\pz}{\Lam{\py}{\pe'}}{\pE_2[\hole]}])$.
        But $\px$ is dead in $\pz$ by \Cref{thm:var-dead}, and so $\px$ is dead
        in $\pE_1[\Let{\pz}{\Lam{\py}{\pe'}}{\pE_2[\pz]}]$.
    \end{itemize}
\end{itemize}
\end{proof}

% OLD

\subsection{Discussion}

The proof of \Cref{thm:semusg-correct-live-3} is quite a bit more involved than
\Cref{thm:semusg-correct-live}, so we declare only partial victory on Goal 1
from \Cref{sec:problem}.
On the other hand, we prove a stronger property that would not be observable in
our simpler call-by-name semantics.
For obvious reasons, it seems preferable to stick to the simpler semantics if
the property in question (\eg, deadness) can be understood there as well.

Still, with \Cref{defn:deadness3} we were at least able to break
down the proof into manageable intermediate steps.
That is a huge step forward compared to the operational deadness definition of
\Cref{defn:deadness2} where we weren't even able to come up with a suitable
correctness relation.

In hindsight, we could trace back the intermediate steps to come up with at
least one proof strategy with \Cref{defn:deadness2}:
The structural induction principle is an important enabling factor and
the focus on maximal traces in \Cref{fig:semvan-correctness} suggests that
we should likely strive for a correctness relation characterising a property of
maximal traces.
This was not obvious to us when we first set out to do the proof; the gap was
too large to see how to get to the other side due to the structural mismatch.
A nice consequence of successfully avoiding said mismatch, which we set
out to in Goal 4.

\subsection{Improvement}

Proving that dead bindings can be soundly rewritten is a nice litmus test
for the semantics.
But does it really make the program faster, or at least not slower?

We can affirm that indirectly:
A diverging program $\mathit{loop}$ takes more steps than a stuck program
$\mathit{panic}$, hence the former runs ``slower'' than the latter.
The stuck and the diverging program are not semantically equivalent,
but $\Let{x}{\mathit{panic}}{\pe}$ is semantically equivalent to
$\Let{x}{\mathit{loop}}{\pe}$ when $x$ is dead in $\pe$.
Since $(\betastep)$ runs sub-programs to completion, we would observe
execution of ${\mathit{panic}}$ or ${\mathit{loop}}$, which is the only
way in which we could have made the program slower or faster.
Since there is no semantic difference, the performance of the program must be
unaffected.

This argument is quite vague.
In order to put it on firm ground, we define an \emph{improvement relation}
$(\lessequiv)$ in the style of \citet{MoranSands:99} in \Cref{fig:improv}.

\begin{figure}
\[\begin{array}{c}
 \ruleform{ A ⊦_n τ_1 \lesssim τ_2 \qquad A ⊦_n (v_{1}, μ_{1}) \lessdot (v_2, μ_{2}) \qquad A ⊦_n μ_1 \lessapprox μ_2 \qquad d_1 \lessequiv d_2 }
 \\
 \\[-0.75em]
 \inferrule*[right=\implrcons]
    {\later (A ⊦_n τ_1 \lesssim τ_2)}
    {A ⊦_{n} \wild \cons τ_1 \lesssim \wild \cons τ_2}
 \hspace{0.75em}
 \inferrule*[right=\imprcons]
    {\later (A ⊦_{n+1} τ_1 \lesssim τ_2) \hspace{1ex} \fin(τ_2)}
    {A ⊦_{n} τ_1 \lesssim \wild \cons τ_2}
 \hspace{0.75em}
 \inferrule*[right=\implcons]
    {\later (A ⊦_{n} τ_1 \lesssim τ_2) \hspace{1ex} \fin(τ_1)}
    {A ⊦_{n+1} \wild \cons τ_1 \lesssim τ_2}
 \\
 \\[-0.75em]
 \inferrule*[right=\impstuck]
    {\quad}
    {A ⊦_{0} \stuckend{} \lesssim \stuckend{}}
 \quad
 \inferrule*[right=\impval]
    {\pa \not\in \dom(μ_{i}) \quad A ⊦_{n} (v_{1},μ_{1}) \lessdot (v_{2}, μ_{2})
     \\\\ A ∪ \{\pa\} ⊦_{n} μ_{1}[\pa\mapsto\memo(\pa,\ret(v_{1}))] \lessapprox μ_{2}[\pa\mapsto\memo(\pa,\ret(v_{2}))]}
    {A ⊦_{n} \goodend{v_{1}, μ_1} \lesssim \goodend{v_{2}, μ_2}}
 \\
 \\[-0.75em]
 \inferrule*[right=\impcon]
    {\later (\many{A ⊦_{n} μ_1(\pa_1)(μ_{1}) \lesssim μ_2(\pa_2)(μ_{2})})}
    {A ⊦_n (\ConV(K, \many{\pa_1}), μ_1)\mathrm{\lessdot} (\ConV(K, \many{\pa_2}), μ_2)}
 \hspace{0.75em}
 \inferrule*[right=\impfun]
    {\forall \pa ∈ A.\  \later (A ⊦_{n} f_1(\pa)(μ_1) \mathrm{\lesssim} f_2(\pa)(μ_2))}
    {A ⊦_{n} (\FunV(f_1), μ_1) \lessdot (\FunV(f_2), μ_2)}
 \\
 \\[-0.75em]
 \inferrule*[right=\impheap]
    {\forall \pa \in A.\ \later(A ⊦_{n} μ_1(\pa)(μ_1) \lesssim μ_2(\pa)(μ_2))}
    {A ⊦_n μ_1 \lessapprox μ_2}
 \\
 \\[-0.75em]
 \inferrule*[right=\impdenot]
    {\forall μ_1,μ_2.\ \dom(μ_{1}) \vdash_{0} μ_1 \lessapprox μ_2 \wedge  \dom(μ_1) ⊦ d_1,d_2 ⟹  \dom(μ_1) ⊦_{0} d_1(μ_1) \lesssim d_2(μ_2)}
    {d_1 \lessequiv d_2}
\end{array}\]
\vspace{-1em}
\caption{Improvement relation}
\label{fig:improv}
\end{figure}

Structurally, $(\lessequiv)$ is similar to $(\equiv)$, with a few differences necessitated
by the precise counting of steps.
In the rules $(\cons)$, we accumulate \emph{skew credits} $n∈ℕ$ in applications
of $\imprcons$ to spend them on $\implcons$ or when a value is further
scrutinised.
Naturally, it is easier to prove $A ⊦_n τ_1 \lesssim τ_2$ the more credits we
have at our expense and thus the larger $n$ is.
The intricacies of sharing are embodied in rule $\impval$
where we add the value returned by the trace to the heap under a fresh address.
In the $\impheap$ antecedent, this address is added to $A$ and hence available
to the observer for repeated scrutinisation or application to a function.

We write $d_1 \impequiv d_2$ when $d_1 \lessequiv d_2$ and $d_2 \lessequiv d_1$, in
which case both denotations operate in lockstep.

%Already it is the case that $d_1 \lessequiv d_2$ implies $d_1 \equiv d_2$,
%so $(\lessequiv)$ corresponds to the strong notion of improvement in
%\citet{MoranSands:99}.
%The weaker notion can be recovered by defining $\imprcons$ and $\implcons$ by
%coinduction, thus accepting skew credits in $ℕ_ω$ instead of $ℕ$.
%
It is quite a bit harder to agree that this improvement relation is a sub-relation
of the strong contextual improvement relation of \citet{MoranSands:99} and we
leave the equivalent of \Cref{thm:sem-equiv-cont-equiv} to the Appendix.

We can now once again refine our notion of deadness using lockstep
simulation.
It is then easy to see that $\semusg{\wild}$ is correct \wrt to this stronger
notion of deadness, because we can in large parts reuse the proof for
\Cref{thm:semusg-correct-live-3}.
\ifappendix
\else
We give an account of that in the Appendix.
\fi

\begin{toappendix}
As our first action, we have to refine our notion of lazy domain elements.
Specifically, in the Postpone and Speculate properties, we additionally require
that the bigstep traces $\bigstep{d}{μ_2}{v_2}{μ_2'}$ take at most as many
steps as $\bigstep{d}{μ_1}{v_1}{μ_1'}$.
More formally, we write $\bigstepn{n_{i}}{d}{μ_i}{v_i}{μ_i'}$ to indicate that
the derivations take $n_{i}$ steps; then $n_{2} \leq n_{1}$.

Likewise, we have to change the definition of extensionality to encompass
$\impsim$ rather than $\sim$ and prove that $\semevt{\pe}_ρ$ is extensional;
a routine change.
From this it follows that $\lessequiv$ is a preorder and $\impequiv$ is an
equivalence relation.
As in \citet[Section 3.4]{MoranSands:99}, this equivalence relation is contained
in $\equiv$.

With this updated definition, we go on to revise a number of auxiliary lemmas.

\begin{lemmarep}[Weakening of address spaces]
  \label{thm:weaken-address-space-imp}
  Let $A_1,A_2$ be address spaces such that $A_1 ⊆ A_2$. \\
  For all $d$, if $A_1 ⊦ d$ then $A_2 ⊦ d$.
  For all traces $τ_1,τ_2$, if $A_2 ⊦_0 τ_1 \impsim τ_2$ then $A_1 ⊦_0 τ_1 \impsim τ_2$.
\end{lemmarep}
\begin{proof}
  Same as \Cref{thm:weaken-address-space}.
\end{proof}

\begin{lemmarep}[Weakening of skew credits]
  \label{thm:weaken-skew-credits}
  Let $A$ be an address space, $n_1 \leq n_2$.
  For all traces $τ_1,τ_2$, if $A ⊦_{n_1} τ_1 \impsim τ_2$ then $A ⊦_{n_2} τ_1 \impsim τ_2$.
\end{lemmarep}
\begin{proof}
  Any judgment rule in $A ⊦_{n_1} τ_1 \impsim τ_2$ can be re-applied relative to
  $n_2$.
  That is particularly true for applications of $\implcons$, for which the
  converse proposition ($n_1 > n_2$) would \emph{not} hold.
\end{proof}

\begin{lemmarep}
  \label{thm:lazy-force-lockstep}
  Let $μ_1,μ_2$ be lazy heaps with $μ_1 \forcesto μ_2$. Then
  \begin{enumerate}
    \item For all lazy $d$ such that $\dom(μ_{1}) ⊦ d$, we have $\dom(μ_1) ⊦_0 d(μ_1) \lesssim d(μ_2)$.
    \item For all $\pa ∈ \dom(μ_1)$, we have $\dom(μ_1) ⊦_0 μ_1(\pa)(μ_1) \lesssim μ_2(\pa)(μ_2)$.
  \end{enumerate}
\end{lemmarep}
\begin{proof}
%  This lemma makes essential use of the Postpone and Speculate properties.
  By Löb induction.
  \begin{enumerate}
    \item
%      Note that we are free to make use of the latter because $d$ is defined
%      at $μ_1$.
      If either $d(μ_1)$ or $d(μ_2)$ is ultimately stuck, then so is the other,
      because it neither diverges nor evaluates.
      In that case, we see that $\dom(μ_1) ⊦_0 d(μ_1) \lesssim d(μ_2)$.

      Similarly, if either $d(μ_1)$ or $d(μ_2)$ diverges, then so does the
      other, in which case $\dom(μ_1) ⊦_0 d(μ_1) \lesssim d(μ_2)$.

      If either $\bigstepn{n_{1}}{d}{μ_1}{v}{μ_1'}$ or $\bigstepn{n_{2}}{d}{μ_2}{v}{μ_2'}$
      then both judgments hold by the Speculate and Postpone properties (note
      that $d$ is defined on $μ_1$ because $\dom(μ_1) ⊦ d$) and we have $μ_1'
      \forcesto μ_2'$ as well as $n_2 \leq n_1$.
      By \Cref{thm:weaken-skew-credits} it suffices to show
      ${\dom(μ_1) ⊦_0 \goodend{v,μ_1'} \lesssim \goodend{v,μ_2'}}$.
      We do so by applying rule $\impval$ and fix some $\pa \not\in \dom(μ_i)$,
      postponing the $\lessdot$ goal and focusing on the $\impheap$ goal first.

      Clearly, $μ_{1}'[\pa ↦ \memo(\pa, \ret(v))] \forcesto μ_{2}'[\pa ↦ \memo(\pa, \ret(v))]$.
      Writing $μ_{i}'' \triangleq μ_{1}'[\pa ↦ \memo(\pa, \ret(v))]$,
      by (2) of the inductive hypothesis we get that
      $\dom(μ_{1}'') ⊦_{0} μ_1''(\pa)(μ_1'') \lesssim μ_2''(\pa)(μ_2'')$.
      We can relax the address space to $\dom(μ_1) ∪ \{ \pa \}$ by
      \Cref{thm:weaken-address-space-imp},
      so $\dom(μ_1) ∪ \{ \pa \} ⊦_{0} μ_{1}'' \lessapprox μ_{2}''$ follows.

      For the $v=\ConV(K,\many{\pa})$ case,
      we have $\later (\many{\dom(μ_1') ⊦_0 μ_1'(\pa')(μ_1') \lesssim μ_2'(\pa')(μ_2')})$
      for all $\pa' ∈ \dom(μ_1')$ by (2) of the induction hypothesis applied to $μ_1' \forcesto μ_2'$.
      We can relax the address space from $\dom(μ_1')$ to $\dom(μ_1)$ via
      \Cref{thm:weaken-address-space-imp} and apply $\impcon$ to show the goal.

      For the $v=\FunV(f)$ case, we fix an arbitrary $\pa ∈ \dom(μ_1)$ and have
      to show that $\later (\dom(μ_1) ⊦_0 f(\pa)(μ_1') \lesssim f(\pa)(μ_2'))$.
      But $f(\pa)$ is lazy, and
      we have $\later (\many{\dom(μ_1') ⊦_0 f(\pa)(μ_1') \lesssim f(\pa)(μ_2')})$
      by (1) of the induction hypothesis applied to $μ_1' \forcesto μ_2'$.
      Again, we may relax the address space from $\dom(μ_1')$ to $\dom(μ_1)$ via
      \Cref{thm:weaken-address-space} and apply $\impfun$ to show the goal.
    \item
      If $μ_1(\pa) = μ_2(\pa)$ then (2) follows by a similar argument as (1),
      so let $μ_1(\pa) \not= μ_2(\pa)$.
      From $μ_1 \forcesto μ_2$ we have $\bigstepn{n_1}{μ_1(\pa)}{μ_1}{v}{μ_1'}$
      for some $v,μ_1'$, $μ_2(\pa) = \memo(\pa,\ret(v))$ and $μ_1' \forcesto μ_2$.
      This implies that $\bigstepn{1}{μ_2(\pa)}{μ_2}{v}{μ_2}$.
      Since $μ_1(\pa)$ is of the form $\memo(\pa,d_1)$, it will perform at least
      one step to memoise; hence $n_1 \geq 1 = n_2$, and with
      \Cref{thm:weaken-skew-credits} it suffices to show
      ${\dom(μ_1) ⊦_0 \goodend{v,μ_1'} \lesssim \goodend{v,μ_2}}$.
      This follows by a similar argument as in (1), knowing that
      $μ_1' \forcesto μ_2$.
  \end{enumerate}
\end{proof}

\begin{theorem}
  For any address space $A$, $A ⊦_{0} \wild \lesssim \wild$ is a preorder on lazy traces.
\end{theorem}
\begin{proof}
  Reflexivity is easy to see.
  The proof for transitivity is similar to the proof for \Cref{thm:sim-equiv}.
  Skew credits are simple to incorporate.
  The new antecedent in $\impval$ is a bit harder, but ultimately follows from
  applying the induction hypothesis.
\end{proof}

With the assumed adjustment to extensionality discussed above, we conclude

\begin{theorem}
  $(\lessequiv)$ is a preorder on lazy and extensional elements in $\EventD$.
\end{theorem}
\begin{proof}
  Similar to \Cref{thm:equiv-equiv}.
  Note that the extension of $μ_1$ and $μ_3$ to $\dom(μ_2)$ (where $\dom(μ_1)
  ⊦_0 μ_1 \lessapprox μ_3$) can happen such that
  $\dom(μ_1') ⊦_{0} μ_1' \lessapprox μ_3'$, by extending both $μ_1$ and $μ_3$ with
  the same entries.
\end{proof}

\begin{lemma}
  \label{thm:memo-lessapprox}
  Let $A ⊦_n μ_1 \lessapprox μ_2$, $\pa \not∈ \dom(μ_i)$ and $d_1 \lessequiv d_2$.
  Then $A ⊦_n μ_1[\pa↦\memo(\pa,d_1)] \lessapprox μ_2[\pa↦\memo(\pa,d_2)]$.
\end{lemma}
\begin{proof}
  Similar to \Cref{thm:memo-approx}.
\end{proof}

\begin{theorem}
  For all environments $ρ$ and expressions $\pe$,
  $\semevt{\pe}_ρ$ is extensional.
\end{theorem}
\begin{proof}
Mostly the same as \Cref{thm:semevt-ext}. By induction on $\pe$.
\begin{itemize}
  \item \textbf{Case $\px$}:
    $\fn{\wild}{\stuckend{}}$ is extensional in case $\px \not∈ \dom(ρ)$.

    Let $\dom(μ_{1})  ⊦_{0} μ_1 \lessapprox μ_2$ and $ρ(\px) = \pa ∈ \dom(μ_i)$.
    (The latter inclusion follows by the Address Domain property.)
    Then the goal is to show that
    \[
      \later (\dom(μ_1) ⊦ μ_1(\pa)(μ_1) \lesssim μ_2(\pa)(μ_2))
    \]
    in order to apply $\eqlrcons$ and that follows simply by $\dom(μ_{1}) ⊦ μ_1 \approx μ_2$.

  \item \textbf{Case $\Lam{\px}{\pe'}$}:
    Let $\dom(μ_{1})  ⊦_{0} μ_1 \lessapprox μ_2$. The goal is to show
    \[
      \dom(μ_1) ⊦_{0} \goodend{v,μ_1} \lesssim \goodend{v,μ_2}
    \]
    where $v \triangleq \FunV(\fn{\pa}{\AppEE(\px↦\pa) \cons \semevt{\pe'}_{ρ[\px↦\pa]}})$.

    In order to apply rule $\impval$, we have two proof obligations:
    The first is, given a fresh $\pa \not\in \dom(μ_{1})$, we have to show that
    \[
      \dom(μ_1) \cup \{\pa\} ⊦ μ_{1}[\pa ↦ \memo(\pa, \ret(v))] \lessapprox μ_{2}[\pa ↦ \memo(\pa, \ret(v))]
    \]
    % Continue here
    So let us fix $\pa ∈ \dom(μ_1)$ and show
    \[
      \dom(μ_1) \cup \{\pa\} ⊦ (v, μ_{1}) \lessdot (v, μ_{2})
    \]
    \[
      \dom(μ_1) ⊦ \semevt{\pe'}_{ρ[\px↦\pa]}(μ_1) \sim \semevt{\pe'}_{ρ[\px↦\pa]}(μ_2),
    \]
    which follows inductively by extensionality of $\semevt{\pe'}_{ρ[\px↦\pa]}$.

  \item \textbf{Case $K~\many{\px}$}:
    Similar to the lambda case, exploiting $μ_1 \approx μ_2$ for the field addresses.

  \item \textbf{Case $\pe'~\px$}:
    $\fn{\wild}{\stuckend{}}$ is extensional in case $\px \not∈ \dom(ρ)$.

    Otherwise, by induction it suffices to show that for extensional
    $d \triangleq \semevt{\pe'}_ρ$, the $\apply$ expression
    $d \betastep \fn{(\FunV(f))}{f(ρ(\px))}$ is extensional.

    Let $μ_1 \approx μ_2$ such that $\dom(μ_1) ⊦ \semevt{\pe}_ρ$.
    The interesting case is when $\bigstep{d}{μ_i}{\FunV(f_i)}{μ'_i}$.
    We have to show that $\dom(μ_1) ⊦ f_1(ρ(\px))(μ_1') \sim f_2(ρ(\px))(μ_2')$,
    the rest follows by congruence.
    From extensionality of $d$ (which is the result of applying the inductive hypothesis),
    we know by $\eqfun$ that $\dom(μ_1) ⊦ f_1(\pa)(μ_1') \sim f_2(\pa)(μ_2')$
    for any $\pa$ such that $\pa ∈ \dom(μ_1)$.
    Since $ρ(\px) ∈ \dom(μ_1)$ by the Address Domain property, we can show the goal.

  \item \textbf{Case $\Case{\pe_s}{\Sel[r]}$}:
    Similar to the application case.

  \item \textbf{Case $\Let{\px}{\pe_1}{\pe_2}$}:
    Let $μ_1 \equiv μ_2$, $\pa \not∈ \dom(μ_1)$ and
    abbreviate
    $ρ' \triangleq ρ[\px↦\pa]$,
    $d \triangleq \semevt{\pe_1}_{ρ'}$,
    $μ_i' \triangleq μ_i[\pa ↦ \memo(\pa,d)]$.

    For extensionality of $\semevt{\Let{\px}{\pe_1}{\pe_2}}_ρ$, we have to show
    \[
       \dom(μ_1) ⊦ \semevt{\pe_2}_{ρ'}(μ_1') \sim \semevt{\pe_2}_{ρ'}(μ_2')
    \]

    Since $d$ is extensional by the inductive hypothesis (so $d \equiv d$),
    we have $μ_1' \approx μ_2'$ by \Cref{thm:memo-approx}.
    We apply the inductive hypothesis once more to have
    extensionality of $\semevt{\pe_2}_{ρ'}$.
    Applying that to $μ_1' \approx μ_2'$, we have
    \[
       \dom(μ_1') ⊦ \semevt{\pe_2}_{ρ'}(μ_1') \sim \semevt{\pe_2}_{ρ'}(μ_2')
    \]
    and we can weaken the address space to $\dom(μ_1)$ by
    \Cref{thm:weaken-address-space} to show the goal.
\end{itemize}
\end{proof}

\begin{definition}[Strong contextual improvement, as in \citep{MoranSands:99}]
  We say that $\pe_2$ contextually improves $\pe_1$ (written $\pe_2
  \lessequiv_{\mathit{ctx}} \pe_1$) if for all evaluation contexts $\pE$ such
  that $\pE[\pe_i]$ are closed,
  \[
     \len(\semevt{\pE[\pe_2]}_{[]}([])) \leq \len(\semevt{\pE[\pe_1]}_{[]}([]))
  \]
  and $\pe_1 \equiv_{\mathit{ctx}} \pe_2$.
\end{definition}

\begin{definition}[Denotational deadness, improving]
  \label{defn:deadness-imp}
  An address $\pa$ is \emph{dead} in a denotation $d$ if and only if,
  for all $\dom(μ_{1}) ⊦_{0} μ_1 \impapprox μ_2$ such that $\pa$ is dead in $\rng(μ_i)$ later,
  $\dom(μ_1) ∪ \{\pa\} ⊦ d$ and $d_1,d_2$, we have
  \[
    \dom(μ_1) ⊦_0 d(μ_1[\pa↦d_1]) \impsim d(μ_2[\pa↦d_2]).
  \]
  A variable $\px$ is \emph{dead} in an expression $\pe$ if and only if
  any $\pa$ is dead in $\semevt{\pe}_{ρ[\px↦\pa]}$ for any $ρ$ such
  that $\pa \not∈ \rng(ρ)$.
  Otherwise, $\px$ is \emph{live}.
\end{definition}

\begin{lemma}[Heap partitioning, improving]
  \label{thm:heap-partitioning-imp}
  Let $μ$ be a heap and $\pa ∈ \dom(μ)$.
  Then there exists a heap $μ_h$ with domain $\dom(μ) \setminus \{ \pa \}$
  and $\dom(μ) ⊦_{0} μ \impapprox μ_h[\pa↦μ(\pa)]$.
\end{lemma}
\begin{proof}
  We use the same $\mathit{hide}_\pa$ function as in
  \Cref{thm:heap-partitioning} and define $μ_h \triangleq \mathit{hide}_\pa(μ)$.
  By Löb induction, we prove
  \[
    \dom(μ) ⊦_{0} μ \impapprox μ_h[\pa↦μ(\pa)]
  \]
  (We have already proven that $μ_h$ is defined, lazy and extensional.)
  So we assume that the property holds later.

  For $\dom(μ) ⊦_{0} μ \impapprox μ_h[\pa↦μ(\pa)]$, we additionally have to consider $\pa' = \pa$.
  In that case, $\dom(μ) ⊦ μ(\pa)(μ) \impsim μ(\pa)(μ_h[\pa↦μ(\pa)])$ (later)
  follows by extensionality of $μ(\pa)$ applied to the induction hypothesis.
\end{proof}

\begin{lemma}
  \label{thm:adom-dead}
  Let $A ⊦ d$. Then any $\pa \not∈ A$ is dead in $d$.
\end{lemma}
\begin{proof}
  Assume that $A$, $\pa$ and $d$ are chosen as in the proposition.
  To show that $\pa$ is dead in $d$, assume $μ_1 \approx μ_2$ such
  that $\pa$ is dead in $\rng(μ_i)$ and $\dom(μ_1) ∪ \{\pa\} ⊦ d$
  and fix arbitrary $d_1,d_2$.
  We have to show that
  \[
    \dom(μ_1) ⊦ d(μ_1[\pa↦d_1]) \sim d(μ_2[\pa↦d_2])
  \]

  W.l.o.g., we assume that $A ⊆ \dom(μ_i)$.
  Otherwise, we can extend $μ_i$ to heaps $μ_i'$ (mapping to equal entries, for
  example $d$) such that $A ⊆ \dom(μ_i')$.
  Then $μ_i \forcesto μ_i'$ and we can apply \Cref{thm:lazy-force-bisimilar}.

  By \Cref{thm:heap-partitioning}, there exist $μ_i'$ such that $μ_i[\pa↦d_i] \approx μ_i'[\pa↦d_i]$
  and $\pa \not∈ \dom(μ_i')$, so $μ_i' \forcesto μ_i'[\pa↦d_i]$.
  By extensionality of $d$, we know that
  \[
    \dom(μ_1') ⊦ d(μ_1') \sim d(μ_2')
  \]
  Since $A ⊆ \dom(μ_i')$ and $A ⊦ d$, we can apply \Cref{thm:lazy-force-bisimilar}
  to $μ_i' \forcesto μ_i'[\pa↦d_i]$ to have
  \[
    \dom(μ_1') ⊦ d(μ_1'[\pa↦d_i]) \sim d(μ_2'[\pa↦d_i])
  \]
  And again by extensionality applied to $μ_i[\pa↦d_i] \approx μ_i'[\pa↦d_i]$
  and transitivity, we show the goal.
\end{proof}

\end{toappendix}

\subsection{Evaluation Cardinality}

Since our new semantics is able to express evaluation cardinality and thunk
update, we may add a new $\mathbf{let1}$ construct to our language that opts out
of memoisation:
\[
 \begin{array}{rcl}
  ε ∈ \Events   & ::= & ... \mid \BindOE(\px,\pa↦d) \\
  \\[-0.5em]
  \semevt{\Letn{\px}{\pe_1}{\pe_2}}_ρ(μ) & = &
    \begin{letarray}
      \text{let} & ρ' = ρ[\px ↦ \pa] \quad \text{where $\pa \not∈ \dom(μ)$} \\
      \text{in}  & \BindOE(\px,\pa↦\semevt{\pe_1}_{ρ'}) \cons \semevt{\pe_2}_{ρ'}(μ[\pa ↦ \highlight{\semevt{\pe_1}_{ρ'}}])
    \end{letarray} \\
 \end{array}
\]
Any program in which we switch from memoised $\mathbf{let}$ to $\mathbf{let1}$
is semantically equivalent after we adjust the definition of lazy heaps
accordingly.
This is a simple consequence of the fact that
$μ[\pa↦\memo(\pa,d)] \approx μ[\pa↦d]$ and compositionality.

However, omitting thunk memoisation has measurable effect on performance
if the same variable is evaluated repeatedly!
We should rather show that whenever $x$ is \emph{evaluated at most
once}, it is an improvement to rewrite $\Let{x}{\pe_1}{\pe_2}$ to
$\Letn{x}{\pe_1}{\pe_2}$.

We can sharpen this statement by making use of \emph{tick algebra}
\citep{MoranSands:99}.
For that, we need to add a notion of ticks to
our language (postfix, to mirror $\memo$), a routine extension:
\[
 \begin{array}{rcl}
  ε ∈ \Events   & ::= & ... \mid \TickE \\
  \\[-0.5em]
  \semevt{\pe \tick}_ρ(μ) & = & \semevt{\pe}_{ρ}(μ) \betastep \fn{v}{\fn{μ}{\TickE \cons \goodend{v,μ}}} \\
 \end{array}
\]
% TODO: Fishy
Whenever $x$ is ``evaluated at most once'', we
have $\semevt{\Let{x}{\pe_1}{\pe_2}}_ρ \impequiv
      \semevt{\Letn{x}{\pe_1\tick}{\pe_2}}_ρ$.
In \Cref{sec:problem}, specifically \Cref{thm:semusg-correct-2}, we have
proclaimed that $\pe$ evaluates $x$ at most once whenever
$\semusg{\pe}_{\tr_Δ}(x) ⊑ 1$.
We can now make good on that claim:

\begin{toappendix}
\begin{figure}
\[\begin{array}{c}
 \ruleform{ \correct_L(τ_1,τ_2) }
 \\
 \\[-0.5em]
 \inferrule*[right=\corstuck]
    {\quad}
    {\correct_L(\stuckend{},\stuckend{})}
 \quad
 \inferrule*[right=\corcons]
    {\later (\correct_L(τ_1,τ_2))}
    {\correct_L(\wild :: τ_1,\wild :: τ_2)}
 \\
 \\[-0.5em]
 \inferrule*[right=\corfun]
    {\dom(μ_1) = \dom(μ_2) \quad \forall \pa ∈ (\dom(μ_i) \setminus L).\ \correct_L(f_1(\pa)(μ_1), f_2(\pa)(μ_2))}
    {\correct_L(\goodend{\FunV(f_1),μ_1},\goodend{\FunV(f_2),μ_2})}
 \\
 \\[-0.5em]
\end{array}\]
\caption{Correctness relation for \Cref{thm:semusg-by-name}}
  \label{fig:semusg-correct}
\end{figure}

\begin{lemma}
  \label{thm:pe1-dead}
  %If $ω*\tr(\px) \not⊑ \semusg{\Let{\px}{\pe_1}{\pe_2}}_{\tr}$, then $\px$ is
  If $\semusg{\Let{\px}{\pe_1}{\pe_2}}_{\tr_Δ}(\px) ⊑ 1$, then $\px$ is
  dead in either $\pe_1$ or $\pe_2$.
\end{lemma}
\begin{proof}
  Unfold $\semusg{\wild}$ once and consider the case that $\px$ is not dead in $\pe_1$,
  so we have $\tr(\px) ⊑ \semusg{\pe_1}_{\tr}$.

  We have
  $ω*\tr_1(\px) \not⊑ \semusg{\pe_2}_{\tr_1}$ where
  $\tr_1 \triangleq \fix(\fn{\tr_1}{\tr ⊔ [\px ↦ [\px↦1]+\semusg{\pe_1}_{\tr_1}]})$.

  Clearly, $\tr_1(\px) = ω$, so
  $\tr_1(\px) \not⊑ \semusg{\pe_2}_{\tr_1}$, so $\px$ must be dead in $\pe_2$.
\end{proof}

\begin{definition}
  $d$ is \emph{$L$-once} if and only if, for all $μ$ such that $\rng(μ)$ is
  $L$-once, $L$ is dead in $μ(\dom(μ) \setminus L)$ and for all $\pa ∈ L$ such
  that $\pa$ is dead in $d'$,
  \[
    \correct_{L}(d(μ[\pa↦ d'\tick]), d(μ[\pa↦\memo(\pa,d')])).
  \]
  (Where $(d'\tick )(μ) \triangleq d'(μ) \betastep \fn{v}{\fn{μ}{\TickE \cons \goodend{v,μ}}}$.)
\end{definition}
\end{toappendix}

\begin{theoremrep}[Update avoidance for $\semusg{\wild}$]
  \label{thm:semusg-by-name}
  Let $\Let{\px}{\pe_1}{\pe_2}$ be an expression
  such that \\
  ${\semusg{\pe_1}_{\tr_Δ}(\px) + \semusg{\pe_2}_{\tr_Δ}(\px) ⊑ 1}$.
  Then
    $\semevt{\Let{\px}{\pe_1}{\pe_2}}_ρ \impequiv
     \semevt{\Letn{\px}{\pe_1\tick}{\pe_2}}_ρ$.
\end{theoremrep}
\begin{proof}
  Unless explicitly stated otherwise, we will always apply the improvement
  judgments in a symmetrical manner.
  That is, whenever we have to prove $a ⊑ b$, the proof should also hold for
  $b ⊑ a$ (where $(⊑)$ is either of $(\lessequiv),(\lessapprox),(\lesssim)$).

  Since $u_1 + u_2 ⊑ 1$ implies that either $u_i$ must $0$ and the other $1$,
  either $\px$ must be dead in $\pe_1$ or $\pe_2$.
  If $\px$ is dead in $\pe_2$, the rewrite follows by deadness.
  So we assume that $\px$ is dead in $\pe_1$.

  We prove $(\impequiv)$ by rule $\impdenot$.
  After unfolding the definitions of $\semevt{\wild}$ for $\mathbf{let}$ and
  $\mathbf{let1}$, the goal is to show
  \[
    \dom(μ_i) ⊦_0 \semevt{\pe_2}_{ρ[\px↦\pa]}(μ_1[\pa↦d_1]) \impsim \semevt{\pe_2}_{ρ[\px↦\pa]}(μ_2[\pa↦d_2])
  \]
  where $\pa \not∈\dom(μ_i)$,
  $\dom(μ_i) ∪ \{\pa\} ⊦ \semevt{\pe_2}_{ρ[\px↦\pa]}$,
  $\dom(μ_{1}) ⊦_{0} μ_1 \impapprox μ_2$.
  Clearly,
  \[
    \dom(μ_i) ⊦_0 \semevt{\pe_2}_{ρ[\px↦\pa]}(μ_1[\pa↦\highlight{d_2}]) \impsim \semevt{\pe_2}_{ρ[\px↦\pa]}(μ_2[\pa↦d_2])
  \]
  holds by reflexivity because $\dom(μ_{1}) ∪ \{\pa\} ⊦_{0} μ_1[\pa↦d_2] \impapprox μ_2[\pa↦d_2]$, so the goal becomes
  \[
    \dom(μ_i) ⊦_0 τ_1 \impsim τ_2,
  \]
  abbreviating $τ_i \triangleq \semevt{\pe_2}_{ρ[\px↦\pa]}(μ_{\highlight{\scriptstyle 1}}[\pa↦d_i])$.

  \noindent
  We will prove the correctness predicate $\correct_{\{\pa\}}(τ_1,τ_2)$ instead, where
  $\correct$ is defined in \Cref{fig:semusg-correct}.

  Since $\pa \not ∈ \dom(μ_1)$,
  $\corfun$ implies $\impfun$ on $\dom(μ_i)$ and the goal follows.

  We generalise our assumption about $\semusg{\wild}$ as follows:
  There exists a $\tr$ (\ie, $\tr_Δ$) such that
  $ω*\tr(\px) \not⊑ \semusg{\Let{\px}{\pe_1}{\pe_2}}_{\tr}$.

  Then
  $ω*\tr_1(\px) \not⊑ \semusg{\pe_2}_{\tr_1}$ where
  $\tr_1 \triangleq \fix(\fn{\tr_1}{\tr ⊔ [\px ↦ [\px↦1]+\semusg{\pe_1}_{\tr_1}]})$.
  Let us also abbreviate $ρ_1 \triangleq ρ[\px↦\pa]$.

  We show that $\semevt{\pe_2}_{ρ[\px↦\pa]}$ is $\{\pa\}$-once
  assuming that $ω*\tr_1(\px) \not⊑ \semusg{\pe_2}_{\tr_1}$.
  By induction on $\pe_2$, generalising to $L$-onceness and
  maintaining that $\tr_1(\px) \not⊑ \tr_1(\py) \Longleftrightarrow ρ(\py) \not∈ L$
  for all $\py∈ \dom(ρ)$.
  \begin{itemize}
    \item \textbf{Case $\pe_2 = \py$}:
      Fix $μ$ such that $\rng(μ)$ is $L$-once and for all $\pa' ∈ L$ such that $\pa'$ is dead in $d'$,
      we need to show that
      \[
        \correct_{L}(μ_1(ρ(\py))(μ_1), μ_2(ρ(\py))(μ_2)).
      \]
      where $μ_1 \triangleq μ[\pa'↦d'\tick ], μ_2 = μ[\pa'↦\memo(\pa',d')]$.
      This is easy to see when $ρ(\py) \not= \pa'$, so assume $ρ(\py) = \pa'$:
      \[
        \correct_{L}(d'(μ_1) \cons \TickE, d'(μ_2) \betastep \fn{v}{(\fn{μ'}{\UpdateE(...) \cons \goodend{v,μ'[\pa'↦\memo(\pa',\ret(v))]}})}).
      \]
      We can strip away the trailing transition, leaving behind just the heap update.

      Now, without loss of generality, we will assume that $L$ is dead in
      $\rng(μ)$.
      If that was not the case, we could apply \Cref{heap-partitioning}
      to hide all entries in which $L$ is live ``inside $d'$'' and allocate them
      on first execution.
      % Turn into lemma. Sounds useful

      When $L$ is dead in $\rng(μ)$, we can rewrite the entry for $\pa'$ freely
      without affecting $\correct$.
      In the interesting case we have $\bigstep{d'}{μ_i}{\FunV(f_i)}{μ_i'}$.
      Because of deadness, we must have
      \[
        \correct_{L}(d'(μ_1), d'(μ_2)).
      \]
      That implies
      \[
        \correct_{L}(f_1(\pa_a)(μ_1'), f_2(\pa_a)(μ_2')).
      \]
      for all $\pa_a ∈ \dom(μ_i') \setminus L$.
      Now it suffices to show the situation after heap update
      \begin{equation}
        \label{eqn:usg-var-goal}
        \correct_{L}(f_2(\pa_a)(μ_2'), f_2(\pa_a)(μ_2'[\pa'↦\memo(\pa',\ret(\FunV(f_2)))])).
      \end{equation}
      In particular, note that
      $μ_2 \forcesto μ_3 \triangleq μ[\pa'↦\memo(\pa',\ret(\FunV(f_2)))]$
      (again, we can properly hide any free address that $f_2$ needs in
      $μ_3(\pa')$), so $\bigstep{d'}{μ_3}{\FunV(f_2)}{μ_3'}$ with
      $μ_2' \forcesto μ_3'$ by the Postpone property.
      By deadness we can follow
      \[
        \correct_{L}(d'(μ_2), d'(μ_3)).
      \]
      and since $\bigstep{d'}{μ_3}{\FunV(f_2)}{μ_3'}$ (NB: $f_2$ due to the
      forcing relationship), that implies
      \[
        \correct_{L}(f_2(\pa_a)(μ_2'), f_2(\pa_a)(μ_3')).
      \]
      for all $\pa_a ∈ \dom(μ_i') \setminus L$.

      Due to forcing, it must be that $μ_3'(\pa) = μ_3(\pa)$.
      Again by forcing, $μ_2'(\pa)$ must be either
      $μ_2(\pa)$ or $μ_3'(\pa)$.

      Since the number of steps taken for $f_2(\pa_a)(μ_2'[\pa↦μ_3'(\pa)])$ must be between $f_2(\pa_a)(μ_2')$
      and $f_2(\pa_a)(μ_3')$ (due to forcing), this shows the goal \Cref{eqn:usg-var-goal}.

    \item \textbf{Case $\pe_2 = \pe~\py$}:
      Fix $μ$ such that $\rng(μ)$ is $L$-once and for all $\pa' ∈ L$ such that $\pa'$ is dead in $d'$,
      we need to show that
      \[
        \correct_{L}(\semevt{\pe~\py}_{ρ_1}(μ[\pa'↦ d'\tick]), \semevt{\pe~\py}_{ρ_1}(μ[\pa'↦\memo(\pa',d')])).
      \]

      From $ω*\tr_1(\px) \not⊑ \semusg{\pe}_{\tr_1} + ω*\tr_1(\py)$ we can see that
      $ω*\tr_1(\px) \not⊑ \semusg{\pe}_{\tr_1}$ and $\tr_1(\px) \not⊑ \tr_1(\py)$ by
      monotonicity of $+$ and $*$.
      By induction, we know that the correctness statement holds for
      $\semevt{\pe}_{ρ_1}$.
      We also know that $ρ(\py) \not∈ L$ because $\tr_1(\px) \not⊑ \tr_1(\py)$.

      In the interesting case that $\bigstep{\semevt{\pe}_{ρ_1}}{μ[\pa'↦...]}{\FunV(f_i)}{μ_i'}$,
      we have
      \[
        \correct_{L}(f_1(ρ(\py))(μ_1), f_2(ρ(\py))(μ_2))
      \]
      for all $\pa' ∈ L$, because $ρ(\py) \not∈ L$.
      This shows the goal.

    \item \textbf{Case $\pe_2 = \Lam{\py}{\pe}$}:
      Fix $μ$ such that $\rng(μ)$ is $L$-once and for all $\pa' ∈ L$ such that $\pa'$ is dead in $d'$.
      To apply $\corfun$, we need to show that
      \[
        \correct_{L}(\semevt{\pe}_{ρ_1[\py↦\pa_a]}(μ[\pa'↦ d'\tick]), \semevt{\pe}_{ρ_1[\py↦\pa_a]}(μ[\pa'↦\memo(\pa,d')]))
      \]
      for all $\pa_a ∈ (\dom(μ) \setminus L)$.

      Since $\pa_a \not∈ L$ we maintain the invariant on $ρ_1[\py↦\pa_a]$ and
      $\tr_1(\px) \not⊑ \bot = \tr_1[\py↦\bot](\py)$ and may apply the induction hypothesis
      to the above situation since $ω*\tr_1[\py↦\bot](\px) \not⊑ \semusg{\pe}_{\tr_1[\py↦⊥]}$.
      This shows the goal.

    \item \textbf{Case $\pe_2 = \Let{\py}{\pe_1'}{\pe_2'}$}:
      Fix $μ$ such that $\rng(μ)$ is $L$-once and for all $\pa' ∈ L$ such that $\pa'$ is dead in $d'$.
      We need to show that
      \[
        \correct_{L}(\semevt{\Let{\py}{\pe_1'}{\pe_2'}}_{ρ_1}(μ[\pa'↦ d'\tick]), \semevt{\Let{\py}{\pe_1'}{\pe_2'}}_{ρ_1}(μ[\pa'↦\memo(\pa,d')])).
      \]
      We unfold the definition of $\semevt{\wild}$ and see that we need to prove
      \[
        \correct_{L}(\semevt{\pe_2'}_{ρ_2}(μ[\pa'↦ d'\tick][\pa_l↦d_1]), \semevt{\pe_2'}_{ρ_2}(μ[\pa'↦\memo(\pa,d')][\pa_l↦d_1])),
      \]
      where $ρ_2 \triangleq ρ_1[\py↦\pa_l]$,
      $d_1 \triangleq \memo(\pa_l,\semevt{\pe_1'}_{ρ_1[\py↦\pa_l]})$.

      From $\px \not= \py$ (otherwise shadowing) we see that
      $\tr_1(\px) = \tr_2(\px)$, \\
      where
      ${\tr_2 \triangleq \fix(\fn{\tr_2}{\tr_1 ⊔ [\py ↦
      [\py↦1]+\semusg{\pe_1'}_{\tr_2}]})}$.
      It is clear that $ω*\tr_2(\px) = ω*\tr_1(\px) \not⊑ \semusg{\pe_2}_{\tr_1} = \semusg{\pe_2'}_{\tr_2}$,
      so we want try to apply the induction hypothesis to this situation,
      but that requires us to show that the invariant on $L$ is satisfied
      as well as that $d_1$ is $L$-once.

      We proceed by cases over $\tr_1(\px) = \tr_2(\px) ⊑ \semusg{\pe_1'}_{\tr_2}$.
      \begin{itemize}
        \item \textbf{Case $ω*\tr_2(\px) ⊑ \semusg{\pe_1'}_{\tr_2}$}:
          Then $ω*\tr_2(\px) ⊑ \tr_2(\py) \not⊑ \semusg{\pe_2'}_{\tr_2}$ and
          hence $\py$ is dead in $\pe_2'$, which allows us to rewrite its bindings
          in $μ$ with something that is dead in $L$.
          (It is easy to see that deadness is compatible with $\correct$ in this
          way.)
          That again makes it possible to apply the induction hypothesis
          (noting that now $\tr(\px) \not⊑ \tr_2(\py)$ and hence $\pa_l \not∈ L$),
          showing the goal.
        \item \textbf{Case $\tr_2(\px) \not⊑ \semusg{\pe_1'}_{\tr_2}$}:
          Then $\px$ is dead in $\pe_1$, $\tr_2(\px) \not⊑  \tr_2(\py)$.
          This implies that $L$ is dead in $\semevt{\pe_1'}_{ρ_2}$:
          For any $\pa'$, either there exists $\py$ such that $\pa' = ρ(\py)$
          in which case $\tr_2(\px) ⊑  \tr_2(\py)$ leads to a contradiction.
          Otherwise, there is no variable in scope that refers to $\pa'$ and hence
          $\pa'$ is dead in $\semevt{\pe_1'}_{ρ_2}$ by \Cref{thm:addr-dom-sem}.

          From $\tr_2(\px) \not⊑ \tr_2(\py)$ we conclude $ρ_2(\py) = \pa_l
          \not∈L$ and $L$ is compatible with the latter inequality.
          $L$-deadness implies that $d_1$ is $L$-once as well.
          Hence we may apply the induction hypothesis to show the goal.
        \item \textbf{Case $\tr_2(\px) ⊑ \semusg{\pe_1'}_{\tr_2}$}:
          Then $\tr_2(\px) ⊑ \tr_2(\py)$ and hence
          $ω*\tr_2(\py) \not⊑ \semusg{\pe_2}_{\tr_2}$.

          To apply the induction hypothesis, we need to show that $μ$ is also
          $L'$-once, where $L' \triangleq L ∪ \{\pa_l\}$.
          Since $\pa_l$ is dead in $\rng(\pa_l)$, that is the case.
          Furthermore, we need to show that $\semevt{\pe_1'}_{ρ_2}$ is
          $L'$-once, but that follows by the induction hypothesis applied
          to $L'$ and $ω*\tr_2(\px) ⊑ \semusg{\pe_1}_{\tr_2}$.

          So it is the case that $\rng(μ[\pa_l↦\semevt{\pe_1'}_{ρ_2}])$ is
          $L'$-once.
          (Concerns about definedness can be overcome with
          \Cref{thm:heap-partitioning}.)
          Hence apply the induction hypothesis.

          We still need to show that we can weaken $L'$ back to $L$.
          Since $L'$ was extended for $\pa_l$ which is free in $\dom(μ_i)$,
          all rule applications can be rewritten without conflict.
      \end{itemize}
  \end{itemize}
\end{proof}

But $\semusg{\Let{x}{\pe_1}{\pe_2}}_{\tr_Δ}(x) ⊑ 1$ is just a sufficient condition for
the semantic property of ``evaluates $x$ at most once''; it is not a suitable
\emph{definition} of that property by far.
For example, $((\Lam{y}{x})~x)$ evaluates $x$ at most once, but is not recognised
as such by $\semusg{\wild}$.

Intuitively, we need a function $\mathit{count}_\pa : \Traces \to ℕ_ω$ that counts
$\LookupE(\pa)$ actions at a particular address $\pa$ over the course of a
head-normal form reduction and take the upper bound of this function in
arbitrary evaluation contexts.

We distribute these reponsibilities between the two functions $\usg$ and $\ctx$
in \Cref{fig:usg-abs}:

%Then, ``$\pa$ is evaluated at most once'' roughly corresponds to
%$\mathit{count}_\pa(\semevt{\pE[\pe_2]}_{[\px↦\pa]}([\pa↦\semevt{\pe_1}_{[\px↦\pa]}])) \leq 1$
%in arbitrary contexts such that $\pE$ does not capture $\px$ nor preoccupies $\pa$.

%This sketch has two flaws:
%The first is that the proposition and thus $\mathit{count}$ needs to know the
%address it should count, so we were forced to unfold the definition of
%$\semevt{\Let{\px}{\pe_1}{\pe_2}}_ρ(μ)$ without knowing $μ$, leading to new side
%conditions begging for an intricate definition, such as ``$\pE$ preoccupies
%$\pa$''.
%The second flaw is that $\pe_1$ can't have any free variables besides $\px$ this
%way, so it is not a faithful model of $\mathbf{let}$-binding and perhaps too
%weak to prove lockstep bisimilarity.

%Both flaws are a result of the need to escape and re-enter the syntactic (or,
%perhaps \emph{static}) realm of $\pE$ and $\pe_2$ to name the semantic (or
%\emph{dynamic}) address which the proposition needs to refer to.

%One solution to this problem is to extend the semantics function to evaluation
%contexts, with functionality
%$\semevt{\pE} : ((\Var \to \Addresses) \to \EventD) \to ((\Var \to \Addresses) \to \EventD)$
%such that $\semevt{\pE[\pe]}_ρ = \semevt{\pE}_ρ(\semevt{\pe})$.
%This equation has a solution because of compositionality
%and would allow for a faithful model of $\mathbf{let}$-binding in the hole of
%the context.
%A closed definition can be found in \Cref{fig:semevt-context}.
%
%\begin{toappendix}
%\begin{figure}
%\[
% \begin{array}{rcl}
%  \multicolumn{3}{c}{ \ruleform{ \semevt{\wild} \colon \Ctx → ((\Var \pfun \Addresses) → \EventD) → (\Var \pfun \Addresses) → \EventD } } \\
%  \\[-0.5em]
%  \semevt{\hole}_ρ(S) & = & S_ρ \\
%  \\[-0.5em]
%  \semevt{\pE~\px}_ρ(S)   & = & \begin{cases}
%    \AppIE(ρ(x)) \cons \apply(\semevt{\pE}_ρ(S), ρ(\px)) & \px ∈ \dom(ρ) \\
%    \fn{\wild}{\stuckend{}}  & \text{otherwise} \\
%  \end{cases} \\
%  \\[-0.5em]
%  \semevt{\Let{\px}{\pe}{\pE}}_ρ(S)(μ) & = &
%    \begin{letarray}
%      \text{let} & ρ' = ρ[\px ↦ \pa] \quad \text{where $\pa \not∈ \dom(μ)$} \\
%                 & d_1^\later = \semevt{\pe}_{ρ'} \\
%      \text{in}  & \BindE(\px,\pa↦d_1^\later) \cons \semevt{\pE}_{ρ'}(S)(μ[\pa ↦ \memo(\pa,d_1^\later)])
%    \end{letarray} \\
%  \\[-0.5em]
%  \semevt{\Let{\px}{\pE_1}{\pE_2[\px]}}_ρ(S)(μ) & = &
%    \begin{letarray}
%      \text{let} & ρ' = ρ[\px ↦ \pa] \quad \text{where $\pa \not∈ \dom(μ)$} \\
%                 & d_1^\later = \semevt{\pE_1}_{ρ'}(S) \\
%      \text{in}  & \BindE(\px,\pa↦d_1^\later) \cons \semevt{\pE_2[\px]}_{ρ'}(μ[\pa ↦ \memo(\pa,d_1^\later)])
%    \end{letarray} \\
%  \\[-0.5em]
%  \semevt{\Case{\pE}{\Sel[r]}}_ρ(S) & = &
%    \begin{letarray}
%      \text{let} & \alts = \fn{(K_i, \many{\pa})}{\CaseEE(K_i,\many{\px_i↦\pa}) \cons \semevt{\pe_{r_i}}_{ρ[\many{\px_i↦\pa}]}} \\
%      \text{in} & \CaseIE(\semevt{\pE}_ρ(S)) \cons \select(\semevt{\pe_s}_ρ, \alts)  \\
%    \end{letarray}
% \end{array}
%\]
%\caption{Eventful Semantics of Evaluation Contexts}
%\label{fig:semevt-context}
%\end{figure}
%\end{toappendix}

\begin{figure}
\[\begin{array}{c}
 \arraycolsep=3pt
 \begin{array}{rclcl@{\quad}rclcl}
  μ & ∈ & \Heaps^{\Look} & =   & \Addresses \pfun \later\Domain{\Look}
  &
  d & ∈ & \Domain{\Look} & =   & \Heaps^{\Look} \to \LookTraces \\
  u & ∈ & \Usg & =   & \{ 0 ⊏ 1 ⊏ ω \} ⊂ ℕ_ω
  &
  τ & ∈ & \LookTraces & ::=  & (l ∈ \Addresses \to \Usg) \lcons \someend{v,μ} \\
  \\[-0.75em]
 \end{array} \\
 \begin{array}{rclcl}
  v & ∈ & \Values{\Look} & ::=   & \FunV(f ∈ \Addresses \to \Domain{\Look}) \\
 \end{array} \\
 \\[-0.5em]
 \begin{array}{lcl}
  \multicolumn{3}{c}{ \ruleform{ \usg_{\Events} : \Events \to (\Addresses \to \Usg) \quad \usg_{\Traces} : \Traces \to \LookTraces } } \\
  \\[-0.5em]
  \usg_{\Events}(ε) & = & \begin{cases}
      [\pa↦1] & ε = \LookupT(\pa) \\
      \constfn{0} & \text{otherwise}
    \end{cases} \\
  l_1 +_1 (l_2 \lcons \someend{\tilde{v},\tm}) & = & (l_1+l_2) \lcons \someend{\tilde{v},\tm} \\
  \usg_{\Traces}(ε \cons τ) & = & \usg_{\Events}(ε) +_1 \usg_{\Traces}(τ) \\
  \usg_{\Traces}(\goodend{\FunV(f),μ}) & = & \constfn{0} \lcons \someend{\FunV(\usg_{\EventD} \circ f), \usg_\Heaps(μ)} \\
  \usg_{\Traces}(\stuckend{}) & = & \constfn{0} \lcons \someend{\bot_{\Values{\UsgD}}, (\constfn{\bot_\EventD}) \circ μ} \\
  \usg_{\Heaps}(μ) & = & \usg_\EventD \circ μ \\
  \usg^{⊣}_{\Heaps}(\tm) & = & \bigcup \{ μ \mid \usg_{\Heaps}(μ) ⊑ \tm \} \\
  \usg_{\EventD}(d) & = & \usg_\Traces \circ d \circ \usg^{⊣}_{\Heaps} \\
  \\[-0.5em]
 \end{array} \\
 \begin{array}{lcl}
  \multicolumn{3}{c}{ \ruleform{ \ctx_{\wild} : \pow{\Addresses} \to \LookTraces \to \Addresses \to \Usg } } \\
  \\[-0.5em]
  \ctx_A(l \lcons \someend{\FunV(\tilde{f}),\tm}) & = & l +
    \Lub_{\pa∈A} \{ ω*\ctx_A(\tilde{f}(\pa)(\tm)) \}  \\
%  \multicolumn{3}{c}{ \ruleform{ α : \Traces \to \UsgD } } \\
%  \\[-0.5em]
%  α(τ) & = & fst(intra(usg_\Traces(blub(τ)(\fn{\px}{[\px↦1]}))(τ))) \\
 \end{array}
\end{array}\]
\vspace{-1em}
\caption{Usage abstraction}
\label{fig:usg-abs}
\end{figure}

\begin{definition}[Usage cardinality]
  \label{defn:usg-card}
  A denotation $d$ evaluates an address $\pa$ \emph{at most $u$ times} (where
  $u∈\Usg$) if and only if, for all $μ$ such that $\pa$ is dead in $μ$,
  \[
    \ctx_{\dom(μ)}(\usg_\Traces(d(μ)))(\pa) ⊑ u.
  \]
  An expression $\pe$ evaluates $\px$ \emph{at most $u$ times} if and only if,
  for $\pa \not∈ \rng(ρ)$, $\semevt{\pe}_{ρ[\px↦\pa]}$ evaluates
  $\pa$ at most $u$ times.
\end{definition}

%Happily and unlike \Cref{defn:deadness3}, this definition does
%not need to relate executions of different heaps, thus it is
%a simple \emph{trace property} rather than a \emph{program
%property}~\citep{Cousot:21} (also called
%\emph{hyperproperty} by~\citet{ClarksonSchneider:10}).

The \emph{semantic usage abstraction} $\usg_\Traces : \Traces \to \LookTraces$
is the most precise usage analysis and is induced componentwise on
$\Traces, \Heaps,\EventD$ and $\EventV$ by the event abstraction $\usg_\Events$
in \Cref{fig:usg-abs}.

A $\LookTraces$ trace can then be folded by $\ctx_\LookTraces$ into
an $\Addresses \to \Usg$ mapping.%
\footnote{It is worth noting that for brevity we play fast and loose with guardedness
conditions for $\LookTraces$.
I.e., $\Let{loop}{\Let{\px}{\ttrue}{loop}}{loop}$ generates a diverging trace
for which $\usg_\Traces$ would only produce an $\Addresses \to \Usg$ mapping
after an infinite amount of time. A rigorous treatment would have each event
emit its own $l$ in a guarded fashion.}
Thus, $\ctx_\LookTraces$ continues a $\LookTraces$ in all possible evaluation
contexts, returning the least upper bound of all continued traces.

\begin{toappendix}
%\begin{figure}
%\[\begin{array}{c}
% \ruleform{ \correct_{\pa,u}(τ_1,τ_2) }
% \\
% \\[-0.5em]
% \inferrule*[right=\corstuck]
%    {\quad}
%    {\correct_{\pa,u}(\stuckend{},\stuckend{})}
% \quad
% \inferrule*[right=\corcons]
%    {\later (\correct_{\pa,u}(τ_1,τ_2))}
%    {\correct_{\pa,u}(\wild :: τ_1,\wild :: τ_2)}
% \\
% \\[-0.5em]
% \inferrule*[right=\corfun]
%    {\dom(μ_1) = \dom(μ_2) \quad \forall \pa' ∈ \dom(μ_i).\ \usg_\Traces(\correct_{\pa,u}(f_1(\pa)(μ_1), f_2(\pa)(μ_2))}
%    {\correct_{\pa,u}(\goodend{\FunV(f_1),μ_1},\goodend{\FunV(f_2),μ_2})}
% \\
% \\[-0.5em]
%\end{array}\]
%\caption{Correctness relation for \Cref{thm:usg-by-name}}
%  \label{fig:semusg-correct2}
%\end{figure}
\end{toappendix}

%Rather than to let syntactic evaluation contexts dictate how $\pe_2$ might be
%evaluated to produce (essentially first-order) traces that we count
%$\LookupE(\pa)$ actions in, $\usg$ lifts the counting of addresses through the
%whole domain, including function values.

%\begin{lemma}[Heap extension preserves improvement]
%  If $A ⊦_0 d_1(μ_1) \lesssim d_2(μ_2)$, then $A ⊦_0 d_1(μ_1[\pa↦d_1]) \lesssim d_2(μ_2[\pa↦d_2])$.
%\end{lemma}

\begin{theoremrep}[Update avoidance]
  \label{thm:usg-by-name}
  Let $\Let{\px}{\pe_1}{\pe_2}$ be an expression such that $\pe_2$ evaluates $\px$
  at most once and $\px$ is dead in $\pe_1$.
  Then
    $\semevt{\Let{\px}{\pe_1}{\pe_2}}_ρ \impequiv
     \semevt{\Letn{\px}{\pe_1\tick}{\pe_2}}_ρ$.
\end{theoremrep}
\begin{proof}
  Not done yet. It should be similar to \Cref{thm:semusg-by-name},
  with a suitably adjusted correctness predicate.
  We hope to provide a complete the proof by October.
%  The setup is quite similar to \Cref{thm:semusg-by-name}:
%
%  We prove $(\impequiv)$ by rule $\impdenot$.
%  After unfolding the definitions of $\semevt{\wild}$ for $\mathbf{let}$ and
%  $\mathbf{let1}$, the goal is to show
%  \[
%    \dom(μ_i) ⊦_0 \semevt{\pe_2}_{ρ[\px↦\pa]}(μ_1[\pa↦d_1]) \impsim \semevt{\pe_2}_{ρ[\px↦\pa]}(μ_2[\pa↦d_2])
%  \]
%  where $\pa \not∈\dom(μ_i)$,
%  $\dom(μ_i) ∪ \{\pa\} ⊦ \semevt{\pe_2}_{ρ[\px↦\pa]}$,
%  $μ_1 \impapprox μ_2$.
%  Clearly,
%  \[
%    \dom(μ_i) ⊦_0 \semevt{\pe_2}_{ρ[\px↦\pa]}(μ_1[\pa↦\highlight{d_2}]) \impsim \semevt{\pe_2}_{ρ[\px↦\pa]}(μ_2[\pa↦d_2])
%  \]
%  holds by reflexivity because $μ_1[\pa↦d_2] \impapprox μ_2[\pa↦d_2]$, so the goal becomes
%  \[
%    \dom(μ_i) ⊦_0 τ_1 \impsim τ_2,
%  \]
%  abbreviating $τ_i \triangleq \semevt{\pe_2}_{ρ[\px↦\pa]}(μ_{\highlight{\scriptstyle 1}}[\pa↦d_i])$.
%
%  \noindent
%  We will prove the correctness predicate $\correct_{\{\pa\}}(τ_1,τ_2)$ instead, where
%  $\correct$ is defined in \Cref{fig:semusg-correct2}.
%
%  Since $\pa \not ∈ \dom(μ_1)$,
%  $\corfun$ implies $\impfun$ on $\dom(μ_i)$ and the goal follows.
%
%  (Same setup until here.)
%
%  Since $\pe_2$ evaluates $\px$ at most once and $\pa$ is dead in
%  $μ_1[\pa↦d_i]$ by assumption that $x$ is dead in $\pe_1$, we have
%  $\ctx_\LookTraces(\usg_\Traces(τ_i))(\pa) ⊑ 1$.
%  Let us also abbreviate $ρ_1 \triangleq ρ[\px↦\pa]$.

%  We show that $\semevt{\pe_2}_{ρ[\px↦\pa]}$ is $\{\pa\}$-once
%  assuming that $ω*\tr_1(\px) \not⊑ \semusg{\pe_2}_{\tr_1}$.
%  By induction on $\pe_2$, generalising to $L$-onceness and
%  maintaining that $\tr_1(\px) \not⊑ \tr_1(\py) \Longleftrightarrow ρ(\py) \not∈ L$
%  for all $\py∈ \dom(ρ)$.
%  \begin{itemize}
%    \item \textbf{Case $\pe_2 = \py$}:
%      Fix $μ$ such that $\rng(μ)$ is $L$-once and for all $\pa' ∈ L$ such that $\pa'$ is dead in $d'$,
%      we need to show that
%      \[
%        \correct_{L}(μ_1(ρ(\py))(μ_1), μ_2(ρ(\py))(μ_2)).
%      \]
%      where $μ_1 \triangleq μ[\pa'↦d'\tick ], μ_2 = μ[\pa'↦\memo(\pa',d')]$.
%      This is easy to see when $ρ(\py) \not= \pa'$, so assume $ρ(\py) = \pa'$:
%      \[
%        \correct_{L}(d'(μ_1) \cons \TickE, d'(μ_2) \betastep \fn{v}{(\fn{μ'}{\UpdateE(...) \cons \goodend{v,μ'[\pa'↦\memo(\pa',\ret(v))]}})}).
%      \]
%      We can strip away the trailing transition, leaving behind just the heap update.
%
%      Now, without loss of generality, we will assume that $L$ is dead in
%      $\rng(μ)$.
%      If that was not the case, we could apply \Cref{heap-partitioning}
%      to hide all entries in which $L$ is live ``inside $d'$'' and allocate them
%      on first execution.
%      % Turn into lemma. Sounds useful
%
%      When $L$ is dead in $\rng(μ)$, we can rewrite the entry for $\pa'$ freely
%      without affecting $\correct$.
%      In the interesting case we have $\bigstep{d'}{μ_i}{\FunV(f_i)}{μ_i'}$.
%      Because of deadness, we must have
%      \[
%        \correct_{L}(d'(μ_1), d'(μ_2)).
%      \]
%      That implies
%      \[
%        \correct_{L}(f_1(\pa_a)(μ_1'), f_2(\pa_a)(μ_2')).
%      \]
%      for all $\pa_a ∈ \dom(μ_i') \setminus L$.
%      Now it suffices to show the situation after heap update
%      \begin{equation}
%        \label{eqn:usg-var-goal}
%        \correct_{L}(f_2(\pa_a)(μ_2'), f_2(\pa_a)(μ_2'[\pa'↦\memo(\pa',\ret(\FunV(f_2)))])).
%      \end{equation}
%      In particular, note that
%      $μ_2 \forcesto μ_3 \triangleq μ[\pa'↦\memo(\pa',\ret(\FunV(f_2)))]$
%      (again, we can properly hide any free address that $f_2$ needs in
%      $μ_3(\pa')$), so $\bigstep{d'}{μ_3}{\FunV(f_2)}{μ_3'}$ with
%      $μ_2' \forcesto μ_3'$ by the Postpone property.
%      By deadness we can follow
%      \[
%        \correct_{L}(d'(μ_2), d'(μ_3)).
%      \]
%      and since $\bigstep{d'}{μ_3}{\FunV(f_2)}{μ_3'}$ (NB: $f_2$ due to the
%      forcing relationship), that implies
%      \[
%        \correct_{L}(f_2(\pa_a)(μ_2'), f_2(\pa_a)(μ_3')).
%      \]
%      for all $\pa_a ∈ \dom(μ_i') \setminus L$.
%
%      Due to forcing, it must be that $μ_3'(\pa) = μ_3(\pa)$.
%      Again by forcing, $μ_2'(\pa)$ must be either
%      $μ_2(\pa)$ or $μ_3'(\pa)$.
%
%      Since the number of steps taken for $f_2(\pa_a)(μ_2'[\pa↦μ_3'(\pa)])$ must be between $f_2(\pa_a)(μ_2')$
%      and $f_2(\pa_a)(μ_3')$ (due to forcing), this shows the goal \Cref{eqn:usg-var-goal}.
%
%    \item \textbf{Case $\pe_2 = \pe~\py$}:
%      Fix $μ$ such that $\rng(μ)$ is $L$-once and for all $\pa' ∈ L$ such that $\pa'$ is dead in $d'$,
%      we need to show that
%      \[
%        \correct_{L}(\semevt{\pe~\py}_{ρ_1}(μ[\pa'↦ d'\tick]), \semevt{\pe~\py}_{ρ_1}(μ[\pa'↦\memo(\pa',d')])).
%      \]
%
%      From $ω*\tr_1(\px) \not⊑ \semusg{\pe}_{\tr_1} + ω*\tr_1(\py)$ we can see that
%      $ω*\tr_1(\px) \not⊑ \semusg{\pe}_{\tr_1}$ and $\tr_1(\px) \not⊑ \tr_1(\py)$ by
%      monotonicity of $+$ and $*$.
%      By induction, we know that the correctness statement holds for
%      $\semevt{\pe}_{ρ_1}$.
%      We also know that $ρ(\py) \not∈ L$ because $\tr_1(\px) \not⊑ \tr_1(\py)$.
%
%      In the interesting case that $\bigstep{\semevt{\pe}_{ρ_1}}{μ[\pa'↦...]}{\FunV(f_i)}{μ_i'}$,
%      we have
%      \[
%        \correct_{L}(f_1(ρ(\py))(μ_1), f_2(ρ(\py))(μ_2))
%      \]
%      for all $\pa' ∈ L$, because $ρ(\py) \not∈ L$.
%      This shows the goal.
%
%    \item \textbf{Case $\pe_2 = \Lam{\py}{\pe}$}:
%      Fix $μ$ such that $\rng(μ)$ is $L$-once and for all $\pa' ∈ L$ such that $\pa'$ is dead in $d'$.
%      To apply $\corfun$, we need to show that
%      \[
%        \correct_{L}(\semevt{\pe}_{ρ_1[\py↦\pa_a]}(μ[\pa'↦ d'\tick]), \semevt{\pe}_{ρ_1[\py↦\pa_a]}(μ[\pa'↦\memo(\pa,d')]))
%      \]
%      for all $\pa_a ∈ (\dom(μ) \setminus L)$.
%
%      Since $\pa_a \not∈ L$ we maintain the invariant on $ρ_1[\py↦\pa_a]$ and
%      $\tr_1(\px) \not⊑ \bot = \tr_1[\py↦\bot](\py)$ and may apply the induction hypothesis
%      to the above situation since $ω*\tr_1[\py↦\bot](\px) \not⊑ \semusg{\pe}_{\tr_1[\py↦⊥]}$.
%      This shows the goal.
%
%    \item \textbf{Case $\pe_2 = \Let{\py}{\pe_1'}{\pe_2'}$}:
%      Fix $μ$ such that $\rng(μ)$ is $L$-once and for all $\pa' ∈ L$ such that $\pa'$ is dead in $d'$.
%      We need to show that
%      \[
%        \correct_{L}(\semevt{\Let{\py}{\pe_1'}{\pe_2'}}_{ρ_1}(μ[\pa'↦ d'\tick]), \semevt{\Let{\py}{\pe_1'}{\pe_2'}}_{ρ_1}(μ[\pa'↦\memo(\pa,d')])).
%      \]
%      We unfold the definition of $\semevt{\wild}$ and see that we need to prove
%      \[
%        \correct_{L}(\semevt{\pe_2'}_{ρ_2}(μ[\pa'↦ d'\tick][\pa_l↦d_1]), \semevt{\pe_2'}_{ρ_2}(μ[\pa'↦\memo(\pa,d')][\pa_l↦d_1])),
%      \]
%      where $ρ_2 \triangleq ρ_1[\py↦\pa_l]$,
%      $d_1 \triangleq \memo(\pa_l,\semevt{\pe_1'}_{ρ_1[\py↦\pa_l]})$.
%
%      From $\px \not= \py$ (otherwise shadowing) we see that
%      $\tr_1(\px) = \tr_2(\px)$, \\
%      where
%      ${\tr_2 \triangleq \fix(\fn{\tr_2}{\tr_1 ⊔ [\py ↦
%      [\py↦1]+\semusg{\pe_1'}_{\tr_2}]})}$.
%      It is clear that $ω*\tr_2(\px) = ω*\tr_1(\px) \not⊑ \semusg{\pe_2}_{\tr_1} = \semusg{\pe_2'}_{\tr_2}$,
%      so we want try to apply the induction hypothesis to this situation,
%      but that requires us to show that the invariant on $L$ is satisfied
%      as well as that $d_1$ is $L$-once.
%
%      We proceed by cases over $\tr_1(\px) = \tr_2(\px) ⊑ \semusg{\pe_1'}_{\tr_2}$.
%      \begin{itemize}
%        \item \textbf{Case $ω*\tr_2(\px) ⊑ \semusg{\pe_1'}_{\tr_2}$}:
%          Then $ω*\tr_2(\px) ⊑ \tr_2(\py) \not⊑ \semusg{\pe_2'}_{\tr_2}$ and
%          hence $\py$ is dead in $\pe_2'$, which allows us to rewrite its bindings
%          in $μ$ with something that is dead in $L$.
%          (It is easy to see that deadness is compatible with $\correct$ in this
%          way.)
%          That again makes it possible to apply the induction hypothesis
%          (noting that now $\tr(\px) \not⊑ \tr_2(\py)$ and hence $\pa_l \not∈ L$),
%          showing the goal.
%        \item \textbf{Case $\tr_2(\px) \not⊑ \semusg{\pe_1'}_{\tr_2}$}:
%          Then $\px$ is dead in $\pe_1$, $\tr_2(\px) \not⊑  \tr_2(\py)$.
%          This implies that $L$ is dead in $\semevt{\pe_1'}_{ρ_2}$:
%          For any $\pa'$, either there exists $\py$ such that $\pa' = ρ(\py)$
%          in which case $\tr_2(\px) ⊑  \tr_2(\py)$ leads to a contradiction.
%          Otherwise, there is no variable in scope that refers to $\pa'$ and hence
%          $\pa'$ is dead in $\semevt{\pe_1'}_{ρ_2}$ by \Cref{thm:addr-dom-sem}.
%
%          From $\tr_2(\px) \not⊑ \tr_2(\py)$ we conclude $ρ_2(\py) = \pa_l
%          \not∈L$ and $L$ is compatible with the latter inequality.
%          $L$-deadness implies that $d_1$ is $L$-once as well.
%          Hence we may apply the induction hypothesis to show the goal.
%        \item \textbf{Case $\tr_2(\px) ⊑ \semusg{\pe_1'}_{\tr_2}$}:
%          Then $\tr_2(\px) ⊑ \tr_2(\py)$ and hence
%          $ω*\tr_2(\py) \not⊑ \semusg{\pe_2}_{\tr_2}$.
%
%          To apply the induction hypothesis, we need to show that $μ$ is also
%          $L'$-once, where $L' \triangleq L ∪ \{\pa_l\}$.
%          Since $\pa_l$ is dead in $\rng(\pa_l)$, that is the case.
%          Furthermore, we need to show that $\semevt{\pe_1'}_{ρ_2}$ is
%          $L'$-once, but that follows by the induction hypothesis applied
%          to $L'$ and $ω*\tr_2(\px) ⊑ \semusg{\pe_1}_{\tr_2}$.
%
%          So it is the case that $\rng(μ[\pa_l↦\semevt{\pe_1'}_{ρ_2}])$ is
%          $L'$-once.
%          (Concerns about definedness can be overcome with
%          \Cref{thm:heap-partitioning}.)
%          Hence apply the induction hypothesis.
%
%          We still need to show that we can weaken $L'$ back to $L$.
%          Since $L'$ was extended for $\pa_l$ which is free in $\dom(μ_i)$,
%          all rule applications can be rewritten without conflict.
%      \end{itemize}
%  \end{itemize}
\end{proof}

Similarly, one could prove that a binding is dead when it is evaluated at most 0
times.
An alternative correctness proof for $\semusg{\wild}$ can then be formulated by
\emph{abstract interpretation}, proving it correct simultaneously as a deadness
and a sharing analysis:

\begin{theoremrep}[$\semusg{\wild}$ approximates semantic usage]
  \label{thm:semusg-correct-3}
  Let $\pe$ be an expression, $\px$ a variable, $\pa$ an address
  that is dead in a heap $μ$ and $ρ$ an environment such that $\pa \not∈
  \rng(ρ)$.
  Then
  \[
    \ctx_{\dom(μ)}(\usg_\Traces(\semevt{\pe}_{ρ[\px↦\pa]}(μ)))(\pa) ⊑ \semusg{\pe}_{\tr_Δ}(\px)
  \]
\end{theoremrep}
\begin{proof}
  Not done yet (deadline is in 30 minutes). Here's what we've got so far:

  We first get rid of the concrete heap by the following estimate:
  \[
    \ctx_{D}(\usg_\Traces(\semevt{\pe}_{ρ[\px↦\pa]}(μ)))(\pa) ⊑ \ctx_{D}(\usg_\EventD(\semevt{\pe}_{ρ[\px↦\pa]})(\usg_\Heaps(μ)))(\pa)
  \]
  for $D \triangleq \dom(μ)$.
  This holds because $μ ∈ \usg^⊣_\Heaps(\usg_\Heaps(μ))$, exploiting that
  $\usg_\Heaps$ is left adjoint to $\usg^⊣_\Heaps$.

  By induction on $\pe$, we show that
  \[
    \ctx_{D}(\usg_\EventD(\semevt{\pe}_{ρ[\px↦\pa]})(\tm))(\pa) ⊑ \semusg{\pe}_{\tr}(\px)
  \]
  for all $\px$, maintaining the invariant that $\pa \not∈ \rng(ρ)$ and $\pa$ is
  dead in $\tm(D)$
  (so for all $\pa' ∈ D.\ \ctx_{D}(\tm(\pa')(\tm))(\pa) = 0$),
  as well as
  \begin{equation}
    \label{eqn:semusg-rho}
    ∀\py.\ \tr_Δ(\py)(\px) + \ctx_D(\tm(ρ[\px↦\pa](\py))(\tm))(\pa) ⊑ \tr(\py)(\px)
  \end{equation}
  Initially, $D = \dom(μ), \tr = \tr_Δ$.
  We abbreviate
  $u_l \triangleq \ctx_{D}(\usg_\EventD(\semevt{\pe}_ρ)(\tm))(\pa)$,
  ${u_r \triangleq \semusg{\pe}_{\tr}(\px)}$ and
  ${ρ_1 \triangleq ρ[\px↦\pa]}$.
  \begin{itemize}
    \item \textbf{Case $\pe = \py$}:
      The stuck case is clear, because then $u_l = 0$.

      When $\px = \py$:
      \[\begin{WithArrows}
              & u_l \Arrow{definition of $\semevt{\px}$} \\
        {}={} & 1 + \ctx_{D}(\tm(ρ_1(\px))(\tm))(\pa) \Arrow{definition of $\tr_Δ(\px)$} \\
        {}={} & \tr_Δ(\px)(\px) + \ctx_{D}(\tm(ρ_1(\px))(\tm))(\pa) \Arrow{\Cref{eqn:semusg-rho}} \\
        {}⊑{} & \tr(\px)(\px) \Arrow{definition of $\semusg{\px}_{\tr}$} \\
        {}={} & u_r
      \end{WithArrows}\]

      When $\px \not= \py$:
      \[\begin{WithArrows}
              & u_l \Arrow{definition of $\semevt{\px}, ρ(\px) \not=ρ(\py)$} \\
        {}={} & 0 + \ctx_{D}(\tm(ρ_1(\px))(\tm))(\pa) \Arrow{definition of $\tr_Δ(\px)$} \\
        {}={} & \tr_Δ(\py)(\px) + \ctx_{D}(\tm(ρ_1(\py))(\tm))(\pa) \Arrow{\Cref{eqn:semusg-rho}} \\
        {}⊑{} & \tr(\px)(\px) \Arrow{definition of $\semusg{\py}_{\tr}$} \\
        {}={} & u_r
      \end{WithArrows}\]
      %$\begin{WithArrows}
      %        & u_l & \\
      %  {}={} & \ctx_{D}(\tm(ρ_1(\px))(\tm))(\pa) & \hspace{5em} \Lbag\text{definition of $\semevt{\py}, ρ(\px) \not=ρ(\py)$}\Rbag \\
      %  {}={} & \tr_Δ(\px)(\px) + \ctx_{D}(\tm(ρ_1(\px))(\tm))(\pa) & \hspace{5em} \Lbag\text{definition of $\tr_Δ(\px)$}\Rbag \\
      %  {}⊑{} & \tr(\px)(\px) & \hspace{5em} \Lbag\text{\Cref{eqn:semusg-rho}}\Rbag \\
      %  {}={} & u_r & \hspace{5em} \Lbag\text{definition of $\semusg{\px}_{\tr}$}\Rbag
      %\end{WithArrows}$

    \item \textbf{Case $\pe = \pe'~\py$}:
      If $\tr(\py)(\px) \not= 0$, then $u_r = ω$, the top element, and the goal follows.

      Otherwise, $\tr(\py)(\px) = 0$
      and hence
      $\ctx_D(\tm(ρ_1(\py))(\tm))(\pa) = 0$ by \Cref{eqn:semusg-rho}.

      Let us define $l \lcons \someend{\FunV(\tilde{f}),\tm'} \triangleq \usg_\EventD(\semevt{\pe'}_{ρ_1})(\tm)$
      and $D' \triangleq D ∪ \{ ρ_1(\py) \}$.
      We get
      \begin{DispWithArrows*}
            & u_l \Arrow{definition of $\semevt{\pe'~\py}$, drop $\AppIE$} \\
        ={} & \ctx_{D}(\usg_\EventD(\apply(\semevt{\pe'}_{ρ_1},ρ_1(\py)))(\tm)) \Arrow{definition of $\apply,\betastep,l,\tilde{f},\tm$} \\
        ={} & l + \ctx_{D}(\tilde{f}(ρ_1(\py))(\tm')) \Arrow{$D ⊆ D'$ and monotonicity of $\ctx$} \\
        ⊑{} & l + \ctx_{D'}(\tilde{f}(ρ_1(\py))(\tm')) \Arrow{$ω*$ is increasing} \\
        ⊑{} & l + ω*\ctx_{D'}(\tilde{f}(ρ_1(\py))(\tm')) \Arrow{$ρ_1(\py) ∈ D'$} \\
        ⊑{} & \ctx_{D'}(l \lcons \someend{\tilde{f},\tm'}) \Arrow{definition of $\usg_\EventD(\semevt{\pe'}_{ρ_1})$} \\
        ={} & \ctx_{D'}(\usg_\EventD(\semevt{\pe'}_{ρ_1})(\tm)) \Arrow{induction hypothesis for $D'$} \\
        ⊑{} & 0 + \semusg{\pe'}_{\tr}(\px) \Arrow{$\tr(\py)(\px) = 0$} \\
        ={} & u_r
      \end{DispWithArrows*}

    \item \textbf{Case $\pe = \Lam{\py}{\pe'}$}:
      Unfolding $\ctx_D$ and $\semusg{\Lam{\py}{\pe'}}_{\tr}$ once, we have
      \[\begin{array}{l}
        u_l = \Lub_{\pa'∈D} \{ ω*\ctx_{D}(\usg_\EventD(\semevt{\pe'}_{ρ_1[\py↦\pa']})(\tm))(\pa) \} \\
        u_r = ω*\semusg{\pe'}_{\tr[\py↦\bot]}(\px) \\
      \end{array}\]
      So we show
      \[
        \ctx_{D}(\usg_\EventD(\semevt{\pe'}_{ρ_1[\py↦\pa']})(\tm))(\pa) ⊑
        \semusg{\pe'}_{\tr[\py↦\bot]}(\px)
      \]
      for all $\pa' ∈ D$, the rest follows by definition of the upper bound and
      monotonicity of $ω*$.

      We want to invoke the induction hypothesis for $D$ in this situation.
      In order to do that, we have to show \Cref{eqn:semusg-rho} for $\py$:
      \begin{DispWithArrows*}
            & \tr_Δ(\py)(\px) + \ctx_D(\tm(ρ_1[\py↦\pa'](\py))(\tm))(\pa) \Arrow{definition of $\tr_Δ$, $\px\not=\py$} \\
        ={} & 0 + \ctx_D(\tm(\pa')(\tm))(\pa) \Arrow{$\pa'∈D$} \\
        ={} & 0 + 0 \Arrow{simplify} \\
        ={} & 0  \Arrow{simplify} \\
        ={} & \tr[\py↦\bot]
      \end{DispWithArrows*}
      The induction hypothesis shows the goal.

    \item \textbf{Case $\pe = \Let{\py}{\pe_1'}{\pe_2'}$}:
      Unfolding $\ctx_D$ and $\semusg{\Let{\py}{\pe_1'}{\pe_2'}}_{\tr}$ once, we have
      \[\begin{array}{l}
        u_l = \ctx_{D}(\usg_\EventD(\semevt{\pe_2'}_{ρ_1[\py↦\pa']})(\tm[\pa'↦\td_1)(\pa) \} \\
        u_r = \semusg{\pe_2'}_{\tr'}(\px) \\
      \end{array}\]
      where $\td_1 \triangleq \usg_\EventD(\semevt{\pe_1'}_{ρ_1[\py↦\pa']}),
             \tr' \triangleq \fix(\fn{\tr'}{\tr ⊔ [\py ↦ [\py↦1]+\semusg{\pe_1'}_{\tr'}]})$.

      When $\semusg{\pe_1'}_{\tr'}(\px) = 0$, we need to add $\pa'$ to $D$.
      \sg{Do we really? It's not necessary I think, in contrast to the lambda case.} \\
      Before we can apply the induction hypothesis, we must verify that
      \Cref{eqn:semusg-rho} holds for the new entry $\py$ (the others hold by an
      address space argument).
      Since we need to apply the induction hypothesis at $\pe_1$ for that, we need
      to show that $\ctx_D(\td_1)$ is a lower bound to the least fixpoint
      $\tr'(\py)(\px) = (\Glb \{ d \mid \semusg{\pe_1}_{\tr[\py ↦ d]} ⊑ d \})(\px)$.\\
      For that it suffices to show $\ctx_D(\td_1) ⊑ d$ whenever $\semusg{\pe_1}_{\tr[\py ↦ d]} ⊑ d$.
      We note that in this case, the condition on $\py$ holds so we can apply
      the induction hypothesis to get $\ctx_D(\td_1) ⊑ $show the .
      \begin{DispWithArrows*}
            & \tr_Δ(\py)(\px) + \ctx_D(\tm[\pa'↦d_1](ρ_1[\py↦\pa'](\py))(\tm[\pa'↦d_1]))(\pa) \Arrow{definition of $\tr_Δ$, $\px\not=\py$} \\
        ={} & 0 + \ctx_D(d_1(\tm[\pa'↦d_1]))(\pa) \Arrow{} \\
        % Want to apply induction hypothesis, but can't without this proof.
        % Perhaps we can by fixpoint induction!
        % E.g., for initial \tr'(\py) = \bot, it is clear.
        % Now imagine we have proven for \tr'_1 and want to show for iteration \tr'_2.
        % Case 0 is clear. Case 1: try with \tr' =
        ={} & \tr'(\py)
      \end{DispWithArrows*}

      \begin{DispWithArrows*}
            & \tr_Δ(\py)(\px) + \ctx_D(\tm[\pa'↦d_1](ρ_1[\py↦\pa'](\py))(\tm[\pa'↦d_1]))(\pa) \Arrow{definition of $\tr_Δ$, $\px\not=\py$} \\
        ={} & 0 + \ctx_D(d_1(\tm[\pa'↦d_1]))(\pa) \Arrow{} \\
        % Want to apply induction hypothesis, but can't without this proof.
        % Perhaps we can by fixpoint induction!
        % E.g., for initial \tr'(\py) = \bot, it is clear.
        % Now imagine we have proven for \tr'_1 and want to show for iteration \tr'_2.
        % Case 0 is clear. Case 1: try with \tr' =
        ={} & \tr'(\py)
      \end{DispWithArrows*}

      We proceed by cases over $\tr(\px) ⊑ \semusg{\pe_1'}_{\tr'}$.
      \begin{itemize}
        \item \textbf{Case $ω*\tr_2(\px) ⊑ \semusg{\pe_1'}_{\tr_2}$}:
        \item \textbf{Case $\tr_2(\px) \not⊑ \semusg{\pe_1'}_{\tr_2}$}:
        \item \textbf{Case $\tr_2(\px) ⊑ \semusg{\pe_1'}_{\tr_2}$}:
      \end{itemize}
  \end{itemize}
\end{proof}

A more elegant proof would continually abstract $\ctx_A \circ \usg_\Traces$
without relating to the structure of $\semevt{\wild}$ or $\semusg{\wild}$ at
all, encoding the strengthened inductive hypothesis (\ie, the logical relation)
of the proof above in a chain of Galois connections. Then the above proof
is simply by direct, structural induction on $\pe$ and equational reasoning
-- \citet{Cousot:21} calls this process ``calculational design''.
We want to attempt such a proof in the future and are confident that it will
yield insightful abstraction functions, for example encoding the transition
from heap-based call-by-need to call-by-name or the transition from
interprocedural to intraprocedural analysis.
